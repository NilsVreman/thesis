% Dokumentdatablad för doktorsavhandlingar
% ==============================
% Uppgifterna fylls i på engelska inom {}.
% övriga uppgifter är beroende av rapportens art.
%
% \name          Ex: FINAL REPORT , MASTER THESIS , INTERNAL REPORT
% \date          Månad och årtal, ex: October 1985
% \num           Nummer (av Mika), dvs endast siffrorna.
% \author        Namn på författare
% \supervisor    Handledare
% \title         Rapportens titel (ev med engelsk övers.)
% \keywords      Eventuella nyckelord
% \language      Det språk rapporten är skriven på
% \pages         Totalt antal sidor
% \begin{abstract}...\end{abstract} Abstract på engelska
%
%------------------------------------------------------------------------------
%
\documentclass{docdataphd}
\usepackage[utf8]{inputenc}
\usepackage{schoolbook} % använd samma som i avhandlingen

\begin{document}
\name{DOCTORAL DISSERTATION}
\date{June 2023}
\num{1141}
\author{Nils Vreman}
\supervisor{Martina Maggio}
\sponsor{}
\title{Analysis of Embedded Controllers Subject to Computational Overruns}
\keywords{Real-time systems, Embedded systems, Microcontroller, Fault-tolerance}
\classification{}
\supplement{}
\isbn{978-91-8039-688-2 (print), 978-91-8039-687-5 (web)}  
\lang{English}
\pages{1--200}
\security{}
\recipient{}
\begin{abstract}
    \vspace{-4mm}
    \normalsize
% Setting the context
Microcontrollers have become an integral part of modern everyday embedded systems, such as smart bikes, cars, and drones.
Typically, microcontrollers operate under real-time constraints, which require the timely execution of programs on the resource-constrained hardware.
As embedded systems are becoming increasingly more complex, microcontrollers run the risk of violating their timing constraints, i.e., overrunning the program deadlines.
Breaking these constraints can cause severe damage to both the embedded system and the humans interacting with the device.
% What is missing from the context
Therefore, it is crucial to analyse embedded systems properly to ensure that they do not pose any significant danger if the microcontroller overruns a few deadlines.

% What methods have I used to fill this hole
However, there are very few tools available for assessing the safety and performance of embedded control systems when considering the implementation of the microcontroller.
This thesis aims to fill this gap in the literature by presenting five papers on the analysis of embedded controllers subject to computational overruns.
Details about the real-time operating system's implementation are included into the analysis, such as what happens to the controller's internal state representation when the timing constraints are violated.
The contribution includes theoretical and computational tools for analysing the embedded system's stability, performance, and real-time properties.

% What are the main results of the thesis
The embedded controller is analysed under three different types of timing violations: blackout events (when no control computation is completed during long periods), weakly-hard constraints (when the number of deadline overruns is constrained over a window), and stochastic overruns (when violations of timing constraints are governed by a probabilistic process).
These scenarios are combined with different implementation policies to reduce the gap between the analysis and its practical applicability.
The analyses are further validated with a comprehensive experimental campaign performed on both a set of physical processes and multiple simulations.

% What is the impact of this thesis?
In conclusion, the findings of this thesis reveal that the effect deadline overruns have on the embedded system heavily depends the implementation details and the system's dynamics.
Additionally, the stability analysis of embedded controllers subject to deadline overruns is typically conservative, implying that additional insights can be gained by also analysing the system's performance.
\end{abstract}
\end{document}
