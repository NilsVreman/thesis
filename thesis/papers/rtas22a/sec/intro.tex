Computer-controlled systems are prime instances of real-time systems~\cite{Oshana:2006, akesson:2020}.
Due to the tight interconnection between the environment, hardware, and software, designing such systems has been considered challenging.
Part of this challenge resides in the design of the real-time software, specifically considering both the normal operation~\cite{Lozoya:2013,Aminifar:2011} (e.g., correct output computation) and system malfunctions~\cite{Caccamo:2002,Ramanathan:1997} (e.g., temporary overloads).
For this reason, the research community has undertaken a significant effort to merge design choices on the algorithmic side and on the real-time implementation side.

In computer-controlled systems, algorithms are developed using control theory~\cite{Astrom:2008}.
The theory offers strong and practically relevant formal guarantees, but also makes strict assumptions on the real-time execution of the implemented algorithm.
These assumptions are naturally translated into periodic tasks with hard deadlines.
However, meeting every single deadline in a periodic control task is not necessary \cite{Ramamritham:1996,Ramanathan:1997}.
Instead, the timing requirements are the result of design choices and engineering trade-offs between resource utilisation and performance \cite{Lozoya:2013,Cervin:2004}.

Co-design approaches have been proposed to maximise the control performance while minimising the real-time resource utilisation~\cite{Marti:2001,Rehbinder:2000}.
However, the tight integration between control algorithms and the real-time implementation has limitations, due to the complexity of the resulting systems.
This complexity generally translates into
\begin{enumerate*}[label=(\roman*)]
    \item complex design methodologies,
    \item conservative results, and
    \item strong assumptions on the system properties.
\end{enumerate*}

Differently from existing approaches, in this paper we avoid the additional complexity by \emph{automatically} adapting the real-time implementation of the controller.
The approach is inspired by the concept of autotuning, originating in the control literature~\cite{Astrom:1984, Hagglund:1983}.
Autotuning was developed to simplify the PID control design process by automatically optimising the controller's design parameters.
This idea is here translated to the real-time implementation, where instead the predesigned control algorithm is automatically adapted for the real-time architecture to minimise the design effort.

To enable an automatic adaptation of the control algorithm for a wide range of systems, we make as few assumptions about the implementation platform as possible.
The system requirements that we pose are nonintrusive and seen in industrial applications~\cite{akesson:2020}.
The real-time operating system is required to be able to: schedule periodic tasks with implicit deadlines, abort (\tK{}) instances of tasks that miss their deadlines, roll back the state of aborted tasks~\cite{Zhang:2003,Seong:2001}, and handle controller inputs and outputs at task release times~\cite{Kirsch:2012, Ernst:2018}.
While previous co-design approaches require, e.g., probabilistic or weakly hard descriptions of deadline misses~\cite{Pazzaglia:2019,Chakraborty:2014}, our adaptation approach works for any deadline miss model.
From the point of view of the controller, we assume a linear time-invariant control law, but we require no prior information about the control design, nor about the system to be controlled.
Similarly to~\cite{Pazzaglia:2021}, our adaptive control implementation is applicable to general linear discrete-time controllers and not only static ones.

% paragraph on the analysis and the empirical campaign
To evaluate the performance of our adaptive implementation, we propose a stochastic analysis of the control system.
The analysis assumes a probabilistic model of the deadline misses; however, it is agnostic to how the model is obtained.
We utilise this analysis to evaluate our approach on both a \emph{real} system and a benchmark set of $268$ \emph{simulated} control systems from the process industrial domain.
We use the former to evaluate the practical relevance of the proposed approach and the latter to evaluate its general applicability.
We complement our performance analysis with a worst-case stability analysis.
In all of our tests, the adaptive implementation improves both the performance and the worst-case stability of the system.

The paper provides the following two main contributions:
\begin{itemize}
    \item It proposes a novel, modular and intuitive control law implementation that adapts the control action upon the occurrence of deadline misses in the periodic controller task. The adaptive implementation has a small overhead and is applicable to all linear dynamic controllers.
    \item It proposes a probabilistic analysis of the resulting control system subject to deadline misses.
    The analysis is based on a comparison with the ideal system without deadline misses and is used to evaluate the performance of both a real system and numerous simulated control systems.
    The results show that the adaptive implementation significantly improves the system performance and robustness, compared to the nominal implementation.
\end{itemize}

The remainder of this paper is outlined as follows. 
In Section~\ref{sec:sys-model} we present the relevant control and real-time system background. 
Section~\ref{sec:problem-descr} presents and discusses the previous literature on the topic, identifying the limitations of the state-of-the-art that we are addressing.
In Section~\ref{sec:darc} we first propose our adaptive implementation of the control law, and then we integrate the proposed controller with a probabilistic analysis method to enable its evaluation.
Section~\ref{sec:results} presents an empirical evaluation of the adaptive control law based on the proposed analysis for both a physical system and numerous simulated systems.
Finally, Section~\ref{sec:conclusion} concludes the paper.
