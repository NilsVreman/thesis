\begin{figure}
    \centerline{\begin{tikzpicture}
\small
\node[draw] (ctrler) at (0,0) {
    \begin{tikzpicture}[
    task/.style={draw,
                 thick,
                 minimum width=3.2cm,
                 minimum height=1.8cm,
                 fill=white,
                 node distance =1.3cm},
    >=stealth]
        \node[] (hw) at (0,0) {\textbf{ECU}};
        \node[task, below of=hw, xshift=-7,yshift=7, opacity=.3] (task-3) [align=center]{};
        \node[task, below of=hw, xshift=-5,yshift=5, opacity=.5] (task-2) [align=center]{};
        \node[task, below of=hw, xshift=-3,yshift=3, opacity=.7] (task-1) [align=center]{};
        \node[task, below of=hw] (task-c)[align=left]{Control Task $\ctrler$ \\[0.5mm]
                                                    \small
                                                      \texttt{y = read\_input();}\\
                                                      \texttt{u = ctrl\_comp();}\\
                                                      \texttt{write\_output(u);}};
    \end{tikzpicture}
};

\node[draw,minimum height=2cm] (plant) at (5,0) [align=center]{
    Physical Plant $\plant$ \\[1mm]
    $\begin{aligned}
    x_{k+1} &= \Ap x_k \!+\! \Bp u_k \!+\! \Wp w_k\\
    y_k &= \Cp x_k \!+\! \Dp u_k 
    \end{aligned}$
};

\draw[->] ([yshift=-0.5cm] plant.west) -- node[yshift=-0.2cm]{$y$} ([yshift=-0.5cm]ctrler.east);
\draw[->] ([yshift= 0.5cm]ctrler.east) -- node[yshift=0.2cm]{$u$} ([yshift= 0.5cm] plant.west);
\draw[->]([yshift=0.8cm]plant.north)--node[xshift=0.2cm]{$w$}(plant);

\end{tikzpicture}
}
    \caption{Typical structure of a computer-controlled system. \emph{Left:} the Electronic Control Unit (ECU) implementing a real-time system, including a task executing the controller. \emph{Right:} the physical plant controlled by the actuation variable $u$, affected by the disturbance $w$, and producing the measurement~$y$.}
    \label{fig:ctrl-loop}
\end{figure}

This section introduces the necessary background and models needed for the remainder of the paper.
We discuss the real-time implementation of a general linear controller and how it can affect the performance of the system. 
We start by describing the behaviour of the system under \emph{ideal} conditions. 
Based on this, we state how \emph{deadline misses} in the real-time implementation affect the system's behaviour.

\subsection{Control Systems under Ideal Operations}
The objective of a control system is to regulate a physical process, usually called a \emph{plant}, so that it behaves as desired. 
Figure~\ref{fig:ctrl-loop} shows the structure of a control system, where an Electronic Control Unit (ECU) implements a real-time system. 
Among the different tasks executed in the system, there is a task responsible for the control computations, denoted as the control task.
Every job released by this task performs the following actions:
\begin{enumerate*}[label=(\roman*)]
    \item Read measurements from the sensors.
    \item Use the sensor information to update its state and compute a control action. 
    \item Write the control action to the actuators.
\end{enumerate*}
The executed algorithm is generally designed using control theory, where the effective application of the algorithm relies on assumptions about the real-time execution.  

The dynamic behaviour of a plant is commonly described using a state-space model~\cite{Astrom:2008}. 
Such models are constituted of two sets of equations: one describing the dynamics of the plant and another one describing the relation between the plant state and the available measurements. 
Describing physical phenomena, those equations are for the most part continuous and nonlinear. 
However, for control design and implementation purposes, they are commonly transformed into a discrete-time linear time-invariant (LTI) state-space model:
\begin{equation}
    \label{eq:plant}
    \plant: \,\, \left\{
    \begin{aligned}
        x_{k+1} &= \Ap x_k + \Bp u_k + \Wp w_k \\
        y_k &= \Cp x_k + \Dp u_k 
    \end{aligned}
    \right.
\end{equation}
Here, the variable $k$ counts the number of discrete time steps that have passed since the system started executing. 
Furthermore, the variable $x_k \in \R^{n_x}$ represents the \emph{plant state}, $u_k \in \R^{n_u}$ corresponds to the \emph{control signal} computed in order to affect the plant, $w_k \in \R^{n_w}$ models disturbances and reference signals, and $y_k \in \R^{n_y}$ is the \emph{measurement signal} available to the controller. 
For what concerns the matrices, $\Ap \in \R^{n_x \times n_x}$ captures the relation between the current state and the next state, while $\Bp \in \R^{n_x \times n_u}$ and $\Wp \in \R^{n_x \times n_w}$ respectively capture how the control signal and the exogenous signals affect the state at the next time step.
Furthermore, $\Cp \in \R^{n_y \times n_x}$ and $\Dp \in \R^{n_y \times n_u}$ respectively describe how the current state and control signal relate to the measurements.

To control the behaviour of the plant, a \emph{controller} $\ctrler$ is synthesised to follow some desired properties, such as: 
\begin{enumerate*}[label=(\roman*)]
    \item stability, 
    \item speed of convergence, 
    \item control effort, and 
    \item disturbance rejection.
\end{enumerate*}
The stability requirement enforces that none of the signals diverge and is a necessary condition for all controllers.
Moreover, a controller that fulfils all requirements will make the output converge to the reference value within a specified time, while minimising the control effort and the effect of possible disturbances.

Controllers are commonly implemented as fixed-rate, periodically executing tasks, following the Logical Execution Time (LET) paradigm~\cite{Henzinger:2003,Kirsch:2012, Ernst:2018}.
Adopting this paradigm, the sensors are read at the beginning of the task period, and the control action is written to the actuators at the end of the period. 
This minimises the effect of fluctuations in the execution pattern of the control algorithm (called jitter) at the cost of introducing a one-step delay in the actuation.

While they are sometimes specified as transfer functions~\cite{Astrom:2008}, controllers are often \emph{implemented} as discrete-time state-space systems.
More specifically, we assume that the controller is an LTI system that takes the measurement $y_k$ as input and produces the control signal $u_{k+1}$ as output:\footnote{Note that we adopt a positive feedback convention.}
\begin{equation}
    \label{eq:ctrler}
    \ctrler: \,\, \left\{
    \begin{aligned}
        z_{k+1} &= \Ac z_k + \Bc y_k \\
        u_{k+1} &= \Cc z_{k} + \Dc y_k.
    \end{aligned}
    \right.
\end{equation}
Here, $z_k \in \R^{n_z}$ represents the internal state of the controller.
The second equation, specifying the control action at step $k+1$, captures the one-step delay introduced by the LET paradigm.
The matrices $\Ac \in \R^{n_z \times n_z}$, $\Bc \in \R^{n_z \times n_y}$, $\Cc \in \R^{n_u \times n_z}$, and $\Dc \in \R^{n_u \times n_y}$ govern the behaviour of the controller.

In conjunction with Equation~\eqref{eq:ctrler}, we define two types of controllers: \emph{static} and \emph{dynamic}.

\begin{definition}[Static Controller]%
    We denote a \emph{static controller} as any controller $\ctrler$ that is \emph{stateless} (i.e., it has no internal state $z$;\, $n_z = 0$).
\end{definition}

\begin{definition}[Dynamic Controller]%
    We denote a \emph{dynamic controller} as any controller $\ctrler$ that is \emph{stateful} (i.e., it has an internal state $z$;\, $n_z\geq 1$).
\end{definition}

From the definitions above, we note that a static controller can be written as a fixed gain matrix times the input (i.e., $\ctrler \,:\, u_{k+1} = \Dc y_k$), while a dynamic controller is equivalent to~\eqref{eq:ctrler} with non-empty matrices $\Ac$, $\Bc$, $\Cc$, and $\Dc$.
Examples of static controllers include proportional (P) controllers, state feedback controllers, and linear--quadratic regulators (LQR), while dynamic controllers include proportional--integral--derivative (PID) controllers, lead--lag compensators, and linear--quadratic--Gaussian (LQG) regulators~\cite{Astrom:2008}.

\subsection{Control Systems Subject to Deadline Misses}

We assume that the controller $\ctrler$ is implemented as a periodic task with period $T$ and implicit deadlines. 
Intuitively, each execution period of the control task corresponds to one time step $k$.
At the start of period $k$, the task releases a \emph{job} that should be completed before the deadline at time $(k+1) T$.

Faults in the real-time system can affect the timely execution of the control task~\cite{Steinbauer:2013}.
We denote the outcome of a job's execution as either a deadline \emph{hit} or \emph{miss}, corresponding to whether the job completed its execution before its deadline or not. 
The source of a deadline miss could be a temporary CPU overload~\cite{Baruah:1997}, cache misses~\cite{Milligan:1996, Wang:2012},  or unexpected preemption from hardware interrupts or higher priority tasks~\cite{Stankovic:1995}.
However, the analysis and adaptation methods presented in this paper are independent of the origin of the deadline miss.

To study what happens when the controller misses a deadline, we have to define how the system behaves in such circumstances.
In particular, three aspects need to be considered: 
\begin{enumerate*}[label=(\roman*)]
    \item how the controller state is updated, 
    \item how the actuator handles the lack of a new control signal~\cite{Schenato:2009}, and
    \item how the operating system handles a job that misses its deadline~\cite{Pazzaglia:2019, Cervin:2005}.
\end{enumerate*}
The first item refers to what happens to the internal state $z$ when the controller is unable to finish its execution ahead of its dedicated deadline.
Henceforth, we assume that when the controller misses its deadline, the controller state is \emph{not} updated (implying $z_{k+1} = z_k$). 
This is motivated by the possibility to roll back $z_k$ to a previous state~\cite{akesson:2020, Seong:2001, Zhang:2003} and by the impossibility to guarantee that the state update was finished if the job was only partially completed.

Regarding the second item, mainly two actuator models have previously been considered in the literature: \emph{Zero} and \emph{Hold} \cite{Schenato:2009}.
Under the Zero model, if the job released at time $kT$ misses its deadline at time  $(k+1)T$, the actuator outputs $u_{k+1} = 0$.
This strategy is uncommon in practice, because it performs well only in very specific cases~\cite{Vreman:2021ecrts}. 
Instead, the more common actuator model is to hold the control signal in the case of a missed deadline: $u_{k+1} = u_k$.
Although our proposed adaptation is in itself independent of the actuator model, the Hold model has been adopted in the analysis and examples of this paper.

% For the final item, there exists at least three different employable strategies dealing with missed deadlines: 
For what concerns the handling of the job that missed the deadline, there exist at least three different employable strategies:
\begin{enumerate*}[label=(\alph*)] % roman already used for the three aspects
    \item \emph{Kill}
    \item \emph{Skip-Next}, and 
    \item \emph{Queue}.
\end{enumerate*}
When using the Kill strategy, the job that missed its deadline gets terminated, the controller state is rolled back, and the next job is released. 
The Skip-Next strategy does not terminate the job that missed its deadline. 
Instead, it lets the job continue its execution, not releasing subsequent jobs until the active one has finished executing.
Queue behaves similarly to the Skip-Next strategy: it does not kill the current job, but it does release the subsequent jobs to the job queue.
Both the Skip-Next and Queue strategies allow jobs to work with outdated input data, since they do not terminate jobs that miss their deadline.
Thus, they introduce a lag in the actuation of the control law, which in most cases reduces the control performance with respect to what could be achieved with the Kill strategy.

For the remainder of this paper we will use the Kill strategy, since it
\begin{enumerate*}[label=(\alph*)]
    \item introduces explicit breakpoints in which to adapt the controller,
    \item supplies the controller with the latest sensor measurement after an overrun,
    \item helps free computational resources in overrun situations, and
    \item is a common choice in both industry and literature~\cite{akesson:2020, Bernat:2001, Hertneck:2019}.
\end{enumerate*}
Furthermore, we note that in~\cite{Vreman:2021ecrts} it has been observed that the actuator model is of greater relevance than the handling of the deadline overrun.
We also note that an effective implementation of the kill strategy that includes state roll-back requires the implementation of checkpointing mechanisms~\cite{Zhang:2003,Seong:2001}, which might not be implemented by default in a real-time system.

We conclude this section by mentioning that various models have been proposed to describe tasks
that may experience deadline misses, e.g., soft~\cite{Marchand:2008} or weakly hard models~\cite{Bernat:2001, hammadeh:2017}.
For the adaptive controller in this work, we only assume that the number of consecutive deadline misses $\counter$ is bounded by a finite quantity $\counter_\mathit{max} < \infty$.
This assumption is a practical necessity, since accepting an infinite run of deadline misses would completely disconnect the plant from the controller~\cite{Maggio:2020}.

Solely for performance analysis, in Section~IV-B we adopt a stochastic model of the sequences of deadline hits and misses. 
Such a model can be computed from a probabilistic task set model using existing techniques~\cite{Chen:2017, Chen:2018, markovic:2021}. 
However, we remark that the proposed adaptive controller implementation is independent of the stochastic deadline miss model.

\subsection{Control System Stability under Deadline Misses}
\label{sec:stability}
Guaranteeing the stability of control systems is essential in control engineering.
A linear control system without exogenous inputs is \emph{stable} if and only if the system's state always converges to zero irrespective of its initial value.
Assuming ideal conditions, i.e., no deadline misses, stability can be verified by checking whether all the closed-loop system's eigenvalues lie inside the unit circle.
However, in the presence of deadline misses, the classical stability criteria are no longer sufficient, since the dynamics of the control system is time-varying and changes according to the specific pattern of deadline misses.
For such cases, switched system stability analysis (also known as switching stability) is a viable extension of classical stability~\cite{Liberzon:2003}.
In this paper, we analyse the switching stability using both a time-averaged (Markov Jump Linear Analysis~\cite{Fang:2002}) and a worst-case (Joint Spectral Radius~\cite{Rota:1960}) approach.

Modelling the deadline misses as a stationary random process with known statistical properties allows us to calculate an analytical time-averaged performance index of the closed-loop system.
If the performance index is finite, then the system is guaranteed to be stable in the mean-square sense (meaning that the state will not diverge with probability~1).
We develop our approach to compute the time-averaged performance in Section~\ref{sec:analysis}.

If the statistical properties of the deadline misses are unknown or uncertain, switching stability can still be analysed under worst-case conditions.
The Joint Spectral Radius (JSR) generalises the spectral radius of a matrix (i.e., the largest absolute eigenvalue) to a set of matrices; thus, the JSR characterises the largest asymptotic growth (or contraction) rate of the states. % arbitrary matrix product of the matrices in the set.
If the JSR of the set of closed-loop matrices representing $i = \{0,1,\ldots,q_{max}\}$ deadline misses (followed by a hit) is below $1$ then the system is switching stable.
Conversely, if it is above $1$, there exists at least one sequence of deadline misses and hits that makes the system unstable.
There exist both toolboxes and methods for calculating the JSR for control systems subject to deadline misses~\cite{Jungers:2014, Maggio:2020, Vreman:2021}.
