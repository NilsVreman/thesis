\def \lqr {\figdir/example-figs/lqr.csv}
\def \lqg {\figdir/example-figs/lqg.csv}
\def \lqrnom {\figdir/example-figs/lqr-nominal.csv}
\def \lqgnom {\figdir/example-figs/lqg-nominal.csv}

\begin{tikzpicture}
    \footnotesize

    %Main axes
    \begin{axis}[%
            height=\timeplotheight,
            width=\columnwidth,
            xmin=0,
            xmax=5,
            ymin=-0.4,
            ymax=1.1,
            xlabel={Time (s)}, 
            ylabel={State},
            ytick={-0.4,-0.2,0,0.2,0.4,0.6,0.8,1.0,1.2},
            yticklabels={-0.4,-0.2,0,0.2,0.4,0.6,0.8,1.0,1.2},
            ylabel near ticks,
            grid=major,
            grid style={dashed,black!20},
            legend cell align=left,
            scatter/classes={a={mark=x, mark size=3, color=misscolour}},
            % legend columns = 2
            ]
        
        % LQR-nominal x
        \addplot[lqrnomcolour, very thick ]
                table [x=T, y=X, col sep=comma] {\lqrnom};
        
        % LQR x
        \addplot[lqrcolour, dash pattern={on 3pt off 3pt}, very thick ]
                table [x=T, y=X, col sep=comma] {\lqr};
        
        % LQG-nominal x
        \addplot[lqgnomcolour, very thick ]
                table [x=T, y=X, col sep=comma] {\lqgnom};
                
        % LQG x
        \addplot[lqgcolour, dash pattern={on 3pt off 3pt}, very thick ]
                table [x=T, y=X, col sep=comma] {\lqg};
        
        % Misses
        % scatter src=explicit symbolic gets the scatter definition in axis options.
        \addplot+[scatter, only marks, thick, scatter src=explicit symbolic] coordinates{
            (0.3, -0.4) [a]
            (0.7, -0.4) [a]
            (0.9, -0.4) [a]
            (1.6, -0.4) [a]
            (2.1, -0.4) [a]
            (2.2, -0.4) [a]
            (2.3, -0.4) [a]
            (2.9, -0.4) [a]
            (3.2, -0.4) [a]
            (3.5, -0.4) [a]
            (3.8, -0.4) [a]
            (4.1, -0.4) [a]
            (5.0, -0.4) [a]
        };

        % Legend
        \legend{$x_2$ ($\ctrler_1 $ -- ideal), $x_2$ ($\ctrler_1 $ -- with misses), $x_2$ ($\ctrler_2 $ -- ideal), $x_2$ ($\ctrler_2 $ -- with misses), Overrun};
    \end{axis}

\end{tikzpicture}
