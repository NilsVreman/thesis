In this work, we analyse a single real-time task. 
For the remainder of this paper, a real-time task $\tau$ is an entity composed of a sequence of jobs $(j_i)_{i \in \N_{\geq}}$, representing code that is executed repeatedly on a given hardware platform (not necessarily according to any temporal pattern or periodicity). 
A task is characterised by its relative deadline $d$, representing the time after which each job should be completed.

The index $i$ counts the job number. 
For a given job $j_i$, we denote with $a_i$ its release time (the time in which the job becomes active in the hardware platform), and with $f_i$ its completion time (the time in which the job terminates its execution). 
We also use $d_i$ to represent the absolute deadline of the $i$-th job, meaning that $d_i=a_i+d$.

In general, a job can either complete its execution before its deadline or overrun it, resulting respectively in a deadline \emph{hit} or \emph{miss} (collectively denoted by the job's \emph{outcome}).
%
\begin{definition}[Deadline Hit]%
\label{def:hit}%
    The $i$-th job of a task $\tau$ is said to hit its deadline if $f_i \leq d_i$.
\end{definition}
%
\begin{definition}[Deadline Miss]%
\label{def:miss}%
    The $i$-th job of a task $\tau$ is said to miss its deadline if $f_i > d_i$.
\end{definition}
%
The weakly-hard task model~\cite{Bernat:2001, Bernat:1998} provides guarantees on the sequence of outcomes of a real-time task via four constraints, each specifying how deadline misses and hits are interleaved for a window of $k \geq 1$ consecutive jobs.
%
\begin{definition}[Weakly-Hard Task]%
\label{def:weakly-hard}%
    A weakly-hard task $\tau$ is a task that satisfies (at least) one of the following constraints:
    \begin{enumerate}[label=(\roman*)]
        \item \label{item:AnyHit} $\tau \vdash \anyhit{}$ (\tAH{}): in any window of $k$ consecutive jobs, the minimum number of hits is $x$;
        \item \label{item:RowHit} $\tau \vdash \rowhit{}$ (\tRH{}): in any window of $k$ consecutive jobs, the minimum number of consecutive hits is $x$;
        \item \label{item:AnyMiss} $\tau \vdash \anymiss{}$ (\tAM{}): in any window of $k$ consecutive jobs, the maximum number of misses is $x$; and
        \item \label{item:RowMiss} $\tau \vdash \rowmiss{}$ (\tRM{}): in any window of $k$ consecutive jobs, the maximum number of consecutive misses is $x$;
    \end{enumerate}
    for some values of $x\in \N_{\geq}$, $k \in \N_{>}$, where $x\leq k$. We use the $\vdash$ symbol to indicate that all the possible sequences of outcomes of $\tau$ satisfy the right hand side.
\end{definition}
%
The types of constraints in Definition~\ref{def:weakly-hard} have received different attention in the real-time systems literature. 
In particular, the \tAM{} constraint has been extensively studied, and is commonly addressed as the $(m,K)$ weakly-hard task model~\cite{Hamdaoui:1995, Hammadeh:2017a, Hammadeh:2017b, Sun:2017, Ahrendts:2018, Pazzaglia:2018}. 
However, these constraints have been studied separately, while a task can simultaneously satisfy many, possibly of different types.

Exploiting different types of constraints -- and possibly different parameters for the same type of constraint -- leads to a better outcome for the analysis of the system. 
This follows from the space of possible sequences being pruned, thus allowing us to focus on proving that the real-time system behaves correctly in the relevant cases. 
In the following, we denote a set of $L$ weakly-hard constraints with $\Lambda = \left\{ \lambda_1, \lambda_2, \ldots, \lambda_L \right\} $. 
To characterise the possible sequences of outcomes that satisfy a constraint, we borrow some elementary concepts from language theory, in particular the \emph{binary alphabet}~\cite{Hopcroft:2006}.
%
\begin{definition}[Alphabet $\Sigma$ of Job Outcomes]%
\label{def:alphabet}%
    We define the alphabet of job outcomes $\Sigma = \left\{ 0, 1 \right\}$, where $0$ indicates a deadline miss and $1$ represents a deadline hit.
\end{definition}
%
Using well-established notation, we denote the \emph{character} $c_i \in \Sigma$ as the outcome of job $j_i$.
A \emph{word} $w$ of length $\abs{w} = N$ is a sequence of characters $w = \seq{ c_1, c_2, \ldots, c_N }$ that specifies a sequence of consecutive job outcomes for a task.
Without loss of generality, we assume that all words are preceded and followed only by hits.
We denote the sub-word of a word $w$ from index $a$ to $b$ with $w\funof{a, b} = \seq{ c_a, c_{a+1}, \ldots, c_b }$.
Finally, $\Sigma^N$ denotes the set of all possible words of length $N$.

With a slight abuse of notation, we use $w \vdash \lambda$ to indicate that the word $w$ satisfies the constraint $\lambda$.
Obtaining the set of words satisfying $\lambda$ follows directly from the definitions of the alphabet and the constraint itself~\cite{Bernat:2001, Bernat:1998}.
%
\begin{definition}[Satisfaction Set $\sset{N}{\lambda}$]%
    \label{def:satisfaction-set}%
    The set of all length $N$ words $w$, satisfying the weakly-hard constraint $\lambda$, is denoted by $\sset{N}{\lambda}$.
    Formally, $\sset{N}{\lambda} = \left\{ w \in \Sigma^N \, | \, w \vdash \lambda \right\},\, N \geq 1$.
\end{definition}
%
Trivially, all words in $\sset{M}{\lambda}$ are sub-words of words existing in $\sset{N}{\lambda}$, if $M \leq N$.
To simplify notation we define the set containing all words of infinite length as $\sset{}{\lambda} \equiv \sset{\infty}{\lambda}$.

Using satisfaction sets, it is possible to formally define a partial ordering between two constraints $\lambda_i$ and $\lambda_j$.
We denote the logical conjunction with $\land$ and the logical disjunction with $\lor$.
The following notions of constraint \emph{domination} and \emph{equivalence}~\cite{Bernat:2001, Bernat:1998} are used extensively throughout the remainder of the paper (jointly denoted \emph{constraint dominance}).
%
\begin{definition}[Constraint Domination]%
\label{def:domination}%
    Given two arbitrary weakly-hard constraints $\lambda_i$ and $\lambda_j$, $\lambda_i$ dominates $\lambda_j$ (denoted $\lambda_i \prec \lambda_j$) if all words satisfying $\lambda_i$ also satisfy $\lambda_j$, i.e., $\sset{}{\lambda_i} \subset \sset{}{\lambda_j}$.
    Correspondingly, $\lambda_i \preceq \lambda_j \Leftrightarrow \sset{}{\lambda_i} \subseteq \sset{}{\lambda_j}$.
\end{definition}
%
\begin{definition}[Constraint Equivalence]%
\label{def:equivalence}%
    Given two arbitrary weakly-hard constraints $\lambda_i$ and $\lambda_j$, $\lambda_i$ is equivalent to $\lambda_j$ if they respectively dominate each other.
    Formally, $\lambda_i \equiv \lambda_j \Leftrightarrow \lambda_i \preceq \lambda_j \land \lambda_j \preceq \lambda_i$. Two constraints are equivalent if they share the same satisfaction set, i.e., $\lambda_i \equiv \lambda_j \Leftrightarrow \sset{}{\lambda_i} = \sset{}{\lambda_j}$.
\end{definition}

The notion of constraint dominance has attracted attention from different areas, and is still occasionally researched~\cite{Wu:2020, Tu:2007}.
To provide dominance results, we first define the \emph{weakest} and \emph{hardest} constraints~\cite{Bernat:2001, Bernat:1998}.
%
\begin{definition}[Weakest Constraint $\lweak$]%
\label{def:weakest}%
    The weakest constraint $\lweak$ is defined as the constraint satisfied by \emph{any} word.
    Formally, $\sset{N}{\lweak} = \Sigma^N,\,\forall N \in \N_{>}$.
\end{definition}
%
\begin{definition}[Hardest Constraint $\lhard$]%
\label{def:hardest}%
    The hardest constraint $\lhard$ is defined as the constraint satisfied \emph{solely} by the word containing all deadline hits.
    Formally, $\sset{N}{\lhard} = \{ 1^N \},\,\forall N \in \N_{>}$.
\end{definition}
Using these definitions, we now review known constraint dominance relations.
We refer the reader to~\cite{Bernat:1998} or any referenced paper for the corresponding proofs.

\begin{lemma}[Known Equivalence Relations]%
\label{thm:equivalence-known}%
    The following equivalence relations hold:
    \begin{enumerate}[label=(\roman*)]
        \item \label{thm:equiv-anyhit-anymiss} $\anyhit{} \equiv \overline{\binom{k-x}{k}}$, an \tAH{} constraint with $x$ deadline hits in a window of $k$ jobs is equivalent to an \tAM{} constraint with $k-x$ hits in a window of $k$ jobs,
        \item \label{thm:equiv-rowmiss-rowmiss} $\rowmiss{} \equiv \overline{\left< x \right>},\,\, \forall k \geq 1$, a \tRM{} constraint is independent of the window size, i.e., it is equivalent to the same constraint with any $k$ value,
        \item \label{thm:equiv-anymiss-rowmiss} $\overline{\left< x \right>} \equiv \overline{\binom{x}{x+1}}$, a \tRM{} constraint with $x$ deadline misses is equivalent to an \tAM{} with $x$ possible misses in a window of $x+1$ jobs~\cite{Maggio:2020},
        \item ${\genfrac{<}{>}{0pt}{}{1}{k}} \equiv {\binom{1}{k}}$, (trivially) a \tRH{} constraint is equivalent to an \tAH{} when looking at the same window length and a single deadline,
        \item \label{thm:equiv-rowhit-hardest} $\rowhit{} \equiv \lhard \Leftrightarrow x > \sfrac{k}{2}$, a \tRH{} constraint is equivalent to the hardest constraint when $x> \sfrac{k}{2}$.
    \end{enumerate}
\end{lemma}

Using these equivalence relations, we can always translate \tAM{} and \tRM{} constraints into a corresponding \tAH{} constraint.
However, there is no clear equivalence between \tAH{} and \tRH{} constraints (beside the trivial case of a single deadline and the same window length).
Finding such a relation is important because it would allow us to treat sets of different types of constraints reducing the analysis to a single type and therefore improving efficiency.
This motivates our formal analysis of the relation between hit-related constraints, presented in Section~\ref{sec:theorems}.

Denoting with $\floor{\cdot}$ and $\ceil{\cdot}$ respectively the floor and ceiling operators, we can then define some domination relations.

\begin{lemma}[Known Domination Relations]%
\label{thm:dominance-known}%
    The following domination relations hold:
    \begin{enumerate}[label=(\roman*)]
        \item \label{thm:dom-anyhit-anyhit} $\anyhit{1} \preceq \anyhit{2} \Leftrightarrow x_2 \leq \max \left\{ a, b \right\}$, where $a=\floor{\frac{k_2}{k_1}}x_1$ and $b=k_2 - \ceil{\frac{k_2}{k_1}}(k_1-x_1)$; the \tAH{} constraint with parameters $x_1$ and $k_1$ dominates all \tAH{} constraints with parameters $x_2$ and $k_2$ if and only if $x_2 \leq \max \left\{ a, b \right\}$ with $a$ and $b$ defined as above.

        \item \label{thm:dom-rowhit-rowhit} For any two constraints $\rowhit{1}, \rowhit{2} \nequiv \lhard$, $\rowhit{1} \preceq \rowhit{2} \Leftrightarrow ( k_2 < k_1 \land k_2 \leq x_1 - \ceil{ \frac{k_1-k_2}{2} } ) \lor \left( k_2 \geq k_1 \land x_2 \leq x_1 \right)$; this specifies the domination between two \tRH{} constraints depending on their constraint parameters.

        \item \label{thm:dom-rowhit-anyhit-fixk} $\genfrac{<}{>}{0pt}{}{x_1}{k} \preceq \binom{x_2}{k} \Rightarrow \{ x_2 \leq 4x_1 - k - 2,\,\, x_2 \leq x_1,\,\, x_2 \geq 0 \}$; for a fixed and equal window $k$, if a \tRH{} constraint with consecutive deadlines hits $x_1$ dominates an \tAH{} constraint with $x_2$ deadlines hits, then the indicated relation between the constraint parameters hold.

        \item \label{thm:rowmiss-rowmiss-relation} $\overline{ \left< x_1 \right>} \preceq \overline{\left< x_2 \right>} \Leftrightarrow x_1 \leq x_2$; a \tRM{} constraint with a lower number of deadline misses dominates a \tRM{} with a higher number of deadline misses.

        \item \label{thm:anymiss-anymiss-relation} $\overline{\binom{x+p}{k+p}} \preceq \overline{\binom{x}{k}}$ if $p > 0$; \tAM{} constraints can be dominated by other \tAM{} constraints when particular relations hold for values of their parameters~\cite{Tu:2007}.
    \end{enumerate}
\end{lemma}

The ability to translate constraints into \tAH{} equivalents makes Lemma~\ref{thm:dominance-known}\ref{thm:dom-anyhit-anyhit} very powerful to compare different weakly hard constraints. 
Finally, Lemma~\ref{thm:dominance-known}\ref{thm:dom-rowhit-anyhit-fixk} is the only known result that relates the \tRH{} constraints with the other types. 
However, its applicability is limited to the case in which the two constraints share the same window size.
From the presentation of the existing constraint dominance relations, we gather that there is an important piece missing to achieve a comprehensive weakly-hard analysis.
