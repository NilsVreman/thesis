This section contains the theoretical contribution of the paper. 
In~\ref{sec:theorems:single}, we present some novel results on the relation between the \tRH{} and \tAH{} constraints. 
The results introduce the final theoretical pieces allowing us to relate all the weakly-hard constraint types to the \tAH{} constraint, and thus to pave the way towards an efficient analysis implementation. 
In~\ref{sec:theorems:set}, we extend the theoretical results to handle sets of constraints, possibly containing constraints of different types.

\subsection{Relating \tRH{} and \tAH{} constraints}
\label{sec:theorems:single}

Our first theoretical contribution is the proof of a condition regarding the domination of a \tRH{} constraint over a \tAH{} constraint, precisely
$$
    \genfrac{<}{>}{0pt}{1}{x_1}{k_1} \preceq \genfrac{(}{)}{0pt}{1}{x_2}{k_2} \Leftrightarrow x_2\leq x_1\,\floor{k_2/p}+\max\,\{0 , x_1-p+(k_2\ourmod{}p) \}
$$ with $p=k_1-x_1+1$.
The proof is based on restricting the \tAH{} constraint's minimum number of hits in order to ensure that its satisfaction set includes the one of the \tRH{} constraint.


\begin{theorem}[\tRH{}--\tAH{} Domination]%
\label{thm:dom-rowhit-anyhit}%
    Let $\mathcal{S}$ be the satisfaction set of the \tRH{} constraint $\lambda_1 = \rowhit{1}$, and $k_2\geq{}x_2$ be non-negative integers. Then the following are equivalent:
    \begin{enumerate}[label=(\roman*)]
        \item Every sequence in $\mathcal{S}$ satisfies the \tAH{} constraint $\anyhit{2}$;
        \item $x_2\leq x_1\,\floor{k_2/p}+\max\,\{0 , x_1-p+(k_2\ourmod{}p) \} $, where $p=k_1-x_1+1$.
    \end{enumerate}
\end{theorem}

\begin{proof} 
    We split the proof in two separate parts.
    First, we are going to prove that \textit{$\lnot$(ii)}$\;\Rightarrow{}\lnot{}$\textit{(i)}, and then we will prove that \textit{(ii)}$\;\Rightarrow{}$\textit{(i)}, concluding the argument.

    \textit{$\lnot$(ii)}$\;\Rightarrow{}\lnot{}$\textit{(i)}: Consider the binary sequence that alternates between $x_1$ consecutive $1$'s and $p-x_1$ consecutive $0$'s, where $p$ is as in \textit{(ii)}:
    \begin{equation}\label{eq:ss1}
        \bar{s}=\ldots{}\underbrace{1\ldots{}1}_{\text{$x_1$}}\overbrace{0\ldots{}0}^{\text{$p-x_1$}}\underbrace{1\ldots{}1}_{\text{$x_1$}}\overbrace{0\ldots{}0}^{\text{$p-x_1$}}\ldots{}
    \end{equation}
    First observe that $\bar{s}\in\mathcal{S}$.
    Using the definitions of floor and modulo operator, for any integer value (including $p=k_1+1$) we can rewrite $k_2$ as $k_2=\floor{k_2/p}+\funof{k_2\ourmod{}p}$.
    From the definition of sequence $\bar{s}$ in Equation~\eqref{eq:ss1}, $\bar{s}$ certainly contains a sub-word of length $k_2$ with
    \[
        x_1\floor{k_2/p}+\max\{0,x_1-p+\funof{k_2\ourmod{}p}\}
    \]
    $1$'s.
    If the inequality in \textit{(ii)} does not hold, then $\bar{s}$ does not satisfy the \tAH{} constraint $\lambda_2 = \anyhit{2}$ (the sub-word of length $k_2$ above would contain fewer than $x_2$ $1$s).

    \textit{(ii)}$\;\Rightarrow{}$\textit{(i)}: Let $s$ be any sequence in $\mathcal{S}$.
    Now let $s'$ be equal to $s$, except that every maximal sub-word of $1$s with fewer than $x_1$ elements has been replaced with a sub-word of zeros:
    \[
        s'_i=\begin{cases}
            1&\text{if $s_i$ is part of a sub-word of at least $x_1$ $1$s,}\\
            0&{\text{otherwise}}.
        \end{cases}
    \]
    First observe that $s\in\mathcal{S}$ implies $s'\in\mathcal{S}$.
    This is because maximal sub-words of $1$s with fewer than $x_1$ elements do not contribute to the satisfaction of a \tRH{} constraint (from the perspective of this constraint, such sub-words may as well be zeros).
    Also note that if $s'$ satisfies an \tAH{} constraint, so does $s$.
    This is because $s$ can be obtained from $s'$ by flipping $0$s to $1$s, which cannot lead to a violation of an \tAH{} constraint.
    Therefore, it is sufficient to show that if \textit{(ii)} holds, any such $s'$ satisfies the \tAH{} constraint in \textit{(i)}.
    By construction, $s'$ alternates between sub-words of $1$'s with at least $x_1$ elements, and sub-words of zeros of at most $p-x_1$ elements
    \[
        s'=\ldots{}\underbrace{1\ldots{}1}_{\text{$\geq{}x_1$}}\overbrace{0\ldots{}0}^{\text{$\leq{}p-x_1$}}\underbrace{1\ldots{}1}_{\text{$\geq{}x_1$}}\overbrace{0\ldots{}0}^{\text{$\leq{}p-x_1$}}\ldots{}
    \]
    It then follows that every sub-word of length $k_2$ in $s'$ has at least as many $1$'s as every sub-word of length $k_2$ in the sequence $\bar{s}$ from \eqref{eq:ss1}.
    Since $\bar{s}$ satisfies the \tAH{} constraint, so does $s'$, and therefore so does every $s\in\mathcal{S}$ as required.
\end{proof}


The second theoretical contribution of the paper is the proof of a condition regarding the domination of an \tAH{} constraint over a \tRH{} constraint, specifically
$$
    \genfrac{(}{)}{0pt}{1}{x_1}{k_1} \preceq \genfrac{<}{>}{0pt}{1}{x_2}{k_2} \Leftrightarrow x_2 \leq \min \left\{ \floor{\sfrac{k_2}{(z_1+1)}} ,\, \ceil{\sfrac{x_1}{z_1}} \right\}
$$ 
where $z_1 = k_1-x_1$.

\begin{theorem}[\tAH{}--\tRH{} Domination]%
\label{thm:dom-anyhit-rowhit}%
    Let $\mathcal{S}$ be the satisfaction set of the \tAH{} constraint $\anyhit{1}$, and $k_2\geq{}x_2$ be non-negative integers. Then the following are equivalent:
    \begin{enumerate}[label=(\roman*)]
        \item Every sequence in $\mathcal{S}$ satisfies the \tRH{} constraint $\rowhit{2}$;
        \item $x_2 \leq \min \left\{ \floor{k_2/(z_1+1)} ,\, \ceil{x_1/z_1} \right\}$, where $z_1 = k_1-x_1$.
    \end{enumerate}
\end{theorem}

\begin{proof}
    We split the proof in two separate parts.
    First, we are going to prove that \textit{$\lnot$(ii)}$\;\Rightarrow{}\lnot{}$\textit{(i)}, and then we will prove that \textit{$\lnot$(i)}$\;\Rightarrow{}\lnot{}$\textit{(ii)}, concluding the argument.

    \textit{$\lnot$(ii)}$\;\Rightarrow{}\lnot{}$\textit{(i)}: We split the proof into three cases.

    \textit{Case 1: $0<k_2\leq{}z_1$.}
    Let $\bar{s}=\ldots{}s_ds_ds_d\ldots{}$ (i.e. the sequence constructed by repeating the sub-word $s_d$), where
    \[
        s_d=\underbrace{1\ldots{}1}_{\text{$x_1$}}\overbrace{0\ldots{}0}^{\text{$z_1$}}.
    \]
    Observe that $\bar{s}\in\mathcal{S}$.
    Since $\floor{k_2/\funof{z_1+1}}=0$, \textit{$\lnot$(ii)} implies that $x_2>0$.
    This implies \textit{$\lnot$(i)} because $\bar{s}$ contains at least $k_2$ consecutive 0s, and therefore cannot satisfy the \tRH{} constraint $\rowhit{2}$.

    \textit{Case 2: $k_2>z_1\wedge{}\ceil{x_1/z_1}\geq{}\floor{k_2/\funof{z_1+1}}$.}
    Let $s_d$ be a sequence of length $k_2$ consisting of $k_2-z_1$ 1s and $z_1$ 0s, with the 1s arranged into $z_1+1$ sub-words
    $$
        s_d = \underbrace{1\ldots{}1}_{\text{$l_1$}}0\underbrace{1\ldots{}1}_{\text{$l_2$}}0\ldots{}0\underbrace{1\ldots{}1}_{\text{$l_{z_1+1}$}},
    $$
    where the lengths of the sub-words $l_k$ satisfy
    $$
        l_k\in\left\{\floor*{\frac{k_2-z_1}{z_1+1}},\ceil*{\frac{k_2-z_1}{z_1+1}}\right\}.
    $$
    Let $\bar{s}=\ldots{}111s_d111\ldots{}$ (i.e. a sequence of all 1s except for a single sub-word $s_d$).
    Since this sequence contains only $z_1$ 0s, $\bar{s}\in\mathcal{S}$.
    The conclusion now follows since
    $$
        \ceil*{\frac{k_2-z_1}{z_1+1}}=\floor*{\frac{k_2-z_1-1}{z_1+1}}+1=\floor*{\frac{k_2}{z_1+1}},
    $$
    and so if $x_2>\floor{k_2/\funof{z_1+1}}$, then this $\bar{s}$ does not satisfy the \tRH{} constraint $\rowhit{2}$.

    \textit{Case 3: $k_2>z_1\wedge{}\ceil{x_1/z_1}<\floor{k_2/\funof{z_1+1}}$.}
    Let $s_d$ be a sequence of length $k_1$ consisting of $x_1$ 1s and $z_1$ 0s, with the 1s arranged into $z_1$ sub-words
    $$
        s_d = \underbrace{1\ldots{}1}_{\text{$l_1$}}0\underbrace{1\ldots{}1}_{\text{$l_2$}}0\ldots{}0\underbrace{1\ldots{}1}_{\text{$l_{z_1}$}}0,
    $$
    where the lengths of the sub-words $l_k$ satisfy $l_k\in\{\floor{x_1/z_1},\ceil{x_1/z_1}\}$.
    Let $\bar{s}=\ldots{}s_ds_ds_d\ldots{}$ (i.e. the sequence constructed by repeating the sub-word $s_d$).
    Observe that every sub-word of length $k_1$ in $\bar{s}$ contains exactly $x_1$ 1s, and therefore $\bar{s}\in\mathcal{S}$.
    Observe also that $\bar{s}$ contains no sub-words of more than $\ceil{x_1/z_1}$ consecutive 1s, and therefore if $x_2>\ceil{x_1/z_1}$, $\bar{s}$ does not satisfy the \tRH{} constraint $\rowhit{2}$.

    \textit{$\lnot{}$(i)}$\;\Rightarrow{}\lnot{}$\textit{(ii)}: Under the hypothesis of \textit{$\lnot{}$(i)}, there exists a sequence $s\in\mathcal{S}$ such that $s$ does not satisfy the \tRH{} constraint $\rowhit{2}$. 

    Let $s'$ be the sequence obtained from $s$ by removing all 0s from the start of $s$, and then replacing all sub-words of 0s with length greater than one with a single 0 (for example, if $s=011001010001\ldots$, then $s'=11010101\ldots$.
    Clearly $s'\in\mathcal{S}$ since this process only removes 0s, and $s'$ also does not satisfy the \tRH{} constraint.
    Consider now the sub-word $s_d$ formed from the first $k_2$ elements of $s'$.\footnote{Strictly speaking if $s$ is too short, then the sequence $s'$ resulting from this process might have length less than $\min\{k_1,k_2\}$ which would mean that the statement $s'\in\mathcal{S}$ is ill defined.
    In this case 0s should only be removed until $s'$ has length $\min\{k_1,k_2\}$.
    This will still result in a sequence that satisfies the \tAH{} constraint but violates the \tRH{} constraint.
    All the given arguments remain valid for such an $s'$, since they only depend on inequalities based on the number of 0s in particular sub-words of length $k_1$ as guaranteed by the \tAH{} constraint (note in \textit{Case 1} it is perfectly valid for $l_1=0$).}
    This sub-word will take the form
    \[
        s_d = \begin{cases}
            \underbrace{1\ldots{}1}_{\text{$l_1$}}0\underbrace{1\ldots{}1}_{\text{$l_2$}}0\ldots{}0\underbrace{1\ldots{}1}_{\text{$l_{n}$}},&\text{or}\\
            \underbrace{1\ldots{}1}_{\text{$l_1$}}0\underbrace{1\ldots{}1}_{\text{$l_2$}}0\ldots{}0\underbrace{1\ldots{}1}_{\text{$l_{n}$}}0,
        \end{cases}
    \]
    depending on whether the final element is 0 or 1.
    Note that the lengths of the sub-words of 1s satisfy $0\leq{}l_k<x_2$.
    We will now show that the existence of such a sub-word implies \textit{$\lnot{}$(ii)} by considering two cases.

    \textit{Case 1: $k_2\leq{}k_1+l_1$.}
    In this case the sub-word $s_d$ contains at most $z_1$ 0s, and so $n\leq{}z_1+1$.
    The pigeonhole principle then demonstrates that there must be an integer $1\leq{}k\leq{}n$ such that
    $$
        l_k\geq{}\ceil*{\frac{k_2-z_1}{n}}.
    $$
    To see this, note that $s_d$ has at least $k_2-z_1$ 1s, and these must be allocated into $n$ pigeonholes corresponding to the $n$ sub-words of 1s.
    This implies that
    $$
        x_2>l_k\geq{}\ceil*{\frac{k_2-z_1}{n}}\geq{}\ceil*{\frac{k_2-z_1}{z_1+1}}=\floor*{\frac{k_2}{z_1+1}}
    $$
    and $\floor{\sfrac{k_2}{(z_1+1)}}\geq{} \min\{\floor{\sfrac{k_2}{(z_1+1)}},\ceil*{\sfrac{x_1}{z_1}}\}$ as required.

    \textit{Case 2: $k_2>k_1+l_1$.}
    Let $s_d'$ denote the sub-word obtained by removing the first $l_1$ elements of $s_d$, and also removing elements from the end of $s_d$, until $s_d'$ has length $k_1$.
    This sub-word takes the form
    $$
        s_d' = \begin{cases}
            0\underbrace{1\ldots{}1}_{\text{$l_2$}}0\underbrace{1\ldots{}1}_{\text{$l_3$}}0\ldots{}0\underbrace{1\ldots{}1}_{\text{$l_{m+1}$}},&\text{or}\\
            0\underbrace{1\ldots{}1}_{\text{$l_2$}}0\underbrace{1\ldots{}1}_{\text{$l_3$}}0\ldots{}0\underbrace{1\ldots{}1}_{\text{$l_{m+1}$}}0,
        \end{cases}
    $$
    depending on whether the final element is 0 or 1.
    Since $s_d'$ satisfies the \tAH{} constraint $\anyhit{1}$, it contains at most $z_1$ zeros, and so $m\leq{}z_1$.
    Therefore, in this case the pigeonhole principle implies that at least one of the lengths $l_k$ must satisfy
    $$
        l_k\geq{}\ceil*{\frac{x_1}{m}}\geq{}\ceil*{\frac{x_1}{z_1}}.
    $$
    This implies $x_2>l_k\geq{}\ceil*{\sfrac{x_1}{z_1}}\geq{}\min\{\floor*{\sfrac{k_2}{(z_1+1)}},\ceil*{\sfrac{x_1}{z_1}}\}$ as required.
\end{proof}


The two theorems above complete the relation graph between the different types of weakly-hard constraints. 
Now that we have a complete picture, we can start investigating sets $\Lambda$ of $L$ constraints, $\Lambda = \{ \lambda_1, \ldots, \lambda_L \}$.

\subsection{Handling sets of weakly-hard constraints $\Lambda$}
\label{sec:theorems:set}

We extend the theory to the case in which $\tau$ is subject to an arbitrary set of constraints of the form presented in Definition~\ref{def:weakly-hard}.
First, we extend the satisfaction from Definition~\ref{def:satisfaction-set} and obtain
%
\begin{equation}
    \label{eq:set-satisfaction-set}
    \sset{N}{\Lambda} = \bigcap_{\lambda \in \Lambda} \sset{N}{\lambda}
\end{equation}
%
where $\bigcap$ is the generalised intersection. 
We use $\tau \vdash \Lambda$ to denote that $\tau$ satisfies all the constraints in the set $\Lambda$.
This implies that each word $w \in \sset{N}{\Lambda}$ must belong to the satisfaction set of all the constraints in $\Lambda$. 
Trivially, Equation~\eqref{eq:set-satisfaction-set} allows us to extended Definitions~\ref{def:satisfaction-set} and~\ref{def:domination} to define constraint dominance for sets of constraints.

Constraint dominance significantly reduces the problem complexity when working with sets of weakly-hard constraints, $\Lambda$. 
If the constraint set supports different types of weakly-hard constraints, it can be beneficial to find an equivalent set of constraints with minimal cardinality.

To minimise the number of constraints in the problem formulation, the constraint dominance is utilised in order to find the minimal cardinality, equivalent subset.
Utilising the comprehensive picture the theorems provide, we propose the notion of a \emph{dominant set}, thus simplifying the analysis of weakly-hard systems subject to multiple constraints.
%
\begin{definition}[Dominant Set]%
    \label{def:dominant-set}%
    The dominant set $\MDS$ of a set of weakly-hard constraints $\Lambda$ is defined as the smallest cardinality subset of $\Lambda$ representing an equivalent set of constraints.
    Formally, $\MDS \subseteq \Lambda$ where
    \begin{enumerate}[label=(\roman*)]
        \item $\lambda_i, \lambda_j \in \MDS \,\,\, \Rightarrow \,\,\, \lambda_i \nequiv \lambda_j,\,\, \forall i \neq j$,
        \item $\lambda_i, \lambda_j \in \MDS \,\,\, \Rightarrow \,\,\, \lambda_i \npreceq \lambda_j,\,\, \forall i \neq j$,
        \item $\lambda_i \in \Lambda \setminus \MDS \,\,\, \Rightarrow \,\,\, \exists \lambda_j \in \MDS\,\,s.t.\,\,\lambda_j \preceq \lambda_i$.
    \end{enumerate}
\end{definition}
%
From Definition~\ref{def:domination}, a weakly-hard constraint $\lambda_i$ dominates $\lambda_j$ if and only if $\sset{}{\lambda_i} \subseteq \sset{}{\lambda_j}$.
Thus, excluding all the dominated constraints from $\Lambda$ does not change the resulting satisfaction set.
The equivalence between the constraint set and its dominant set is trivial considering the respective satisfaction sets:
%
\begin{equation*}
    \sset{}{\MDS} = \bigcap_{\lambda \in \MDS} \sset{}{\lambda} = \bigcap_{\lambda \in \Lambda} \sset{}{\lambda} = \sset{}{\Lambda}.
\end{equation*}

In the following section, we present our tool, \tool{}, and use the theorems presented in this section and the dominance between constraints to simplify the analysis of sets of weakly-hard constraints.
