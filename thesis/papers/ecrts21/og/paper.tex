\documentclass[a4paper,UKenglish,cleveref, autoref, thm-restate]{lipics-v2021}

\bibliographystyle{plainurl}

%%%%%%%%%%%%%%%%%%%%%%%%%%%%%%%%%%%%%%%%%%%%%%%%%%%%%%%%%%%%%%%%%%%%%%%%%%%%%%%%
% In this file, only packages are allowed. These packages should be explained to
% greatest possible extent.
%%%%%%%%%%%%%%%%%%%%%%%%%%%%%%%%%%%%%%%%%%%%%%%%%%%%%%%%%%%%%%%%%%%%%%%%%%%%%%%%

\usepackage{amsmath, amsthm}            % Math related packages
\usepackage{subdepth}                   % Fixing subscript problems
%\usepackage{microtype}                  % For char. and variable expansion
%\usepackage{cite}                      % For citations
%\usepackage{pifont}                     % For check marks in tables
\usepackage{xcolor}                     % For colours
\usepackage[inline]{enumitem}           % for inline enumerations
\usepackage{graphicx}                   % Resizebox
\usepackage{listings}                   % For labeling verbatim env
%\usepackage{balance}                    % balance columns on last page

%%% Tikz
\usepackage{tikz}
\usepackage{pgfplots}
\usepackage{pgfplotstable}
\usepgfplotslibrary{groupplots}
\usepgfplotslibrary{patchplots}
\usetikzlibrary{external}
\tikzexternalize[prefix=tikzed/] 
\pgfplotsset{compat=newest}
\pgfplotsset{colormap/viridis}

%\usepackage{adjustbox}
%\usepackage[T1]{fontenc}
%\usepackage[scaled=0.85]{beramono}

%%% Math redefines
\newtheorem{theorem}{Theorem}
\newtheorem{corollary}{Corollary}
\newtheorem{problem}{Problem}
\newtheorem{lemma}{Lemma}
\newtheorem{remark}{Remark}
\newtheorem{definition}{Definition}
\newtheorem{rule_}{Constraint}


%% Code
\definecolor{codegreen}{rgb}{0,0.6,0}
\definecolor{codegray}{rgb}{0.5,0.5,0.5}
\definecolor{codepurple}{rgb}{0.58,0,0.82}
\definecolor{backcolour}{rgb}{0.95,0.95,0.92}
\lstdefinestyle{mystyle}{
    backgroundcolor=\color{black!10}, commentstyle=\color{codegreen},
    frame=single,
    keywordstyle=\color{black},
    numberstyle=\tiny\color{codegray},
    stringstyle=\color{codepurple},
    basicstyle=\ttfamily\linespread{1.15}\footnotesize,
    breakatwhitespace=false,         
    breaklines=true,                 
    captionpos=b,                    
    keepspaces=true,                 
    numbers=left,                    
    numbersep=5pt,                  
    showspaces=false,                
    showstringspaces=false,
    showtabs=false,                  
    tabsize=2,
    xrightmargin=1cm,xleftmargin=1cm,
    language=Python,
    keywordstyle=\color{blue},
    morekeywords={read_sensor_ch,compute_control,send_actuator_ch,sleep_until},
}
\lstset{style=mystyle}

%%%%%%%%%%%%%%%%%%%%%%%%%%%%%%%%%%%%%%%%%%%%%%%%%%%%%%%%%%%%%%%%%%%%%%%%%%%%%%%%
% In this file, only commands are allowed. These commands should be explained to
% greatest possible extent.
%%%%%%%%%%%%%%%%%%%%%%%%%%%%%%%%%%%%%%%%%%%%%%%%%%%%%%%%%%%%%%%%%%%%%%%%%%%%%%%%

% Real-time
\newcommand{\counter}{q}

% Plots
\newcommand{\timeplotheight}{4.9cm}

% Tables
\definecolor{Gray}{gray}{0.9}
\newcolumntype{a}{>{\centering\arraybackslash\columncolor{Gray}}m{0.075\columnwidth}}
\newcolumntype{b}{>{\centering\arraybackslash\columncolor{white}}m{0.075\columnwidth}}

% Set number of bins for aggregated plants histograms
\newcommand{\binsaggregatedhist}[0]{65}

%%% Colors
\definecolor{misscolour}{RGB}{255, 0, 0}
\definecolor{lqrcolour}{RGB}{0,28,255}
\definecolor{lqrnomcolour}{RGB}{0,28,255}
\definecolor{lqgcolour}{RGB}{0,139,0}
\definecolor{lqgnomcolour}{RGB}{0,139,0}
\definecolor{adacolour}{RGB}{239,133,16}

\tikzexternalize[prefix=tikz/] % needs to be in the main

\title{Stability and Performance Analysis of Control Systems Subject to Bursts of Deadline Misses}

\author{Nils Vreman}{Lund University, Department of Automatic Control, Sweden}{nils.vreman@control.lth.se}{https://orcid.org/0000-0002-6732-9500}{}
\author{Anton Cervin}{Lund University, Department of Automatic Control, Sweden}{anton.cervin@control.lth.se}{https://orcid.org/0000-0003-4889-8772}{}
\author{Martina Maggio}{Saarland University, Department of Computer Science, Germany \and Lund University, Department of Automatic Control, Sweden}{maggio@cs.uni-saarland.de}{https://orcid.org/0000-0002-1143-1127}{}

\authorrunning{N. Vreman, A. Cervin, and M. Maggio}

\titlerunning{Stability and Performance Analysis of Control Systems \ldots}

\ccsdesc{Computer systems organization~Embedded and cyber-physical systems}
\ccsdesc{Computer systems organization~Real-time systems}
\ccsdesc{Computer systems organization~Dependable and fault-tolerant systems and networks}


\keywords{Fault-Tolerant Control Systems, Weakly Hard Task Model}

\acknowledgements{The authors are members of the ELLIIT Strategic Research Area at Lund University. This project has received funding from the European Union's Horizon 2020 research and innovation programme under grant agreement Number 871259 (ADMORPH project). This publication reflects only the authors' view and the European Commission is not responsible for any use that may be made of the information it contains.
}

\Copyright{Nils Vreman, Anton Cervin, Martina Maggio}

\EventEditors{Bj\"{o}rn B. Brandenburg}
\EventNoEds{1}
\EventLongTitle{33rd Euromicro Conference on Real-Time Systems (ECRTS 2021)}
\EventShortTitle{ECRTS 2021}
\EventAcronym{ECRTS}
\EventYear{2021}
\EventDate{July 5--9, 2021}
\EventLocation{Virtual Conference}
\EventLogo{}
\SeriesVolume{196}
\ArticleNo{15}

\begin{document}

\maketitle

\begin{abstract}
Control systems are by design robust to various disturbances, ranging from noise to unmodelled dynamics. 
Recent work on the weakly hard model---applied to controllers---has shown that control tasks can also be inherently robust to deadline misses. 
However, existing exact analyses are limited to the stability of the closed-loop system. 
In this paper we show that stability is important but cannot be the only factor to determine whether the behaviour of a system is acceptable also under deadline misses. 
We focus on systems that experience bursts of deadline misses and on their recovery to normal operation. 
We apply the resulting comprehensive analysis (that includes both stability and performance) to a Furuta pendulum, comparing simulated data and data obtained with the real plant. 
We further evaluate our analysis using a benchmark set composed of 133 systems, which is considered representative of industrial control plants. 
Our results show the handling of the control signal is an extremely important factor in the performance degradation that the controller experiences---a clear indication that only a stability test does not give enough indication about the robustness to deadline misses.
\end{abstract}

\section{Introduction}
\label{sec:intro}
A recent survey on the state of industrial practice in real-time systems showed that a significant fraction of real-time tasks are allowed to miss a finite number of deadlines~\cite{Akesson:2020}.
%
The research community spent years defining and analysing models of tasks that can miss deadlines, from soft real-time systems~\cite{Buttazzo:2005}, to tasks with a skip-factor~\cite{Koren:1995}, from calculating the miss ratio based on execution time probability distributions~\cite{Manolache:2004}, to approximating the deadline miss probability~\cite{vonDerBruggen:2018, Bozhko:2021, vonderBrueggen:2021} for a given system.

One of such models in which tasks may miss deadlines is the weakly-hard task model~\cite{Bernat:2001}. 
Weakly-hard tasks behave according to patterns of hit and missed deadlines that are (mainly) window-based.
The originally proposed constraint models specifies alternatively (for a window of subsequent jobs):
\begin{enumerate*}[label=(\roman*)]
    \item the minimum number of deadlines that are hit,
    \item the minimum number of consecutive deadlines that are hit,
    \item the maximum number of deadlines that may be missed, or
    \item the maximum number of consecutive deadlines that may be missed.
\end{enumerate*}
The third of these models -- often called the $(m,K)$ model -- gained attention in the research community, generating results on scheduling algorithms~\cite{Hamdaoui:1995}, real-time and schedulability analysis~\cite{Sun:2017, Pazzaglia:2021b, Hammadeh:2017}, verification~\cite{Huang:2019b, Behrouzian:2020} and runtime monitoring~\cite{Wu:2020} of constraint satisfaction, derivation of task model parameters~\cite{Xu:2015}, together with applications to domains like telecommunication~\cite{Ahrendts:2018, Huang:2019a} and control systems~\cite{Ramanathan:1999, Pazzaglia:2018, Vreman:2021, Pazzaglia:2021}. 
The fourth model has also proved relevant to perform analyses of the stability of control systems~\cite{Maggio:2020}. 
Furthermore, the relation between weakly-hard constraint types has been partially investigated~\cite{Tu:2007, Wu:2020}.
However, this investigation remains partial as some of the constraints are not connected and their dominance (i.e., the comparison of how strictly does the task model constrain the task execution for different types of constraints) is not assessed.

The practical usefulness of weakly-hard models will remain limited, unless it is possible to build tools to enforce and monitor the satisfaction of weakly-hard constraints for execution platforms.
Many real-time platforms offer the possibility to invoke ``protected'' task executions, ensuring that deadlines are met at the cost of increasing the execution cost.
This is a very simple mechanism to secure that the weakly-hard constraint is satisfied in an execution platform.
However, this requires writing monitoring code, that generates transition points to this protected execution mode when a constraint might otherwise be violated.
Generating this code in a scalable way requires abstracting from the constraint and representing the execution of tasks with compact, but expressive, models.

To date, the literature has focused on the $(m,K)$ constraint, neglecting the others, despite their relevance in application domains such as control~\cite{Maggio:2020, Linsenmayer:2017, Vreman:2021}.
As a result, the mentioned tools and models are not available for all the constraint types.
This paper aims at both solving this problem and answering some open issues, namely:
\begin{enumerate*}[label=(\roman*)]
    \item guaranteeing consecutive deadline hits, and not only following patterns of deadline misses; and
    \item dealing with systems that satisfy multiple weakly-hard constraint simultaneously.
\end{enumerate*}

The first issue comes from the consideration that in practice it may be easier to guarantee that some prescribed job will hit their deadline rather than ensuring that the number of misses follows a given pattern. 
This is the case of the mentioned protected execution environment. 
As an example, mixed-criticality allows the scheduler to raise the criticality level and thus guarantee that the highly-critical tasks meet the corresponding deadlines~\cite{Burns:2013}. 
We can treat the weakly-hard task as highly critical and raise the criticality level when a deadline hit must be enforced. 
Alternatively, we can increase the budget of a reservation-based scheduler~\cite{Casini:2019}.
Despite the fact that guaranteeing hits is often easier than enforcing miss patterns, the first two types of weakly hard tasks, that constrain the number of hits, have not been receiving much attention from the research community.

Furthermore, we would like to analyse tasks that satisfy multiple constraints simultaneously.
Most analysis methods only take into account a single constraint, e.g.,~\cite{Pazzaglia:2018} or~\cite{Maggio:2020} for the stability of control systems.
In some cases, one of the two constraints \emph{dominates} the other, meaning that satisfying the dominant constraint also guarantees the satisfaction of the dominated one. 
But this is not always the case.
Consider for example two constraints $\lambda_1$ and $\lambda_2$, where $\lambda_1$ specifies that the task may miss a maximum of 2 deadlines in every window of 5 consecutive jobs, and $\lambda_2$ that it may miss a maximum of 3 deadlines in every window of 7 consecutive jobs.
On the one hand the sequence 0011100, where 0 represents a deadline miss and 1, satisfies $\lambda_1$ but fails $\lambda_2$, meaning that $\lambda_2$ does not dominate $\lambda_1$.
On the other hand the sequence 0001111 satisfies $\lambda_2$ but fails $\lambda_1$, and so $\lambda_1$ does not dominate $\lambda_2$ either.
If the analysis can only be conducted with a single constraint, the choice of which constraint is to be used is left to the practitioner, while it would be best to consider \emph{both} constraints simultaneously.

Finally, we bring forward the question of \emph{scalability}. 
Many of the research results, for example in the control domain~\cite{Pazzaglia:2018, Linsenmayer:2017, Linsenmayer:2021}, use short windows. 
However, for practical applications it may be relevant to use a large window size, as done for example in the experimental analysis in~\cite{Behrouzian:2020}. 
In fact, the original motivation behind the weakly-hard task model~\cite{Bernat:2001} uses a practical example from the avionics domain in which a deadline may be missed $11$ times in every consecutive $295$ jobs. 
It seems reasonable that systems that are built and certified (for example in the automotive domain) would not experience many deadline misses, and that using a short window size would lead to very conservative results.

To address these questions and empower researchers with a tool to apply their analysis techniques, this paper presents \tool, a software library for weakly hard tasks that treats scalability as a first-class citizen. 
More precisely, the contributions of the paper are the following:

\begin{itemize}

    \item We provide a theoretical contribution on the relation between weakly hard tasks that constrain the number of hits and the number of consecutive hits in a window (Section~\ref{sec:theorems}). 
        This relation allows us to relate all the types of constraints with one another, and provide some ordering among them. 

    \item We leverage an automata-based representation to describe the behaviour of a task subject to a weakly-hard constraint~\cite{Horssen:2016, Linsenmayer:2021}.
        In constrast to other approaches, our description exploits a mapping between a single transition in the automaton and a deadline (Section~\ref{sec:code}).
        This enables uses such as automatic generation of monitors to check weakly-hard constraint satisfaction on the fly.

    \item We extend the automaton to describe a task subject to a finite set of weakly-hard constraints (Section~\ref{sec:theorems}).
        In this way, we are able to address the analysis of systems that satisfy multiple constraints, possibly of different types, that do not dominate one another.
        As far as we know, this is the first paper that presents an analysis of a set of weakly-hard constraints.

\end{itemize}

We conduct an extensive performance evaluation campaign with a two-fold purpose (Section~\ref{sec:experiment}). 
First, we analyse the scalability of our library compared to the state of the art whenever possible, i.e., for single constraints. 
Second, we look at sets of constraints and perform a sensitivity analysis, to determine which parameters affect the execution time of the automaton construction for a set of constraints. 

\tool{} can be used for monitoring tasks subject to multiple weakly-hard constraints, analysing satisfaction sets, schedulability analysis, or connecting the weakly-hard model to applied fields like control theory.
In particular, recent papers~\cite{Pazzaglia:2018, Maggio:2020, Vreman:2021, Linsenmayer:2021, Linsenmayer:2017} connected the weakly-hard model with control proofs considering stability and performance guarantees, and \tool{} can generate general automata-based monitoring code ensuring the satisfaction of said properties.


\section{Related Work}
\label{sec:related}
The work presented in this paper is closely related to two broad research areas, namely, the analysis of 
\begin{enumerate*}[label=(\roman*)]
    \item weakly hard systems and
    \item fault-tolerant control systems.
\end{enumerate*}

\textbf{Weakly Hard Systems:}
Deadline misses can be seen as sporadic events caused by
unforeseen delays in the system. Such delays could for instance
be induced by overload activations~\cite{Xu:2015, Ernst:2014}
or cache misses~\cite{Altmeyer:2014, Davis:2013}. The idea behind
weakly hard analysis is that deadline misses are permitted under
predefined constraints. Such systems have been analysed
extensively from a real-time scheduling
perspective~\cite{Bernat:1997, Caccamo:1997, Choi:2019,
Hammadeh:2019}.  The weakly hard models have gained traction in
the research community as a tool to understand and analyse
systems with sporadic faults~\cite{Soudbakhsh:2013, Bund:2014,
Frehse:2014, Bund:2015, Hammadeh:2017a, Hammadeh:2017b, Sun:2017,
Ahrendts:2018, Soudbakhsh:2018, Pazzaglia:2018,
Gaukler:2019a}. In a recent paper, Gujarati et
al.~\cite{Gujarati:2019} analysed and compared different methods
for estimating the overall reliability of control systems using
the weakly hard task model. Furthermore, the authors
of~\cite{Broman:2019} proposed a toolchain for analysing the
strongest, satisfied weakly hard constraints as a function of the
worst-case execution time.

\textbf{Fault-Tolerant Control Systems:} 
Real-time systems are sensitive to faults. Due to their
safety-critical nature, it is arguably more important
to guarantee fault-tolerance with respect to other
classes of systems. Some of these faults can be
described using the weakly hard model. Due to the
nature of control systems, special analysis techniques
can combine fault models and the physical characteristics of
systems.

Fault-tolerance has been investigated in
many of its aspects, e.g., fault-aware scheduling 
algorithms~\cite{Rowe:2013, Buttazzo:2000b} and the analysis of systems with unreliable components~\cite{Teich:2015}. Furthermore, 
restart-based design~\cite{Caccamo:2017a, Caccamo:2018} has been used as a technique to guarantee resilience. The fault models are frequently assumed to target overload-prone 
systems, or systems with components subject to sporadic failures. Bursts of faults have been observed to affect real systems~\cite{Phan:2015, Vreman:2020}.
Gujarati et al.~\cite{Gujarati:2018} proposed an analysis 
method for networked control systems that uses active replication and quantifies the resilience of the control
system to stochastic errors. 
Maggio et al.~\cite{Maggio:2020} developed a tool for determining the stability of a control system where the control task behaves according to the weakly hard 
model. From the control perspective, there has been extensive research into both analysis and mitigation of real-time faults in feedback systems~\cite{Ramanathan:1997, Chakraborty:2014b, Chakraborty:2018}. Very often, this research produced tools to analyse the effect of computational delays~\cite{Cervin:2019} and of choosing specific scheduling policies or parameters~\cite{Palopoli:2000, Cervin:2005}, possibly including deadline misses. In a few instances, researchers looked at how to improve the performance of control systems in conjunction with scheduling information~\cite{Buttazzo:2007}. One such effort analyses modifications to the code of classic and simple control systems to handle overruns that reset the period of execution of the control task~\cite{Pazzaglia:2021}.
Abdi et al.~\cite{Caccamo:2017b} proposed a control design method for safe system-level restart, mitigating 
unknown faults during runtime execution, while keeping the system inside a safe operating space. 
Pazzaglia et al.~\cite{Pazzaglia:2019} used the scenario theory to derive a control design method accounting for potential 
deadline misses, and discussed the effect of different deadline handling strategies.
Linsenmayer et al.~\cite{Linsenmayer:2020} worked on the stabilisation of weakly-hard linear control systems for networked control systems, with some extension for nonlinear systems~\cite{Hertneck:2019}. In the considered setup, faults compromise network transmissions, but do not interfere with the controller computation (assuming that the computation is triggered). The work also focused on stability, with no control performance evaluation.

To the best of our knowledge, no previous work has devised a combined stability and performance analysis to understand how faults (even when they can be tolerated) affect the plant that should be controlled when different deadline handling strategies are used.



\section{System Behaviour in Nominal Conditions}
\label{sec:model}
To provide a comprehensive analysis framework, we need to examine what occurs in each time interval $(\pi_k)_{k \in \N_{\geq}}$, with $\pi_k = [a_0 + k\cdot \Ts, a_0 + (k+1)\cdot \Ts)$. 
In this context, an extension of the weakly-hard model is required to account for the given deadline miss handling strategy, denoted with the symbol $\strat$.
%
\begin{definition}[Extended weakly-hard model $\tau \vdash \lambda^{\strat}$]%
    \label{def:new-mk}%
    A task $\tau$ may satisfy any combination of the four \emph{extended weakly-hard constraints} (\ewhc{}) $\lambda^{\strat}$:
    \begin{enumerate}[label=(\roman*)]
        \item $\tau \vdash \eanymiss{}{\strat}$: in any window of $k$ consecutive jobs, at most $x$ intervals lack a job completion;
        \item $\tau \vdash \eanyhit{}{\strat}$:  in any window of $k$ consecutive jobs, at least $x$ intervals have a job completion;
        \item $\tau \vdash \erowmiss{}{\strat}$: in any window of $k$ consecutive jobs, at most $x$ \emph{consecutive} intervals lack a job completion;
        \item $\tau \vdash \erowhit{}{\strat}$: in any window of $k$ consecutive jobs, at least $x$ \emph{consecutive} intervals have a job completion
    \end{enumerate}
    with $x\in \N_{\geq}$, $k \in \N_{>}$, and $x\leq k$, while using strategy $\strat$ to handle potential deadline misses.
\end{definition}
%
The definition above differs from the original weakly-hard model of~\cite{Bernat:2001}, since
\begin{enumerate*}[label=(\roman*)]
    \item it explicitly introduces the handling strategy $\strat$; and
    \item it focuses on the presence of a new control command at the end of each time interval $\pi_k$, instead of checking the deadline miss events, which guarantees its applicability also for strategies different than \tK{}.
\end{enumerate*}

We now require an expressive alphabet $\Sigma\left(\strat\right)$ to characterize the behaviour of task $\tau$ in each possible time interval.
For both \tK{} and \tS{} strategies, each interval $\pi_k$ contains at most one activated and one completed job.
This restricts the possible behaviours to three cases:
\begin{enumerate}[label=(\roman*)]
    \item a time interval in which the same job is both released and completed is denoted by $\cH$ (\emph{hit});
    \item a time interval in which no job is completed is denoted by $\cM$ (\emph{miss});
    \item a time interval in which no job is released, but a job (released in a previous interval) is completed, is denoted by $\cR$ (\emph{recovery}).
\end{enumerate}
%
By checking all unique combinations of job activations and completions in each interval, we obtain the alphabets for \tK{} and \tS{} as $\Sigma\left(\tK{}\right) = \{ \cM, \cH \}$ and $\Sigma\left(\tS{}\right) = \{ \cM, \cH, \cR \}$, respectively.
The recovery character $\cR$ is used in the \tS{} alphabet to identify the late \emph{completion} of a job.
As a consequence, $\cR$ is treated equivalently to $\cH$ when verifying the extended weakly hard constraints (\ewhc{}).


The algebra presented in Section~\ref{ssec:whalgebra} is extended to the new alphabet.
We assign a character of the alphabet $\Sigma\left(\strat\right)$ to each interval $\pi_k$.
A word $w = \seq{c_1,c_2,\dots,c_N}$ is used to represent a sequence of $N$ outcomes for task $\tau$, with $c_k \in \Sigma\left(\strat\right)$ representing the outcome associated to the interval $\pi_k$. 
To enforce only feasible sequences, we introduce an order constraint for the $\cR$ character with the following Rule.
%
\begin{rule_}[Outcome ordering]%
    \label{rule:R}%
    For any word $w \in \Sigma\left(\tS{}\right)^N$, $\cR$ may only directly follow $\cM$, or be the initial element of the word.
\end{rule_}

The extended weakly-hard model also inherits all the properties of the original weakly-hard model.
In particular, the satisfaction set of $\lambda^\strat$ can be defined for $N\geq 1$ as $\sset{N}{\lambda^{\strat}} = \{ w \in \Sigma\left(\strat\right)^N \mid w \vdash \lambda^{\strat} \}$, and the constraint domination still holds as $\lambda^{\strat}_{i} \preceq \lambda^{\strat}_{j}$ if $\sset{}{\lambda^{\strat}_{i}} \subseteq \sset{}{\lambda^{\strat}_{j}}$.


\section{System Behaviour with Deadline Misses}
\label{sec:deadline}
The analysis above holds when the control task meets all its
deadlines. However, the presence of deadline misses changes the
behaviour of the system. The stability of controllers with a
number of consecutive deadline misses has been
investigated in~\cite{Maggio:2020}. The results of this
investigation attested that, due to their inherent robustness,
many control systems can withstand at least a small number of
consecutive misses.

To analyse the system, we need to clarify three aspects about the
miss behaviour:

\begin{enumerate}[label=(\roman*)]
\item What happens to the control signal.
\item What happens to the control task.
\item The computational model used for the analysis (how many
deadlines can we miss, and in what pattern).
\end{enumerate}

For the first item, the actuator can either output a
\emph{zero} ($u_k = 0_{\numoutputctl \times 1}$), or \emph{hold}
the previous value ($u_k = u_{k-1}$). The choice depends
on both the plant dynamics and on the controller, as no strategy
in general dominates the other one \cite{schenato09}. For controllers
with integral action, it makes sense to hold the previous control
value, under the presumption that the system is still
disturbed and that a non-zero control signal is needed to keep the 
plant close to its operating point. On the other hand, the zero strategy
may be preferred for plants with unstable or integrator dynamics,
where outputting a zero control action may be the safer option. 

Considering the second item, at least three different strategies can be employed to deal
with a control task that misses its deadline~\cite{Cervin:2005}:
\begin{enumerate*}[label=(\roman*)]
\item \emph{Kill},
\item \emph{Skip-Next},
\item and \emph{Queue$\funof{\lambda}$}
($\lambda \in \left\{ 1, 2, 3, \ldots \right\}$).
\end{enumerate*}
%
When the Kill strategy is used, the job that missed its deadline
is terminated, its changes are rolled back, and the next job is
released. Following the Skip-Next strategy, the job that missed
its deadline continues its execution. No new control task jobs
are released until the currently running one completes its
execution. Queue$\funof{\lambda}$ behaves similarly to Skip-Next in allowing the
current job to complete execution, but also allows the activation
of new jobs (the queue of active jobs holds up to the most recent
$\lambda$ instances of the control task). In this paper we only
analyse Kill and Skip-Next. In fact, the results presented
in~\cite{Cervin:2005,Maggio:2020} suggest that Queue$\funof{\lambda}$ is not a feasible
strategy to handle misses. The presence of two or more active jobs
in the same period creates a chain effect that is hard to recover from and that
deteriorates stability and performance.
%verloads the system, that loses the ability of
%withstanding even a few misses.

The last item refers to models of computation. The weakly hard
task model~\cite{Hamdaoui:1995, Bernat:2001} is usually
considered expressive enough to analyse the behaviour of tasks
that miss their deadlines. The authors of~\cite{Bernat:2001}
propose four definitions for a weakly hard real-time task $\tau$:
\begin{definition}[Weakly Hard Task Models~\cite{Bernat:2001}]
    \label{def:wh-models}
    A task $\tau$ may satisfy any of these four weakly hard constraints:
    \begin{enumerate}[label=(\roman*)]
        \item $\tau \vdash \binom{n}{\!\:\!\:\numtotalanalysed\!\:\!\:}$: there are at least $n$ hits for every $\numtotalanalysed$ jobs,
        \item $\tau \vdash \binom{\overline{m}}{\numtotalanalysed}$: there are at most $m$ misses for every $\numtotalanalysed$ jobs, 
        \item $\tau \vdash \genfrac{<}{>}{0pt}{}{n}{\!\:\!\:\numtotalanalysed\!\:\!\:}$: there are at least $n$ consecutive hits for every $\numtotalanalysed$ jobs,
        \item $\tau \vdash \genfrac{<}{>}{0pt}{}{\overline{m}}{\numtotalanalysed}$: there are at most $m$ consecutive misses for every $\numtotalanalysed$ jobs.
    \end{enumerate}
\end{definition}

There has been a lot of research on the second model, often also
called $m$-$K$ model~\cite{Koren:1995, Ramanathan:1997, Soudbakhsh:2013, Bund:2014,
Frehse:2014, Bund:2015, Hammadeh:2017a, Hammadeh:2017b, Sun:2017,
Ahrendts:2018, Soudbakhsh:2018, Pazzaglia:2018, Pazzaglia:2019,
Gaukler:2019a} (with $m$ being the maximum number of misses in a
window of $K$ activations). Recently there has also been an
analysis of the stability of control systems when the control
task behaves according to the fourth model~\cite{Maggio:2020}.

If the misses are due to faults or security attacks, usually the
control task experiences an interval of consecutive misses. When the
fault is resolved, the control task starts hitting its deadlines
again. From the performance standpoint, a consecutive number of
misses degrades the control quality. We are interested in what
degradation is acceptable and how much time should occur between
two potential failures. Specifically, we look at how many deadline
hits should follow a given number of consecutive misses for the
system to \emph{recover}. None of the four models above allow us
to formulate this requirement (as they specify either consecutive 
hits or misses but not both), which leads us to introduce a different
weakly hard model of computation, together with its analysis, in
Section~\ref{sec:analysis}.


\section{Burst Interval Analysis}
\label{sec:analysis}
In this section, we analyse the stability and performance of a real-time control system that experiences bursts of deadline misses. 
Section~\ref{sec:fault} introduces the fault model, Section~\ref{sec:derivation} derives the control system behaviour subject to different real-time policies and delves into both the stability and performance analysis.

\subsection{Fault Model}%
\label{sec:fault}

Faults can happen during the normal execution of tasks on a platform.
Informally, as a result of a fault, tasks miss their deadlines.
When the fault is resolved, then the original situation is recovered (possibly after a transient initial phase).

Specifically, given a system $\clsys$, we define a \emph{burst interval} $\miss$ as an interval of controller activations in which the control task executing $\ctrler$ consecutively misses $\nummisses$ deadlines, regardless of the strategy used to handle the misses.
We assume that the burst interval $\miss$ is followed by a \emph{recovery interval} $\recovery$, defined as an interval in which the control task consecutively hits $\numhits$ deadlines.

During the burst interval, the deadline misses of the control task are handled using a \emph{deadline handling strategy} $\dstrat$ (\tK{}, $K$, or \tS{}, $S$).
The control signal $u_t$ is selected in accordance with the \emph{actuation strategy} $\hstrat$ (\tZ{}, $Z$, or \tH{}, $H$).
We denote the combination of $\dstrat$ and $\hstrat$ with~$\strat = \left(\dstrat,\hstrat\right)$.
For example $\strat$ could be $SZ$ to indicate that the \tS{} deadline handling strategy is paired with the \tZ{} actuation strategy.
The system \emph{recovers} once it operates close to steady-state.

%%%%%%%%%%%%%%% Nils' Version %%%%%%%%%%%%%%%
From an industrial viewpoint, the proposed fault model is highly relevant.
The common approach is to treat faults as pseudo-independent events adhering to predefined constraints on their incidence rate~\cite{reliabook,montgomery2009introduction, Teich:2017}.
However, during the operation of a control system, faults can be caused by events like network connection problems (e.g., cutting the connection between the sensor and the controller), security attacks, contention on resources.
Studies in the automotive sector, for example, indicate that deadline misses can occur in bursts~\cite{Quinton:2014,Xu:2015}.
In these cases, the controller does not execute properly for a given amount of time (e.g., until the connection is restored, the attack is terminated, or the resource contention is reduced).
The analysis methods we propose allow us to address such situations and to provide tighter bounds on the closed-loop stability and performance than under the previously proposed weakly hard models.
Moreover, following a burst interval, we are interested in analysing the length of the recovery interval $\recovery$ that is needed to return to normal operation under each implementation strategy $\strat$.
Hence, we here extend the weakly hard models of computation with a fifth alternative and then devote the remainder of the paper to its analysis.
 
\begin{definition}[Weakly Hard Fault Model with Burst of Misses]%
    \label{def:task-model}%
    A real-time task $\tau$ may satisfy the weakly hard task model
    \begin{enumerate}[label=(\roman*)]
        \setcounter{enumi}{4}
        \item $\tau \vdash \genfrac{\{}{\}}{0pt}{}{\nummisses}{\numtotalanalysed}$: there are at most $m$ consecutive misses, followed by $\numtotalanalysed-m$ consecutive hits for every $\numtotalanalysed$ jobs.
    \end{enumerate}
    This means that a real-time task $\tau$ behaves according to the model
    %
    $
        \tau \vdash
        \genfrac{\{}{\}}{0pt}{}{\nummisses}{\numtotalanalysed},
    $
    %
    if, whenever $\tau$ experiences a burst interval $\miss$ consisting of $\nummisses$ consecutive deadline misses, it is always followed by a recovery interval $\recovery$ consisting of $\numhits=\numtotalanalysed-\nummisses$ consecutive deadline hits.
\end{definition}

\subsection{Closed-Loop System Dynamics}%
\label{sec:derivation}
In this section we derive the system dynamics for a closed-loop control system under the assumption that we enter a burst interval of length $\nummisses$ after time instant $k$, and after $\nummisses$ deadline misses we start completing the control job in time.

\textbf{Normal Operation: }%
Under \emph{normal operating conditions} the system is not experiencing any deadline misses.
In other words, the system evolves according to the closed-loop system dynamics~\eqref{eq:feedback-basic}.

\textbf{\tKZ{}: }%
%
If a control task deadline miss occurs at time instant $t$, the plant states $x_t$ still evolve as normal.
However, the controller terminates its execution prematurely by killing the job, thus not updating its states ($z_{t+1} = z_t$).
The controller output is determined by the actuation strategy and is here zero ($u_{t+1} = 0$).
Now, consider a burst interval of length $\nummisses$ after time instant $t$.
Recalling that $\tilde{x}_t = \left[ {x_t}^\T {z_t}^\T {u_t}^\T \right]^\T$, we can write the evolution of the closed-loop system for the sequence of $m$ deadline misses followed by a single deadline hit as the product of a matrix representing the behaviour of the system for a hit and a matrix representing the behaviour in case of miss elevated to the power of $m$ to indicate $m$ steps of the system evolution.

The resulting closed-loop system in state-space form is
%
\begin{equation} 
    \setlength\arraycolsep{2pt}
    \label{eq:KZ}
    \begin{bmatrix}
        x_{t+\nummisses+1} \\
        z_{t+\nummisses+1} \\
        u_{t+\nummisses+1}
    \end{bmatrix} = 
    \underbrace{ \clmat{} \begin{bmatrix}
        \Ap                                         & 0_{n_x \times n_z}  & \Bp \\
        0_{n_z \times n_x}    & I                                         & 0_{n_z \times n_u} \\
        0_{n_u \times n_x}   & 0_{n_u \times n_z}   & 0_{n_u \times n_u}
    \end{bmatrix}^{\nummisses}}_{\clmat{}_{KZ}\funof{\nummisses}}
    \begin{bmatrix}
        x_{t} \\
        z_{t} \\
        u_{t}
    \end{bmatrix},
\end{equation}
%
where $\clmat{}_{KZ}\funof{\nummisses}$ represents the system matrix for $\nummisses$ misses under the \tKZ{} strategy, followed by a single hit (the matrix $\clmat{}$ that is multiplied to the left of the equation).
The matrix $\clmat{}$ is the same specified in~\eqref{eq:matrixA}, and represents the first hit that follows the $\nummisses$ misses, hence, we determine how $\tilde{x}_t$ influences $\tilde{x}_{t+m+1}$ ($\nummisses$ misses and one hit).

\textbf{\tKH{}: }%
%
Changing the actuation strategy to \tH{}, slightly alters the system matrix we derived for the \tKZ{} case.
The plant states $x_t$ evolve as normal and the control states $z_t$ are still not updated ($z_{t+1} = z_t$).
However, due to the change in actuation strategy, the last actuated value is instead held ($u_{t+1} = u_t$).
The resulting closed-loop state-space form can be seen in~\eqref{eq:KH}, where $\clmat{}_{KH}\funof{\nummisses}$ is used to represent the system matrix for $\nummisses$ misses under the \tKH{} strategy and matrix $\clmat{}$ is specified in~\eqref{eq:matrixA}.
%
\begin{equation}
    \setlength\arraycolsep{2pt}
    \label{eq:KH}
    \begin{bmatrix}
        x_{t+\nummisses+1} \\
        z_{t+\nummisses+1} \\
        u_{t+\nummisses+1}
    \end{bmatrix} = 
    \underbrace{ \clmat{} \begin{bmatrix}
        \Ap                                         & 0_{n_x \times n_z}  & \Bp \\
        0_{n_z \times n_x}    & I                                         & 0_{n_z \times n_u} \\
        0_{n_u \times n_x}   & 0_{n_u \times n_z}   & I
    \end{bmatrix}^{\nummisses}}_{\clmat{}_{KH}\funof{\nummisses}}
    \begin{bmatrix}
        x_{t} \\
        z_{t} \\
        u_{t}
    \end{bmatrix}
\end{equation}

\textbf{\tSZ{}: }%
%
When the control task misses a deadline under the \tS{} strategy, the job missing the deadline is allowed to continue its execution until completion.
However, no subsequent job of the control task is released until the current job has finished executing.
If the currently active job terminates during period $t$, the next control job is released at the start of the $t+1$-th period.
We can then write the evolution of the system where the control job experiences $\nummisses$ misses before completing its execution, meaning that there is a subsequent hit that uses old information for the error measurements.
While the controller executed only once to completion, the plant evolved for $\nummisses+1$ steps.
The resulting closed-loop state-space form can be seen in~\eqref{eq:SZ}, where $\clmat{}_{SZ}\funof{\nummisses}$ is used to represent the system matrix under the \tSZ{} strategy for $\nummisses$ misses and one completion using old measurements.
%
\begin{equation}
\setlength\arraycolsep{2pt}
\label{eq:SZ}
    \begin{bmatrix}
        x_{t+\nummisses+1} \\
        z_{t+\nummisses+1} \\
        u_{t+\nummisses+1}
    \end{bmatrix} = 
    \underbrace{\begin{bmatrix}
        \Ap^{\nummisses+1}  & 0_{n_x \times n_z}  & \Ap^\nummisses \Bp \\
        -\Bc\Cp             & \Ac                                       & -\Bc\Dp \\
        -\Dc\Cp             & \Cc                                       & -\Dc\Dp
    \end{bmatrix}}_{\clmat{}_{SZ}\funof{\nummisses}}
    \begin{bmatrix}
        x_{t} \\
        z_{t} \\
        u_{t}
    \end{bmatrix}
\end{equation}

\textbf{\tSH{}: }%
%
Similar to \tSZ{}, one job finishes execution after $\nummisses$ consecutive misses.
However, the actuation strategy holds the previous control value during the entire burst interval.
Therefore, the plant evolution is affected by a cumulative sum over the prior control values.
The resulting closed-loop state-space form can be seen in~\eqref{eq:SH}, where $\clmat{}_{SH}\funof{\nummisses}$ is used to represent the system matrix for $\nummisses$ misses under the \tSH{} strategy.
%
\begin{equation}
\setlength\arraycolsep{2pt}
\label{eq:SH}
    \begin{bmatrix}
        x_{t+\nummisses+1} \\
        z_{t+\nummisses+1} \\
        u_{t+\nummisses+1}
    \end{bmatrix} = 
    \underbrace{\begin{bmatrix}
        \Ap^{\nummisses+1}  & 0_{n_x \times n_z}  & \sum_{i=0}^\nummisses \Ap^i\Bp \\
        -\Bc\Cp             & \Ac                                       & -\Bc\Dp \\
        -\Dc\Cp             & \Cc                                       & -\Dc\Dp
    \end{bmatrix}}_{\clmat{}_{SH}\funof{\nummisses}}
    \begin{bmatrix}
        x_{t} \\
        z_{t} \\
        u_{t}
    \end{bmatrix}
\end{equation}

Equations~\eqref{eq:KZ}--\eqref{eq:SH} are inspired by the analysis in~\cite{Maggio:2020}, but we have we introduced two generalisations.
The first one is that our controller is specified as a general state-space system; therefore our method is able to address \emph{all} linear controllers.
The second generalisation is that we could include estimates of the plant states in the controller.
We can thus properly handle the presence of an observer.\footnote{In~\cite{Maggio:2020} the controller state is specified as part of the plant (e.g., when the proportional and integral controller is introduced). This implies that the state is computed although the controller did not execute. Our formulation fixes this by separating the plant execution and the controller states.}
Furthermore, we simplify the calculations by reducing the number of states $\tilde{x}_t$ of the closed-loop matrices.

\paragraph*{Stability}%

We now describe how the system matrices above can be used to analyse stability. 
Recall that a closed-loop control system is stable if and only if the (fixed) system matrix $\clmat{}$ is Schur stable. 
This criterion is also valid for cyclic patterns, where $\clmat{}$ represents the product of all closed-loop state matrices experienced in a full burst--recovery cycle. 
Hence, we can search for the shortest recovery interval length $\numhits$ such that
%
\begin{equation}
\label{eq:stability-cond}
    \underset{i}{\max}{\abs{\eig{i}{\clmat{}^{\numhits-1}\clmat{}_{\strat}\funof{\nummisses}}}} < 1, \quad \strat \in  \{ KZ, KH, SZ, SH \}.
\end{equation}
%
Recall that $\clmat{}_{\strat}\funof{m}$ already includes one hit, thus the left multiplication with $\clmat{}^{n-1}$. 
This is a sufficient condition and not necessary, meaning that a miss occurring during the recovery interval does not immediately imply that the closed-loop system is destabilised. 
We summarise the analysis in the following definition.

\begin{definition}[Static-Cyclic Stability Analysis]%
    We denote the stability analysis from~\eqref{eq:stability-cond} with the term \emph{\nilsstability{} stability analysis}.
    The system under analysis cycles through a sequence of $\nummisses$ misses followed by a sequence of $\numhits$ hits, indefinitely.
\end{definition}
%
The \nilsstability{} analysis assumes a repeating burst--recovery cycle with no interruptions.
This works well for instance in case the misses are due to a permanent overload condition caused by a mode switch (for example from low to high criticality mode in mixed-critical systems).
However, the setting is not very general.
To foster generality, we complement the stability evaluation with a less restrictive stability analysis, based on the proposed task model in Definition~\ref{def:task-model}.

\begin{definition}[\switchingstability{} stability analysis]%
    To guarantee \emph{\switchingstability{} stability}, a system has to be stable under arbitrary switching between all the possible $\nummisses$ realisations (i.e., closed-loop matrices) that comply with all task models $\tau \vdash \genfrac{\{}{\}}{0pt}{}{\nummisses_\subset}{\numtotalanalysed}, \nummisses_\subset \in \{1, \dots,\nummisses\}$ and also include the case in which the system does not miss deadlines.
\end{definition}
In other words, a system is \switchingstability{} stable if and only if it is stable under arbitrary switching of the closed-loop matrices in the set
%
\begin{equation}
    \left\{ \clmat{}^{\numtotalanalysed-1}\clmat{}_\strat\funof{1},\, \clmat{}^{\numtotalanalysed-2}\clmat{}_\strat\funof{2},\, \ldots,\, \clmat{}^{\numtotalanalysed-\nummisses}\clmat{}_\strat\funof{\nummisses},\, \clmat{} \right\}.
\end{equation}
%
Switching stability is unfortunately quite involved.\footnote{We have devoted some research effort into the investigation of a suitable stability analysis for control tasks subject to a set of weakly-hard constraints (of the type presented in Defintion~\ref{def:wh-models}). A summary of our findings can be found at \url{https://arxiv.org/abs/2101.11312}.}
However, many excellent tools have been developed to simplify this analysis (e.g., \code{MJSR}~\cite{Maggio:2020} or the \code{JSR toolbox}~\cite{Jungers:2014} for MATLAB).

\paragraph*{Performance}%
%\label{sec:performance}

We now show how the cost function in Equation~\eqref{eq:covartocost} can be used as a time-varying performance metric.
Before a burst interval, we assume that the system is in the neighbourhood of its steady-state covariance $P_\infty$ and performance $J_\infty$.

When a burst interval of $\nummisses$ missed deadlines occurs, the system will be disrupted and its covariance matrix will evolve according to
%
\begin{equation}
\label{eq:covariance-evolution}
    P_{t+\nummisses+1} = \clmat{}_{\strat}\funof{m} P_t \left( \clmat{}_{\strat}\funof{m} \right)^\T + \clmat{}^{j_{\numhits}} R_w \left( \clmat{}^{j_{\numhits}} \right)^\T,
\end{equation}
where
\begin{equation}
\label{eq:evolutionparameters}
\setlength\arraycolsep{2pt}
    \begin{aligned}
        R_w &= 
        \begin{bmatrix} 
            \sum_{i=0}^{j_{\nummisses}} \Ap^i\,\Wp\,R\,\Wp^\T\,(\Ap^i)^\T & 0_{n_x \times n_z+n_u} \\ 
            0_{n_z + n_u \times n_x}               & 0_{n_z + n_u \times n_z+ n_u}
        \end{bmatrix}, \\
        j_{\nummisses} &= 
        \begin{cases} 
            \nummisses - 1 & \text{ if } \dstrat = K\,\, \text{(\tK{}),} \\ 
            \nummisses & \text{ if } \dstrat = S\,\,\,\, \text{(\tS{}),} 
        \end{cases} \\
        j_{\numhits} &= 
        \begin{cases} 
            1 \phantom{||222} & \text{ if } \dstrat = K\,\, \text{(\tK{}),} \\ 
            0 & \text{ if } \dstrat = S\,\,\,\, \text{(\tS{}).}
        \end{cases}
        \end{aligned}
\end{equation}

$\Ap$ and $\Wp$ are matrices from the plant evolution in~\eqref{eq:background:plant}, $R$ is the noise intensity from~\eqref{eq:covevolution}, and $\clmat{}$ is the closed-loop matrix from~\eqref{eq:matrixA}.
The cost will simultaneously change following~\eqref{eq:covartocost}.
In the recovery interval, the covariance is again governed by the normal closed-loop evolution described in~\eqref{eq:covevolution}.
The system is said to have recovered once the cost is arbitrarily close to the steady-state cost.
We evaluate this as
%
\begin{equation}
    \label{eq:cost-threshold}
    \abs{\frac{J_{\infty}-J_t}{J_\infty}} < \varepsilon,
\end{equation}
where $\varepsilon> 0$ is the \emph{recovery threshold}.

\begin{definition}[Performance Recovery Interval]%
\label{def:recovery-lenght-interval}%
    We define the recovery length interval $\recoverylengthinterval_{\strat}$ as the smallest $\numhits$ such that~\eqref{eq:cost-threshold} is satisfied for all $t \geq \numhits$ when using $\strat$ to handle deadline misses.
\end{definition}

\begin{definition}[Maximum normalised cost]%
\label{def:maximum-system-cost}%
    We denote the maximum normalised cost of the system by
    %
    \begin{equation}
        J_{M,\strat} = \max_t \frac{J_{t, \strat}}{J_\infty},
    \end{equation}
    %
    where $J_{t,\strat}$ is the cost computed according to~\eqref{eq:covartocost} when using $\strat$ to handle the deadline misses.
\end{definition}
\begin{figure}[t]
    \centering
    \resizebox{\textwidth}{!}{%
\begin{tikzpicture}

\begin{axis}[%clip=false,
thick, 
xlabel={Time},
xtick={1,2,3,5,...,17},
ylabel={Normalised Cost},
ylabel near ticks,
xlabel near ticks,
width=12cm, height=4.5cm,
xmin=0.5,
ymin=0, ymax=6.25,
enlarge x limits=0]

\fill[red!15] (axis cs:0.01,0) rectangle (axis cs:3,6.25);
\fill[green!15] (axis cs:3,0) rectangle (axis cs:13,6.25);
\fill[blue!30] (axis cs:0,0.9) rectangle (axis cs:20,1.1);

\node[rectangle, draw, fill=white, align=center] at (axis cs:14.5,4.5) {%
    \tikz{%
        \fill[white] (0,0) rectangle (5,50);%
        \draw[blue, ultra thick] (0,25) -- (5,25);%
        \fill[blue] (2.5,25) ellipse (1 and 12.5);}%
        $\,\,J_{k,\mathcal{H}}/J_\infty$};
\draw[black] (axis cs:0,0) rectangle (axis cs:17,6.25);

\addplot table [col sep=comma, x=T, y=J] {figs/ecrts21/data/example-data.csv};
\draw[dashed, black] (axis cs:0,5.4692) -- (axis cs:5,5.4692) node[right, xshift=1mm, blue] {$J_{M,\mathcal{H}}$};

\draw[<->, blue, thick] (axis cs:5,0) -- (axis cs:5,5.46921);
\draw[<->, red, thick] (axis cs:0.5,3) -- node[above] {$m$} (axis cs:3,3);
\draw[<->, green!70!black, thick] (axis cs:3,3) -- node[above] {$n^*_{\mathcal{H}}$} (axis cs:13,3);

\end{axis}

\end{tikzpicture}
}%
 
    \caption{Illustration of normalised cost ($J_t/J_\infty$), performance recovery interval $\recoverylengthinterval_{\strat}$ and maximum normalised cost $J_{M,\strat}$ on a data trace. The example uses $\strat = \text{\tKZ{}}$ and $\varepsilon=0.1$.}
    \label{fig:recoveryandpeak}
\end{figure}
Figure~\ref{fig:recoveryandpeak} gives a graphical representation of $\recoverylengthinterval_{\strat}$ and $J_{M,\strat}$ for an execution trace in which the controller experiences 3 misses and uses \tKZ{} as strategy $\strat$.

Compared to the stability analysis, the performance analysis also takes into account state deviations and uncertainty due to disturbances.
In Section~\ref{sec:derivation} we used the system dynamics to analyse the stability of the system.
The disturbance term $w_t$ was neglected as it does not influence the system stability.
However, its presence (as the presence of any disturbance) changes the dynamic behaviour of the system.
For the performance metric, the state covariance matrix $P_t$ evolves according to both the noise intensity and the system dynamics~\eqref{eq:covariance-evolution}.
The result is that the performance analysis provides us with a conservative (but more realistic) recovery interval, that takes system uncertainties into consideration.

To find the length of the recovery interval, we evolve the state covariance during a burst interval, using a specific strategy $\strat$ according to~\eqref{eq:covariance-evolution}.
Thereafter, the state covariance is evolved under normal operation, according to~\eqref{eq:covevolution}, until~\eqref{eq:cost-threshold} is satisfied, allowing us to find the performance recovery interval $\recoverylengthinterval_{\strat}$.


\section{Experimental Results}
\label{sec:results}
In this section, we apply the analysis presented in Section~\ref{sec:analysis} to a set of case studies, analysing stability and performance. 
We first present detailed results with a Furuta pendulum, both in simulation and with real hardware, using the same controller. 
The simulated results are compared to the real physical plant. 
This shows that the performance analysis does capture the important trends for real control systems.
We then present some aggregate results obtained with a set of 133 different plants from a control benchmark.
One noteworthy aspect is that the Furuta pendulum model is linearised for the control design and the pendulum stabilised around an unstable equilibrium---the top position---while the control benchmark includes (by design) stable systems. 
The difference between simulation results and real experiments for stable linear systems should in principle be smaller than for unstable nonlinear systems, making our pendulum the ideal stress test for the similarity of simulated and real data.

\subsection{Furuta Pendulum}
\label{sec:example}

We here analyse the behaviour of a Furuta pendulum~\cite{Furuta:1992}, a rotational inverted pendulum in which a rotating arm is connected to a pendulum. 
The rotation of the arm induces a swing movement on the pendulum. 
The pendulum has two equilibria: a stable position in which the pendulum is downright, and an unstable position in which the pendulum is upright. 
Our objective is to keep the pendulum in the up position, by moving the rotating arm.

The Furuta pendulum is a highly nonlinear process. 
In order to design a control strategy to keep the pendulum in the top position, it is necessary to linearise the dynamics of the system around the desired equilibrium point. 
We consider this as a stress test to check the divergence between simulation results and real hardware results, because of the instability of the equilibrium and the nonlinearity of the dynamics. 
In fact, the controller necessarily acts with information that is valid only around the upright position, and there is only a range of states in which the linearised model closely describes the behaviour of the physical plant.

We design a linear-quadratic regulator (LQR) to control the plant. 
Every $\Ts=10\,$ms the plant is sampled and the control signal is actuated.
Based on state-of-the-art models~\cite{Cazzolato:2011} and on our control design, the plant model $\plant$ is
\begin{equation}
    \label{eq:furuta}
    \setlength\arraycolsep{2pt}
    \begin{aligned}
        \plant \,\, : \,\, & \left\{ 
        \begin{aligned}
            x_{t+1} &= 
            \begin{bmatrix*}[c] 
                1.002 & 0.0100 & 0 & 0 \\
                0.3133 & 1.002 & 0 & 0 \\
                -2.943\cdot 10^{-5} & -9.808\cdot 10^{-8} & 1 & 0.01 \\
                -0.0059 & -2.943\cdot 10^{-5} & 0 & 1
            \end{bmatrix*} x_t + 
            \begin{bmatrix*}[c]
                -0.0036 \\
                -0.7127 \\
                \textcolor{white}{+}0.0096 \\
                \textcolor{white}{+}1.9120
            \end{bmatrix*} u_t + I w_t,\\
            y_t &= I x_t, 
        \end{aligned} \right. \\
    \end{aligned}
\end{equation}
the controller $\ctrler$ takes the form
\begin{equation}
    \label{eq:furuta2}
    \begin{aligned}
        \ctrler \,\, : \,\,\, & u_{t+1} = 
        \begin{bmatrix*}[c]
        8.8349 & 1.5804 & 0.2205 & 0.3049
        \end{bmatrix*} x_t 
    \end{aligned}
\end{equation}
and is designed and analysed using the following parameters (see Section~\ref{sec:cldynamics}):
\begin{equation}
    \label{eq:furuta3}
    \begin{aligned}
        Q_e &= \diag{100,1,10,10}, \,\,\, Q_u = 100, \,\,\, R = \diag{0, 0, 10, 1}.
    \end{aligned}
\end{equation}

We first apply the stability analyses presented in Section~\ref{sec:derivation} to our model.
\begin{figure}[t]
    \centerline{\resizebox{0.95\textwidth}{!}{\pgfplotstableread[header=false,col sep=comma]{\figdir/data/stability-furuta/KH-stability.csv}\kh%
\pgfplotstableread[header=false,col sep=comma]{\figdir/data/stability-furuta/KZ-stability.csv}\kz%
\pgfplotstableread[header=false,col sep=comma]{\figdir/data/stability-furuta/SH-stability.csv}\sh%
\pgfplotstableread[header=false,col sep=comma]{\figdir/data/stability-furuta/SZ-stability.csv}\sz%

% they all have the same number of columns and rows
\pgfplotstablegetrowsof{\kh}
\pgfmathtruncatemacro{\numrows}{\pgfplotsretval}
\pgfplotstablegetcolsof{\kh}
\pgfmathtruncatemacro{\numcols}{\pgfplotsretval}

\centering

\begin{tikzpicture}[scale=0.096]
    \small

    \draw[thick] (0,0) -- (0,50) -- (25,50) -- (25,0) -- cycle;
    \draw[dashed] (-1,0.5) node[left] {$1$} -- (0,0.5);
    \draw[dashed] (-1,9.5) node[left] {$10$} -- (0,9.5);
    \draw[dashed] (-1,19.5) node[left] {$20$} -- (0,19.5);
    \draw[dashed] (-1,29.5) node[left] {$30$} -- (0,29.5);
    \draw[dashed] (-1,39.5) node[left] {$40$} -- (0,39.5);
    \draw[dashed] (-1,49.5) node[left] {$50$} -- (0,49.5);

    \draw[dashed] (0.5,-1) node[below] {$1$} -- (0.5,0);
    \draw[dashed] (4.5,-1) node[below] {$5$} -- (4.5,0);
    \draw[dashed] (9.5,-1) node[below] {$10$} -- (9.5,0);
    \draw[dashed] (14.5,-1) node[below] {$15$} -- (14.5,0);
    \draw[dashed] (19.5,-1) node[below] {$20$} -- (19.5,0);
    \draw[dashed] (24.5,-1) node[below] {$25$} -- (24.5,0);

    \foreach \Y [evaluate=\Y as \PrevY using {int(\numrows-\Y)},
    evaluate=\Y as \NewY using {int(\numrows-\Y+1)}] in {1,...,\numrows}{
        \foreach \X  [evaluate=\X as \PrevX using {int(\X-1)}] in {1,...,\numcols}{
            \ReadOutElement{\kh}{\PrevY}{\PrevX}{\Current}
            % our class
            \ifnum\Current=0 \def\colorcell{white} \fi
            \ifnum\Current=2 \def\colorcell{blue!30} \fi
            \ifnum\Current=3 \def\colorcell{blue} \fi

            \draw[black,densely dotted,fill=\colorcell] (\PrevX,\numrows-\PrevY-1) rectangle +(1,1);
        }
        }
        % title and axis
        \node at (12.5,52.5) {\textbf{Kill\&Hold}};
        %\node[rotate=90] at (-7.5,25) {$m$};
        \node[] at (12.5,-6.5) {$n$};
        \node[rotate=90] at (-7.5,25) {$m$};
\end{tikzpicture}
\hspace{-1mm}
\begin{tikzpicture}[scale=0.096]
    \small

    \draw[thick] (0,0) -- (0,50) -- (25,50) -- (25,0) -- cycle;
    \draw[dashed] (-1,0.5) node[left] {$1$} -- (0,0.5);
    \draw[dashed] (-1,9.5) node[left] {$10$} -- (0,9.5);
    \draw[dashed] (-1,19.5) node[left] {$20$} -- (0,19.5);
    \draw[dashed] (-1,29.5) node[left] {$30$} -- (0,29.5);
    \draw[dashed] (-1,39.5) node[left] {$40$} -- (0,39.5);
    \draw[dashed] (-1,49.5) node[left] {$50$} -- (0,49.5);

    \draw[dashed] (0.5,-1) node[below] {$1$} -- (0.5,0);
    \draw[dashed] (4.5,-1) node[below] {$5$} -- (4.5,0);
    \draw[dashed] (9.5,-1) node[below] {$10$} -- (9.5,0);
    \draw[dashed] (14.5,-1) node[below] {$15$} -- (14.5,0);
    \draw[dashed] (19.5,-1) node[below] {$20$} -- (19.5,0);
    \draw[dashed] (24.5,-1) node[below] {$25$} -- (24.5,0);

    \foreach \Y [evaluate=\Y as \PrevY using {int(\numrows-\Y)},
    evaluate=\Y as \NewY using {int(\numrows-\Y+1)}] in {1,...,\numrows}{
        \foreach \X  [evaluate=\X as \PrevX using {int(\X-1)}] in {1,...,\numcols}{
            \ReadOutElement{\sh}{\PrevY}{\PrevX}{\Current}
            % our class
            \ifnum\Current=0 \def\colorcell{white} \fi
            \ifnum\Current=2 \def\colorcell{pink} \fi
            \ifnum\Current=3 \def\colorcell{pink!75!black} \fi

            \draw[black,densely dotted,fill=\colorcell] (\PrevX,\numrows-\PrevY-1) rectangle +(1,1);
        }
        }
        % title and axis
        \node at (12.5,52.5) {\textbf{Skip\&Hold}};
        %\node[rotate=90] at (-7.5,25) {$m$};
        \node[] at (12.5,-6.5) {$n$};
\end{tikzpicture}
\hspace{-1mm}
\begin{tikzpicture}[scale=0.096]
    \small

    \draw[thick] (0,0) -- (0,50) -- (25,50) -- (25,0) -- cycle;
    \draw[dashed] (-1,0.5) node[left] {$1$} -- (0,0.5);
    \draw[dashed] (-1,9.5) node[left] {$10$} -- (0,9.5);
    \draw[dashed] (-1,19.5) node[left] {$20$} -- (0,19.5);
    \draw[dashed] (-1,29.5) node[left] {$30$} -- (0,29.5);
    \draw[dashed] (-1,39.5) node[left] {$40$} -- (0,39.5);
    \draw[dashed] (-1,49.5) node[left] {$50$} -- (0,49.5);

    \draw[dashed] (0.5,-1) node[below] {$1$} -- (0.5,0);
    \draw[dashed] (4.5,-1) node[below] {$5$} -- (4.5,0);
    \draw[dashed] (9.5,-1) node[below] {$10$} -- (9.5,0);
    \draw[dashed] (14.5,-1) node[below] {$15$} -- (14.5,0);
    \draw[dashed] (19.5,-1) node[below] {$20$} -- (19.5,0);
    \draw[dashed] (24.5,-1) node[below] {$25$} -- (24.5,0);

    \foreach \Y [evaluate=\Y as \PrevY using {int(\numrows-\Y)},
    evaluate=\Y as \NewY using {int(\numrows-\Y+1)}] in {1,...,\numrows}{
        \foreach \X  [evaluate=\X as \PrevX using {int(\X-1)}] in {1,...,\numcols}{
            \ReadOutElement{\kz}{\PrevY}{\PrevX}{\Current}
            % our class
            \ifnum\Current=0 \def\colorcell{white} \fi
            \ifnum\Current=2 \def\colorcell{cyan!30} \fi
            \ifnum\Current=3 \def\colorcell{cyan} \fi

            \draw[black,densely dotted,fill=\colorcell] (\PrevX,\numrows-\PrevY-1) rectangle +(1,1);
        }
        }
        % title and axis
        \node at (12.5,52.5) {\textbf{Kill\&Zero}};
        \node[] at (12.5,-6.5) {$n$};
\end{tikzpicture}
\hspace{-1mm}
\begin{tikzpicture}[scale=0.096]
    \small

    \draw[thick] (0,0) -- (0,50) -- (25,50) -- (25,0) -- cycle;
    \draw[dashed] (-1,0.5) node[left] {$1$} -- (0,0.5);
    \draw[dashed] (-1,9.5) node[left] {$10$} -- (0,9.5);
    \draw[dashed] (-1,19.5) node[left] {$20$} -- (0,19.5);
    \draw[dashed] (-1,29.5) node[left] {$30$} -- (0,29.5);
    \draw[dashed] (-1,39.5) node[left] {$40$} -- (0,39.5);
    \draw[dashed] (-1,49.5) node[left] {$50$} -- (0,49.5);

    \draw[dashed] (0.5,-1) node[below] {$1$} -- (0.5,0);
    \draw[dashed] (4.5,-1) node[below] {$5$} -- (4.5,0);
    \draw[dashed] (9.5,-1) node[below] {$10$} -- (9.5,0);
    \draw[dashed] (14.5,-1) node[below] {$15$} -- (14.5,0);
    \draw[dashed] (19.5,-1) node[below] {$20$} -- (19.5,0);
    \draw[dashed] (24.5,-1) node[below] {$25$} -- (24.5,0);

    \foreach \Y [evaluate=\Y as \PrevY using {int(\numrows-\Y)},
    evaluate=\Y as \NewY using {int(\numrows-\Y+1)}] in {1,...,\numrows}{
        \foreach \X  [evaluate=\X as \PrevX using {int(\X-1)}] in {1,...,\numcols}{
            \ReadOutElement{\sz}{\PrevY}{\PrevX}{\Current}
            % our class
            \ifnum\Current=0 \def\colorcell{white} \fi
            \ifnum\Current=2 \def\colorcell{red!30} \fi
            \ifnum\Current=3 \def\colorcell{red} \fi

            \draw[black,densely dotted,fill=\colorcell] (\PrevX,\numrows-\PrevY-1) rectangle +(1,1);
        }
        }
        % title and axis
        \node at (12.5,52.5) {\textbf{Skip\&Zero}};
        %\node[rotate=90] at (-7.5,25) {$m$};
        \node[] at (12.5,-6.5) {$n$};
\end{tikzpicture}
}}
    \caption{Miss-constrained stability (dark coloured area) and \nilsstability{} stability (light coloured area) when different strategies $\strat$ are used in the example and the weakly hard model in Definition~\ref{def:task-model} is considered.
        Each square represents a window of size $\ell = \nummisses+\numhits$.
        The dark area satisfies both the \switchingstability{} and \nilsstability{} stability whilst the light area only provides \nilsstability{} stability.
        The white squares denote potentially unstable combinations of $\nummisses$ and $\numhits$.}
    \label{fig:stability_extended}
\end{figure}
Figure~\ref{fig:stability_extended} shows the results. Each square in the figure represents a combination of (at most) $m$ deadline misses (on the vertical axis) and (at least) $n$ deadline hits (on the horizontal axis).
If a square is coloured with a dark colour, the corresponding combination of misses and hits is both \nilsstability{} and \switchingstability{} stable, found using the \code{JSR Toolbox}~\cite{Jungers:2014}. 
The light squares in the figure show combinations for which the system only satisfies the \nilsstability{} stability condition. 
The white squares mark configurations for which stability cannot be guaranteed.

We remark on the presence of peaks in the \nilsstability{} stability region of $\strat = KH$ at $\numhits = \{1, 5, 9, 13, 19\}$.
Similar peaks are also found for the other strategies, but for different values of $\numhits$.
These peaks indicate that the system would be stable if that particular burst and recovery interval length would be repeated indefinitely.
However, this assumption is not robust to variations in the burst or recovery interval lengths as can be seen from the \switchingstability{} stability region being more conservative with its guarantees.
Instead, the peaks in the \nilsstability{} region can be explained by stable modes occurring due to the natural frequencies of the open-loop (for the \tZ{} actuation mode) and closed-loop (for the \tH{} actuation mode) systems.
It is also interesting to note that \tK{} seems to consistently yield a larger stability region than \tS{}, while neither \tZ{} nor \tH{} dominate each other in terms of stability guarantees. An example of the latter fact was given already in~\cite{schenato09}.

For the performance analysis, we considered a one-shot burst fault of a specific length $m$, followed by a long period of normal execution. 
Assuming that the pendulum starts close to the upright equilibrium, with stationary cost $J_\infty$, we calculate how the covariance $P_t$ and performance cost $J_t$ evolve during and after the burst interval using Equations~\eqref{eq:covariance-evolution}--\eqref{eq:evolutionparameters}.\footnote{The analysis is implemented using \code{JitterTime}~\cite{Cervin:2019}, \url{https://www.control.lth.se/jittertime}.} 
These calculations assume an ideal, linear model of the pendulum. 
The simulation results for different strategies and bursts of length $m=20$ are shown in the upper half of Figure~\ref{fig:cost_simvsreal}. 
For \tH{}, it is seen that the cost grows exponentially during the initial fault interval (the first $20\,\Ts=0.2\,$s). 
This is true also for \tZ{}, although the growth rate is too small to be visible.
The reason for the poor performance of \tH{} is that any non-zero held control signal will actively push the pendulum away from its unstable upright equilibrium even further than either disturbances or noise would already do without a proper control action.
%
\begin{figure}
    \centering
    \begin{tikzpicture}
\small

\def \fkhj {\figdir/data/experiment-JT-data/KH.csv}
\def \fkzj {\figdir/data/experiment-JT-data/KZ.csv}
\def \fshj {\figdir/data/experiment-JT-data/SH.csv}
\def \fszj {\figdir/data/experiment-JT-data/SZ.csv}
\def \fkhr {\figdir/data/experiment-data/kill-hold-20-480.csv}
\def \fkzr {\figdir/data/experiment-data/kill-zero-20-480.csv}
\def \fshr {\figdir/data/experiment-data/skip-hold-20-480.csv}
\def \fszr {\figdir/data/experiment-data/skip-zero-20-480.csv}

\begin{groupplot}[%clip mode=individual,
     group style = {group size = 4 by 2, horizontal sep = 7mm, vertical sep=5mm},
     height = 0.3\textwidth,
     width = 0.3\textwidth,
     enlargelimits=false,
     ymin=0,
     grid = both,
     grid style = {dashed, black!20},
     xlabel near ticks,
     ylabel near ticks,
     cyan,
     ultra thick,
     mark=none,
     mark options={fill=white}]

    \nextgroupplot[ylabel = \textbf{$J_k/J_\infty$ Simulated}, ymax=57, ymin=-5, ytick={1,10,30,50}]
    \node[clip=false, anchor=south] at (axis cs:1,57) {\textbf{\textcolor{white}{p}Kill\&Hold\textcolor{white}{p}}};
    \pgfplotstableread[header=true,col sep=comma]{\fkhj}\extdata;
    \addplot[mark max, blue] table[x expr=\thisrow{T}*0.01, y=J, col sep=comma] {\extdata};

    \nextgroupplot[ymax=57, ymin=-5,ytick={1,10,30,50}]
    \node[clip=false, anchor=south] at (axis cs:1,57) {\textbf{Skip\&Hold}};
    \pgfplotstableread[header=true,col sep=comma]{\fshj}\extdata;
    \addplot[mark max, pink!75!black] table[x expr=\thisrow{T}*0.01, y=J, col sep=comma] {\extdata};

    \nextgroupplot[ymax=10, ytick={1,3,5,7}]
    \node[clip=false, anchor=south] at (axis cs:1,10) {\textbf{\textcolor{white}{p}Kill\&Zero\textcolor{white}{p}}};
    \pgfplotstableread[header=true,col sep=comma]{\fkzj}\extdata;
    \addplot[mark max, cyan] table[x expr=\thisrow{T}*0.01, y=J, col sep=comma] {\extdata};

    \nextgroupplot[ymax=10, ytick={1,3,5,7}]
    \node[clip=false, anchor=south] at (axis cs:1,10) {\textbf{Skip\&Zero}};
    \pgfplotstableread[header=true,col sep=comma]{\fszj}\extdata;
    \addplot[mark max, red] table[x expr=\thisrow{T}*0.01, y=J, col sep=comma] {\extdata};
      
    \nextgroupplot[xlabel = {Time [s]}, ylabel = \textbf{$J_k/J_\infty$ Real}, ymax=57, ymin=-5, ytick={1,10,30,50}]
    \pgfplotstableread[header=true,col sep=comma]{\fkhr}\extdata;
    \addplot[mark max, blue] table[x expr=\thisrow{T}*0.01, y=J, col sep=comma] {\extdata};

    \nextgroupplot[xlabel = {Time [s]}, ymax=57, ymin=-5,ytick={1,10,30,50}]
    \pgfplotstableread[header=true,col sep=comma]{\fshr}\extdata;
    \addplot[mark max, pink!75!black] table[x expr=\thisrow{T}*0.01, y=J, col sep=comma] {\extdata};

    \nextgroupplot[xlabel = {Time [s]}, ymax=10, ytick={1,3,5,7}]
    \pgfplotstableread[header=true, col sep=comma]{\fkzr}\extdata;
    \addplot[mark max, cyan] table[x expr=\thisrow{T}*0.01, y=J, col sep=comma] {\extdata};

    \nextgroupplot[xlabel = {Time [s]}, ymax=10, ytick={1,3,5,7}]
    \pgfplotstableread[header=true,col sep=comma]{\fszr}\extdata;
    \addplot[mark max, red] table[x expr=\thisrow{T}*0.01, y=J, col sep=comma] {\extdata};

\end{groupplot}
\end{tikzpicture}

    \caption{Normalised performance cost $J_t/J_\infty$ obtained with the Furuta pendulum.
        The upper part of the figure shows simulated data, while the lower part of the figure shows the corresponding values obtained averaging the results of 500 experiments with the real process and hardware.
        Each experiment corresponds to a 500 jobs of the controller (20 misses and 480 hits).}
    \label{fig:cost_simvsreal}
\end{figure}


The large spike in cost comes when the controller is reactivated at time $0.2\,$s. 
Here, the \tH{} strategy again shows much worse performance than \tZ{}, with the peak cost being almost an order of magnitude worse. 
The difference between \tK{} and \tS{} is relatively small, with the latter strategy consistently performing slightly worse than the former. 
This is due to the small extra delay caused by using old data in the \tS{} strategy.

We conducted experiments on a Furuta pendulum, using the same controller for the real plant rather than its model.\footnote{A video, showing experiments with the real system and bursts of deadline misses can be viewed at \url{https://youtu.be/0P0K_7lvKVU}. The video shows a comparison of all the strategies for bursts of $(m = 20, n=480)$. Furthermore, we have included additional experiments with $(m=50, n=450)$ and $(m=75, n=425)$ for the \tSH{} strategy. The results of the additional experiments with higher values of $m$ are not described in the paper, as stability could not be guaranteed (and in fact the pendulum is not at all times kept in the upright position).}
Initially, we performed 500 experiments with 500 jobs each and no deadline misses, to determine the nominal variance of the system---i.e., the stationary variance used to find the static cost $J_\infty$.
For each strategy $\strat$ we then ran 500 identically set up experiments. 
In each experiment, the control task operated according to the task model from Definition~\ref{def:task-model}, experiencing a burst of length $\nummisses=20$ misses, followed by by a recovery interval with $\numhits=480$ deadline hits.

Due to system model uncertainties (e.g., friction) being significant, the rotation angle around the arm axis displayed a considerable variance.
We removed the state from the covariance calculations, since the arm angle majorly impacted the variance despite its inconsequential significance on the system dynamics (the pendulum can be stabilised with the arm being around any position, provided that the pendulum itself is kept in the upright position).
Including the rotation angle would not change the shape of the performance degradation seen in Figure~\ref{fig:cost_simvsreal}. 
However, it would make the results obtained with different strategies $\strat$ not comparable (in some of them, the rotation angle could have varied less across the 500 experiments). 
The covariance matrix $P_t$ was derived by calculating the variance of the closed-loop state vector $\tilde{x}_t$ according to Equation~\eqref{eq:covarcalc}, in each time step $t$. 

The resulting performance cost can be seen in the lower half of Figure~\ref{fig:cost_simvsreal}, where the cost $J_t$ was calculated according to Equation~\eqref{eq:covartocost} and normalised using the stationary cost $J_\infty$. 
Comparing the simulated (upper) and real (lower) performance costs in Figure~\ref{fig:cost_simvsreal}, we notice the similarities between the simulated analysis and the analysis performed on the physical plant. 
%
Particularly, the strategies involving \tH{} actuation show similar behaviours. 
For these strategies, the simulated and real values are very close for the transient burst interval, the secondary cost peak (seen around time $0.4\,$s), and the maximum normalised cost $J_{M, \strat}$.
However, the real cost is recovering slower than in the simulations---an effect that arises due to the nonlinear effects present in the real process, but unmodelled in the simulated environment.
%
Instead, comparing the \tZ{} actuation strategies, the performance cost of the physical experiments during the burst interval seem to improve compared to the simulations.
This is again likely due to the unmodelled dynamics (e.g., friction) appearing in the physical experiment but not in the simulations.
The stiction component of the friction reduces the variance of the states when the actuation signal becomes zero.
With longer burst intervals, a similar behaviour as for the \tH{} actuation strategies would appear.
Despite this difference, both the recovery interval, the secondary cost peak (around $0.4\,$s), and the maximum normalised costs $J_{M,\strat}$ are comparable.

We conclude that the results of the experiments performed on the physical process support the validity of the performance analysis presented in Section~\ref{sec:derivation}.

\subsection{Control Benchmark}
\label{sec:aggregateresults}

In Section~\ref{sec:example} we extensively discussed the results obtained with a single plant (the Furuta pendulum), with the aim of showing that simulating the performance cost yields interesting and relevant results.
As the main novelty of this paper lays in the introduction of the performance analysis as an additional tool to evaluate the behaviour of control systems that can miss deadlines, we here focus on performance.

\begin{figure}[t]
    \centering
    \resizebox{0.95\textwidth}{!}{\begin{tikzpicture}[bar width=1mm]
\small

\begin{groupplot}[group style = {group size = 4 by 9, vertical sep=8mm, horizontal sep=8mm},
    height = 3.05cm,
    width = 4cm,
    ybar,
    ymin = 0,
    ylabel style={at={(-0.35,1)}, anchor=east},
    grid = major,
    grid style = {dashed, black!20},
    x tick label style = {rotate = 60, anchor=north east, inner sep = 0mm},
    xtick distance=1, enlarge x limits=0.05,
    extra x ticks={1}]

    \nextgroupplot[title={\textbf{\textcolor{white}{p}Kill\&Hold\textcolor{white}{p}}}, ymax=70, ylabel={\textbf{$\,\,$Batch 1}},
    xtick distance=3, enlarge x limits=0.02]
    \addplot[thin, black, fill=blue] table[x=ID, y=HITS, col sep=comma]
    {\figdir/data/processed/B1_KH_10.csv};

    \nextgroupplot[title={\textbf{Skip\&Hold}}, ymax=70,
    xtick distance=3, enlarge x limits=0.02]
    \addplot[thin, black, fill=pink!75!black] table[x=ID, y=HITS, col sep=comma]
    {\figdir/data/processed/B1_SH_10.csv};

    \nextgroupplot[title={\textbf{\textcolor{white}{p}Kill\&Zero\textcolor{white}{p}}}, ymax=70,
    xtick distance=3, enlarge x limits=0.02]
    \addplot[thin, black, fill=cyan] table[x=ID, y=HITS, col sep=comma]
    {\figdir/data/processed/B1_KZ_10.csv};

    \nextgroupplot[title={\textbf{Skip\&Zero}}, ymax=70,
    xtick distance=3, enlarge x limits=0.02]
    \addplot[thin, black, fill=red] table[x=ID, y=HITS, col sep=comma]
    {\figdir/data/processed/B1_SZ_10.csv};

    % ------------------------------------------------------

    \nextgroupplot[ymax=120, ylabel={\textbf{$\,\,$Batch 2}},
    xtick distance=3, enlarge x limits=0.02]
    \addplot[thin, black, fill=blue] table[x=ID, y=HITS, col sep=comma]
    {\figdir/data/processed/B2_KH_10.csv};

    \nextgroupplot[ymax=120,
    xtick distance=3, enlarge x limits=0.02]
    \addplot[thin, black, fill=pink!75!black] table[x=ID, y=HITS, col sep=comma]
    {\figdir/data/processed/B2_SH_10.csv};

    \nextgroupplot[ymax=120,
    xtick distance=3, enlarge x limits=0.02]
    \addplot[thin, black, fill=cyan] table[x=ID, y=HITS, col sep=comma]
    {\figdir/data/processed/B2_KZ_10.csv};

    \nextgroupplot[ymax=120,
    xtick distance=3, enlarge x limits=0.02]
    \addplot[thin, black, fill=red] table[x=ID, y=HITS, col sep=comma]
    {\figdir/data/processed/B2_SZ_10.csv};

    % ------------------------------------------------------

    \nextgroupplot[ymax=90, ylabel={\textbf{$\,\,$Batch 3}}]
    \addplot[thin, black, fill=blue] table[x=ID, y=HITS, col sep=comma]
    {\figdir/data/processed/B3_KH_10.csv};

    \nextgroupplot[ymax=90]
    \addplot[thin, black, fill=pink!75!black] table[x=ID, y=HITS, col sep=comma]
    {\figdir/data/processed/B3_SH_10.csv};

    \nextgroupplot[ymax=90]
    \addplot[thin, black, fill=cyan] table[x=ID, y=HITS, col sep=comma]
    {\figdir/data/processed/B3_KZ_10.csv};

    \nextgroupplot[ymax=90]
    \addplot[thin, black, fill=red] table[x=ID, y=HITS, col sep=comma]
    {\figdir/data/processed/B3_SZ_10.csv};

    % ------------------------------------------------------

    \nextgroupplot[ymax=70,
    ylabel={\textbf{$\,\,$Batch 4}}]
    \addplot[thin, black, fill=blue] table[x=ID, y=HITS, col sep=comma]
    {\figdir/data/processed/B4_KH_10.csv};

    \nextgroupplot[ymax=70]
    \addplot[thin, black, fill=pink!75!black] table[x=ID, y=HITS, col sep=comma]
    {\figdir/data/processed/B4_SH_10.csv};

    \nextgroupplot[ymax=70]
    \addplot[thin, black, fill=cyan] table[x=ID, y=HITS, col sep=comma]
    {\figdir/data/processed/B4_KZ_10.csv};

    \nextgroupplot[ymax=70]
    \addplot[thin, black, fill=red] table[x=ID, y=HITS, col sep=comma]
    {\figdir/data/processed/B4_SZ_10.csv};

    % ------------------------------------------------------

    \nextgroupplot[ymax=100,
    ylabel={\textbf{$\,\,$Batch 5}}]
    \addplot[thin, black, fill=blue] table[x=ID, y=HITS, col sep=comma]
    {\figdir/data/processed/B5_KH_10.csv};

    \nextgroupplot[ymax=100]
    \addplot[thin, black, fill=pink!75!black] table[x=ID, y=HITS, col sep=comma]
    {\figdir/data/processed/B5_SH_10.csv};

    \nextgroupplot[ymax=100]
    \addplot[thin, black, fill=cyan] table[x=ID, y=HITS, col sep=comma]
    {\figdir/data/processed/B5_KZ_10.csv};

    \nextgroupplot[ymax=100]
    \addplot[thin, black, fill=red] table[x=ID, y=HITS, col sep=comma]
    {\figdir/data/processed/B5_SZ_10.csv};

    % ------------------------------------------------------

    \nextgroupplot[ymax=120,
    ylabel={\textbf{$\,\,$Batch 6}}]
    \addplot[thin, black, fill=blue] table[x=ID, y=HITS, col sep=comma]
    {\figdir/data/processed/B6_KH_10.csv};

    \nextgroupplot[ymax=120]
    \addplot[thin, black, fill=pink!75!black] table[x=ID, y=HITS, col sep=comma]
    {\figdir/data/processed/B6_SH_10.csv};

    \nextgroupplot[ymax=120]
    \addplot[thin, black, fill=cyan] table[x=ID, y=HITS, col sep=comma]
    {\figdir/data/processed/B6_KZ_10.csv};

    \nextgroupplot[ymax=120]
    \addplot[thin, black, fill=red] table[x=ID, y=HITS, col sep=comma]
    {\figdir/data/processed/B6_SZ_10.csv};


    % ------------------------------------------------------

    \nextgroupplot[ymax=120, ylabel={\textbf{$\,\,$Batch 7}},
    xtick distance=5, enlarge x limits=0.02]
    \addplot[thin, black, fill=blue, bar width=0.6mm] table[x=ID, y=HITS, col sep=comma]
    {\figdir/data/processed/B7_KH_10.csv};

    \nextgroupplot[ymax=120,
    xtick distance=5, enlarge x limits=0.02]
    \addplot[thin, black, fill=pink!75!black, bar width=0.6mm] table[x=ID, y=HITS, col sep=comma]
    {\figdir/data/processed/B7_SH_10.csv};

    \nextgroupplot[ymax=120,
    xtick distance=5, enlarge x limits=0.02]
    \addplot[thin, black, fill=cyan, bar width=0.6mm] table[x=ID, y=HITS, col sep=comma]
    {\figdir/data/processed/B7_KZ_10.csv};

    \nextgroupplot[ymax=120,
    xtick distance=5, enlarge x limits=0.02]
    \addplot[thin, black, fill=red, bar width=0.6mm] table[x=ID, y=HITS, col sep=comma]
    {\figdir/data/processed/B7_SZ_10.csv};

    % ------------------------------------------------------

    \nextgroupplot[ymax=70,
    ylabel={\textbf{$\,\,$Batch 8}}]
    \addplot[thin, black, fill=blue] table[x=ID, y=HITS, col sep=comma]
    {\figdir/data/processed/B8_KH_10.csv};

    \nextgroupplot[ymax=70]
    \addplot[thin, black, fill=pink!75!black] table[x=ID, y=HITS, col sep=comma]
    {\figdir/data/processed/B8_SH_10.csv};

    \nextgroupplot[ymax=70]
    \addplot[thin, black, fill=cyan] table[x=ID, y=HITS, col sep=comma]
    {\figdir/data/processed/B8_KZ_10.csv};

    \nextgroupplot[ymax=70]
    \addplot[thin, black, fill=red] table[x=ID, y=HITS, col sep=comma]
    {\figdir/data/processed/B8_SZ_10.csv};


    % ------------------------------------------------------

    \nextgroupplot[ymax=60,
    ylabel={\textbf{$\,\,$Batch 9}}]
    \addplot[thin, black, fill=blue] table[x=ID, y=HITS, col sep=comma]
    {\figdir/data/processed/B9_KH_10.csv};

    \nextgroupplot[ymax=60]
    \addplot[thin, black, fill=pink!75!black] table[x=ID, y=HITS, col sep=comma]
    {\figdir/data/processed/B9_SH_10.csv};

    \nextgroupplot[ymax=60]
    \addplot[thin, black, fill=cyan] table[x=ID, y=HITS, col sep=comma]
    {\figdir/data/processed/B9_KZ_10.csv};

    \nextgroupplot[ymax=60]
    \addplot[thin, black, fill=red] table[x=ID, y=HITS, col sep=comma]
    {\figdir/data/processed/B9_SZ_10.csv};

\end{groupplot}

\end{tikzpicture}
}
    \caption{Performance Recovery Interval $\recoverylengthinterval_{\strat}$ needed to recover from a burst of 10 deadline misses for different strategies and all the plants in the 9 batches for PI controllers designed according to~\cite{Garpinger:2015}.}
    \label{fig:overview10}
\end{figure}
\afterpage{\clearpage}

We use a set of representative process industrial plants~\cite{Astrom:2004}, developed to benchmark PID design algorithms in the control literature.
The set includes 9 different batches of stable plants, each presenting different features that can be encountered in process industrial plants, for a total of 133 plants.\footnote{%
In our analysis, we present results with 134 plants. In fact, the test set was used in~\cite{Garpinger:2015} to assess a control design method, and an additional plant was added to the set during this assessment. We included this additional plant in our analysis.}
For each batch, all systems have the same structure, but different parameters.
For example, the fourth batch is a stable system with a set of repeated eigenvalues, and a single parameter specifying the system order, which can take six possible values ($3$, $4$, $5$, $6$, $7$, or $8$).
Almost all the plants have a single independent parameter.
The only exception is Batch 7, for which we can specify two different configuration parameters, the first one having 4 possible values and the second one having 9 potential alternatives, with a total of 36 possible configurations.

The analysis methodology presented in this paper is valid for \emph{all} linear control systems.
In Section~\ref{sec:example}, we introduced an LQR controller to analyse the Furuta pendulum. To demonstrate the generality of the analysis, here, we focus on the most common controller class: proportional and integral (PI) controllers. 
These controllers constitute the vast majority of all the control loops in the process industry.\footnote{A 2001 survey by Honeywell~\cite{Desborough2001} states that 97\% of the existing industrial controllers are PI controllers.}
We also performed the analysis for proportional, integral, and derivative (PID) controllers obtaining similar results.
Introducing our tuning for PID controllers requires additional clarifications and details, which we omit due to space limitations.

For each plant we derived a PI controller according to the methodology presented in~\cite{Garpinger:2015}.
In order to showcase the applicability of our analysis to different linear systems, controllers, and noise models, we analyse the resulting closed-loop systems for $\nummisses \in [1,20]$, under the assumption that the systems are affected by brown noise (in comparison to the white noise applied to the Furuta Pendulum).
The brown noise model integrates the white noise and is thus applicable to systems where the noise is more dominant at lower frequencies (e.g., oscillations from nearby machinery).
Figure~\ref{fig:overview10} shows the results for $\nummisses=10$.

The first result that the figure shows is that the plant dynamics plays an important role in how the system reacts to misses.
For example, the plants in Batch 4 and Batch 8 need around 20 hits to recover from a burst of 10 misses.
On the contrary, the plants in Batch~6 and Batch 7 need a higher number of hits to recover from the same burst interval.
The second result that is apparent from the figure is that the \tH{} actuation strategy recovers much better (performance-wise) than \tZ{}.
The reason why \tH{} outperforms \tZ{} can be explained by the brown noise.
The control signal will actively counteract the integrated noise dynamics, meaning that zeroing the control signal removes the compensation against the integrated noise.
%
Finally, comparing the deadline handling strategies, \tK{} performs marginally better than \tS{}.
Under \tK{}, the controller uses fresh data at the beginning of the recovery interval, while \tS{} uses old data.
However, we assumed ideal rollback (i.e., zero additional computation time for the rollback and clean state) for the \tK{} strategy.
In real systems, rollback is difficult to realise and the advantage provided by \tK{} over \tS{} may therefore become unimportant.
%
These findings are consistent throughout all the plants in the experimental set, regardless of the burst interval length $\nummisses$.

The plant dynamics and noise affect the behaviour and performance of the strategies.
Comparing the results of Section~\ref{sec:example} with the aggregate results, it becomes apparent that the actuation strategy (\tZ{} or \tH{}) affects control performance significantly more than the deadline handling strategy.
For the Furuta pendulum (an unstable, nonlinear plant influenced by white noise) \tZ{} performed the best, but for the process industrial systems (stable, linear plants influenced by brown noise) \tH{} outperformed \tZ{}.
These results were apparent even with no consideration taken to the deadline handling strategies.
Thus, we conclude that the plant and noise model should be the ruling factor when choosing the actuation strategy, while the deadline handling strategy is mainly limited by the constraints imposed by the real-time implementation.


\section{Conclusions}
\label{sec:conc}
In this paper we analysed control systems and their behaviour in the presence of bursts of deadline misses. 
We provided a comprehensive set of tools to determine how robust a given control system is to faults that hinder the computation to complete in time, with different handling strategies. 
%
Our analysis tackles both stability and performance. In fact, we have shown that analysing the stability of the system is not enough to properly quantify the robustness to deadline misses, as the performance loss could be significant even for stable systems. We introduced two performance metrics, linked to the recovery of a system from a burst of deadline misses.

A limitation of the presented performance analysis is that it only applies to linear control systems. However, the approach can easily be extended to analyse \emph{time-varying} linear systems and can also be used for local analysis of a nonlinear system that should follow a given reference trajectory. In fact, to illustrate the applicability to real (e.g., nonlinear) systems, we applied the  analysis to a Furuta pendulum and compared the results of simulations obtained with a model of the process to the real execution data.
The results support our claim that the proposed performance analysis is a valid approximation of the real-world system performance.

We performed additional tests on a large batch of industrial plants, using modern control design techniques. 
From our experimental campaign, we conclude that the choice of actuation strategy affects the control performance significantly more than the choice of deadline handling strategy.

%\bibliographystyle{abbrv}
\bibliography{paper}

\end{document}
