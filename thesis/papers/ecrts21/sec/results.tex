In this section, we apply the analysis presented in Section~\ref{sec:analysis} to a set of case studies, analysing stability and performance. 
We first present detailed results with a Furuta pendulum, both in simulation and with real hardware, using the same controller. 
The simulated results are compared to the real physical plant. 
This shows that the performance analysis does capture the important trends for real control systems.
We then present some aggregate results obtained with a set of 133 different plants from a control benchmark.
One noteworthy aspect is that the Furuta pendulum model is linearised for the control design and the pendulum stabilised around an unstable equilibrium---the top position---while the control benchmark includes (by design) stable systems. 
The difference between simulation results and real experiments for stable linear systems should in principle be smaller than for unstable nonlinear systems, making our pendulum the ideal stress test for the similarity of simulated and real data.

\subsection{Furuta Pendulum}
\label{sec:example}

We here analyse the behaviour of a Furuta pendulum~\cite{Furuta:1992}, a rotational inverted pendulum in which a rotating arm is connected to a pendulum. 
The rotation of the arm induces a swing movement on the pendulum. 
The pendulum has two equilibria: a stable position in which the pendulum is downright, and an unstable position in which the pendulum is upright. 
Our objective is to keep the pendulum in the up position, by moving the rotating arm.

The Furuta pendulum is a highly nonlinear process. 
In order to design a control strategy to keep the pendulum in the top position, it is necessary to linearise the dynamics of the system around the desired equilibrium point. 
We consider this as a stress test to check the divergence between simulation results and real hardware results, because of the instability of the equilibrium and the nonlinearity of the dynamics. 
In fact, the controller necessarily acts with information that is valid only around the upright position, and there is only a range of states in which the linearised model closely describes the behaviour of the physical plant.

We design a linear-quadratic regulator (LQR) to control the plant. 
Every $\Ts=10\,$ms the plant is sampled and the control signal is actuated.
Based on state-of-the-art models~\cite{Cazzolato:2011} and on our control design, the plant model $\plant$ is
\begin{equation}
    \label{eq:furuta}
    \setlength\arraycolsep{2pt}
    \begin{aligned}
        \plant \,\, : \,\, & \left\{ 
        \begin{aligned}
            x_{k+1} &= 
            \begin{bmatrix*}[c] 
                1.002 & 0.0100 & 0 & 0 \\
                0.3133 & 1.002 & 0 & 0 \\
                -2.943\cdot 10^{-5} & -9.808\cdot 10^{-8} & 1 & 0.01 \\
                -0.0059 & -2.943\cdot 10^{-5} & 0 & 1
            \end{bmatrix*} x_k + 
            \begin{bmatrix*}[c]
                -0.0036 \\
                -0.7127 \\
                \textcolor{white}{+}0.0096 \\
                \textcolor{white}{+}1.9120
            \end{bmatrix*} u_k + I w_k,\\
            y_k &= I x_k, 
        \end{aligned} \right. \\
    \end{aligned}
\end{equation}
the controller $\ctrler$ takes the form
\begin{equation}
    \label{eq:furuta2}
    \begin{aligned}
        \ctrler \,\, : \,\,\, & u_{k+1} = 
        \begin{bmatrix*}[c]
        8.8349 & 1.5804 & 0.2205 & 0.3049
        \end{bmatrix*} x_k 
    \end{aligned}
\end{equation}
and is designed and analysed using the following parameters (see Section~\ref{sec:cldynamics}):
\begin{equation}
    \label{eq:furuta3}
    \begin{aligned}
        Q_e &= \diag{100,1,10,10}, \,\,\, Q_u = 100, \,\,\, R = \diag{0, 0, 10, 1}.
    \end{aligned}
\end{equation}

We first apply the stability analyses presented in Section~\ref{sec:derivation} to our model.
\begin{figure}[t]
    \centerline{\resizebox{0.95\textwidth}{!}{\pgfplotstableread[header=false,col sep=comma]{\figdir/data/stability-furuta/KH-stability.csv}\kh%
\pgfplotstableread[header=false,col sep=comma]{\figdir/data/stability-furuta/KZ-stability.csv}\kz%
\pgfplotstableread[header=false,col sep=comma]{\figdir/data/stability-furuta/SH-stability.csv}\sh%
\pgfplotstableread[header=false,col sep=comma]{\figdir/data/stability-furuta/SZ-stability.csv}\sz%

% they all have the same number of columns and rows
\pgfplotstablegetrowsof{\kh}
\pgfmathtruncatemacro{\numrows}{\pgfplotsretval}
\pgfplotstablegetcolsof{\kh}
\pgfmathtruncatemacro{\numcols}{\pgfplotsretval}

\centering

\begin{tikzpicture}[scale=0.096]
    \small

    \draw[thick] (0,0) -- (0,50) -- (25,50) -- (25,0) -- cycle;
    \draw[dashed] (-1,0.5) node[left] {$1$} -- (0,0.5);
    \draw[dashed] (-1,9.5) node[left] {$10$} -- (0,9.5);
    \draw[dashed] (-1,19.5) node[left] {$20$} -- (0,19.5);
    \draw[dashed] (-1,29.5) node[left] {$30$} -- (0,29.5);
    \draw[dashed] (-1,39.5) node[left] {$40$} -- (0,39.5);
    \draw[dashed] (-1,49.5) node[left] {$50$} -- (0,49.5);

    \draw[dashed] (0.5,-1) node[below] {$1$} -- (0.5,0);
    \draw[dashed] (4.5,-1) node[below] {$5$} -- (4.5,0);
    \draw[dashed] (9.5,-1) node[below] {$10$} -- (9.5,0);
    \draw[dashed] (14.5,-1) node[below] {$15$} -- (14.5,0);
    \draw[dashed] (19.5,-1) node[below] {$20$} -- (19.5,0);
    \draw[dashed] (24.5,-1) node[below] {$25$} -- (24.5,0);

    \foreach \Y [evaluate=\Y as \PrevY using {int(\numrows-\Y)},
    evaluate=\Y as \NewY using {int(\numrows-\Y+1)}] in {1,...,\numrows}{
        \foreach \X  [evaluate=\X as \PrevX using {int(\X-1)}] in {1,...,\numcols}{
            \ReadOutElement{\kh}{\PrevY}{\PrevX}{\Current}
            % our class
            \ifnum\Current=0 \def\colorcell{white} \fi
            \ifnum\Current=2 \def\colorcell{blue!30} \fi
            \ifnum\Current=3 \def\colorcell{blue} \fi

            \draw[black,densely dotted,fill=\colorcell] (\PrevX,\numrows-\PrevY-1) rectangle +(1,1);
        }
        }
        % title and axis
        \node at (12.5,52.5) {\textbf{Kill\&Hold}};
        %\node[rotate=90] at (-7.5,25) {$m$};
        \node[] at (12.5,-6.5) {$n$};
        \node[rotate=90] at (-7.5,25) {$m$};
\end{tikzpicture}
\hspace{-1mm}
\begin{tikzpicture}[scale=0.096]
    \small

    \draw[thick] (0,0) -- (0,50) -- (25,50) -- (25,0) -- cycle;
    \draw[dashed] (-1,0.5) node[left] {$1$} -- (0,0.5);
    \draw[dashed] (-1,9.5) node[left] {$10$} -- (0,9.5);
    \draw[dashed] (-1,19.5) node[left] {$20$} -- (0,19.5);
    \draw[dashed] (-1,29.5) node[left] {$30$} -- (0,29.5);
    \draw[dashed] (-1,39.5) node[left] {$40$} -- (0,39.5);
    \draw[dashed] (-1,49.5) node[left] {$50$} -- (0,49.5);

    \draw[dashed] (0.5,-1) node[below] {$1$} -- (0.5,0);
    \draw[dashed] (4.5,-1) node[below] {$5$} -- (4.5,0);
    \draw[dashed] (9.5,-1) node[below] {$10$} -- (9.5,0);
    \draw[dashed] (14.5,-1) node[below] {$15$} -- (14.5,0);
    \draw[dashed] (19.5,-1) node[below] {$20$} -- (19.5,0);
    \draw[dashed] (24.5,-1) node[below] {$25$} -- (24.5,0);

    \foreach \Y [evaluate=\Y as \PrevY using {int(\numrows-\Y)},
    evaluate=\Y as \NewY using {int(\numrows-\Y+1)}] in {1,...,\numrows}{
        \foreach \X  [evaluate=\X as \PrevX using {int(\X-1)}] in {1,...,\numcols}{
            \ReadOutElement{\sh}{\PrevY}{\PrevX}{\Current}
            % our class
            \ifnum\Current=0 \def\colorcell{white} \fi
            \ifnum\Current=2 \def\colorcell{pink} \fi
            \ifnum\Current=3 \def\colorcell{pink!75!black} \fi

            \draw[black,densely dotted,fill=\colorcell] (\PrevX,\numrows-\PrevY-1) rectangle +(1,1);
        }
        }
        % title and axis
        \node at (12.5,52.5) {\textbf{Skip\&Hold}};
        %\node[rotate=90] at (-7.5,25) {$m$};
        \node[] at (12.5,-6.5) {$n$};
\end{tikzpicture}
\hspace{-1mm}
\begin{tikzpicture}[scale=0.096]
    \small

    \draw[thick] (0,0) -- (0,50) -- (25,50) -- (25,0) -- cycle;
    \draw[dashed] (-1,0.5) node[left] {$1$} -- (0,0.5);
    \draw[dashed] (-1,9.5) node[left] {$10$} -- (0,9.5);
    \draw[dashed] (-1,19.5) node[left] {$20$} -- (0,19.5);
    \draw[dashed] (-1,29.5) node[left] {$30$} -- (0,29.5);
    \draw[dashed] (-1,39.5) node[left] {$40$} -- (0,39.5);
    \draw[dashed] (-1,49.5) node[left] {$50$} -- (0,49.5);

    \draw[dashed] (0.5,-1) node[below] {$1$} -- (0.5,0);
    \draw[dashed] (4.5,-1) node[below] {$5$} -- (4.5,0);
    \draw[dashed] (9.5,-1) node[below] {$10$} -- (9.5,0);
    \draw[dashed] (14.5,-1) node[below] {$15$} -- (14.5,0);
    \draw[dashed] (19.5,-1) node[below] {$20$} -- (19.5,0);
    \draw[dashed] (24.5,-1) node[below] {$25$} -- (24.5,0);

    \foreach \Y [evaluate=\Y as \PrevY using {int(\numrows-\Y)},
    evaluate=\Y as \NewY using {int(\numrows-\Y+1)}] in {1,...,\numrows}{
        \foreach \X  [evaluate=\X as \PrevX using {int(\X-1)}] in {1,...,\numcols}{
            \ReadOutElement{\kz}{\PrevY}{\PrevX}{\Current}
            % our class
            \ifnum\Current=0 \def\colorcell{white} \fi
            \ifnum\Current=2 \def\colorcell{cyan!30} \fi
            \ifnum\Current=3 \def\colorcell{cyan} \fi

            \draw[black,densely dotted,fill=\colorcell] (\PrevX,\numrows-\PrevY-1) rectangle +(1,1);
        }
        }
        % title and axis
        \node at (12.5,52.5) {\textbf{Kill\&Zero}};
        \node[] at (12.5,-6.5) {$n$};
\end{tikzpicture}
\hspace{-1mm}
\begin{tikzpicture}[scale=0.096]
    \small

    \draw[thick] (0,0) -- (0,50) -- (25,50) -- (25,0) -- cycle;
    \draw[dashed] (-1,0.5) node[left] {$1$} -- (0,0.5);
    \draw[dashed] (-1,9.5) node[left] {$10$} -- (0,9.5);
    \draw[dashed] (-1,19.5) node[left] {$20$} -- (0,19.5);
    \draw[dashed] (-1,29.5) node[left] {$30$} -- (0,29.5);
    \draw[dashed] (-1,39.5) node[left] {$40$} -- (0,39.5);
    \draw[dashed] (-1,49.5) node[left] {$50$} -- (0,49.5);

    \draw[dashed] (0.5,-1) node[below] {$1$} -- (0.5,0);
    \draw[dashed] (4.5,-1) node[below] {$5$} -- (4.5,0);
    \draw[dashed] (9.5,-1) node[below] {$10$} -- (9.5,0);
    \draw[dashed] (14.5,-1) node[below] {$15$} -- (14.5,0);
    \draw[dashed] (19.5,-1) node[below] {$20$} -- (19.5,0);
    \draw[dashed] (24.5,-1) node[below] {$25$} -- (24.5,0);

    \foreach \Y [evaluate=\Y as \PrevY using {int(\numrows-\Y)},
    evaluate=\Y as \NewY using {int(\numrows-\Y+1)}] in {1,...,\numrows}{
        \foreach \X  [evaluate=\X as \PrevX using {int(\X-1)}] in {1,...,\numcols}{
            \ReadOutElement{\sz}{\PrevY}{\PrevX}{\Current}
            % our class
            \ifnum\Current=0 \def\colorcell{white} \fi
            \ifnum\Current=2 \def\colorcell{red!30} \fi
            \ifnum\Current=3 \def\colorcell{red} \fi

            \draw[black,densely dotted,fill=\colorcell] (\PrevX,\numrows-\PrevY-1) rectangle +(1,1);
        }
        }
        % title and axis
        \node at (12.5,52.5) {\textbf{Skip\&Zero}};
        %\node[rotate=90] at (-7.5,25) {$m$};
        \node[] at (12.5,-6.5) {$n$};
\end{tikzpicture}
}}
    \caption{Miss-constrained stability (dark coloured area) and \nilsstability{} stability (light coloured area) when different strategies $\strat$ are used in the example and the weakly hard model in Definition~\ref{def:task-model} is considered.
        Each square represents a window of size $\numtotalanalysed = \nummisses+\numhits$.
        The dark area satisfies both the \switchingstability{} and \nilsstability{} stability whilst the light area only provides \nilsstability{} stability.
        The white squares denote potentially unstable combinations of $\nummisses$ and $\numhits$.}
    \label{fig:stability_extended}
\end{figure}
Figure~\ref{fig:stability_extended} shows the results. Each square in the figure represents a combination of (at most) $m$ deadline misses (on the vertical axis) and (at least) $n$ deadline hits (on the horizontal axis).
If a square is coloured with a dark colour, the corresponding combination of misses and hits is both \nilsstability{} and \switchingstability{} stable, found using the \code{JSR Toolbox}~\cite{Jungers:2014}. 
The light squares in the figure show combinations for which the system only satisfies the \nilsstability{} stability condition. 
The white squares mark configurations for which stability cannot be guaranteed.

We remark on the presence of peaks in the \nilsstability{} stability region of $\strat = KH$ at $\numhits = \{1, 5, 9, 13, 19\}$.
Similar peaks are also found for the other strategies, but for different values of $\numhits$.
These peaks indicate that the system would be stable if that particular burst and recovery interval length would be repeated indefinitely.
However, this assumption is not robust to variations in the burst or recovery interval lengths as can be seen from the \switchingstability{} stability region being more conservative with its guarantees.
Instead, the peaks in the \nilsstability{} region can be explained by stable modes occurring due to the natural frequencies of the open-loop (for the \tZ{} actuation mode) and closed-loop (for the \tH{} actuation mode) systems.
It is also interesting to note that \tK{} seems to consistently yield a larger stability region than \tS{}, while neither \tZ{} nor \tH{} dominate each other in terms of stability guarantees. An example of the latter fact was given already in~\cite{schenato09}.

For the performance analysis, we considered a one-shot burst fault of a specific length $m$, followed by a long period of normal execution. 
Assuming that the pendulum starts close to the upright equilibrium, with stationary cost $J_\infty$, we calculate how the covariance $P_k$ and performance cost $J_k$ evolve during and after the burst interval using Equations~\eqref{eq:covariance-evolution}--\eqref{eq:evolutionparameters}.\footnote{The analysis is implemented using \code{JitterTime}~\cite{Cervin:2019}, \url{https://www.control.lth.se/jittertime}.} 
These calculations assume an ideal, linear model of the pendulum. 
The simulation results for different strategies and bursts of length $m=20$ are shown in the upper half of Figure~\ref{fig:cost_simvsreal}. 
For \tH{}, it is seen that the cost grows exponentially during the initial fault interval (the first $20\,\Ts=0.2\,$s). 
This is true also for \tZ{}, although the growth rate is too small to be visible.
The reason for the poor performance of \tH{} is that any non-zero held control signal will actively push the pendulum away from its unstable upright equilibrium even further than either disturbances or noise would already do without a proper control action.
%
\begin{figure}
    \centering
    \begin{tikzpicture}
\small

\def \fkhj {\figdir/data/experiment-JT-data/KH.csv}
\def \fkzj {\figdir/data/experiment-JT-data/KZ.csv}
\def \fshj {\figdir/data/experiment-JT-data/SH.csv}
\def \fszj {\figdir/data/experiment-JT-data/SZ.csv}
\def \fkhr {\figdir/data/experiment-data/kill-hold-20-480.csv}
\def \fkzr {\figdir/data/experiment-data/kill-zero-20-480.csv}
\def \fshr {\figdir/data/experiment-data/skip-hold-20-480.csv}
\def \fszr {\figdir/data/experiment-data/skip-zero-20-480.csv}

\begin{groupplot}[%clip mode=individual,
     group style = {group size = 4 by 2, horizontal sep = 7mm, vertical sep=5mm},
     height = 0.3\textwidth,
     width = 0.3\textwidth,
     enlargelimits=false,
     ymin=0,
     grid = both,
     grid style = {dashed, black!20},
     xlabel near ticks,
     ylabel near ticks,
     cyan,
     ultra thick,
     mark=none,
     mark options={fill=white}]

    \nextgroupplot[ylabel = \textbf{$J_k/J_\infty$ Simulated}, ymax=57, ymin=-5, ytick={1,10,30,50}]
    \node[clip=false, anchor=south] at (axis cs:1,57) {\textbf{\textcolor{white}{p}Kill\&Hold\textcolor{white}{p}}};
    \pgfplotstableread[header=true,col sep=comma]{\fkhj}\extdata;
    \addplot[mark max, blue] table[x expr=\thisrow{T}*0.01, y=J, col sep=comma] {\extdata};

    \nextgroupplot[ymax=57, ymin=-5,ytick={1,10,30,50}]
    \node[clip=false, anchor=south] at (axis cs:1,57) {\textbf{Skip\&Hold}};
    \pgfplotstableread[header=true,col sep=comma]{\fshj}\extdata;
    \addplot[mark max, pink!75!black] table[x expr=\thisrow{T}*0.01, y=J, col sep=comma] {\extdata};

    \nextgroupplot[ymax=10, ytick={1,3,5,7}]
    \node[clip=false, anchor=south] at (axis cs:1,10) {\textbf{\textcolor{white}{p}Kill\&Zero\textcolor{white}{p}}};
    \pgfplotstableread[header=true,col sep=comma]{\fkzj}\extdata;
    \addplot[mark max, cyan] table[x expr=\thisrow{T}*0.01, y=J, col sep=comma] {\extdata};

    \nextgroupplot[ymax=10, ytick={1,3,5,7}]
    \node[clip=false, anchor=south] at (axis cs:1,10) {\textbf{Skip\&Zero}};
    \pgfplotstableread[header=true,col sep=comma]{\fszj}\extdata;
    \addplot[mark max, red] table[x expr=\thisrow{T}*0.01, y=J, col sep=comma] {\extdata};
      
    \nextgroupplot[xlabel = {Time [s]}, ylabel = \textbf{$J_k/J_\infty$ Real}, ymax=57, ymin=-5, ytick={1,10,30,50}]
    \pgfplotstableread[header=true,col sep=comma]{\fkhr}\extdata;
    \addplot[mark max, blue] table[x expr=\thisrow{T}*0.01, y=J, col sep=comma] {\extdata};

    \nextgroupplot[xlabel = {Time [s]}, ymax=57, ymin=-5,ytick={1,10,30,50}]
    \pgfplotstableread[header=true,col sep=comma]{\fshr}\extdata;
    \addplot[mark max, pink!75!black] table[x expr=\thisrow{T}*0.01, y=J, col sep=comma] {\extdata};

    \nextgroupplot[xlabel = {Time [s]}, ymax=10, ytick={1,3,5,7}]
    \pgfplotstableread[header=true, col sep=comma]{\fkzr}\extdata;
    \addplot[mark max, cyan] table[x expr=\thisrow{T}*0.01, y=J, col sep=comma] {\extdata};

    \nextgroupplot[xlabel = {Time [s]}, ymax=10, ytick={1,3,5,7}]
    \pgfplotstableread[header=true,col sep=comma]{\fszr}\extdata;
    \addplot[mark max, red] table[x expr=\thisrow{T}*0.01, y=J, col sep=comma] {\extdata};

\end{groupplot}
\end{tikzpicture}

    \caption{Normalised performance cost $J_k/J_\infty$ obtained with the Furuta pendulum.
        The upper part of the figure shows simulated data, while the lower part of the figure shows the corresponding values obtained averaging the results of 500 experiments with the real process and hardware.
        Each experiment corresponds to a 500 jobs of the controller (20 misses and 480 hits).}
    \label{fig:cost_simvsreal}
\end{figure}


The large spike in cost comes when the controller is reactivated at time $0.2\,$s. 
Here, the \tH{} strategy again shows much worse performance than \tZ{}, with the peak cost being almost an order of magnitude worse. 
The difference between \tK{} and \tS{} is relatively small, with the latter strategy consistently performing slightly worse than the former. 
This is due to the small extra delay caused by using old data in the \tS{} strategy.

We conducted experiments on a Furuta pendulum, using the same controller for the real plant rather than its model.\footnote{A video, showing experiments with the real system and bursts of deadline misses can be viewed at \url{https://youtu.be/0P0K_7lvKVU}. The video shows a comparison of all the strategies for bursts of $(m = 20, n=480)$. Furthermore, we have included additional experiments with $(m=50, n=450)$ and $(m=75, n=425)$ for the \tSH{} strategy. The results of the additional experiments with higher values of $m$ are not described in the paper, as stability could not be guaranteed (and in fact the pendulum is not at all times kept in the upright position).}
Initially, we performed 500 experiments with 500 jobs each and no deadline misses, to determine the nominal variance of the system---i.e., the stationary variance used to find the static cost $J_\infty$.
For each strategy $\strat$ we then ran 500 identically set up experiments. 
In each experiment, the control task operated according to the task model from Definition~\ref{def:task-model}, experiencing a burst of length $\nummisses=20$ misses, followed by by a recovery interval with $\numhits=480$ deadline hits.

Due to system model uncertainties (e.g., friction) being significant, the rotation angle around the arm axis displayed a considerable variance.
We removed the state from the covariance calculations, since the arm angle majorly impacted the variance despite its inconsequential significance on the system dynamics (the pendulum can be stabilised with the arm being around any position, provided that the pendulum itself is kept in the upright position).
Including the rotation angle would not change the shape of the performance degradation seen in Figure~\ref{fig:cost_simvsreal}. 
However, it would make the results obtained with different strategies $\strat$ not comparable (in some of them, the rotation angle could have varied less across the 500 experiments). 
The covariance matrix $P_k$ was derived by calculating the variance of the closed-loop state vector $\tilde{x}_k$ according to Equation~\eqref{eq:covarcalc}, in each time step $k$. 

The resulting performance cost can be seen in the lower half of Figure~\ref{fig:cost_simvsreal}, where the cost $J_k$ was calculated according to Equation~\eqref{eq:covartocost} and normalised using the stationary cost $J_\infty$. 
Comparing the simulated (upper) and real (lower) performance costs in Figure~\ref{fig:cost_simvsreal}, we notice the similarities between the simulated analysis and the analysis performed on the physical plant. 
%
Particularly, the strategies involving \tH{} actuation show similar behaviours. 
For these strategies, the simulated and real values are very close for the transient burst interval, the secondary cost peak (seen around time $0.4\,$s), and the maximum normalised cost $J_{M, \strat}$.
However, the real cost is recovering slower than in the simulations---an effect that arises due to the nonlinear effects present in the real process, but unmodelled in the simulated environment.
%
Instead, comparing the \tZ{} actuation strategies, the performance cost of the physical experiments during the burst interval seem to improve compared to the simulations.
This is again likely due to the unmodelled dynamics (e.g., friction) appearing in the physical experiment but not in the simulations.
The stiction component of the friction reduces the variance of the states when the actuation signal becomes zero.
With longer burst intervals, a similar behaviour as for the \tH{} actuation strategies would appear.
Despite this difference, both the recovery interval, the secondary cost peak (around $0.4\,$s), and the maximum normalised costs $J_{M,\strat}$ are comparable.

We conclude that the results of the experiments performed on the physical process support the validity of the performance analysis presented in Section~\ref{sec:derivation}.

\subsection{Control Benchmark}
\label{sec:aggregateresults}

In Section~\ref{sec:example} we extensively discussed the results obtained with a single plant (the Furuta pendulum), with the aim of showing that simulating the performance cost yields interesting and relevant results.
As the main novelty of this paper lays in the introduction of the performance analysis as an additional tool to evaluate the behaviour of control systems that can miss deadlines, we here focus on performance.

\begin{figure}[t]
    \centering
    \resizebox{0.95\textwidth}{!}{\begin{tikzpicture}[bar width=1mm]
\small

\begin{groupplot}[group style = {group size = 4 by 9, vertical sep=8mm, horizontal sep=8mm},
    height = 3.05cm,
    width = 4cm,
    ybar,
    ymin = 0,
    ylabel style={at={(-0.35,1)}, anchor=east},
    grid = major,
    grid style = {dashed, black!20},
    x tick label style = {rotate = 60, anchor=north east, inner sep = 0mm},
    xtick distance=1, enlarge x limits=0.05,
    extra x ticks={1}]

    \nextgroupplot[title={\textbf{\textcolor{white}{p}Kill\&Hold\textcolor{white}{p}}}, ymax=70, ylabel={\textbf{$\,\,$Batch 1}},
    xtick distance=3, enlarge x limits=0.02]
    \addplot[thin, black, fill=blue] table[x=ID, y=HITS, col sep=comma]
    {\figdir/data/processed/B1_KH_10.csv};

    \nextgroupplot[title={\textbf{Skip\&Hold}}, ymax=70,
    xtick distance=3, enlarge x limits=0.02]
    \addplot[thin, black, fill=pink!75!black] table[x=ID, y=HITS, col sep=comma]
    {\figdir/data/processed/B1_SH_10.csv};

    \nextgroupplot[title={\textbf{\textcolor{white}{p}Kill\&Zero\textcolor{white}{p}}}, ymax=70,
    xtick distance=3, enlarge x limits=0.02]
    \addplot[thin, black, fill=cyan] table[x=ID, y=HITS, col sep=comma]
    {\figdir/data/processed/B1_KZ_10.csv};

    \nextgroupplot[title={\textbf{Skip\&Zero}}, ymax=70,
    xtick distance=3, enlarge x limits=0.02]
    \addplot[thin, black, fill=red] table[x=ID, y=HITS, col sep=comma]
    {\figdir/data/processed/B1_SZ_10.csv};

    % ------------------------------------------------------

    \nextgroupplot[ymax=120, ylabel={\textbf{$\,\,$Batch 2}},
    xtick distance=3, enlarge x limits=0.02]
    \addplot[thin, black, fill=blue] table[x=ID, y=HITS, col sep=comma]
    {\figdir/data/processed/B2_KH_10.csv};

    \nextgroupplot[ymax=120,
    xtick distance=3, enlarge x limits=0.02]
    \addplot[thin, black, fill=pink!75!black] table[x=ID, y=HITS, col sep=comma]
    {\figdir/data/processed/B2_SH_10.csv};

    \nextgroupplot[ymax=120,
    xtick distance=3, enlarge x limits=0.02]
    \addplot[thin, black, fill=cyan] table[x=ID, y=HITS, col sep=comma]
    {\figdir/data/processed/B2_KZ_10.csv};

    \nextgroupplot[ymax=120,
    xtick distance=3, enlarge x limits=0.02]
    \addplot[thin, black, fill=red] table[x=ID, y=HITS, col sep=comma]
    {\figdir/data/processed/B2_SZ_10.csv};

    % ------------------------------------------------------

    \nextgroupplot[ymax=90, ylabel={\textbf{$\,\,$Batch 3}}]
    \addplot[thin, black, fill=blue] table[x=ID, y=HITS, col sep=comma]
    {\figdir/data/processed/B3_KH_10.csv};

    \nextgroupplot[ymax=90]
    \addplot[thin, black, fill=pink!75!black] table[x=ID, y=HITS, col sep=comma]
    {\figdir/data/processed/B3_SH_10.csv};

    \nextgroupplot[ymax=90]
    \addplot[thin, black, fill=cyan] table[x=ID, y=HITS, col sep=comma]
    {\figdir/data/processed/B3_KZ_10.csv};

    \nextgroupplot[ymax=90]
    \addplot[thin, black, fill=red] table[x=ID, y=HITS, col sep=comma]
    {\figdir/data/processed/B3_SZ_10.csv};

    % ------------------------------------------------------

    \nextgroupplot[ymax=70,
    ylabel={\textbf{$\,\,$Batch 4}}]
    \addplot[thin, black, fill=blue] table[x=ID, y=HITS, col sep=comma]
    {\figdir/data/processed/B4_KH_10.csv};

    \nextgroupplot[ymax=70]
    \addplot[thin, black, fill=pink!75!black] table[x=ID, y=HITS, col sep=comma]
    {\figdir/data/processed/B4_SH_10.csv};

    \nextgroupplot[ymax=70]
    \addplot[thin, black, fill=cyan] table[x=ID, y=HITS, col sep=comma]
    {\figdir/data/processed/B4_KZ_10.csv};

    \nextgroupplot[ymax=70]
    \addplot[thin, black, fill=red] table[x=ID, y=HITS, col sep=comma]
    {\figdir/data/processed/B4_SZ_10.csv};

    % ------------------------------------------------------

    \nextgroupplot[ymax=100,
    ylabel={\textbf{$\,\,$Batch 5}}]
    \addplot[thin, black, fill=blue] table[x=ID, y=HITS, col sep=comma]
    {\figdir/data/processed/B5_KH_10.csv};

    \nextgroupplot[ymax=100]
    \addplot[thin, black, fill=pink!75!black] table[x=ID, y=HITS, col sep=comma]
    {\figdir/data/processed/B5_SH_10.csv};

    \nextgroupplot[ymax=100]
    \addplot[thin, black, fill=cyan] table[x=ID, y=HITS, col sep=comma]
    {\figdir/data/processed/B5_KZ_10.csv};

    \nextgroupplot[ymax=100]
    \addplot[thin, black, fill=red] table[x=ID, y=HITS, col sep=comma]
    {\figdir/data/processed/B5_SZ_10.csv};

    % ------------------------------------------------------

    \nextgroupplot[ymax=120,
    ylabel={\textbf{$\,\,$Batch 6}}]
    \addplot[thin, black, fill=blue] table[x=ID, y=HITS, col sep=comma]
    {\figdir/data/processed/B6_KH_10.csv};

    \nextgroupplot[ymax=120]
    \addplot[thin, black, fill=pink!75!black] table[x=ID, y=HITS, col sep=comma]
    {\figdir/data/processed/B6_SH_10.csv};

    \nextgroupplot[ymax=120]
    \addplot[thin, black, fill=cyan] table[x=ID, y=HITS, col sep=comma]
    {\figdir/data/processed/B6_KZ_10.csv};

    \nextgroupplot[ymax=120]
    \addplot[thin, black, fill=red] table[x=ID, y=HITS, col sep=comma]
    {\figdir/data/processed/B6_SZ_10.csv};


    % ------------------------------------------------------

    \nextgroupplot[ymax=120, ylabel={\textbf{$\,\,$Batch 7}},
    xtick distance=5, enlarge x limits=0.02]
    \addplot[thin, black, fill=blue, bar width=0.6mm] table[x=ID, y=HITS, col sep=comma]
    {\figdir/data/processed/B7_KH_10.csv};

    \nextgroupplot[ymax=120,
    xtick distance=5, enlarge x limits=0.02]
    \addplot[thin, black, fill=pink!75!black, bar width=0.6mm] table[x=ID, y=HITS, col sep=comma]
    {\figdir/data/processed/B7_SH_10.csv};

    \nextgroupplot[ymax=120,
    xtick distance=5, enlarge x limits=0.02]
    \addplot[thin, black, fill=cyan, bar width=0.6mm] table[x=ID, y=HITS, col sep=comma]
    {\figdir/data/processed/B7_KZ_10.csv};

    \nextgroupplot[ymax=120,
    xtick distance=5, enlarge x limits=0.02]
    \addplot[thin, black, fill=red, bar width=0.6mm] table[x=ID, y=HITS, col sep=comma]
    {\figdir/data/processed/B7_SZ_10.csv};

    % ------------------------------------------------------

    \nextgroupplot[ymax=70,
    ylabel={\textbf{$\,\,$Batch 8}}]
    \addplot[thin, black, fill=blue] table[x=ID, y=HITS, col sep=comma]
    {\figdir/data/processed/B8_KH_10.csv};

    \nextgroupplot[ymax=70]
    \addplot[thin, black, fill=pink!75!black] table[x=ID, y=HITS, col sep=comma]
    {\figdir/data/processed/B8_SH_10.csv};

    \nextgroupplot[ymax=70]
    \addplot[thin, black, fill=cyan] table[x=ID, y=HITS, col sep=comma]
    {\figdir/data/processed/B8_KZ_10.csv};

    \nextgroupplot[ymax=70]
    \addplot[thin, black, fill=red] table[x=ID, y=HITS, col sep=comma]
    {\figdir/data/processed/B8_SZ_10.csv};


    % ------------------------------------------------------

    \nextgroupplot[ymax=60,
    ylabel={\textbf{$\,\,$Batch 9}}]
    \addplot[thin, black, fill=blue] table[x=ID, y=HITS, col sep=comma]
    {\figdir/data/processed/B9_KH_10.csv};

    \nextgroupplot[ymax=60]
    \addplot[thin, black, fill=pink!75!black] table[x=ID, y=HITS, col sep=comma]
    {\figdir/data/processed/B9_SH_10.csv};

    \nextgroupplot[ymax=60]
    \addplot[thin, black, fill=cyan] table[x=ID, y=HITS, col sep=comma]
    {\figdir/data/processed/B9_KZ_10.csv};

    \nextgroupplot[ymax=60]
    \addplot[thin, black, fill=red] table[x=ID, y=HITS, col sep=comma]
    {\figdir/data/processed/B9_SZ_10.csv};

\end{groupplot}

\end{tikzpicture}
}
    \caption{Performance Recovery Interval $\recoverylengthinterval_{\strat}$ needed to recover from a burst of 10 deadline misses for different strategies and all the plants in the 9 batches for PI controllers designed according to~\cite{Garpinger:2015}.}
    \label{fig:overview10}
\end{figure}
\afterpage{\clearpage}

We use a set of representative process industrial plants~\cite{Astrom:2004}, developed to benchmark PID design algorithms in the control literature.
The set includes 9 different batches of stable plants, each presenting different features that can be encountered in process industrial plants, for a total of 133 plants.\footnote{%
In our analysis, we present results with 134 plants. In fact, the test set was used in~\cite{Garpinger:2015} to assess a control design method, and an additional plant was added to the set during this assessment. We included this additional plant in our analysis.}
For each batch, all systems have the same structure, but different parameters.
For example, the fourth batch is a stable system with a set of repeated eigenvalues, and a single parameter specifying the system order, which can take six possible values ($3$, $4$, $5$, $6$, $7$, or $8$).
Almost all the plants have a single independent parameter.
The only exception is Batch 7, for which we can specify two different configuration parameters, the first one having 4 possible values and the second one having 9 potential alternatives, with a total of 36 possible configurations.

The analysis methodology presented in this paper is valid for \emph{all} linear control systems.
In Section~\ref{sec:example}, we introduced an LQR controller to analyse the Furuta pendulum. To demonstrate the generality of the analysis, here, we focus on the most common controller class: proportional and integral (PI) controllers. 
These controllers constitute the vast majority of all the control loops in the process industry.\footnote{A 2001 survey by Honeywell~\cite{Desborough2001} states that 97\% of the existing industrial controllers are PI controllers.}
We also performed the analysis for proportional, integral, and derivative (PID) controllers obtaining similar results.
Introducing our tuning for PID controllers requires additional clarifications and details, which we omit due to space limitations.

For each plant we derived a PI controller according to the methodology presented in~\cite{Garpinger:2015}.
In order to showcase the applicability of our analysis to different linear systems, controllers, and noise models, we analyse the resulting closed-loop systems for $\nummisses \in [1,20]$, under the assumption that the systems are affected by brown noise (in comparison to the white noise applied to the Furuta Pendulum).
The brown noise model integrates the white noise and is thus applicable to systems where the noise is more dominant at lower frequencies (e.g., oscillations from nearby machinery).
Figure~\ref{fig:overview10} shows the results for $\nummisses=10$.

The first result that the figure shows is that the plant dynamics plays an important role in how the system reacts to misses.
For example, the plants in Batch 4 and Batch 8 need around 20 hits to recover from a burst of 10 misses.
On the contrary, the plants in Batch~6 and Batch 7 need a higher number of hits to recover from the same burst interval.
The second result that is apparent from the figure is that the \tH{} actuation strategy recovers much better (performance-wise) than \tZ{}.
The reason why \tH{} outperforms \tZ{} can be explained by the brown noise.
The control signal will actively counteract the integrated noise dynamics, meaning that zeroing the control signal removes the compensation against the integrated noise.
%
Finally, comparing the deadline handling strategies, \tK{} performs marginally better than \tS{}.
Under \tK{}, the controller uses fresh data at the beginning of the recovery interval, while \tS{} uses old data.
However, we assumed ideal rollback (i.e., zero additional computation time for the rollback and clean state) for the \tK{} strategy.
In real systems, rollback is difficult to realise and the advantage provided by \tK{} over \tS{} may therefore become unimportant.
%
These findings are consistent throughout all the plants in the experimental set, regardless of the burst interval length $\nummisses$.

The plant dynamics and noise affect the behaviour and performance of the strategies.
Comparing the results of Section~\ref{sec:example} with the aggregate results, it becomes apparent that the actuation strategy (\tZ{} or \tH{}) affects control performance significantly more than the deadline handling strategy.
For the Furuta pendulum (an unstable, nonlinear plant influenced by white noise) \tZ{} performed the best, but for the process industrial systems (stable, linear plants influenced by brown noise) \tH{} outperformed \tZ{}.
These results were apparent even with no consideration taken to the deadline handling strategies.
Thus, we conclude that the plant and noise model should be the ruling factor when choosing the actuation strategy, while the deadline handling strategy is mainly limited by the constraints imposed by the real-time implementation.
