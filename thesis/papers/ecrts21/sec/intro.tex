Feedback control systems have been used as prime examples of hard real-time systems ever since the term was coined.
However, in the past twenty years, it has become increasingly clear that the hard real-time task model is overly strict for most control systems.
Requiring that \emph{all} deadlines of a periodic control task must be met can lead to very conservative designs with low utilisation, low sampling rates, and---in the end---worse than necessary control performance.
Following this line of reasoning, researchers started looking into task models in which tasks can sporadically miss some deadlines, and defined concepts like the ``skip factor''~\cite{Koren:1995}, i.e., the number of correctly executed jobs that must occur between two failed instances.
Task models with failed jobs eventually led to the definition of the weakly hard task model~\cite{Bernat:2001}, that specify constraints on the sequence of jobs that complete their execution correctly and the ones that miss their deadlines.
Adopting the weakly hard model allows a control task to opportunistically execute more frequently, which in general improves reference tracking and disturbance rejection~\cite{Chakraborty:2014a, Linsenmayer:2017, Pazzaglia:2018}.

A recent industrial survey has shown that practitioners are used to work with systems that experience deadline misses~\cite[Questions 14 and 15]{Akesson:2020}.
In a significant percentage of cases, these systems are subject to blackout events that can persist for more than ten consecutive task periods.
Examples of such events are mode switches in mixed-criticality systems, resets due to hardware faults, security attacks, specific types of cache misses, and connectivity issues in networked control systems.
Handling all of these situations by design could require extreme resource over-provisioning.

In this paper we focus precisely on these sporadic system events, which may cause a control task to stall for one or several cycles.
To determine the effect of deadline misses on the control system, it is of utmost importance to analyse the physics of the plant and the effect of control signals not being delivered to it.
For these systems, stability guarantees have been given on the maximum number of tolerable consecutive deadline misses~\cite{Maggio:2020}.
These guarantees only consider \emph{stability} of the closed-loop system as the property to be preserved.
In this paper, we demonstrate that while stability may be preserved, the control system \emph{performance} may be severely affected by the burst of misses.
Performance and stability have been considered simultaneously in the literature.
For example, in~\cite{Chakraborty:2018} a controller is developed that guarantees stability, accepting some level of performance degradation for a given plant.
However, we believe that a lot is left open to investigate, especially with respect to general guarantees.
In particular, in this paper we aim to understand the effect that the deadline handling strategies jointly have on performance and stability, providing a holistic evaluation.
Furthermore, we evaluate our results on both simulated platforms and real control plants.
More precisely, we offer the following contributions:
%
\begin{itemize}
    \item We propose a new type of weakly hard task model, which specifies a \emph{consecutive} deadline miss interval followed by a minimum \emph{consecutive} deadline hit (recovery) interval.
        This model is crucial to properly assess the performance effect of a burst of deadline misses, as the ones reported by practitioners~\cite{Akesson:2020}.

    \item We provide an analysis methodology for stability and performance of control tasks executing under this task model using a variety of implementation choices to handle deadline misses (Kill vs.
        Skip-Next, Zero vs. Hold).
        In particular, we separately consider the two cases in which a miss pattern is repeated (which fits an increased workload situation---for example due to a different mode of execution), and in which it is not possible to specify constraints on the repetition of the miss pattern.

    \item We compare experimental results obtained with a real process---a Furuta pendulum that is stabilised in the upright position---with simulation results based on a linear model of the same process, using the same controller. 
        This shows that simulated data is representative enough to draw conclusions on the controller performance, despite unmodelled nonlinear dynamics and noise.

    \item We present the result of a large scale evaluation campaign of commonly used controllers on a benchmark of 133 industrial plants.
        From this evaluation we conclude that the choice of actuation strategy (i.e., what to do with the control signal when a miss occurs) affects control performance significantly more than the choice of deadline handling strategy (i.e., what to do with the control task when a miss occurs).

\end{itemize}

The rest of this paper is outlined as follows.
In Section~\ref{sec:related} we give a brief overview of related work.
In Section~\ref{sec:model} we present relevant control theory and introduce the stability and performance concepts.
Section~\ref{sec:deadline} describes the weakly hard task models and the strategies that are commonly used to handle deadline misses.
Section~\ref{sec:analysis} presents our extension to the weakly hard task model, and the corresponding stability and performance analysis.
Section~\ref{sec:results} presents our experimental results, and Section~\ref{sec:conc} concludes the paper.
