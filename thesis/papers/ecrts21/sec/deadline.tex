The analysis above holds when the control task meets all its deadlines.
However, the presence of deadline misses changes the behaviour of the system.
The stability of controllers with a number of consecutive deadline misses has been investigated in~\cite{Maggio:2020}.
The results of this investigation attested that, due to their inherent robustness, many control systems can withstand at least a small number of consecutive misses.

To analyse the system, we need to clarify three aspects about the miss behaviour:

\begin{enumerate}[label=(\roman*)]
    \item What happens to the control signal.
    \item What happens to the control task.
    \item The computational model used for the analysis (how many deadlines can we miss, and in what pattern).
\end{enumerate}

For the first item, the actuator can either output a \tZ{} ($u_k = 0_{n_u \times 1}$), or \tH{} the previous value ($u_k = u_{k-1}$).
The choice depends on both the plant dynamics and on the controller, as no strategy in general dominates the other one \cite{schenato09}.
For controllers with integral action, it makes sense to hold the previous control value, under the presumption that the system is still disturbed and that a non-zero control signal is needed to keep the plant close to its operating point.
On the other hand, the \tZ{} strategy may be preferred for plants with unstable or integrator dynamics, where outputting a zero control action may be the safer option.

Considering the second item, at least three different strategies can be employed to deal with a control task that misses its deadline~\cite{Cervin:2005}:
\begin{enumerate*}[label=(\roman*)]
    \item \tK{},
    \item \tS{},
    \item and \tQ{}$\funof{\sigma}$ ($\sigma \in \left\{ 1, 2, 3, \ldots \right\}$).
\end{enumerate*}
%
When the \tK{} strategy is used, the job that missed its deadline is terminated, its changes are rolled back, and the next job is released.
Following the \tS{} strategy, the job that missed its deadline continues its execution.
No new control task jobs are released until the currently running one completes its execution.
\tQ{}$\funof{\sigma}$ behaves similarly to \tS{} in allowing the current job to complete execution, but also allows the activation of new jobs (the queue of active jobs holds up to the most recent $\sigma$ instances of the control task).
In this paper we only analyse \tK{} and \tS{}.
In fact, the results presented in~\cite{Cervin:2005,Maggio:2020} suggest that \tQ{}$\funof{\sigma}$ is not a feasible strategy to handle misses.
The presence of two or more active jobs in the same period creates a chain effect that is hard to recover from and that deteriorates stability and performance.
%verloads the system, that loses the ability of
%withstanding even a few misses.

The last item refers to models of computation.
The weakly hard task model~\cite{Hamdaoui:1995, Bernat:2001} is usually considered expressive enough to analyse the behaviour of tasks that miss their deadlines.
The authors of~\cite{Bernat:2001} propose four definitions for a weakly hard real-time task $\tau$:
\begin{definition}[Weakly Hard Task Models~\cite{Bernat:2001}]%
    \label{def:wh-models}%
    A task $\tau$ may satisfy any of these four weakly hard constraints:
    \begin{enumerate}[label=(\roman*)]
        \item $\tau \vdash \binom{x}{\numtotalanalysed}$: there are at least $x$ hits for every $\numtotalanalysed$ jobs,
        \item $\tau \vdash \overbar{\binom{x}{\numtotalanalysed}}$: there are at most $x$ misses for every $\numtotalanalysed$ jobs, 
        \item $\tau \vdash \genfrac{<}{>}{0pt}{}{x}{\numtotalanalysed}$: there are at least $x$ consecutive hits for every $\numtotalanalysed$ jobs,
        \item $\tau \vdash \overbar{\genfrac{<}{>}{0pt}{}{x}{\numtotalanalysed}}$: there are at most $x$ consecutive misses for every $\numtotalanalysed$ jobs.
    \end{enumerate}
\end{definition}

There has been a lot of research on the second model, often also called $m$-$K$ model~\cite{Koren:1995, Ramanathan:1997, Soudbakhsh:2013, Bund:2014, Frehse:2014, Bund:2015, Hammadeh:2017a, Hammadeh:2017b, Sun:2017, Ahrendts:2018, Soudbakhsh:2018, Pazzaglia:2018, Pazzaglia:2019, Gaukler:2019a} (with $m$ being the maximum number of misses in a window of $K$ activations).
Recently there has also been an analysis of the stability of control systems when the control task behaves according to the fourth model~\cite{Maggio:2020}.

If the misses are due to faults or security attacks, usually the control task experiences an interval of consecutive misses.
When the fault is resolved, the control task starts hitting its deadlines again.
From the performance standpoint, a consecutive number of misses degrades the control quality.
We are interested in what degradation is acceptable and how much time should occur between two potential failures.
Specifically, we look at how many deadline hits should follow a given number of consecutive misses for the system to \emph{recover}.
None of the four models above allow us to formulate this requirement (as they specify either consecutive hits or misses but not both), which leads us to introduce a different weakly hard model of computation, together with its analysis, in Section~\ref{sec:analysis}.
