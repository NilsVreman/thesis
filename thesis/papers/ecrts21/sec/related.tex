The work presented in this paper is closely related to two broad research areas, namely, the analysis of 
\begin{enumerate*}[label=(\roman*)]
    \item weakly hard systems and
    \item fault-tolerant control systems.
\end{enumerate*}

\textbf{Weakly Hard Systems:}
Deadline misses can be seen as sporadic events caused by
unforeseen delays in the system. Such delays could for instance
be induced by overload activations~\cite{Xu:2015, Ernst:2014}
or cache misses~\cite{Altmeyer:2014, Davis:2013}. The idea behind
weakly hard analysis is that deadline misses are permitted under
predefined constraints. Such systems have been analysed
extensively from a real-time scheduling
perspective~\cite{Bernat:1997, Caccamo:1997, Choi:2019,
Hammadeh:2019}.  The weakly hard models have gained traction in
the research community as a tool to understand and analyse
systems with sporadic faults~\cite{Soudbakhsh:2013, Bund:2014,
Frehse:2014, Bund:2015, Hammadeh:2017a, Hammadeh:2017b, Sun:2017,
Ahrendts:2018, Soudbakhsh:2018, Pazzaglia:2018,
Gaukler:2019a}. In a recent paper, Gujarati et
al.~\cite{Gujarati:2019} analysed and compared different methods
for estimating the overall reliability of control systems using
the weakly hard task model. Furthermore, the authors
of~\cite{Broman:2019} proposed a toolchain for analysing the
strongest, satisfied weakly hard constraints as a function of the
worst-case execution time.

\textbf{Fault-Tolerant Control Systems:} 
Real-time systems are sensitive to faults. Due to their
safety-critical nature, it is arguably more important
to guarantee fault-tolerance with respect to other
classes of systems. Some of these faults can be
described using the weakly hard model. Due to the
nature of control systems, special analysis techniques
can combine fault models and the physical characteristics of
systems.

Fault-tolerance has been investigated in
many of its aspects, e.g., fault-aware scheduling 
algorithms~\cite{Rowe:2013, Buttazzo:2000b} and the analysis of systems with unreliable components~\cite{Teich:2015}. Furthermore, 
restart-based design~\cite{Caccamo:2017a, Caccamo:2018} has been used as a technique to guarantee resilience. The fault models are frequently assumed to target overload-prone 
systems, or systems with components subject to sporadic failures. Bursts of faults have been observed to affect real systems~\cite{Phan:2015, Vreman:2020}.
Gujarati et al.~\cite{Gujarati:2018} proposed an analysis 
method for networked control systems that uses active replication and quantifies the resilience of the control
system to stochastic errors. 
Maggio et al.~\cite{Maggio:2020} developed a tool for determining the stability of a control system where the control task behaves according to the weakly hard 
model. From the control perspective, there has been extensive research into both analysis and mitigation of real-time faults in feedback systems~\cite{Ramanathan:1997, Chakraborty:2014b, Chakraborty:2018}. Very often, this research produced tools to analyse the effect of computational delays~\cite{Cervin:2019} and of choosing specific scheduling policies or parameters~\cite{Palopoli:2000, Cervin:2005}, possibly including deadline misses. In a few instances, researchers looked at how to improve the performance of control systems in conjunction with scheduling information~\cite{Buttazzo:2007}. One such effort analyses modifications to the code of classic and simple control systems to handle overruns that reset the period of execution of the control task~\cite{Pazzaglia:2021}.
Abdi et al.~\cite{Caccamo:2017b} proposed a control design method for safe system-level restart, mitigating 
unknown faults during runtime execution, while keeping the system inside a safe operating space. 
Pazzaglia et al.~\cite{Pazzaglia:2019} used the scenario theory to derive a control design method accounting for potential 
deadline misses, and discussed the effect of different deadline handling strategies.
Linsenmayer et al.~\cite{Linsenmayer:2020} worked on the stabilisation of weakly-hard linear control systems for networked control systems, with some extension for nonlinear systems~\cite{Hertneck:2019}. In the considered setup, faults compromise network transmissions, but do not interfere with the controller computation (assuming that the computation is triggered). The work also focused on stability, with no control performance evaluation.

To the best of our knowledge, no previous work has devised a combined stability and performance analysis to understand how faults (even when they can be tolerated) affect the plant that should be controlled when different deadline handling strategies are used.

