In this section, we analyse the stability and performance of a real-time control system that experiences bursts of deadline misses. 
Section~\ref{sec:fault} introduces the fault model, Section~\ref{sec:derivation} derives the control system behaviour subject to different real-time policies and delves into both the stability and performance analysis.

\subsection{Fault Model}%
\label{sec:fault}

Faults can happen during the normal execution of tasks on a platform.
Informally, as a result of a fault, tasks miss their deadlines.
When the fault is resolved, then the original situation is recovered (possibly after a transient initial phase).

Specifically, given a system $\clsys$, we define a \emph{burst interval} $\miss$ as an interval of controller activations in which the control task executing $\ctrler$ consecutively misses $\nummisses$ deadlines, regardless of the strategy used to handle the misses.
We assume that the burst interval $\miss$ is followed by a \emph{recovery interval} $\recovery$, defined as an interval in which the control task consecutively hits $\numhits$ deadlines.

During the burst interval, the deadline misses of the control task are handled using a \emph{deadline handling strategy} $\dstrat$ (\tK{}, $K$, or \tS{}, $S$).
The control signal $u_k$ is selected in accordance with the \emph{actuation strategy} $\hstrat$ (\tZ{}, $Z$, or \tH{}, $H$).
We denote the combination of $\dstrat$ and $\hstrat$ with~$\strat = \left(\dstrat,\hstrat\right)$.
For example $\strat$ could be $SZ$ to indicate that the \tS{} deadline handling strategy is paired with the \tZ{} actuation strategy.
The system \emph{recovers} once it operates close to steady-state.

%%%%%%%%%%%%%%% Nils' Version %%%%%%%%%%%%%%%
From an industrial viewpoint, the proposed fault model is highly relevant.
The common approach is to treat faults as pseudo-independent events adhering to predefined constraints on their incidence rate~\cite{reliabook,montgomery2009introduction, Teich:2017}.
However, during the operation of a control system, faults can be caused by events like network connection problems (e.g., cutting the connection between the sensor and the controller), security attacks, contention on resources.
Studies in the automotive sector, for example, indicate that deadline misses can occur in bursts~\cite{Quinton:2014,Xu:2015}.
In these cases, the controller does not execute properly for a given amount of time (e.g., until the connection is restored, the attack is terminated, or the resource contention is reduced).
The analysis methods we propose allow us to address such situations and to provide tighter bounds on the closed-loop stability and performance than under the previously proposed weakly hard models.
Moreover, following a burst interval, we are interested in analysing the length of the recovery interval $\recovery$ that is needed to return to normal operation under each implementation strategy $\strat$.
Hence, we here extend the weakly hard models of computation with a fifth alternative and then devote the remainder of the paper to its analysis.
 
\begin{definition}[Weakly Hard Fault Model with Burst of Misses]%
    \label{def:task-model}%
    A real-time task $\tau$ may satisfy the weakly hard task model
    \begin{enumerate}[label=(\roman*)]
        \setcounter{enumi}{4}
        \item $\tau \vdash \genfrac{\{}{\}}{0pt}{}{\nummisses}{\numtotalanalysed}$: there are at most $m$ consecutive misses, followed by $\numtotalanalysed-m$ consecutive hits for every $\numtotalanalysed$ jobs.
    \end{enumerate}
    This means that a real-time task $\tau$ behaves according to the model
    %
    $
        \tau \vdash
        \genfrac{\{}{\}}{0pt}{}{\nummisses}{\numtotalanalysed},
    $
    %
    if, whenever $\tau$ experiences a burst interval $\miss$ consisting of $\nummisses$ consecutive deadline misses, it is always followed by a recovery interval $\recovery$ consisting of $\numhits=\numtotalanalysed-\nummisses$ consecutive deadline hits.
\end{definition}

\subsection{Closed-Loop System Dynamics}%
\label{sec:derivation}
In this section we derive the system dynamics for a closed-loop control system under the assumption that we enter a burst interval of length $\nummisses$ after time instant $k$, and after $\nummisses$ deadline misses we start completing the control job in time.

\textbf{Normal Operation: }%
Under \emph{normal operating conditions} the system is not experiencing any deadline misses.
In other words, the system evolves according to the closed-loop system dynamics~\eqref{eq:feedback-basic}.

\textbf{\tKZ{}: }%
%
If a control task deadline miss occurs at time instant $k$, the plant states $x_k$ still evolve as normal.
However, the controller terminates its execution prematurely by killing the job, thus not updating its states ($z_{k+1} = z_k$).
The controller output is determined by the actuation strategy and is here zero ($u_{k+1} = 0$).
Now, consider a burst interval of length $\nummisses$ after time instant $k$.
Recalling that $\tilde{x}_k = \left[ {x_k}^\T {z_k}^\T {u_k}^\T \right]^\T$, we can write the evolution of the closed-loop system for the sequence of $m$ deadline misses followed by a single deadline hit as the product of a matrix representing the behaviour of the system for a hit and a matrix representing the behaviour in case of miss elevated to the power of $m$ to indicate $m$ steps of the system evolution.

The resulting closed-loop system in state-space form is
%
\begin{equation} 
    \setlength\arraycolsep{2pt}
    \label{eq:KZ}
    \begin{bmatrix}
        x_{k+\nummisses+1} \\
        z_{k+\nummisses+1} \\
        u_{k+\nummisses+1}
    \end{bmatrix} = 
    \underbrace{ \clmat{} \begin{bmatrix}
        \Ap                                         & 0_{n_x \times n_z}  & \Bp \\
        0_{n_z \times n_x}    & I                                         & 0_{n_z \times n_u} \\
        0_{n_u \times n_x}   & 0_{n_u \times n_z}   & 0_{n_u \times n_u}
    \end{bmatrix}^{\nummisses}}_{\clmat{}_{KZ}\funof{\nummisses}}
    \begin{bmatrix}
        x_{k} \\
        z_{k} \\
        u_{k}
    \end{bmatrix},
\end{equation}
%
where $\clmat{}_{KZ}\funof{\nummisses}$ represents the system matrix for $\nummisses$ misses under the \tKZ{} strategy, followed by a single hit (the matrix $\clmat{}$ that is multiplied to the left of the equation).
The matrix $\clmat{}$ is the same specified in~\eqref{eq:matrixA}, and represents the first hit that follows the $\nummisses$ misses, hence, we determine how $\tilde{x}_k$ influences $\tilde{x}_{k+m+1}$ ($\nummisses$ misses and one hit).

\textbf{\tKH{}: }%
%
Changing the actuation strategy to \tH{}, slightly alters the system matrix we derived for the \tKZ{} case.
The plant states $x_k$ evolve as normal and the control states $z_k$ are still not updated ($z_{k+1} = z_k$).
However, due to the change in actuation strategy, the last actuated value is instead held ($u_{k+1} = u_k$).
The resulting closed-loop state-space form can be seen in~\eqref{eq:KH}, where $\clmat{}_{KH}\funof{\nummisses}$ is used to represent the system matrix for $\nummisses$ misses under the \tKH{} strategy and matrix $\clmat{}$ is specified in~\eqref{eq:matrixA}.
%
\begin{equation}
    \setlength\arraycolsep{2pt}
    \label{eq:KH}
    \begin{bmatrix}
        x_{k+\nummisses+1} \\
        z_{k+\nummisses+1} \\
        u_{k+\nummisses+1}
    \end{bmatrix} = 
    \underbrace{ \clmat{} \begin{bmatrix}
        \Ap                                         & 0_{n_x \times n_z}  & \Bp \\
        0_{n_z \times n_x}    & I                                         & 0_{n_z \times n_u} \\
        0_{n_u \times n_x}   & 0_{n_u \times n_z}   & I
    \end{bmatrix}^{\nummisses}}_{\clmat{}_{KH}\funof{\nummisses}}
    \begin{bmatrix}
        x_{k} \\
        z_{k} \\
        u_{k}
    \end{bmatrix}
\end{equation}

\textbf{\tSZ{}: }%
%
When the control task misses a deadline under the \tS{} strategy, the job missing the deadline is allowed to continue its execution until completion.
However, no subsequent job of the control task is released until the current job has finished executing.
If the currently active job terminates during period $k$, the next control job is released at the start of the $k+1$-th period.
We can then write the evolution of the system where the control job experiences $\nummisses$ misses before completing its execution, meaning that there is a subsequent hit that uses old information for the error measurements.
While the controller executed only once to completion, the plant evolved for $\nummisses+1$ steps.
The resulting closed-loop state-space form can be seen in~\eqref{eq:SZ}, where $\clmat{}_{SZ}\funof{\nummisses}$ is used to represent the system matrix under the \tSZ{} strategy for $\nummisses$ misses and one completion using old measurements.
%
\begin{equation}
\setlength\arraycolsep{2pt}
\label{eq:SZ}
    \begin{bmatrix}
        x_{k+\nummisses+1} \\
        z_{k+\nummisses+1} \\
        u_{k+\nummisses+1}
    \end{bmatrix} = 
    \underbrace{\begin{bmatrix}
        \Ap^{\nummisses+1}  & 0_{n_x \times n_z}  & \Ap^\nummisses \Bp \\
        -\Bc\Cp             & \Ac                                       & -\Bc\Dp \\
        -\Dc\Cp             & \Cc                                       & -\Dc\Dp
    \end{bmatrix}}_{\clmat{}_{SZ}\funof{\nummisses}}
    \begin{bmatrix}
        x_{k} \\
        z_{k} \\
        u_{k}
    \end{bmatrix}
\end{equation}

\textbf{\tSH{}: }%
%
Similar to \tSZ{}, one job finishes execution after $\nummisses$ consecutive misses.
However, the actuation strategy holds the previous control value during the entire burst interval.
Therefore, the plant evolution is affected by a cumulative sum over the prior control values.
The resulting closed-loop state-space form can be seen in~\eqref{eq:SH}, where $\clmat{}_{SH}\funof{\nummisses}$ is used to represent the system matrix for $\nummisses$ misses under the \tSH{} strategy.
%
\begin{equation}
\setlength\arraycolsep{2pt}
\label{eq:SH}
    \begin{bmatrix}
        x_{k+\nummisses+1} \\
        z_{k+\nummisses+1} \\
        u_{k+\nummisses+1}
    \end{bmatrix} = 
    \underbrace{\begin{bmatrix}
        \Ap^{\nummisses+1}  & 0_{n_x \times n_z}  & \sum_{i=0}^\nummisses \Ap^i\Bp \\
        -\Bc\Cp             & \Ac                                       & -\Bc\Dp \\
        -\Dc\Cp             & \Cc                                       & -\Dc\Dp
    \end{bmatrix}}_{\clmat{}_{SH}\funof{\nummisses}}
    \begin{bmatrix}
        x_{k} \\
        z_{k} \\
        u_{k}
    \end{bmatrix}
\end{equation}

Equations~\eqref{eq:KZ}--\eqref{eq:SH} are inspired by the analysis in~\cite{Maggio:2020}, but we have we introduced two generalisations.
The first one is that our controller is specified as a general state-space system; therefore our method is able to address \emph{all} linear controllers.
The second generalisation is that we could include estimates of the plant states in the controller.
We can thus properly handle the presence of an observer.\footnote{In~\cite{Maggio:2020} the controller state is specified as part of the plant (e.g., when the proportional and integral controller is introduced). This implies that the state is computed although the controller did not execute. Our formulation fixes this by separating the plant execution and the controller states.}
Furthermore, we simplify the calculations by reducing the number of states $\tilde{x}_k$ of the closed-loop matrices.

\paragraph*{Stability}%

We now describe how the system matrices above can be used to analyse stability. 
Recall that a closed-loop control system is stable if and only if the (fixed) system matrix $\clmat{}$ is Schur stable. 
This criterion is also valid for cyclic patterns, where $\clmat{}$ represents the product of all closed-loop state matrices experienced in a full burst--recovery cycle. 
Hence, we can search for the shortest recovery interval length $\numhits$ such that
%
\begin{equation}
\label{eq:stability-cond}
    \underset{i}{\max}{\abs{\eig{i}{\clmat{}^{\numhits-1}\clmat{}_{\strat}\funof{\nummisses}}}} < 1, \quad \strat \in  \{ KZ, KH, SZ, SH \}.
\end{equation}
%
Recall that $\clmat{}_{\strat}\funof{m}$ already includes one hit, thus the left multiplication with $\clmat{}^{n-1}$. 
This is a sufficient condition and not necessary, meaning that a miss occurring during the recovery interval does not immediately imply that the closed-loop system is destabilised. 
We summarise the analysis in the following definition.

\begin{definition}[Static-Cyclic Stability Analysis]%
    We denote the stability analysis from~\eqref{eq:stability-cond} with the term \emph{\nilsstability{} stability analysis}.
    The system under analysis cycles through a sequence of $\nummisses$ misses followed by a sequence of $\numhits$ hits, indefinitely.
\end{definition}
%
The \nilsstability{} analysis assumes a repeating burst--recovery cycle with no interruptions.
This works well for instance in case the misses are due to a permanent overload condition caused by a mode switch (for example from low to high criticality mode in mixed-critical systems).
However, the setting is not very general.
To foster generality, we complement the stability evaluation with a less restrictive stability analysis, based on the proposed task model in Definition~\ref{def:task-model}.

\begin{definition}[\switchingstability{} stability analysis]%
    To guarantee \emph{\switchingstability{} stability}, a system has to be stable under arbitrary switching between all the possible $\nummisses$ realisations (i.e., closed-loop matrices) that comply with all task models $\tau \vdash \genfrac{\{}{\}}{0pt}{}{\nummisses_\subset}{\numtotalanalysed}, \nummisses_\subset \in \{1, \dots,\nummisses\}$ and also include the case in which the system does not miss deadlines.
\end{definition}
In other words, a system is \switchingstability{} stable if and only if it is stable under arbitrary switching of the closed-loop matrices in the set
%
\begin{equation}
    \left\{ \clmat{}^{\numtotalanalysed-1}\clmat{}_\strat\funof{1},\, \clmat{}^{\numtotalanalysed-2}\clmat{}_\strat\funof{2},\, \ldots,\, \clmat{}^{\numtotalanalysed-\nummisses}\clmat{}_\strat\funof{\nummisses},\, \clmat{} \right\}.
\end{equation}
%
Switching stability is unfortunately quite involved.\footnote{We have devoted some research effort into the investigation of a suitable stability analysis for control tasks subject to a set of weakly-hard constraints (of the type presented in Defintion~\ref{def:wh-models}). A summary of our findings can be found at \url{https://arxiv.org/abs/2101.11312}.}
However, many excellent tools have been developed to simplify this analysis (e.g., \code{MJSR}~\cite{Maggio:2020} or the \code{JSR toolbox}~\cite{Jungers:2014} for MATLAB).

\paragraph*{Performance}%
%\label{sec:performance}

We now show how the cost function in Equation~\eqref{eq:covartocost} can be used as a time-varying performance metric.
Before a burst interval, we assume that the system is in the neighbourhood of its steady-state covariance $P_\infty$ and performance $J_\infty$.

When a burst interval of $\nummisses$ missed deadlines occurs, the system will be disrupted and its covariance matrix will evolve according to
%
\begin{equation}
\label{eq:covariance-evolution}
    P_{k+\nummisses+1} = \clmat{}_{\strat}\funof{m} P_k \left( \clmat{}_{\strat}\funof{m} \right)^\T + \clmat{}^{j_{\numhits}} R_w \left( \clmat{}^{j_{\numhits}} \right)^\T,
\end{equation}
where
\begin{equation}
\label{eq:evolutionparameters}
\setlength\arraycolsep{2pt}
    \begin{aligned}
        R_w &= 
        \begin{bmatrix} 
            \sum_{i=0}^{j_{\nummisses}} \Ap^i\,\Wp\,R\,\Wp^\T\,(\Ap^i)^\T & 0_{n_x \times n_z+n_u} \\ 
            0_{n_z + n_u \times n_x}               & 0_{n_z + n_u \times n_z+ n_u}
        \end{bmatrix}, \\
        j_{\nummisses} &= 
        \begin{cases} 
            \nummisses - 1 & \text{ if } \dstrat = K\,\, \text{(\tK{}),} \\ 
            \nummisses & \text{ if } \dstrat = S\,\,\,\, \text{(\tS{}),} 
        \end{cases} \\
        j_{\numhits} &= 
        \begin{cases} 
            1 \phantom{||222} & \text{ if } \dstrat = K\,\, \text{(\tK{}),} \\ 
            0 & \text{ if } \dstrat = S\,\,\,\, \text{(\tS{}).}
        \end{cases}
        \end{aligned}
\end{equation}

$\Ap$ and $\Wp$ are matrices from the plant evolution in~\eqref{eq:background:plant}, $R$ is the noise intensity from~\eqref{eq:covevolution}, and $\clmat{}$ is the closed-loop matrix from~\eqref{eq:matrixA}.
The cost will simultaneously change following~\eqref{eq:covartocost}.
In the recovery interval, the covariance is again governed by the normal closed-loop evolution described in~\eqref{eq:covevolution}.
The system is said to have recovered once the cost is arbitrarily close to the steady-state cost.
We evaluate this as
%
\begin{equation}
    \label{eq:cost-threshold}
    \abs{\frac{J_{\infty}-J_k}{J_\infty}} < \varepsilon,
\end{equation}
where $\varepsilon> 0$ is the \emph{recovery threshold}.

\begin{definition}[Performance Recovery Interval]%
\label{def:recovery-lenght-interval}%
    We define the recovery length interval $\recoverylengthinterval_{\strat}$ as the smallest $\numhits$ such that~\eqref{eq:cost-threshold} is satisfied for all $k \geq \numhits$ when using $\strat$ to handle deadline misses.
\end{definition}

\begin{definition}[Maximum normalised cost]%
\label{def:maximum-system-cost}%
    We denote the maximum normalised cost of the system by
    %
    \begin{equation}
        J_{M,\strat} = \max_k \frac{J_{k, \strat}}{J_\infty},
    \end{equation}
    %
    where $J_{k,\strat}$ is the cost computed according to~\eqref{eq:covartocost} when using $\strat$ to handle the deadline misses.
\end{definition}
\begin{figure}[t]
    \centering
    \resizebox{\textwidth}{!}{%
\begin{tikzpicture}

\begin{axis}[%clip=false,
thick, 
xlabel={Time},
xtick={1,2,3,5,...,17},
ylabel={Normalised Cost},
ylabel near ticks,
xlabel near ticks,
width=12cm, height=4.5cm,
xmin=0.5,
ymin=0, ymax=6.25,
enlarge x limits=0]

\fill[red!15] (axis cs:0.01,0) rectangle (axis cs:3,6.25);
\fill[green!15] (axis cs:3,0) rectangle (axis cs:13,6.25);
\fill[blue!30] (axis cs:0,0.9) rectangle (axis cs:20,1.1);

\node[rectangle, draw, fill=white, align=center] at (axis cs:14.5,4.5) {%
    \tikz{%
        \fill[white] (0,0) rectangle (5,50);%
        \draw[blue, ultra thick] (0,25) -- (5,25);%
        \fill[blue] (2.5,25) ellipse (1 and 12.5);}%
        $\,\,J_{k,\mathcal{H}}/J_\infty$};
\draw[black] (axis cs:0,0) rectangle (axis cs:17,6.25);

\addplot table [col sep=comma, x=T, y=J] {figs/ecrts21/data/example-data.csv};
\draw[dashed, black] (axis cs:0,5.4692) -- (axis cs:5,5.4692) node[right, xshift=1mm, blue] {$J_{M,\mathcal{H}}$};

\draw[<->, blue, thick] (axis cs:5,0) -- (axis cs:5,5.46921);
\draw[<->, red, thick] (axis cs:0.5,3) -- node[above] {$m$} (axis cs:3,3);
\draw[<->, green!70!black, thick] (axis cs:3,3) -- node[above] {$n^*_{\mathcal{H}}$} (axis cs:13,3);

\end{axis}

\end{tikzpicture}
}%
 
    \caption{Illustration of normalised cost ($J_k/J_\infty$), performance recovery interval $\recoverylengthinterval_{\strat}$ and maximum normalised cost $J_{M,\strat}$ on a data trace. The example uses $\strat = \text{\tKZ{}}$ and $\varepsilon=0.1$.}
    \label{fig:recoveryandpeak}
\end{figure}
Figure~\ref{fig:recoveryandpeak} gives a graphical representation of $\recoverylengthinterval_{\strat}$ and $J_{M,\strat}$ for an execution trace in which the controller experiences 3 misses and uses \tKZ{} as strategy $\strat$.

Compared to the stability analysis, the performance analysis also takes into account state deviations and uncertainty due to disturbances.
In Section~\ref{sec:derivation} we used the system dynamics to analyse the stability of the system.
The disturbance term $w_k$ was neglected as it does not influence the system stability.
However, its presence (as the presence of any disturbance) changes the dynamic behaviour of the system.
For the performance metric, the state covariance matrix $P_k$ evolves according to both the noise intensity and the system dynamics~\eqref{eq:covariance-evolution}.
The result is that the performance analysis provides us with a conservative (but more realistic) recovery interval, that takes system uncertainties into consideration.

To find the length of the recovery interval, we evolve the state covariance during a burst interval, using a specific strategy $\strat$ according to~\eqref{eq:covariance-evolution}.
Thereafter, the state covariance is evolved under normal operation, according to~\eqref{eq:covevolution}, until~\eqref{eq:cost-threshold} is satisfied, allowing us to find the performance recovery interval $\recoverylengthinterval_{\strat}$.
