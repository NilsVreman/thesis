\documentclass[a4paper,UKenglish,cleveref, autoref, thm-restate]{lipics-v2021}

\bibliographystyle{plainurl}

%%%%%%%%%%%%%%%%%%%%%%%%%%%%%%%%%%%%%%%%%%%%%%%%%%%%%%%%%%%%%%%%%%%%%%%%%%%%%%%%
% In this file, only packages are allowed. These packages should be explained to
% greatest possible extent.
%%%%%%%%%%%%%%%%%%%%%%%%%%%%%%%%%%%%%%%%%%%%%%%%%%%%%%%%%%%%%%%%%%%%%%%%%%%%%%%%

\usepackage{amsmath, amssymb, amsthm}       % Math related packages
\usepackage{microtype}                      % For char. and variable expansion
\usepackage{cite}                           % For citations
\usepackage{xcolor}                         % For proper colours
\usepackage{colortbl}                       % For colourful tables
\usepackage[inline]{enumitem}               % Used for inline, horizontal lists
\usepackage{url}                            % For URLs
\usepackage{xfrac}                          % For inline fractions
\usepackage{relsize}                        % For \smaller
\usepackage{array}                          % For alignment in tables

%% tikz stuff
\usepackage{tikz}
\usepackage{pgfplots}
\usepgfplotslibrary{groupplots} 
\usetikzlibrary{patterns}                   % For error plots
\usepgfplotslibrary{fillbetween}            % used in error plot
\usetikzlibrary{math}                       % used in heatmap for color shading
\usetikzlibrary{shapes.misc,backgrounds}


\usetikzlibrary{arrows}
\tikzset{every picture/.style={auto, line width=0.7pt,>=latex,font=\small}}

\newcommand{\ewhc}{EWHC}

%%% Math Commands
\newcommand{\Alifted}{\mathcal{L}}
\newcommand{\sos}{\mathrm{SOS}}

\tikzexternalize[prefix=tikz/] % needs to be in the main

\title{Stability and Performance Analysis of Control Systems Subject to Bursts of Deadline Misses}

\author{Nils Vreman}{Lund University, Department of Automatic Control, Sweden}{nils.vreman@control.lth.se}{https://orcid.org/0000-0002-6732-9500}{}
\author{Anton Cervin}{Lund University, Department of Automatic Control, Sweden}{anton.cervin@control.lth.se}{https://orcid.org/0000-0003-4889-8772}{}
\author{Martina Maggio}{Saarland University, Department of Computer Science, Germany \and Lund University, Department of Automatic Control, Sweden}{maggio@cs.uni-saarland.de}{https://orcid.org/0000-0002-1143-1127}{}

\authorrunning{N. Vreman, A. Cervin, and M. Maggio}

\titlerunning{Stability and Performance Analysis of Control Systems \ldots}

\ccsdesc{Computer systems organization~Embedded and cyber-physical systems}
\ccsdesc{Computer systems organization~Real-time systems}
\ccsdesc{Computer systems organization~Dependable and fault-tolerant systems and networks}


\keywords{Fault-Tolerant Control Systems, Weakly Hard Task Model}

\acknowledgements{The authors are members of the ELLIIT Strategic Research Area at Lund University. This project has received funding from the European Union's Horizon 2020 research and innovation programme under grant agreement Number 871259 (ADMORPH project). This publication reflects only the authors' view and the European Commission is not responsible for any use that may be made of the information it contains.
}

\Copyright{Nils Vreman, Anton Cervin, Martina Maggio}

\EventEditors{Bj\"{o}rn B. Brandenburg}
\EventNoEds{1}
\EventLongTitle{33rd Euromicro Conference on Real-Time Systems (ECRTS 2021)}
\EventShortTitle{ECRTS 2021}
\EventAcronym{ECRTS}
\EventYear{2021}
\EventDate{July 5--9, 2021}
\EventLocation{Virtual Conference}
\EventLogo{}
\SeriesVolume{196}
\ArticleNo{15}

\begin{document}

\maketitle

\begin{abstract}
Control systems are by design robust to various disturbances, ranging from noise to unmodelled dynamics. 
Recent work on the weakly hard model---applied to controllers---has shown that control tasks can also be inherently robust to deadline misses. 
However, existing exact analyses are limited to the stability of the closed-loop system. 
In this paper we show that stability is important but cannot be the only factor to determine whether the behaviour of a system is acceptable also under deadline misses. 
We focus on systems that experience bursts of deadline misses and on their recovery to normal operation. 
We apply the resulting comprehensive analysis (that includes both stability and performance) to a Furuta pendulum, comparing simulated data and data obtained with the real plant. 
We further evaluate our analysis using a benchmark set composed of 133 systems, which is considered representative of industrial control plants. 
Our results show the handling of the control signal is an extremely important factor in the performance degradation that the controller experiences---a clear indication that only a stability test does not give enough indication about the robustness to deadline misses.
\end{abstract}

\section{Introduction}
\label{sec:intro}
Robustness is an essential concern in the design of control systems; they must be able to reliably handle nonlinear effects, unmodeled dynamics and noise, as well as delays in signal transmissions and dropped packets.
A lesser known problem concerns the assessment of robustness to \emph{computational issues} when controllers are implemented as periodic tasks in cheap embedded platforms.
Such tasks are expected to execute with real-time guarantees, i.e., their execution must be completed before a well-defined \emph{deadline}, when the control output must be sent to the actuator.
However, it is common in practice~\cite{akesson2020empirical} that tasks do not always complete within their deadline, causing what is called a \emph{deadline miss}.
This may be caused by delays in computation and memory accesses, transient overloads, bugs and other issues.

A popular model to describe real-time systems allowing deadline misses is the \emph{weakly-hard} model~\cite{Bernat:2001}. 
Weakly-hard tasks feature constraints defining a maximum number of deadlines that can be missed (alternatively, a minimum number to be satisfied) in a given number of consecutive periods.
This model is also the focus of this work.
To analyse the effects on the controlled plant, it is necessary to specify also \emph{what happens when the miss is experienced}, both in terms of changes to the control signal and of actions taken to deal with the failed computation~\cite{Pazzaglia:2019}.
An instance that experiences a deadline miss can be allowed to continue executing until completion (and possibly used later), while in other applications it is stopped and discarded instead.

There is however a mismatch between the guarantees that can be obtained for real-time tasks and platforms~\cite{Ernst:2015,choi2019job}, and the analysis available for \emph{control} tasks under the weakly-hard model.
Fewer works deal with \emph{stability} analysis of weakly-hard real-time control tasks, often targeting specific use-cases. 
For instance, the analysis in~\cite{Maggio:2020} is limited to constraints specifying a maximum number of \emph{consecutive} deadline misses.
The results in \cite{Linsenmayer:2017,linsenmayer2020linear}, obtained for networked linear control systems having packet dropouts bounded using the weakly-hard model, can not be generalised for \emph{late completions} or \emph{sets} of weakly-hard constraints.
The authors of~\cite{liang2019security,liang2020leveraging} studied safety guarantees of weakly-hard controllers, considering a miss as a discarded computation with a known periodic pattern.
%
In \cite{huang2020saw, huang2019formal}, an over-approximation-based approach is proposed to check the safety of nonlinear weakly-hard systems, where misses are treated as discarded computations and the actuator holds its previous value.
Convergence rates (providing sufficient stability guarantees) are analysed in~\cite{Gaukler:2019a}.
A Lyapunov-based stability analysis of nonlinear weakly-hard systems is studied in~\cite{hertneck2021efficient}, with deadline misses treated as packet dropouts.
However, the state-of-the-art listed above lack generalisability to more expressive real-time implementations, such as different deadline miss models or handling strategies.

This paper aims at filling the gap, by providing a stability analysis that can be applied to a class of generic weakly-hard models and deadline miss handling strategies.
First, we formally extend the weakly-hard model to explicitly consider the strategy used to handle the miss events. 
By leveraging an automaton representation of the sequences allowed by (a set of) extended weakly-hard constraints, we use Kronecker lifting and the joint spectral radius to properly express its stability conditions.
Using the concept of constraint dominance, we prove analytic bounds on the stability of a weakly-hard system with respect to \emph{less dominant} constraints.
Finally, we analyse the stability of the resulting closed-loop systems using \code{SparseJSR}~\cite{sparsejsr}, which exploits the sparsity pattern that naturally arises in the Kronecker lifted representation.
The proposed analysis calls for modularity and separation of concern, and can be a useful tool to decouple the constraint specification and the control verification.
%, the embedded system designer can extract a set of constraints to be used in the design phase, and the control engineer can verify that the proposed constraints satisfy all control requirements. 


\section{Related Work}
\label{sec:related}
The work presented in this paper is closely related to two broad research areas, namely, the analysis of 
\begin{enumerate*}[label=(\roman*)]
    \item weakly hard systems and
    \item fault-tolerant control systems.
\end{enumerate*}

\textbf{Weakly Hard Systems:}
Deadline misses can be seen as sporadic events caused by
unforeseen delays in the system. Such delays could for instance
be induced by overload activations~\cite{Xu:2015, Ernst:2014}
or cache misses~\cite{Altmeyer:2014, Davis:2013}. The idea behind
weakly hard analysis is that deadline misses are permitted under
predefined constraints. Such systems have been analysed
extensively from a real-time scheduling
perspective~\cite{Bernat:1997, Caccamo:1997, Choi:2019,
Hammadeh:2019}.  The weakly hard models have gained traction in
the research community as a tool to understand and analyse
systems with sporadic faults~\cite{Soudbakhsh:2013, Bund:2014,
Frehse:2014, Bund:2015, Hammadeh:2017a, Hammadeh:2017b, Sun:2017,
Ahrendts:2018, Soudbakhsh:2018, Pazzaglia:2018,
Gaukler:2019a}. In a recent paper, Gujarati et
al.~\cite{Gujarati:2019} analysed and compared different methods
for estimating the overall reliability of control systems using
the weakly hard task model. Furthermore, the authors
of~\cite{Broman:2019} proposed a toolchain for analysing the
strongest, satisfied weakly hard constraints as a function of the
worst-case execution time.

\textbf{Fault-Tolerant Control Systems:} 
Real-time systems are sensitive to faults. Due to their
safety-critical nature, it is arguably more important
to guarantee fault-tolerance with respect to other
classes of systems. Some of these faults can be
described using the weakly hard model. Due to the
nature of control systems, special analysis techniques
can combine fault models and the physical characteristics of
systems.

Fault-tolerance has been investigated in
many of its aspects, e.g., fault-aware scheduling 
algorithms~\cite{Rowe:2013, Buttazzo:2000b} and the analysis of systems with unreliable components~\cite{Teich:2015}. Furthermore, 
restart-based design~\cite{Caccamo:2017a, Caccamo:2018} has been used as a technique to guarantee resilience. The fault models are frequently assumed to target overload-prone 
systems, or systems with components subject to sporadic failures. Bursts of faults have been observed to affect real systems~\cite{Phan:2015, Vreman:2020}.
Gujarati et al.~\cite{Gujarati:2018} proposed an analysis 
method for networked control systems that uses active replication and quantifies the resilience of the control
system to stochastic errors. 
Maggio et al.~\cite{Maggio:2020} developed a tool for determining the stability of a control system where the control task behaves according to the weakly hard 
model. From the control perspective, there has been extensive research into both analysis and mitigation of real-time faults in feedback systems~\cite{Ramanathan:1997, Chakraborty:2014b, Chakraborty:2018}. Very often, this research produced tools to analyse the effect of computational delays~\cite{Cervin:2019} and of choosing specific scheduling policies or parameters~\cite{Palopoli:2000, Cervin:2005}, possibly including deadline misses. In a few instances, researchers looked at how to improve the performance of control systems in conjunction with scheduling information~\cite{Buttazzo:2007}. One such effort analyses modifications to the code of classic and simple control systems to handle overruns that reset the period of execution of the control task~\cite{Pazzaglia:2021}.
Abdi et al.~\cite{Caccamo:2017b} proposed a control design method for safe system-level restart, mitigating 
unknown faults during runtime execution, while keeping the system inside a safe operating space. 
Pazzaglia et al.~\cite{Pazzaglia:2019} used the scenario theory to derive a control design method accounting for potential 
deadline misses, and discussed the effect of different deadline handling strategies.
Linsenmayer et al.~\cite{Linsenmayer:2020} worked on the stabilisation of weakly-hard linear control systems for networked control systems, with some extension for nonlinear systems~\cite{Hertneck:2019}. In the considered setup, faults compromise network transmissions, but do not interfere with the controller computation (assuming that the computation is triggered). The work also focused on stability, with no control performance evaluation.

To the best of our knowledge, no previous work has devised a combined stability and performance analysis to understand how faults (even when they can be tolerated) affect the plant that should be controlled when different deadline handling strategies are used.



\section{System Behaviour in Nominal Conditions}
\label{sec:model}
To provide a comprehensive analysis framework, we need to examine what occurs in each time interval $(\pi_k)_{k \in \N_{\geq}}$, with $\pi_k = [a_0 + k\cdot \Ts, a_0 + (k+1)\cdot \Ts)$.
In this context, an extension of the weakly-hard model is required to account for the given deadline miss handling strategy, denoted with the symbol $\strat$.
%
\begin{definition}[Extended weakly-hard model $\tau \vdash \lambda^{\strat}$]%
    \label{def:new-mk}%
    A task $\tau$ may satisfy any combination of the four \emph{extended weakly-hard constraints} (\ewhc{}) $\lambda^{\strat}$:
    \begin{enumerate}[label=(\roman*)]
        \item $\tau \vdash \elcssanymiss{}{\strat}$: in any window of $\ell$ consecutive jobs, at most $x$ intervals lack a job completion;
        \item $\tau \vdash \elcssanyhit{}{\strat}$:  in any window of $\ell$ consecutive jobs, at least $x$ intervals have a job completion;
        \item $\tau \vdash \elcssrowmiss{}{\strat}$: in any window of $\ell$ consecutive jobs, at most $x$ \emph{consecutive} intervals lack a job completion;
        \item $\tau \vdash \elcssrowhit{}{\strat}$: in any window of $\ell$ consecutive jobs, at least $x$ \emph{consecutive} intervals have a job completion
    \end{enumerate}
    with $x\in \N_{\geq}$, $\ell \in \N_{>}$, and $x\leq \ell$, while using strategy $\strat$ to handle potential deadline misses.
\end{definition}
%
The definition above differs from the original weakly-hard model of~\cite{Bernat:2001}, since
\begin{enumerate*}[label=(\roman*)]
    \item it explicitly introduces the handling strategy $\strat$; and
    \item it focuses on the presence of a new control command at the end of each time interval $\pi_k$, instead of checking the deadline miss events, which guarantees its applicability also for strategies different than \tK{}.
\end{enumerate*}

We now require an expressive alphabet $\Sigma\left(\strat\right)$ to characterize the behaviour of task $\tau$ in each possible time interval.
For both \tK{} and \tS{} strategies, each interval $\pi_k$ contains at most one activated and one completed job.
This restricts the possible behaviours to three cases:
\begin{enumerate}[label=(\roman*)]
    \item a time interval in which the same job is both released and completed is denoted by $\cH$ (\emph{hit});
    \item a time interval in which no job is completed is denoted by $\cM$ (\emph{miss});
    \item a time interval in which no job is released, but a job (released in a previous interval) is completed, is denoted by $\cR$ (\emph{recovery}).
\end{enumerate}
%
By checking all unique combinations of job activations and completions in each interval, we obtain the alphabets for \tK{} and \tS{} as $\Sigma\left(\tK{}\right) = \{ \cM, \cH \}$ and $\Sigma\left(\tS{}\right) = \{ \cM, \cH, \cR \}$, respectively.
The recovery character $\cR$ is used in the \tS{} alphabet to identify the late \emph{completion} of a job.
As a consequence, $\cR$ is treated equivalently to $\cH$ when verifying the extended weakly hard constraints (\ewhc{}).


The algebra presented in Section~\ref{ssec:whalgebra} is extended to the new alphabet.
We assign a character of the alphabet $\Sigma\left(\strat\right)$ to each interval $\pi_k$.
A word $w = \seq{c_1,c_2,\dots,c_N}$ is used to represent a sequence of $N$ outcomes for task $\tau$, with $c_k \in \Sigma\left(\strat\right)$ representing the outcome associated to the interval $\pi_k$. 
To enforce only feasible sequences, we introduce an order constraint for the $\cR$ character with the following Rule.
%
\begin{rule_}[Outcome ordering]%
    \label{rule:R}%
    For any word $w \in \Sigma\left(\tS{}\right)^N$, $\cR$ may only directly follow $\cM$, or be the initial element of the word.
\end{rule_}

The extended weakly-hard model also inherits all the properties of the original weakly-hard model.
In particular, the satisfaction set of $\lambda^\strat$ can be defined for $N\geq 1$ as $\sset{N}{\lambda^{\strat}} = \{ w \in \Sigma\left(\strat\right)^N \mid w \vdash \lambda^{\strat} \}$, and the constraint domination still holds as $\lambda^{\strat}_{i} \preceq \lambda^{\strat}_{j}$ if $\sset{}{\lambda^{\strat}_{i}} \subseteq \sset{}{\lambda^{\strat}_{j}}$.


\section{System Behaviour with Deadline Misses}
\label{sec:deadline}
The analysis above holds when the control task meets all its deadlines.
However, the presence of deadline misses changes the behaviour of the system.
The stability of controllers with a number of consecutive deadline misses has been investigated in~\cite{Maggio:2020}.
The results of this investigation attested that, due to their inherent robustness, many control systems can withstand at least a small number of consecutive misses.

To analyse the system, we need to clarify three aspects about the miss behaviour:

\begin{enumerate}[label=(\roman*)]
    \item What happens to the control signal.
    \item What happens to the control task.
    \item The computational model used for the analysis (how many deadlines can we miss, and in what pattern).
\end{enumerate}

For the first item, the actuator can either output a \emph{zero} ($u_t = 0_{n_u \times 1}$), or \emph{hold} the previous value ($u_t = u_{t-1}$).
The choice depends on both the plant dynamics and on the controller, as no strategy in general dominates the other one \cite{schenato09}.
For controllers with integral action, it makes sense to hold the previous control value, under the presumption that the system is still disturbed and that a non-zero control signal is needed to keep the plant close to its operating point.
On the other hand, the zero strategy may be preferred for plants with unstable or integrator dynamics, where outputting a zero control action may be the safer option.

Considering the second item, at least three different strategies can be employed to deal with a control task that misses its deadline~\cite{Cervin:2005}:
\begin{enumerate*}[label=(\roman*)]
    \item \emph{Kill},
    \item \emph{Skip-Next},
    \item and \emph{Queue$\funof{\sigma}$} ($\sigma \in \left\{ 1, 2, 3, \ldots \right\}$).
\end{enumerate*}
%
When the Kill strategy is used, the job that missed its deadline is terminated, its changes are rolled back, and the next job is released.
Following the Skip-Next strategy, the job that missed its deadline continues its execution.
No new control task jobs are released until the currently running one completes its execution.
Queue$\funof{\sigma}$ behaves similarly to Skip-Next in allowing the current job to complete execution, but also allows the activation of new jobs (the queue of active jobs holds up to the most recent $\sigma$ instances of the control task).
In this paper we only analyse Kill and Skip-Next.
In fact, the results presented in~\cite{Cervin:2005,Maggio:2020} suggest that Queue$\funof{\sigma}$ is not a feasible strategy to handle misses.
The presence of two or more active jobs in the same period creates a chain effect that is hard to recover from and that deteriorates stability and performance.
%verloads the system, that loses the ability of
%withstanding even a few misses.

The last item refers to models of computation.
The weakly hard task model~\cite{Hamdaoui:1995, Bernat:2001} is usually considered expressive enough to analyse the behaviour of tasks that miss their deadlines.
The authors of~\cite{Bernat:2001} propose four definitions for a weakly hard real-time task $\tau$:
\begin{definition}[Weakly Hard Task Models~\cite{Bernat:2001}]%
    \label{def:wh-models}%
    A task $\tau$ may satisfy any of these four weakly hard constraints:
    \begin{enumerate}[label=(\roman*)]
        \item $\tau \vdash \binom{n}{\!\:\!\:\numtotalanalysed\!\:\!\:}$: there are at least $n$ hits for every $\numtotalanalysed$ jobs,
        \item $\tau \vdash \overbar{\binom{m}{\numtotalanalysed}}$: there are at most $m$ misses for every $\numtotalanalysed$ jobs, 
        \item $\tau \vdash \genfrac{<}{>}{0pt}{}{n}{\!\:\!\:\numtotalanalysed\!\:\!\:}$: there are at least $n$ consecutive hits for every $\numtotalanalysed$ jobs,
        \item $\tau \vdash \overbar{\genfrac{<}{>}{0pt}{}{m}{\numtotalanalysed}}$: there are at most $m$ consecutive misses for every $\numtotalanalysed$ jobs.
    \end{enumerate}
\end{definition}

There has been a lot of research on the second model, often also called $m$-$K$ model~\cite{Koren:1995, Ramanathan:1997, Soudbakhsh:2013, Bund:2014, Frehse:2014, Bund:2015, Hammadeh:2017a, Hammadeh:2017b, Sun:2017, Ahrendts:2018, Soudbakhsh:2018, Pazzaglia:2018, Pazzaglia:2019, Gaukler:2019a} (with $m$ being the maximum number of misses in a window of $K$ activations).
Recently there has also been an analysis of the stability of control systems when the control task behaves according to the fourth model~\cite{Maggio:2020}.

If the misses are due to faults or security attacks, usually the control task experiences an interval of consecutive misses.
When the fault is resolved, the control task starts hitting its deadlines again.
From the performance standpoint, a consecutive number of misses degrades the control quality.
We are interested in what degradation is acceptable and how much time should occur between two potential failures.
Specifically, we look at how many deadline hits should follow a given number of consecutive misses for the system to \emph{recover}.
None of the four models above allow us to formulate this requirement (as they specify either consecutive hits or misses but not both), which leads us to introduce a different weakly hard model of computation, together with its analysis, in Section~\ref{sec:analysis}.


\section{Burst Interval Analysis}
\label{sec:analysis}
In this section, we analyse the stability and performance of a real-time control system that experiences bursts of deadline misses. 
Section~\ref{sec:fault} introduces the fault model, Section~\ref{sec:derivation} derives the control system behaviour subject to different real-time policies and delves into both the stability and performance analysis.

\subsection{Fault Model}%
\label{sec:fault}

Faults can happen during the normal execution of tasks on a platform.
Informally, as a result of a fault, tasks miss their deadlines.
When the fault is resolved, then the original situation is recovered (possibly after a transient initial phase).

Specifically, given a system $\clsys$, we define a \emph{burst interval} $\miss$ as an interval of controller activations in which the control task executing $\ctrler$ consecutively misses $\nummisses$ deadlines, regardless of the strategy used to handle the misses.
We assume that the burst interval $\miss$ is followed by a \emph{recovery interval} $\recovery$, defined as an interval in which the control task consecutively hits $\numhits$ deadlines.

During the burst interval, the deadline misses of the control task are handled using a \emph{deadline handling strategy} $\dstrat$ (\tK{}, $K$, or \tS{}, $S$).
The control signal $u_k$ is selected in accordance with the \emph{actuation strategy} $\hstrat$ (\tZ{}, $Z$, or \tH{}, $H$).
We denote the combination of $\dstrat$ and $\hstrat$ with~$\strat = \left(\dstrat,\hstrat\right)$.
For example $\strat$ could be $SZ$ to indicate that the \tS{} deadline handling strategy is paired with the \tZ{} actuation strategy.
The system \emph{recovers} once it operates close to steady-state.

%%%%%%%%%%%%%%% Nils' Version %%%%%%%%%%%%%%%
From an industrial viewpoint, the proposed fault model is highly relevant.
The common approach is to treat faults as pseudo-independent events adhering to predefined constraints on their incidence rate~\cite{reliabook,montgomery2009introduction, Teich:2017}.
However, during the operation of a control system, faults can be caused by events like network connection problems (e.g., cutting the connection between the sensor and the controller), security attacks, contention on resources.
Studies in the automotive sector, for example, indicate that deadline misses can occur in bursts~\cite{Quinton:2014,Xu:2015}.
In these cases, the controller does not execute properly for a given amount of time (e.g., until the connection is restored, the attack is terminated, or the resource contention is reduced).
The analysis methods we propose allow us to address such situations and to provide tighter bounds on the closed-loop stability and performance than under the previously proposed weakly hard models.
Moreover, following a burst interval, we are interested in analysing the length of the recovery interval $\recovery$ that is needed to return to normal operation under each implementation strategy $\strat$.
Hence, we here extend the weakly hard models of computation with a fifth alternative and then devote the remainder of the paper to its analysis.
 
\begin{definition}[Weakly Hard Fault Model with Burst of Misses]%
    \label{def:task-model}%
    A real-time task $\tau$ may satisfy the weakly hard task model
    \begin{enumerate}[label=(\roman*)]
        \setcounter{enumi}{4}
        \item $\tau \vdash \genfrac{\{}{\}}{0pt}{}{\nummisses}{\numtotalanalysed}$: there are at most $m$ consecutive misses, followed by $\numtotalanalysed-m$ consecutive hits for every $\numtotalanalysed$ jobs.
    \end{enumerate}
    This means that a real-time task $\tau$ behaves according to the model
    %
    $
        \tau \vdash
        \genfrac{\{}{\}}{0pt}{}{\nummisses}{\numtotalanalysed},
    $
    %
    if, whenever $\tau$ experiences a burst interval $\miss$ consisting of $\nummisses$ consecutive deadline misses, it is always followed by a recovery interval $\recovery$ consisting of $\numhits=\numtotalanalysed-\nummisses$ consecutive deadline hits.
\end{definition}

\subsection{Closed-Loop System Dynamics}%
\label{sec:derivation}
In this section we derive the system dynamics for a closed-loop control system under the assumption that we enter a burst interval of length $\nummisses$ after time instant $k$, and after $\nummisses$ deadline misses we start completing the control job in time.

\textbf{Normal Operation: }%
Under \emph{normal operating conditions} the system is not experiencing any deadline misses.
In other words, the system evolves according to the closed-loop system dynamics~\eqref{eq:feedback-basic}.

\textbf{\tKZ{}: }%
%
If a control task deadline miss occurs at time instant $k$, the plant states $x_k$ still evolve as normal.
However, the controller terminates its execution prematurely by killing the job, thus not updating its states ($z_{k+1} = z_k$).
The controller output is determined by the actuation strategy and is here zero ($u_{k+1} = 0$).
Now, consider a burst interval of length $\nummisses$ after time instant $k$.
Recalling that $\tilde{x}_k = \left[ {x_k}^\T {z_k}^\T {u_k}^\T \right]^\T$, we can write the evolution of the closed-loop system for the sequence of $m$ deadline misses followed by a single deadline hit as the product of a matrix representing the behaviour of the system for a hit and a matrix representing the behaviour in case of miss elevated to the power of $m$ to indicate $m$ steps of the system evolution.

The resulting closed-loop system in state-space form is
%
\begin{equation} 
    \setlength\arraycolsep{2pt}
    \label{eq:KZ}
    \begin{bmatrix}
        x_{k+\nummisses+1} \\
        z_{k+\nummisses+1} \\
        u_{k+\nummisses+1}
    \end{bmatrix} = 
    \underbrace{ \clmat{} \begin{bmatrix}
        \Ap                                         & 0_{n_x \times n_z}  & \Bp \\
        0_{n_z \times n_x}    & I                                         & 0_{n_z \times n_u} \\
        0_{n_u \times n_x}   & 0_{n_u \times n_z}   & 0_{n_u \times n_u}
    \end{bmatrix}^{\nummisses}}_{\clmat{}_{KZ}\funof{\nummisses}}
    \begin{bmatrix}
        x_{k} \\
        z_{k} \\
        u_{k}
    \end{bmatrix},
\end{equation}
%
where $\clmat{}_{KZ}\funof{\nummisses}$ represents the system matrix for $\nummisses$ misses under the \tKZ{} strategy, followed by a single hit (the matrix $\clmat{}$ that is multiplied to the left of the equation).
The matrix $\clmat{}$ is the same specified in~\eqref{eq:matrixA}, and represents the first hit that follows the $\nummisses$ misses, hence, we determine how $\tilde{x}_k$ influences $\tilde{x}_{k+m+1}$ ($\nummisses$ misses and one hit).

\textbf{\tKH{}: }%
%
Changing the actuation strategy to \tH{}, slightly alters the system matrix we derived for the \tKZ{} case.
The plant states $x_k$ evolve as normal and the control states $z_k$ are still not updated ($z_{k+1} = z_k$).
However, due to the change in actuation strategy, the last actuated value is instead held ($u_{k+1} = u_k$).
The resulting closed-loop state-space form can be seen in~\eqref{eq:KH}, where $\clmat{}_{KH}\funof{\nummisses}$ is used to represent the system matrix for $\nummisses$ misses under the \tKH{} strategy and matrix $\clmat{}$ is specified in~\eqref{eq:matrixA}.
%
\begin{equation}
    \setlength\arraycolsep{2pt}
    \label{eq:KH}
    \begin{bmatrix}
        x_{k+\nummisses+1} \\
        z_{k+\nummisses+1} \\
        u_{k+\nummisses+1}
    \end{bmatrix} = 
    \underbrace{ \clmat{} \begin{bmatrix}
        \Ap                                         & 0_{n_x \times n_z}  & \Bp \\
        0_{n_z \times n_x}    & I                                         & 0_{n_z \times n_u} \\
        0_{n_u \times n_x}   & 0_{n_u \times n_z}   & I
    \end{bmatrix}^{\nummisses}}_{\clmat{}_{KH}\funof{\nummisses}}
    \begin{bmatrix}
        x_{k} \\
        z_{k} \\
        u_{k}
    \end{bmatrix}
\end{equation}

\textbf{\tSZ{}: }%
%
When the control task misses a deadline under the \tS{} strategy, the job missing the deadline is allowed to continue its execution until completion.
However, no subsequent job of the control task is released until the current job has finished executing.
If the currently active job terminates during period $k$, the next control job is released at the start of the $k+1$-th period.
We can then write the evolution of the system where the control job experiences $\nummisses$ misses before completing its execution, meaning that there is a subsequent hit that uses old information for the error measurements.
While the controller executed only once to completion, the plant evolved for $\nummisses+1$ steps.
The resulting closed-loop state-space form can be seen in~\eqref{eq:SZ}, where $\clmat{}_{SZ}\funof{\nummisses}$ is used to represent the system matrix under the \tSZ{} strategy for $\nummisses$ misses and one completion using old measurements.
%
\begin{equation}
\setlength\arraycolsep{2pt}
\label{eq:SZ}
    \begin{bmatrix}
        x_{k+\nummisses+1} \\
        z_{k+\nummisses+1} \\
        u_{k+\nummisses+1}
    \end{bmatrix} = 
    \underbrace{\begin{bmatrix}
        \Ap^{\nummisses+1}  & 0_{n_x \times n_z}  & \Ap^\nummisses \Bp \\
        -\Bc\Cp             & \Ac                                       & -\Bc\Dp \\
        -\Dc\Cp             & \Cc                                       & -\Dc\Dp
    \end{bmatrix}}_{\clmat{}_{SZ}\funof{\nummisses}}
    \begin{bmatrix}
        x_{k} \\
        z_{k} \\
        u_{k}
    \end{bmatrix}
\end{equation}

\textbf{\tSH{}: }%
%
Similar to \tSZ{}, one job finishes execution after $\nummisses$ consecutive misses.
However, the actuation strategy holds the previous control value during the entire burst interval.
Therefore, the plant evolution is affected by a cumulative sum over the prior control values.
The resulting closed-loop state-space form can be seen in~\eqref{eq:SH}, where $\clmat{}_{SH}\funof{\nummisses}$ is used to represent the system matrix for $\nummisses$ misses under the \tSH{} strategy.
%
\begin{equation}
\setlength\arraycolsep{2pt}
\label{eq:SH}
    \begin{bmatrix}
        x_{k+\nummisses+1} \\
        z_{k+\nummisses+1} \\
        u_{k+\nummisses+1}
    \end{bmatrix} = 
    \underbrace{\begin{bmatrix}
        \Ap^{\nummisses+1}  & 0_{n_x \times n_z}  & \sum_{i=0}^\nummisses \Ap^i\Bp \\
        -\Bc\Cp             & \Ac                                       & -\Bc\Dp \\
        -\Dc\Cp             & \Cc                                       & -\Dc\Dp
    \end{bmatrix}}_{\clmat{}_{SH}\funof{\nummisses}}
    \begin{bmatrix}
        x_{k} \\
        z_{k} \\
        u_{k}
    \end{bmatrix}
\end{equation}

Equations~\eqref{eq:KZ}--\eqref{eq:SH} are inspired by the analysis in~\cite{Maggio:2020}, but we have we introduced two generalisations.
The first one is that our controller is specified as a general state-space system; therefore our method is able to address \emph{all} linear controllers.
The second generalisation is that we could include estimates of the plant states in the controller.
We can thus properly handle the presence of an observer.\footnote{In~\cite{Maggio:2020} the controller state is specified as part of the plant (e.g., when the proportional and integral controller is introduced). This implies that the state is computed although the controller did not execute. Our formulation fixes this by separating the plant execution and the controller states.}
Furthermore, we simplify the calculations by reducing the number of states $\tilde{x}_k$ of the closed-loop matrices.

\paragraph*{Stability}%

We now describe how the system matrices above can be used to analyse stability. 
Recall that a closed-loop control system is stable if and only if the (fixed) system matrix $\clmat{}$ is Schur stable. 
This criterion is also valid for cyclic patterns, where $\clmat{}$ represents the product of all closed-loop state matrices experienced in a full burst--recovery cycle. 
Hence, we can search for the shortest recovery interval length $\numhits$ such that
%
\begin{equation}
\label{eq:stability-cond}
    \underset{i}{\max}{\abs{\eig{i}{\clmat{}^{\numhits-1}\clmat{}_{\strat}\funof{\nummisses}}}} < 1, \quad \strat \in  \{ KZ, KH, SZ, SH \}.
\end{equation}
%
Recall that $\clmat{}_{\strat}\funof{m}$ already includes one hit, thus the left multiplication with $\clmat{}^{n-1}$. 
This is a sufficient condition and not necessary, meaning that a miss occurring during the recovery interval does not immediately imply that the closed-loop system is destabilised. 
We summarise the analysis in the following definition.

\begin{definition}[Static-Cyclic Stability Analysis]%
    We denote the stability analysis from~\eqref{eq:stability-cond} with the term \emph{\nilsstability{} stability analysis}.
    The system under analysis cycles through a sequence of $\nummisses$ misses followed by a sequence of $\numhits$ hits, indefinitely.
\end{definition}
%
The \nilsstability{} analysis assumes a repeating burst--recovery cycle with no interruptions.
This works well for instance in case the misses are due to a permanent overload condition caused by a mode switch (for example from low to high criticality mode in mixed-critical systems).
However, the setting is not very general.
To foster generality, we complement the stability evaluation with a less restrictive stability analysis, based on the proposed task model in Definition~\ref{def:task-model}.

\begin{definition}[\switchingstability{} stability analysis]%
    To guarantee \emph{\switchingstability{} stability}, a system has to be stable under arbitrary switching between all the possible $\nummisses$ realisations (i.e., closed-loop matrices) that comply with all task models $\tau \vdash \genfrac{\{}{\}}{0pt}{}{\nummisses_\subset}{\numtotalanalysed}, \nummisses_\subset \in \{1, \dots,\nummisses\}$ and also include the case in which the system does not miss deadlines.
\end{definition}
In other words, a system is \switchingstability{} stable if and only if it is stable under arbitrary switching of the closed-loop matrices in the set
%
\begin{equation}
    \left\{ \clmat{}^{\numtotalanalysed-1}\clmat{}_\strat\funof{1},\, \clmat{}^{\numtotalanalysed-2}\clmat{}_\strat\funof{2},\, \ldots,\, \clmat{}^{\numtotalanalysed-\nummisses}\clmat{}_\strat\funof{\nummisses},\, \clmat{} \right\}.
\end{equation}
%
Switching stability is unfortunately quite involved.\footnote{We have devoted some research effort into the investigation of a suitable stability analysis for control tasks subject to a set of weakly-hard constraints (of the type presented in Defintion~\ref{def:wh-models}). A summary of our findings can be found at \url{https://arxiv.org/abs/2101.11312}.}
However, many excellent tools have been developed to simplify this analysis (e.g., \code{MJSR}~\cite{Maggio:2020} or the \code{JSR toolbox}~\cite{Jungers:2014} for MATLAB).

\paragraph*{Performance}%
%\label{sec:performance}

We now show how the cost function in Equation~\eqref{eq:covartocost} can be used as a time-varying performance metric.
Before a burst interval, we assume that the system is in the neighbourhood of its steady-state covariance $P_\infty$ and performance $J_\infty$.

When a burst interval of $\nummisses$ missed deadlines occurs, the system will be disrupted and its covariance matrix will evolve according to
%
\begin{equation}
\label{eq:covariance-evolution}
    P_{k+\nummisses+1} = \clmat{}_{\strat}\funof{m} P_k \left( \clmat{}_{\strat}\funof{m} \right)^\T + \clmat{}^{j_{\numhits}} R_w \left( \clmat{}^{j_{\numhits}} \right)^\T,
\end{equation}
where
\begin{equation}
\label{eq:evolutionparameters}
\setlength\arraycolsep{2pt}
    \begin{aligned}
        R_w &= 
        \begin{bmatrix} 
            \sum_{i=0}^{j_{\nummisses}} \Ap^i\,\Wp\,R\,\Wp^\T\,(\Ap^i)^\T & 0_{n_x \times n_z+n_u} \\ 
            0_{n_z + n_u \times n_x}               & 0_{n_z + n_u \times n_z+ n_u}
        \end{bmatrix}, \\
        j_{\nummisses} &= 
        \begin{cases} 
            \nummisses - 1 & \text{ if } \dstrat = K\,\, \text{(\tK{}),} \\ 
            \nummisses & \text{ if } \dstrat = S\,\,\,\, \text{(\tS{}),} 
        \end{cases} \\
        j_{\numhits} &= 
        \begin{cases} 
            1 \phantom{||222} & \text{ if } \dstrat = K\,\, \text{(\tK{}),} \\ 
            0 & \text{ if } \dstrat = S\,\,\,\, \text{(\tS{}).}
        \end{cases}
        \end{aligned}
\end{equation}

$\Ap$ and $\Wp$ are matrices from the plant evolution in~\eqref{eq:background:plant}, $R$ is the noise intensity from~\eqref{eq:covevolution}, and $\clmat{}$ is the closed-loop matrix from~\eqref{eq:matrixA}.
The cost will simultaneously change following~\eqref{eq:covartocost}.
In the recovery interval, the covariance is again governed by the normal closed-loop evolution described in~\eqref{eq:covevolution}.
The system is said to have recovered once the cost is arbitrarily close to the steady-state cost.
We evaluate this as
%
\begin{equation}
    \label{eq:cost-threshold}
    \abs{\frac{J_{\infty}-J_k}{J_\infty}} < \varepsilon,
\end{equation}
where $\varepsilon> 0$ is the \emph{recovery threshold}.

\begin{definition}[Performance Recovery Interval]%
\label{def:recovery-lenght-interval}%
    We define the recovery length interval $\recoverylengthinterval_{\strat}$ as the smallest $\numhits$ such that~\eqref{eq:cost-threshold} is satisfied for all $k \geq \numhits$ when using $\strat$ to handle deadline misses.
\end{definition}

\begin{definition}[Maximum normalised cost]%
\label{def:maximum-system-cost}%
    We denote the maximum normalised cost of the system by
    %
    \begin{equation}
        J_{M,\strat} = \max_k \frac{J_{k, \strat}}{J_\infty},
    \end{equation}
    %
    where $J_{k,\strat}$ is the cost computed according to~\eqref{eq:covartocost} when using $\strat$ to handle the deadline misses.
\end{definition}
\begin{figure}[t]
    \centering
    \resizebox{\textwidth}{!}{%
\begin{tikzpicture}

\begin{axis}[%clip=false,
thick, 
xlabel={Time},
xtick={1,2,3,5,...,15},
xlabel near ticks,
width=12cm, height=4.5cm,
xmin=0.5,
ymin=0, ymax=6.25,
enlarge x limits=0]

\fill[red!15] (axis cs:0.01,0) rectangle (axis cs:3,6.25);
\fill[green!15] (axis cs:3,0) rectangle (axis cs:13,6.25);
\fill[blue!30] (axis cs:0,0.9) rectangle (axis cs:20,1.1);

\draw[black] (axis cs:0,0) rectangle (axis cs:15,6.25);

\addplot table [col sep=comma, x=T, y=J] {figs/ecrts21/data/example-data.csv};
\draw[dashed, black] (axis cs:0,5.4692) -- (axis cs:5,5.4692) node[right, xshift=1mm, blue] {$J_{M}$};

\draw[<->, blue, thick] (axis cs:5,0) -- (axis cs:5,5.46921);
\draw[<->, red, thick] (axis cs:0.5,3) -- node[above] {$m$} (axis cs:3,3);
\draw[<->, green!70!black, thick] (axis cs:3,3) -- node[above] {$n^{*}$} (axis cs:13,3);

\end{axis}

\end{tikzpicture}
}%
 
    \caption{Illustration of normalised cost ($J_k/J_\infty$), performance recovery interval $\recoverylengthinterval_{\strat}$ and maximum normalised cost $J_{M,\strat}$ on a data trace. The example uses $\strat = \text{\tKZ{}}$ and $\varepsilon=0.1$.}
    \label{fig:recoveryandpeak}
\end{figure}
Figure~\ref{fig:recoveryandpeak} gives a graphical representation of $\recoverylengthinterval_{\strat}$ and $J_{M,\strat}$ for an execution trace in which the controller experiences 3 misses and uses \tKZ{} as strategy $\strat$.

Compared to the stability analysis, the performance analysis also takes into account state deviations and uncertainty due to disturbances.
In Section~\ref{sec:derivation} we used the system dynamics to analyse the stability of the system.
The disturbance term $w_k$ was neglected as it does not influence the system stability.
However, its presence (as the presence of any disturbance) changes the dynamic behaviour of the system.
For the performance metric, the state covariance matrix $P_k$ evolves according to both the noise intensity and the system dynamics~\eqref{eq:covariance-evolution}.
The result is that the performance analysis provides us with a conservative (but more realistic) recovery interval, that takes system uncertainties into consideration.

To find the length of the recovery interval, we evolve the state covariance during a burst interval, using a specific strategy $\strat$ according to~\eqref{eq:covariance-evolution}.
Thereafter, the state covariance is evolved under normal operation, according to~\eqref{eq:covevolution}, until~\eqref{eq:cost-threshold} is satisfied, allowing us to find the performance recovery interval $\recoverylengthinterval_{\strat}$.


\section{Experimental Results}
\label{sec:results}
%
In this section, we compare our adaptive controller with the nominal controller implementation for different case studies.
We demonstrate the practical usefulness of the proposed controller by examining its impact on real hardware, namely, a ball and beam plant.
We compare the performance of the adaptive control system with the nominal one, according to the analysis presented in Section~\ref{sec:analysis}.
Finally, we complement the results with a worst-case switching stability analysis of the nominal and adaptive controlled systems.

In addition to the evaluation on the physical system, we present aggregate results obtained from a set of control benchmarks, representative of the process industry. %~\cite{Astrom:2004}.
We use this set of plants to evaluate the general applicability of our approach.
To make the evaluation comprehensive, we chose an unstable plant (the ball and beam) for the physical experiments and a set of mainly stable plants for the aggregate results.
Furthermore, we remark that all the considered controllers are dynamic.
As discussed in Section~\ref{sec:related}, to the best of our knowledge, only one previous work considers dynamic controllers~\cite{Pazzaglia:2021}.
In that work, however, a different overrun handling method is used, and a proper comparison is therefore not possible.

\subsection{Real World Evaluation -- Ball and Beam}
\label{sec:realplant}
%
\subsubsection*{System Description and Models}
The ball and beam~\cite{Wellstead:1978} is a common example in the automatic control literature and education, where a ball is free to roll over a beam that in turn is tilted by a servo motor.
The control objective is to make the ball position follow a reference trajectory across the beam by adjusting the voltage sent to the motor. Both the beam angle and the ball position can be measured.
Assuming the sampling period $\Ts = 0.01$~s, a discrete-time plant model $\plant$ was derived as
%
\begin{equation*}
\setlength\arraycolsep{1.25pt}
    \label{eq:bnb-plant}
    \plant : 
    \left\{
    \begin{matrix}
        x_{k+1} &=& 
        \begin{bmatrix}
            1 & 0.015 & 0.0003 & 0\\
            0 & 1 & 0.045 & 0\\
            0 & 0 & 1 & 0 \\
            0 & 0 & 0 & 1
        \end{bmatrix}\,x_k &+ 
        \begin{bmatrix}
            2.9\!\cdot\!10^{-5} \\
            0.0058 \\
            0.256 \\
            0
        \end{bmatrix}\,u_k + w_k \\ \vspace{-3mm} \\
        y_k &=& 
        \begin{bmatrix}
            0.5 & 0 & 0 & -1\\
            0 & 0 & 0.25 & 0
        \end{bmatrix}\,x_k, & 
    \end{matrix}
    \right.
\end{equation*}
where the four components of $x_k$ represent the ball position, ball velocity, beam velocity, and ball reference, respectively. The external signal vector $w_k$ is assumed to be white noise with variance $R = \diag{1,1,1,1}$. Under this state-space model, the objective is to regulate both outputs $y_k$ to zero, with the performance weighting matrix $Q = \diag{1,1}$.

To control $\plant$ we design a cascaded P--PID controller.
Cascaded controllers are frequently applied to systems with multiple measurements where one measured quantity affects another, but not vice versa.
Thus, the plant measurements can be controlled in sequential order (hence the naming \emph{cascaded}) using a controller designed for each measurement signal.
In our case study, this is implemented by a proportional (P) controller designed for controlling the beam's angle and a proportional--integral--derivative (PID) controller for the ball's position.
The controller is run as a periodically executing task with period $\Ts=0.01$~s on a single core CPU where overrun deadlines are killed and the corresponding sensor data is discarded. 
In between actuator calls, the control signal is assumed to be held constant.
The state-space representation of our controller is
%
\begin{equation*}
\setlength\arraycolsep{2pt}
    \label{eq:bnb-ctrler}
    \ctrler : \,\,
    \left\{
    \begin{aligned}
        z_{k+1} &= 
        \begin{bmatrix}
            1 & 0  \\
            0 & 0.9685 \\
        \end{bmatrix}\,z_k + 
        \begin{bmatrix}
            0.025 & 0 \\
            -0.2608 & 0
        \end{bmatrix}\,y_k, \\ \vspace{-4mm} \\
        u_{k+1} &= 
        \begin{bmatrix}
            -0.108 & -0.2608
        \end{bmatrix}\,z_k +
        \begin{bmatrix}
            -2.43 & -3
        \end{bmatrix}\,y_k.
        \end{aligned}
    \right.
\end{equation*}

\begin{table}[t]
    \centering
    \caption{Analytical study of the relative performance degradation of the ball and beam plant $\plant$ using either the nominal $\ctrler^n$ or adaptive controller $\ctrler^a$.}
    \renewcommand{\arraystretch}{1.6}
    \setlength{\tabcolsep}{5pt}
    \begin{tabular}{c | a b a b a b a} \hline
        $p$ & 10\% & 20\% & 30\% & 40\% & 50\% & 60\% & 70\% \\ \hline\hline
        ${\large\sfrac{\Delta J^n}{J}}$ & 2.5\% & 9.2\% & 20.8\% & 39.9\% & 75.3\% & 156\% & 452\% \\ \hline
        ${\large\sfrac{\Delta J^a}{J}}$ & 0.1\% & 0.1\% & 0.3\% & 0.6\% & 1.1\% & 2.5\% & 6.7\% \\ \hline
    \end{tabular}
    \label{tab:cost-sim}
\end{table}


\subsubsection*{Experiments Design}
We apply the performance analysis presented in Section~\ref{sec:analysis} to the plant model $\plant$ controlled using either the ideal ($\ctrler$), nominal ($\ctrler^n$), or adaptive ($\ctrler^a$) implementations from Section~\ref{sec:adaptation}.
We include the effect of deadline misses only on the nominal and adaptive control systems.
The probability distribution $p_\counter$ can be chosen arbitrarily according to the desired task model. 
For simplicity, we assume here that the deadline misses are Bernoulli distributed~\cite{Schenato:2007}, i.e., the probabilities of missing deadlines in each period are independently and identically distributed with probability $p$.
This results in the probability $p_\counter = (1-p)p^\counter$ of $\counter$ consecutive deadline misses followed by a hit.
We assume that no more than $\counter_{max}=20$ consecutive deadlines can be missed.
The latter assumption might seem restrictive, but if the probability of missing a deadline is $30\%$, the probability of missing $20$ consecutive deadlines is less than $4\cdot10^{-11}$.

\begin{table}[t]
    \centering
    \caption{Empirical study of the relative performance degradation of the real ball and beam plant using either the nominal $\ctrler^n$ or adaptive controller $\ctrler^a$.}
    \renewcommand{\arraystretch}{1.6}
    \setlength{\tabcolsep}{5pt}
    \begin{tabular}{c | a b a b a b a} \hline
        $p$ & 10\% & 20\% & 30\% & 40\% & 50\% & 60\% & 70\% \\ \hline\hline
        ${\large\sfrac{\Delta J^n}{J}}$ & 6.3\% & 27.1\% & 22.5\% & 50.5\% & 73.1\% & 260\% & $\infty$ \\ \hline
        ${\large\sfrac{\Delta J^a}{J}}$ & 6.1\% & 7.8\% & 3.2\% & 4.6\% & 3.8\% & 4.9\% & 11.7\% \\ \hline
    \end{tabular}
    \label{tab:cost-real}
\end{table}

\begin{figure}
    \centering
    \begin{tikzpicture}
\begin{groupplot}[%
    group style={group size = 1 by 1,
                 vertical sep = 0.25cm,
                 horizontal sep = 0.75cm},
    width = \columnwidth,
    height = 3.5cm,
    xmin=  500, xmax= 550,
    ymin= -1, ymax= 4,
    ytick = {0,1,2,3},
    title style = {yshift=-0.1cm},
    ylabel style = {yshift=-0.5cm},
    grid style = {dashed, black!20},
    grid = major,
    ]

    %%%%%%%%%%%%%%%%%%%%%%
    %% ideal controller %%
    %%%%%%%%%%%%%%%%%%%%%%
    \nextgroupplot[xlabel={Time (s)},
                   xlabel near ticks,
                   ylabel = {Pos. (cm)},
                   yticklabels = {0,5,10,15}, 
                   ylabel style = {yshift = 10pt},
                   title={Ball Position with the Ideal Controller $\ctrler$}
                  ]
    \addplot[smooth, thick, lqrnomcolour]
            table[col sep=comma, x=T, y=Y1]
            {\figdir/evaluation/ball-and-beam/data/experiment_M_10_P_50_Mode_1.csv};
    % reference plot
    \addplot[]coordinates {(500, 0)(525, 0)(525, 3)(550, 3)};

\end{groupplot}
\end{tikzpicture}

    \caption{Snippet of the test performed on the real ball and beam plant using the ideal controller, i.e., without deadline misses.
        The plot shows one period of the square wave used as reference, the black line. The blue line shows the ball's position.}
    \label{fig:real-plant-ideal}
\end{figure}


\begin{figure*}
    \centering
    \begin{tikzpicture}
\begin{groupplot}[%
    group style={group size = 1 by 4,
                 vertical sep = 0.6cm,
                 horizontal sep = 0.75cm},
    width = \columnwidth,
    height = 4.76cm,
    xmin=  500, xmax= 550,
    ymin= -0.75, ymax= 4,
    ytick = {0,1,2,3},
    title style = {yshift=-0.1cm},
    grid style = {dashed, black!20},
    grid = major,
    ]

    %%%%%%%%%%%%%%%%%%%%%%%%%%%%%%
    %% p=0.3 nominal controller %%
    %%%%%%%%%%%%%%%%%%%%%%%%%%%%%%
    \nextgroupplot[ylabel = {Pos. (cm)},
                   ylabel style = {yshift = -3pt},
                   ylabel right = {$\ctrler^n $},
                   xticklabels={},
                   yticklabels = {0,5,10,15}, ]
    \addplot[smooth, thick, lqgcolour,]
            table[col sep=comma, x=T, y=Y1]
            {\figdir/evaluation/ball-and-beam/data/experiment_M_20_P_30_Mode_2.csv};
    % reference plot
    \addplot[]coordinates {(500, 0)(525, 0)(525, 3)(550, 3)};
    \node[draw, fill=white] at (axis cs:545, 0) {$\rho=30\%$};

    %%%%%%%%%%%%%%%%%%%%%%%%%%%%%%%
    %% p=0.3 adaptive controller %%
    %%%%%%%%%%%%%%%%%%%%%%%%%%%%%%%
    \nextgroupplot[ylabel = {Pos. (cm)},
                   ylabel style = {yshift = -3pt},
                   ylabel right = {$\ctrler^a $},
                   xticklabels={},
                   yticklabels = {0,5,10,15}, ]
    \addplot[smooth, thick, adacolour,]
            table[col sep=comma, x=T, y=Y1]
            {\figdir/evaluation/ball-and-beam/data/experiment_M_20_P_30_Mode_3.csv};
    % reference plot
    \addplot[]coordinates {(500, 0)(525, 0)(525, 3)(550, 3)};
    \node[draw, fill=white] at (axis cs:545, 0) {$\rho=30\%$};

    %%%%%%%%%%%%%%%%%%%%%%%%%%%%%%
    %% p=0.5 nominal controller %%
    %%%%%%%%%%%%%%%%%%%%%%%%%%%%%%
    \nextgroupplot[xticklabels={},
                   yticklabels = {0,5,10,15},
                   ylabel = {Pos. (cm)},
                   ylabel style = {yshift = -3pt},
                   ylabel right = {$\ctrler^n $}, ]
    \addplot[smooth, thick, lqgcolour,]
            table[col sep=comma, x=T, y=Y1]
            {\figdir/evaluation/ball-and-beam/data/experiment_M_20_P_50_Mode_2.csv};
    % reference plot
    \addplot[]coordinates {(500, 0)(525, 0)(525, 3)(550, 3)};
    \node[draw, fill=white] at (axis cs:545, 0) {$\rho=50\%$};

    %%%%%%%%%%%%%%%%%%%%%%%%%%%%%%%
    %% p=0.5 adaptive controller %%
    %%%%%%%%%%%%%%%%%%%%%%%%%%%%%%%
    \nextgroupplot[xlabel = {Time (s)},
                   xlabel near ticks,
                   ylabel = {Pos. (cm)},
                   ylabel style = {yshift = -3pt},
                   ylabel right = {$\ctrler^a $},
                   yticklabels = {0,5,10,15}, ]
    \addplot[smooth, thick, adacolour,]
            table[col sep=comma, x=T, y=Y1]
            {\figdir/evaluation/ball-and-beam/data/experiment_M_20_P_50_Mode_3.csv};
    % reference plot
    \addplot[]coordinates {(500, 0)(525, 0)(525, 3)(550, 3)};
    \node[draw, fill=white] at (axis cs:545, 0) {$\rho=50\%$};

\end{groupplot}
\end{tikzpicture}

    \vspace{-3mm}
    \caption{Snippets of the tests performed on the real ball and beam plant for $p=30\%$ (two left plots) and $p=50\%$ (two right plots).
        The plots show one period of the square wave used as reference, the black line.
        The coloured lines show the ball position, in green for the nominal controller (upper plots) and orange for the adaptive controller (lower plots).}
    \label{fig:real-plant}
\end{figure*}

% performance metric
We measure the relative performance of the nominal and adaptive controllers according to the quantity $\sfrac{\Delta J^\dagger}{J}$ in Equation~\eqref{eq:relcost}.
Since the mean-square deviation from the ideal controller is used to evaluate the relative performance, the \emph{optimal} achievable cost is $0$.
For the real system, we do not feed the system with white noise, but we expose the system to a repeatable exogenous signal in the form of periodic reference changes. 
Furthermore, we evaluate the relative performance degradation empirically from the measured signals using Equation~\eqref{eq:relcost}, with $\mathbb{E}$ being interpreted as the mean value of the real signals.

To complement the performance analysis, we perform a JSR stability analysis on the model to determine the maximum number of consecutive deadline misses that are tolerated while still guaranteeing closed-loop stability.

\subsubsection*{Analytical Evaluation}
The performance results obtained with the analytical study of $\plant$ for different values of $p$ are summarised in Table~\ref{tab:cost-sim}.
From the table, we can see that the adaptive controller drastically improves the relative performance (in comparison to the nominal controller) across all deadline miss probabilities.
Already for small probabilities, the nominal controller significantly degrades the relative performance compared to the ideal controller; e.g., for $p = 30\%$ the relative performance is degraded by $20.8\%$. This can be compared to the  adaptive controller, where the relative performance reduction stays below $5\%$ until the miss probability reaches $70\%$.

Analysing the switching stability, we calculated the JSR for the set of closed-loop matrices corresponding to $i = \{0,1,\ldots,q\}$ consecutive deadline misses followed by one hit ($q \leq q_{max}$).
The nominal control system is guaranteed to be switching stable (i.e., the JSR is below $1$) for a maximum of $q=2$ consecutive deadline misses, while the adaptive control system is guaranteed stable up to $q=8$. 
We conclude that the adaptive controller improves also worst-case robustness against deadline misses for the ball and beam.
However, we emphasise that these results do \emph{not} imply that the system will go unstable if more deadline misses occurs; only that the system is \emph{guaranteed} switching stable if no more than $q$ consecutive deadline misses are ever experienced.

\subsubsection*{Empirical Evaluation}
We conducted experiments on the physical ball and beam plant to evaluate the performance of the controller on a real system.\footnote{A video, showing experiments with the real ball and beam system can be viewed at \texttt{https://youtu.be/6y\_C7NIzXto}. The video provides a real-world comparison between the nominal and adaptive controllers for $p = \left\{30\%, 50\%, 70\%\right\}$.}
Each experiment is run for $10$ minutes, where the control objective is for the ball to follow a square-wave reference across the beam.
The square wave has a period of $50$~s and alternates between position $0$ and $15$~cm.
Differently from the analytical evaluation, it is impossible to obtain the same exogenous signal $w_k$ in the different experiments.
While the reference changes can be exactly repeated, the real stochastic disturbances (in the form of electrical noise, mechanical glitches, etc.) are not repeatable.
This means that the empirical cost relative to the ideal case, as measured by Equation~\eqref{eq:relcost}, is not expected to be zero even in the complete absence of deadline misses.

Figure~\ref{fig:real-plant-ideal} displays a snippet of the ball's position (blue line) under said ideal conditions.
The ball quite successfully follows the reference (black line). 
Here, the fluctuations around the reference are caused by measurement noise and irregularities in the beam surface, where the latter can cause the ball to get lodged in an undesired position and thus result in oscillations.

After measuring the performance of the ideal controller, each controller ($\ctrler^n$ and $\ctrler^a$) was applied to the system, using probabilities $p \in \left\{ 10\%,\, 20\%,\, \ldots,\, 70\% \right\}$ of missing each deadline (with $\counter_{max} = 20$).
The results of the experiments are reported in Table~\ref{tab:cost-real}, where the relative performance degradation $\sfrac{\Delta J^\dagger}{J}$ is computed for both the nominal and adaptive controllers.
To give an intuition for how the physical system behaves, in Figure~\ref{fig:real-plant} we provide a snippet of a time plot portraying the ball's position controlled by either the nominal (upper plots) or adaptive (lower plots) controller.
We distinguish the differences between the nominal and adaptive controllers for a probability $p = 30\%$ (left plots) of missing a deadline.
The nominal controller shows oscillations around the reference value.
When the probability of missing a deadline is increased to $p = 50\%$ (right plots), the nominal controller's oscillations grow more evident, while the adaptive controller appears unaffected (compared to the ideal controller in Figure~\ref{fig:real-plant-ideal}).

% experiments comments
From Table~\ref{tab:cost-real}, we observe that the adaptive controller has a lower performance degradation across all deadline miss probabilities $p \geq 20\%$ compared to the nominal controller.
The performance of the adaptive controller seems virtually unaffected for $p \leq 60\%$, where the baseline relative degradation of approximately  $4\%$ to $8\%$ is due to the natural disturbances in the system.
The nominal controller on the other hand experiences significant performance degradation at higher miss probabilities, and for $p = 70\%$ the system becomes unstable -- we report this as an infinite cost.

In summary, both the analytical and empirical studies show that the adaptive controller $\ctrler^a$ consistently outperforms the nominal controller $\ctrler^n$ for the ball and beam. Furthermore, the adaptive controller can tolerate
a large likelihood of random deadline misses (at least $60\%$) without any noticeable performance degradation.

\subsection{Benchmark Evaluation -- Process Industry}
\label{sec:aggregate}
%
\subsubsection*{System Description and Models}
To evaluate the general applicability of the proposed adaptive controller, we perform an extensive evaluation campaign on a benchmark set of plants.
The set was developed specifically to evaluate various PID designs~\cite{Astrom:2004} in the process industry.
It consists of $134$ unique plants separated into $9$ categories, where each category has its own specific properties frequently recognised in the process industry.
Since the benchmark was developed specifically with process industrial plants in mind, the majority of the plants are stable, i.e., all their eigenvalues lie inside the unit circle.
However, there are also plants with integrating dynamics included in the benchmark, i.e., an eigenvalue in $1$; these plants are generally not considered stable.
For each plant, two controllers -- a PI and a PID controller -- are optimised using known methods~\cite{Garpinger:2008}; hence, $268$ unique control systems are analysed in total.

\subsubsection*{Experiments Design}
Similarly to the ball and beam, we analyse the relative performance of the nominal and adaptive controllers in accordance with the analysis described in Section~\ref{sec:analysis}.
We again consider the probability of missing a deadline to follow a Bernoulli distribution with probabilities $p \in \left\{ 10\%,\, 30\%,\, 50\%,\, 70\% \right\}$ and a maximum of $\counter_{max} = 20$ consecutive deadline misses.
We feed the systems with a stochastic disturbance and analytically evaluate the ability of the controllers to reject it.
Differently from the ball and beam, we analyse the systems when subject to brown noise, i.e., integrated white noise~\cite{Schmidt:1985}.
The brown noise model is generally considered appropriate for process industrial plants since it is dominant for low frequencies (e.g., load disturbances and disturbances from nearby heavy machinery).
We assume that the \emph{same} disturbance process enters the ideal, nominal, and adaptive control systems; this guarantees an unbiased comparison between the different controllers.
For each of the $268$ control systems we calculate the relative performance $\sfrac{\Delta J^\dagger}{J}$ for both the nominal and the adaptive controller.

Similarly to the ball and beam, we complement our performance analysis with a JSR worst-case stability analysis.

\begin{figure}
    \centering
    % Set number of bins for histograms in commands file

\begin{tikzpicture}
\begin{groupplot}[group style = {group size = 1 by 4,
                                 vertical sep=0.4cm},
                  width=\textwidth,
                  grid=both,
                  grid style={dashed,black!20},
                  height=2.8cm,
                  width=\columnwidth,
                  xmin=-5, xmax=2.2,
                  ymin=0, ymax=165,
                  tick align=inside,
                 ]

    %%%%%%%%%%%%%%%
    %%% p = 0.1 %%%
    %%%%%%%%%%%%%%%
    \nextgroupplot[xticklabels = {},
                   legend style = {at = {(0.5,1.1)},
                                   anchor = south,
                                   /tikz/every even column/.append style = {column sep=0.2cm}},
                   legend columns = 3
                  ]
        \addplot[ybar, ybar legend,
                 fill=lqgcolour,
                ] coordinates {(0,0)};
        \addlegendentry{Nominal}
        \addplot[ybar,
                 hist={bins=\binsaggregatedhist},
                 fill=lqgcolour,
                 forget plot,
                ] 
                table [y index=0, col sep=comma] 
                {figs/rtas22a/data/batch-results-10-log.csv};
        \addplot[ybar, ybar legend,
                 fill=adacolour,
                ] coordinates {(0,0)};
        \addlegendentry{Adaptive}
        \addplot[ybar, ybar legend,
                 hist={bins=\binsaggregatedhist},
                 fill=adacolour,
                 forget plot,
                ]
                table [y index=1, col sep=comma] 
                {figs/rtas22a/data/batch-results-10-log.csv};
        % \addlegendentry{$\mathcal{C}^{a}$}
        \addplot[red, dashed, ultra thick] coordinates {(2,0) (2,200)};
        \addlegendentry{Instability threshold}
        \node[draw, fill=white] at (axis cs:1, 115) {$\rho=10\%$};
    
    %%%%%%%%%%%%%%%
    %%% p = 0.3 %%%
    %%%%%%%%%%%%%%%
    \nextgroupplot[xticklabels= {},
                  ]
        \addplot[ybar, hist={bins=\binsaggregatedhist},
                 fill=lqgcolour,
                ] 
                table [y index=0, col sep=comma] 
                {figs/rtas22a/data/batch-results-30-log.csv};
        \addplot[ybar, hist={bins=\binsaggregatedhist},
                 fill=adacolour,
                ]
                table [y index=1, col sep=comma] 
                {figs/rtas22a/data/batch-results-30-log.csv};
        \draw[red,ultra thick, dashed] (axis cs:2,0)--(axis cs:2,200);
        \node[draw, fill=white] at (axis cs:1, 115) {$\rho=30\%$};

    %%%%%%%%%%%%%%%
    %%% p = 0.5 %%%
    %%%%%%%%%%%%%%%
    \nextgroupplot[xticklabels= {},
                   ylabel = {Number of systems},
                   ylabel near ticks,
                   ylabel style = {xshift=1cm},
                  ]
        \addplot[ybar, hist = {bins=\binsaggregatedhist},
                 fill = lqgcolour,
                ] 
                table [y index = 0, col sep = comma] 
                {figs/rtas22a/data/batch-results-50-log.csv};
        \addplot[ybar, hist={bins=\binsaggregatedhist},
                 fill=adacolour,
                ]
                table [y index=1, col sep=comma] 
                {figs/rtas22a/data/batch-results-50-log.csv};
        \draw[red,ultra thick, dashed] (axis cs:2,0)--(axis cs:2,200);
        \node[draw, fill=white] at (axis cs:1, 115) {$\rho=50\%$};

    %%%%%%%%%%%%%%%
    %%% p = 0.7 %%%
    %%%%%%%%%%%%%%%
    \nextgroupplot[xticklabels={}]
        \addplot[ybar, hist = {bins=\binsaggregatedhist},
                 fill = lqgcolour,
                ] 
                table [y index = 0, col sep = comma] 
                {figs/rtas22a/data/batch-results-70-log.csv};
        \addplot[ybar, hist={bins=\binsaggregatedhist},
                 fill=adacolour,
                ]
                table [y index=1, col sep=comma] 
                {figs/rtas22a/data/batch-results-70-log.csv};
        \draw[red,ultra thick, dashed] (axis cs:2,0)--(axis cs:2,200);
        \node[draw, fill=white] at (axis cs:1, 115) {$\rho=70\%$};

\end{groupplot}
\end{tikzpicture}

    \caption{Histograms comparing the relative performance degradation of the nominal and adaptive controllers for the benchmark plants.
    The plots correspond to different deadline miss probabilities $p$.
    The orange bars report the performance obtained with the adaptive controller $\ctrler^a$, while the green bars report the performance obtained with the nominal controller $\ctrler^n$.
    The systems with a performance worse than the stability threshold (red dashed line) resulted in unstable dynamics.}
    \label{fig:aggregate}
\end{figure}

% In Figure~\ref{fig:aggregate} we display a sample of the aggregate results ($p = 10\%,\, 30\%,\, 50\%,\text{ and }70\%$) as histograms -- using both the nominal (green bars) and adaptive (blue bars) controller structures.
\subsubsection*{Experiments Results}
In Figure~\ref{fig:aggregate} we display histograms reporting the relative performance degradation of all the $268$ control systems.
The horizontal axis displays the relative performance $\sfrac{\Delta J^\dagger}{J}$ in logarithmic scale. 
The vertical axis counts the number of control systems with a given relative performance.
The four plots correspond to the different deadline miss probabilities considered.
In each plot, we represent the nominal controllers with green bars and the adaptive controllers with orange bars.
Unstable closed-loop systems have an infinite cost and are thus marked in the rightmost part of the plot, beyond the red dashed threshold.

From Figure~\ref{fig:aggregate} we see that the adaptive controller performs better than the nominal one for \emph{all} the $268$ control systems, regardless of the probability of missing a deadline.
Despite the control systems' dynamics varying significantly (e.g., lag dominated, lead dominated, oscillatory, high system order, integrating), the worst adaptive control system still performs better than the best nominal control system for all $p$.
The improvement is particularly distinguishable for lower probabilities, e.g., $p=10\%$, where the mean relative cost over all the control systems is improved by two orders of magnitude.

Second, when the probability of missing a deadline grows, the relative performance degradation increases accordingly.
For $p=50\%$ and $p=70\%$ some of the systems using the nominal controller become unstable, i.e., $\sfrac{\Delta J^\dagger}{J} = \infty$.
In the case of $p=70\%$, more than $40\%$ of the nominal control systems are unstable.
On the other hand, \emph{all} the adaptive control systems are stable and have a relative cost degradation below $10\%$.
This suggests that $\ctrler^a$ improves both performance and robustness compared to the nominal controller.

\begin{figure}
    \centering
    \begin{tikzpicture}
\begin{axis}[
            group style = {group size = 1 by 2,
                                 vertical sep=0.4cm},
            ybar,ybar legend,
            %  x=0.40cm,
            %  bar width=0.1cm,
            ybar interval,
            width=\textwidth,
            grid=both,
            grid style={dashed,black!20},
            height=3cm,
            width=\columnwidth,
            ymin=0, ymax=70,
            xmin=a0, xmax=a21,
            tick align=inside,
            xtick align=outside,
            xtick pos=lower,
            legend style = {at = {(0.5,1.1)},
                anchor = south,
                /tikz/every even column/.append style = {column sep=0.2cm}},
            legend columns = 2,
            ylabel = {Number of systems},
            xlabel = {$\counter$},
            ylabel near ticks,
            symbolic x coords={a0,a1,a2,a3,a4,a5,a6,a7,a8,a9,a10,a11,a12,a13,a14,a15,a16,a17,a18,a19,a20,a21},
            xticklabels={0,1,2,3,4,5,6,7,8,9,10,11,12,13,14,15,16,17,18,19,20,21}
            ]
        \addplot [fill=lqgcolour, ]
                coordinates { (a0,5) (a1,64) (a2,59) (a3,0) (a4,0) (a5,0) (a6,0) (a7,2) (a8,0) (a9,0) (a10,0) (a11,0) (a12,0) (a13,0) (a14,0) (a15,0) (a16,0) (a17,0) (a18,0) (a19,0) (a20,138) (a21,0) };
        \addlegendentry{$\ctrler^{n}$}
        \addplot [fill=adacolour,]
                coordinates { (a0,0) (a1,0) (a2,0) (a3,1) (a4,0) (a5,1) (a6,0) (a7,1) (a8,2) (a9,3) (a10,6) (a11,20) (a12,1) (a13,5) (a14,11) (a15,6) (a16,8) (a17,2) (a18,6) (a19,10) (a20,185) (a21,0) };
        \addlegendentry{$\ctrler^{a}$}


\end{axis}
\end{tikzpicture}
    \caption{Histogram reporting the number of benchmark systems (out of 268) that are guaranteed switching stable for up to~$\counter$ consecutive deadline misses, according to the JSR analysis.
    For each value of $\counter$, the green bar (left) reports how many $\ctrler^{n}$ controlled systems can tolerate up to $\counter$ consecutive deadline misses and the orange bar (right) reports the corresponding number of $\ctrler^{a}$ controlled systems.
    For readability the y axis is cut at $65$: a total of $138$ plants can tolerate $20$ or more misses with the nominal controller, and a total of $185$ plants can tolerate $20$ or more misses with the adaptive controller.}
    \label{fig:jsr-histogram}
\end{figure}

To verify that the adaptive controller improves the robustness to deadline misses compared to the nominal controller, we complement the evaluation with a JSR analysis.
The histogram in Figure~\ref{fig:jsr-histogram} shows, for each value of $\counter$, the number of control systems that are guaranteed switching stable for a maximum number of consecutive deadline misses $q$, when they are controlled with either the nominal ($\ctrler^n$) or the adaptive ($\ctrler^a$) controller.
Intuitively, the more control systems that can guarantee switching stability for a higher value of $\counter$, the better.
The vertical axis of the histogram is cut at $65$ for legibility: this affects only the columns for $\counter=20$ where the nominal controller can guarantee stability for $138$ plants while the adaptive controller can guarantee stability for $185$ plants.

For the nominal controller, we see that the maximum number of consecutive deadline misses tolerated by the system varies greatly between the different control systems.
In the whole benchmark, $138$ systems were stable for (at least) $20$ consecutive deadline misses, but $123$ systems were guaranteed stable only for one or two misses.
Furthermore, $5$ of the nominal control systems were unstable unless \emph{all} of the control task's deadlines were hit.

For the adaptive controller, on average, a much larger number of consecutive deadline misses can be tolerated.
Out of all the control systems, $185$ were stable for (at least) $q=20$, and the large majority of the remaining control systems are guaranteed to tolerate between $q=10$ and $q=19$ consecutive deadline misses. Additionally, we see that \emph{all} adaptively controlled systems can tolerate at least $3$ deadline misses.

We note that both for the nominal and adaptive controllers, a significant number of control systems are stable for $20$ deadline misses.
This presumably follows from the (mainly) stable nature of the plants in the benchmark, an attribute that generally makes the system more robust.

The results of the evaluation campaign confirm the hypothesis that the proposed adaptive controller improves the control system's performance in the presence of deadline misses.
While we observed some cases in which the nominal controller goes unstable and the adaptive controller is stable, we never observed the opposite.
Additionally, the adaptive controller does not compromise the performance under ideal conditions, and it preserves the major part of the ideal controller's performance when deadline misses are present.


\section{Conclusions}
\label{sec:conc}
In this paper we analysed control systems and their behaviour in the presence of bursts of deadline misses. 
We provided a comprehensive set of tools to determine how robust a given control system is to faults that hinder the computation to complete in time, with different handling strategies. 
%
Our analysis tackles both stability and performance. In fact, we have shown that analysing the stability of the system is not enough to properly quantify the robustness to deadline misses, as the performance loss could be significant even for stable systems. We introduced two performance metrics, linked to the recovery of a system from a burst of deadline misses.

A limitation of the presented performance analysis is that it only applies to linear control systems. However, the approach can easily be extended to analyse \emph{time-varying} linear systems and can also be used for local analysis of a nonlinear system that should follow a given reference trajectory. In fact, to illustrate the applicability to real (e.g., nonlinear) systems, we applied the  analysis to a Furuta pendulum and compared the results of simulations obtained with a model of the process to the real execution data.
The results support our claim that the proposed performance analysis is a valid approximation of the real-world system performance.

We performed additional tests on a large batch of industrial plants, using modern control design techniques. 
From our experimental campaign, we conclude that the choice of actuation strategy affects the control performance significantly more than the choice of deadline handling strategy.

%\bibliographystyle{abbrv}
\bibliography{paper}

\end{document}
