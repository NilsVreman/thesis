Using the alphabet $\Sigma\left(\strat\right)$ and the chosen actuator mode (i.e., zeroing, or holding the previous value), we compute the closed-loop behaviour of the controlled system.
We identify one matrix for each dynamics corresponding to an interval $\pi_t$ associated by $\alpha \in \Sigma\left( \strat \right)$, building the set $\Aa^\strat$.

\textbf{Kill: }%
%
We define $\tilde x_t^{\,\text{K}} = \left[ x_t^\T{}\,\, z_t^\T{}\,\, u_t^\T{} \right]^\T{}$ as the closed-loop state vector and compute the closed-loop dynamics $\clmat{}^{\,\text{K}}_{\cH}$, corresponding to the character $\cH$.
\begin{equation*}
    \tilde x_{t+1}^{\,\text{K}} = \clmat{}^{\,\text{K}}_{\cH}\,\tilde x_t^{\,\text{K}}, \quad
    \clmat{}^{\,\text{K}}_{\cH} = \begin{bmatrix}
        \Ap       & 0    & \Bp       \\
        -\Bc\Cp   & \Ac  & -\Bc\Dp   \\
        -\Dc\Cp   & \Cc  & -\Dc\Dp   \\
    \end{bmatrix}
\end{equation*}
%
For the case of $\cM$, the controller execution terminates prematurely, thus not updating its states ($z_{t+1} = z_t$).
Depending on the actuation mode, the controller output is either zeroed ($u_{t+1} = 0$) or held ($u_{t+1} = u_t$).
The resulting closed-loop system in state-space form is denoted with $\clmat{}^{\,\text{K}}_{\cM}$:
\begin{equation*}
    \tilde x_{t+1}^{\,\text{K}} = \clmat{}^{\,\text{K}}_{\cM}\,\tilde x_t^{\,\text{K}}, \quad
    \clmat{}^{\,\text{K}}_{\cM} = \begin{bmatrix}
        \Ap & 0  & \Bp \\
        0   & I  & 0   \\
        0   & 0  & \Delta
    \end{bmatrix}
\end{equation*}
Here, $\Delta = I$ (identity matrix) if the control signal is held and $\Delta = 0$ if it is zeroed.
The set of dynamic matrices that a controlled system under the Kill strategy may experience is then $\Aa^{\text{Kill}}=\left\{\clmat{}^{\,\text{K}}_{\cH},\clmat{}^{\,\text{K}}_{\cM}\right\}$.

\textbf{Skip-Next: }%
%
For the Skip-Next strategy, we introduce two additional states $\hat x_t$ and $\hat u_t$ storing the old values of $x_t$ and $u_t$ while the controller awaits an update.
The resulting state vector then becomes $\tilde x_t^{\,\text{S}} = \left[ x_t^\T{}\,\,z_t^\T{}\,\, u_t^\T{}\,\, \hat{x}_t^\T{}\,\, \hat{u}_t^\T{} \right]^\T{}$.
When $\pi_t$ is associated to $\cH$, the two additional states mirror the behaviour of the states of which they are storing data.
The resulting closed-loop system is described using $\clmat{}^{\,\text{S}}_{\cH}$:
%
\begin{equation*}
    \tilde x_{t+1}^{\,\text{S}} = \clmat{}^{\,\text{S}}_{\cH}\,\tilde x_t^{\,\text{S}}, \quad
    \clmat{}^{\,\text{S}}_{\cH} = \begin{bmatrix}
        \Ap       & 0    & \Bp      & 0 & 0 \\
        -\Bc\Cp   & \Ac  & -\Bc\Dp  & 0 & 0 \\
        -\Dc\Cp   & \Cc  & -\Dc\Dp  & 0 & 0 \\
        \Ap       & 0    & \Bp      & 0 & 0 \\
        -\Dc\Cp   & \Cc  & -\Dc\Dp  & 0 & 0 \\
    \end{bmatrix}.
\end{equation*}
%
For the case of $\cM$ in $\pi_t$ the two augmented states hold their previous values.
The resulting closed-loop system is described by $\clmat{}^{\,\text{S}}_{\cM}$:
%
\begin{equation*}
    \tilde x_{t+1}^{\,\text{S}} = \clmat{}^{\,\text{S}}_{\cM}\,\tilde x_t^{\,\text{S}}, \quad
    \clmat{}^{\,\text{S}}_{\cM}=\begin{bmatrix}
        \Ap & 0  & \Bp & 0 & 0 \\
        0   & I  & 0   & 0 & 0 \\
        0   & 0  & \Delta   & 0 & 0 \\
        0   & 0  & 0   & I & 0 \\
        0   & 0  & 0   & 0 & I \\
    \end{bmatrix}.
\end{equation*}
%
Finally, for the case of recovery $\cR$ in $\pi_t$, the new control command is calculated using the values stored in $\hat x_t$ and $\hat u_t$.
The resulting closed-loop system is described by $\clmat{}^{\,\text{S}}_{\cR}$:
%
\begin{equation*}
    \tilde x_{t+1}^{\,\text{S}} = \clmat{}^{\,\text{S}}_{\cR} \, \tilde x_{t}^{\,\text{S}}, \quad
    \clmat{}^{\,\text{S}}_{\cR} = \begin{bmatrix}
        \Ap & 0    & \Bp & 0       & 0 \\
        0   & \Ac  & 0   & -\Bc\Cp & -\Bc\Dp \\
        0   & \Cc  & 0   & -\Dc\Cp & -\Dc\Dp \\
        \Ap & 0    & \Bp & 0       & 0 \\
        0   & \Cc  & 0   & -\Dc\Cp & -\Dc\Dp \\
    \end{bmatrix}.
\end{equation*}
%
The set of dynamic matrices under the Skip-Next strategy is then defined as $\Aa^{\text{Skip-Next}}=\left\{\clmat{}^{\,\text{S}}_{\cH},\clmat{}^{\,\text{S}}_{\cM},\clmat{}^{\,\text{S}}_{\cR}\right\}$.

\subsection{Kronecker Lifted Switching System}%
\label{sec:system_dynamics}
%
A straightforward approach to analyse the switching system's ($\Aa^\strat$) stability under any switching pattern constrained by $\Lambda^\strat$ would be configuring a CJSR problem in the form of Equation~\eqref{cjsr}, obtaining $\rho\,(\Aa^\strat,\GG{\Lambda^\strat})$.
However, few efficient CJSR computation methods exist.

Instead, building upon the recent work of~\cite{xu2020approximation}, we seek to obtain an equivalent system model based on \emph{Kronecker lifting}~\cite{horn2012matrix}, characterized by a set of matrices denoted by $\Alifted_{\Lambda^\strat}$ and behaving as an \emph{arbitrary switching system}, such that $\rho\funof{\Alifted_{\Lambda^\strat}} = \rho\funof{\Aa^{\strat},\GG{\Lambda^\strat}}$.
This way, powerful algorithms applicable to arbitrary switching system~\cite{vankeerberghen2014jsr,sparsejsr} can be used to find tight stability bounds.
%
By leveraging the vector $q_t$ of Definition~\ref{def:qt}, we introduce the \emph{lifted discrete-time state}, $\xi_t\in\R^{n\cdot n_{\VV{}}}$, defined as 
\begin{equation*}
    \xi_t = q_t\otimes \tilde x_t,
\end{equation*}
where $n$ is the number of closed-loop states $\tilde x_t$, $n_{\VV{}} = \abs{\VV{\Lambda^\strat}}$, and $\otimes$ is the Kronecker product operator.
By construction, $\xi_t$ is then a vector composed of $n_{\VV{}}$ blocks of size $n$, where at most one block is equal to $\tilde x_t$ and all other blocks are equal to the $0$ vector.
%
\fix{P is problematic here}
Then, we build a set of lifted matrices $P_{\alpha} ( \GG{\Lambda^\strat} )\in\R^{n\cdot n_{\VV{}}\times n\cdot n_{\VV{}}}$ that includes information about both the system dynamics and the possible transitions given a certain outcome $\alpha \in \Sigma\left(\strat\right)$:
%
\begin{equation*}
    P_{\alpha} ( \GG{\Lambda^\strat} ) = F_{\alpha} ( \GG{\Lambda^\strat} )\otimes \clmat{}_\alpha^\strat,\quad \alpha \in \Sigma \left( \strat \right).
\end{equation*}
%
The lifted dynamics of the closed loop system then become%
%
\begin{equation*}
    \xi_{t+1} = P_{\alpha} ( \GG{\Lambda^\strat} )\,\xi_t.
\end{equation*}
%
Formally, we obtain an equivalent closed-loop system composed of a set of lifted dynamic matrices, $\Alifted_{\Lambda^\strat}$.
%
\begin{definition}[Lifted switching set $\Alifted_{\Lambda^\strat}$]%
    \label{def:switching_set}%
    Given a set of dynamic matrices $\Aa^{\strat}$ and an automaton $\GG{\Lambda^\strat}$, the switching set $\Alifted_{\Lambda^\strat}$ is defined as the set of all lifted matrices corresponding to $\Aa^{\strat}$.
    Formally,
    %
    \begin{equation*}
        \Alifted_{\Lambda^\strat} = \left\{ P_{\alpha} ( \GG{\Lambda^\strat} ) \,\, | \,\, \alpha \in \Sigma\left(\strat\right) \right\}.
    \end{equation*}
\end{definition}
%
By leveraging the mixed-product property~\cite{horn2012matrix} of $\otimes$ and by introducing a proper submultiplicative norm, it is possible to prove that $\rho\left(\Alifted_{\Lambda^\strat}\right)= \rho\,(\Aa^{\strat},\GG{\Lambda^\strat})$.
For more details and a formal proof we refer the interested reader to~\cite{xu2020approximation}.


\subsection{Extended Weakly Hard and JSR Properties}
\label{sec:analytic_results}
%
We now provide a general relation between \emph{all} \ewhc{}s in terms of the joint spectral radii.

\begin{theorem}[JSR dominance]%
    \label{th:rho_dominance_general}%
    Given two arbitrary \ewhc{}, $\lambda_1^\strat$ and $\lambda_2^\strat$.
    If $\lambda_1^\strat \preceq \lambda_2^\strat$, then
    \begin{equation*}
        \rho\funof{\Alifted_{\lambda_1^\strat}} \leq \rho\funof{\Alifted_{\lambda_2^\strat}}.
    \end{equation*}

    \begin{proof}
        \fix{Add an explanation that the second part of the equation below holds because of the mixed-product property of $\otimes$, because otherwise $\clmat{}$ might be confusing}
        From Equation~\eqref{jsr}, for a generic \ewhc{} $\lambda^\strat$,
        \begin{equation*}
            \rho\left(\Alifted_{\lambda^\strat}\right) = \lim_{\ell\rightarrow \infty}\rho_\ell\left(\Alifted_{\lambda^\strat}\right), \quad \rho_\ell\left(\Alifted_{\lambda^\strat}\right) = \max_{w \in \sset{\ell}{\lambda^\strat}}\norm{\clmat{}_{w}}^{1/\ell}.
        \end{equation*}
        According to Definition~\ref{def:domination}, it holds that $\lambda^\strat_1 \preceq \lambda^\strat_2$ if and only if $\sset{}{\lambda^\strat_1} \subseteq \sset{}{\lambda^\strat_2}$.
        Thus, if for a word $a$ it holds that $a \in \sset{\ell}{\lambda^\strat_1}$, then it also holds that $a \in \sset{\ell}{\lambda^\strat_2}$.
        The set of all possible $\clmat{}_{a}$ is thus included in the set of all possible $\clmat{}_{b},\, b \in \sset{\ell}{\lambda^\strat_2}$.
        As a consequence it holds that
        \begin{equation*}
            \max_{a \in \sset{\ell}{\lambda^\strat_1}}\norm{\clmat{}_{a}}^{1/\ell} \leq
            \max_{b \in \sset{\ell}{\lambda^\strat_2}}\norm{\clmat{}_{b}}^{1/\ell}, \quad
            \forall \ell\in\mathbb{N}^{>}.
        \end{equation*}
        The theorem follows immediately when $\ell\rightarrow \infty$.
    \end{proof}
\end{theorem}

Theorem~\ref{th:rho_dominance_general} is the first result that provides a clear, analytic, correlation between the control theoretical analysis and real-time implementation.
Primarily, it implies that the constraint dominance from Definition~\ref{def:domination} also carries on to the JSR, giving us a notion of \emph{JSR dominance}.
If stability under a specific \ewhc{} is shown, Theorem~\ref{th:rho_dominance_general} guarantees stability for all \ewhc{} which are harder to satisfy.
The results of Theorem~\ref{th:rho_dominance_general} are strategy independent, further reducing the coupling between the control analysis and implementation approach.
Additionally, the results are independent of the dynamics and apply to open-loop as well as closed-loop systems, including feedforward controllers and any linear feedback control strategy -- provided that an appropriate reformulation of matrices $\Aa^\strat$ is given.

Two Corollaries of Theorem~\ref{th:rho_dominance_general} are derived for the commonly used models $\erowmiss{}{\strat}$ and $\eanymiss{}{\strat}$, highlighting some practical relations between such constraints.
\begin{corollary}[$\eanymiss{}{\strat}$ dominance]%
    \label{cor:rho_dominance_mk}%
    Given two \ewhc{}, $\lambda^\strat_1 = \overbar{\binom{x}{k_1}}^{\strat}$ and $\lambda^\strat_2 = \overbar{\binom{x}{k_2}}^{\strat}$, if $k_1 \leq k_2$ then
    \begin{equation*}
        \rho\bigl(\Alifted_{\lambda^\strat_2}\bigr) \leq \rho\bigl(\Alifted_{\lambda^\strat_1}\bigr).
    \end{equation*}
    %
    \begin{proof}
        According to~\cite{Wu:2020}, $\overbar{\binom{x}{k_2}}^{\strat}\preceq\overbar{\binom{x}{k_1}}^{\strat}$ if $k_1 \leq k_2$.
        The corollary follows directly from~\cite{Wu:2020} and Theorem~\ref{th:rho_dominance_general}.
    \end{proof}
\end{corollary}
%
\begin{corollary}[$\erowmiss{}{\strat}$ dominance]%
    \label{cor:rho_dominance_cons}%
    Given two \ewhc{}, $\lambda^\strat_1 = \erowmiss{}{\strat}$ and $\lambda^\strat_2 =\eanymiss{}{\strat}$, then
    \begin{equation*}
        \rho\bigl(\Alifted_{\lambda^\strat_2}\bigr) \leq \rho\bigl(\Alifted_{\lambda^\strat_1}\bigr).
    \end{equation*}
    %
    \begin{proof}
        According to~\cite{Maggio:2020} it holds that $\erowmiss{}{\strat} \equiv \overbar{\binom{x}{x+1}}^{\strat}$.
        The corollary follows directly from~\cite{Maggio:2020} and Corollary~\ref{cor:rho_dominance_mk}.
    \end{proof}
\end{corollary}
%
The conclusions drawn from Theorem~\ref{th:rho_dominance_general} are theoretical, but its practical applicability lies in the algorithm used to find $\rho^{LB}$ and $\rho^{UB}$, i.e., lower and upper bounds for the JSR value.
Using these bounds we can determine the stability of the corresponding switching systems, as follows:
%
$$
\rho^{LB} \bigl( \Alifted_{\lambda^\strat_1} \bigr) \leq \rho \bigl( \Alifted_{\lambda^\strat_1} \bigr) \leq \rho \bigl( \Alifted_{\lambda^\strat_2} \bigr) \leq \rho^{UB} \bigl( \Alifted_{\lambda^\strat_2} \bigr).
$$
%
Regardless of the algorithm used to find the bounds, we can generally conclude that if $\rho^{UB} \left( \Alifted_{\lambda^\strat} \right) < 1$ the constrained system is switching stable.
Similarly, if $\rho^{LB} \left( \Alifted_{\lambda^\strat} \right) > 1$ the system is unstable.
Thus, if $\lambda^\strat_1 \preceq \lambda^\strat_2$, where $\rho^{UB} \left( \Alifted_{\lambda^\strat_2} \right) < 1$, we know that the system under $\lambda^\strat_1$ constraints is switching stable.
A similar relation holds for the lower bound.

We now extend the results of Theorem~\ref{th:rho_dominance_general} by relating the joint spectral radius of a single constraint to sets of constraints.
\begin{theorem}%
    \label{th:rho_dominance_set_general}%
    Given an arbitrary \ewhc{} $\lambda^\strat$, it holds that
    \begin{equation*}
        \rho\left(\Alifted_{\Lambda^\strat}\right) \leq \rho \left( \Alifted_{\lambda^\strat} \right) ,\,\, \forall \Lambda^\strat \ni \lambda^\strat.
    \end{equation*}
    %
    \begin{proof}
        \fix{Add an explanation that the second part of the equation below holds because of the mixed-product property of $\otimes$, because otherwise $\clmat{}$ might be confusing}
        For an arbitrary \ewhc{} set $\Lambda^\strat = \{\lambda^\strat_1, \dots, \lambda^\strat_N\}$, its satisfaction set is $\sset{\ell}{\Lambda^\strat} = \bigcap_{i \in \{1,\dots, N\}} \sset{\ell}{\lambda^\strat_i}$.
        Thus, for any $\lambda^\strat \in \Lambda^\strat$ it holds that 
        \begin{equation*}
            \sset{\ell}{\Lambda^\strat} \subseteq \sset{\ell}{\lambda^\strat}
        \end{equation*}
        If a word $a$ is in $\sset{\ell}{\Lambda^\strat}$ it also belongs to $\sset{\ell}{\lambda^\strat}$. 
        The set of all possible $\clmat{}_{a}$ is thus included in the set of all possible $\clmat{}_{b},\, b \in \sset{\ell}{\lambda^\strat}$.
        As a consequence it holds that
        \begin{equation*}
            \max_{a \in \sset{\ell}{\Lambda^\strat }} \norm{\clmat{}_{a}}^{1/\ell} \leq
            \max_{b \in \sset{\ell}{\lambda^\strat }} \norm{\clmat{}_{b}}^{1/\ell}, \quad
            \forall \ell\in\mathbb{N}^{>}.
        \end{equation*}
        The theorem follows immediately when $\ell\rightarrow \infty$.
    \end{proof}
\end{theorem}

As in Theorem~\ref{th:rho_dominance_general}, the more we restrict the execution pattern of the control task with sets of constraints, the lower its JSR will be.
Theorem~\ref{th:rho_dominance_set_general} delivers the practical insight that enforcing tighter \ewhc{} to a stable system will \emph{never} destabilise it, as formally stated in the following corollary.
\begin{corollary}%
    \label{cor:rho_dominance_set}%
    Given an arbitrary \ewhc{}, $\lambda^\strat$, if $\rho \left( \Alifted_{\lambda^\strat} \right) < 1$ then
    \begin{equation*}
        \rho\left(\Alifted_{\Lambda^\strat}\right) < 1 ,\,\, \forall \Lambda^\strat \ni \lambda^\strat.
    \end{equation*}

    \begin{proof}
        The corollary follows immediately from Theorem~\ref{th:rho_dominance_set_general} when $\rho \left( \Alifted_{\lambda^\strat} \right) < 1$.
    \end{proof}
\end{corollary}
