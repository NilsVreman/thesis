In this Section we illustrate the dynamics of the closed-loop system under \ewhc{}, \removed{together with}an efficient way to implement a stability analysis and some interesting properties of the resulting system.

\subsection{Closed-loop system}%
\label{sec:matrices}

Using the alphabet $\Sigma\left(\strat\right)$, defined for a given \removed{deadline miss}strategy $\strat$, and the chosen actuator mode (Zero or Hold), we \removed{now}compute the closed-loop system's behaviour, \removed{of plant~\eqref{eq:plant} with controller~\eqref{eq:controller}}\new{recall Equations~\eqref{eq:plant} and~\eqref{eq:controller}}.
In particular, we identify one matrix for each distinct dynamics corresponding to a character $\event \in \Sigma\left( \strat \right)$, building the set $\Aa^\strat$.
Again, we treat the cases of Kill and Skip-Next, but the same approach can be extended to other strategies.

\textbf{Kill: }%
%
The alphabet for the Kill strategy includes two characters, $\Sigma\left(\text{Kill}\right) = \{ \cH, \cM \}$. We define $\tilde x_t^{\,\text{K}} = \left[ x_t^T\,\, z_t^T\,\, u_t^T \right]^T$ as the state vector for the closed-loop system.
We compute the one-step closed-loop system dynamics $A^{\,\text{K}}_{\cH}$, corresponding to the character $\cH$ \removed{(i.e., when the job released in the given period $\pi_t$ successfully completes within its deadline)}as follows.
\begin{equation*}
    \tilde x_{t+1}^{\,\text{K}} = A^{\,\text{K}}_{\cH}\,\tilde x_t^{\,\text{K}}, \quad
     A^{\,\text{K}}_{\cH} = \begin{bmatrix}
        A_p       & 0    & B_p       \\
        -B_cC_p   & A_c  & -B_cD_p   \\
        -D_cC_p   & C_c  & -D_cD_p   \\
    \end{bmatrix}
\end{equation*}
%
On the other hand, for the case of $\cM$, \removed{(i.e., when the job released in $\pi_t$ misses its deadline)}the controller terminates its execution prematurely by killing the job, thus not updating its states ($z_{t+1} = z_t$).
In this case, the controller output is forced to a default value, determined by the actuator mode and is thus either zeroed ($u_{t+1} = 0$) or held ($u_{t+1} = u_t$).
The resulting closed-loop system in state-space form is denoted with $A^{\,\text{K}}_{\cM}$ and defined as follows.
\begin{equation*}
    \tilde x_{t+1}^{\,\text{K}} = A^{\,\text{K}}_{\cM}\,\tilde x_t^{\,\text{K}}, \quad
   A^{\,\text{K}}_{\cM} = \begin{bmatrix}
        A_p & 0  & B_p \\
        0   & I  & 0   \\
        0   & 0  & \Delta
    \end{bmatrix}
\end{equation*}
Here, $I$ is the identity matrix of appropriate size, \removed{and $\Delta$ assume values depending on the actuation mode, i.e.,}$\Delta = I$ if the control signal is held and $\Delta = 0$ if it is zeroed.
The set of dynamic matrices that a controlled system under Kill strategy may experience is $\Aa^{\text{Kill}}=\left\{A^{\,\text{K}}_{\cH},A^{\,\text{K}}_{\cM}\right\}$.

\textbf{Skip-Next: }%
%
\removed{When considering \ewhc{} under the Skip-Next strategy, characterizing the resulting closed-loop dynamics requires a different state vector model.
In this case, when a control task released in an arbitrary interval $\pi_t$ misses a deadline, it is allowed to continue its execution until completion.
Once completed, the control state will then be updated with a control command computed using \emph{old} measurement values.
Thus,}\new{For the Skip-Next strategy,} we introduce two additional states $\hat x_t$ and $\hat u_t$ that store \removed{those}\new{the corresponding} old values \new{of $x_t$ and $u_t$} while the controller awaits an update.
The resulting state vector then becomes $\tilde x_t^{\,\text{S}} = \left[ x_t^T\,\,z_t^T\,\, u_t^T\,\, \hat{x}_t^T\,\, \hat{u}_t^T \right]^T$.

When $\pi_t$ is associated to $\cH$ (i.e., a job is both activated and completed in $\pi_t$), the two \removed{augmented}\new{additional} states mirror the behaviour of the states of which they are storing data.
The resulting closed-loop system is described using $A^{\,\text{S}}_{\cH}$ as follows:
\begin{equation*}
    \tilde x_{t+1}^{\,\text{S}} = A^{\,\text{S}}_{\cH}\,\tilde x_t^{\,\text{S}}, \,\,\,
    A^{\,\text{S}}_{\cH} = \begin{bmatrix}
        A_p       & 0    & B_p      & 0 & 0 \\
        -B_cC_p   & A_c  & -B_cD_p  & 0 & 0 \\
        -D_cC_p   & C_c  & -D_cD_p  & 0 & 0 \\
        A_p       & 0    & B_p      & 0 & 0 \\
        -D_cC_p   & C_c  & -D_cD_p  & 0 & 0 \\
    \end{bmatrix}.
\end{equation*}
%
For the case of $\cM$ in $\pi_t$ (i.e., when a job is not completed) the two augmented states hold their previous values. The
resulting closed-loop system is described by $A^{\,\text{S}}_{\cM}$:
%
\begin{equation*}
    \tilde x_{t+1}^{\,\text{S}} = A^{\,\text{S}}_{\cM}\,\tilde x_t^{\,\text{S}}, \,\,\,
    A^{\,\text{S}}_{\cM}=\begin{bmatrix}
        A_p & 0  & B_p & 0 & 0 \\
        0   & I  & 0   & 0 & 0 \\
        0   & 0  & \Delta   & 0 & 0 \\
        0   & 0  & 0   & I & 0 \\
        0   & 0  & 0   & 0 & I \\
	\end{bmatrix}.
\end{equation*}
%
Finally, for the case of recovery $\cR$ in $\pi_t$ \removed{(i.e., when no control job is activated, but an old one is completed)}the new control command is calculated using the old measurement and control values stored in $\hat x_t$ and $\hat u_t$.
The resulting closed-loop system is described by $A^{\,\text{S}}_{\cR}$ as follows:
%
\begin{equation*}
    \tilde x_{t+1}^{\,\text{S}} = A^{\,\text{S}}_{\cR} \, \tilde x_{t}^{\,\text{S}},\,\,\,
    A^{\,\text{S}}_{\cR} = \begin{bmatrix}
        A_p & 0    & B_p & 0       & 0 \\
        0   & A_c  & 0   & -B_cC_p & -B_cD_p \\
        0   & C_c  & 0   & -D_cC_p & -D_cD_p \\
        A_p & 0    & B_p & 0       & 0 \\
        0   & C_c  & 0   & -D_cC_p & -D_cD_p \\
    \end{bmatrix}.
\end{equation*}
%
The set of dynamic matrices under Skip-Next strategy can then be finally defined as $\Aa^{\text{Skip-Next}}=\left\{A^{\,\text{S}}_{\cH},A^{\,\text{S}}_{\cM},A^{\,\text{S}}_{\cR}\right\}$.

\subsection{Kronecker lifted switching system}%
\label{sec:system_dynamics}

\removed{We have so far derived the possible switching patterns under a set of constraints $\Lambda^\strat$ in terms of a graph $\GG{\Lambda^\strat}$ (Section~\ref{sec:model}), and the closed-loop switching dynamics $\Aa^\strat$ of the controlled system (Section~\ref{sec:matrices}).}%
To analyse the system stability under any switching pattern constrained by $\Lambda^\strat$, we \removed{are then required to}combine the set of system dynamics $\Aa^\strat$ with the FSM describing the allowed switches $\GG{\Lambda^\strat}$.

A straightforward approach would be configuring a CJSR problem in the form of Equation~\eqref{cjsr}, obtaining $\rho\,(\Aa^\strat,\GG{\Lambda^\strat})$.
%
Conversely, building upon the recent work of~\cite{xu2020approximation}, we seek to obtain an equivalent \new{system} model \removed{of the system}based on Kronecker lifting.
The model obtained with this approach \new{is} characterized by a set of matrices \removed{that we}denoted \removed{with}\new{by} $\Alifted_{\Lambda^\strat}$ \new{and} behaves as an \emph{arbitrary switching system}, such that \removed{the JSR of the set $\Alifted_{\Lambda^\strat}$ is equal to the CJSR of the constrained system $\rho\left(\Aa^{\strat},\GG{\Lambda^\strat}\right)$, i.e.,}$\rho\,(\Alifted_{\Lambda^\strat})= \rho\,(\Aa^{\strat},\GG{\Lambda^\strat})$.
\removed{In this way}\new{Therefore}, \removed{we can use}powerful algorithms applicable to arbitrary switching system -- such as the ones implemented in~\cite{vankeerberghen2014jsr} and \removed{the {\tt SparseJSR} algorithm}\cite{sparsejsr} -- \new{can be used} to find tight stability bounds \removed{of our target controlled system}.
For the sake of brevity, we \removed{will give}only \new{provide} an intuitive explanation \removed{here about}\new{of} the Kronecker lifting approach presented in~\cite{xu2020approximation}.
\removed{The interested reader is referred to~\cite{xu2020approximation} for a more detailed analysis.}The Kronecker product is formally defined as follows~\cite{horn2012matrix}.

\begin{definition}[Kronecker product]%
    The Kronecker product between two real-valued matrices $A$ and $B$ \emph{of any size} is\removed{defined as}:%
    \begin{equation*}
        A \otimes B :=
        %\left[\begin{array}{cccc}
        \begin{bmatrix}
            a_{11}\cdot B   & a_{12}\cdot B & \hdots \\
            a_{21}\cdot B   & a_{22}\cdot B & \hdots \\
            \vdots          & \vdots        & \ddots\\
        %\end{array}\right].
      \end{bmatrix}.
    \end{equation*}
\end{definition}
\removed{%
Differently from the common matrix product, the Kronecker product does not require the equivalence between the number of rows of $A$ and columns of $B$.

The Kronecker product is used here to build a set of lifted matrices ($P_\event$) that includes the information of both the system dynamics (closed-loop matrix $A^\strat_\event$) and the possible transitions (transition matrix $F_\event$) of a given outcome $\event\in\Sigma\left(\strat\right)$. 
To avoid heavy notation, we will consider $A^\strat_\event\equiv A_\event$ henceforth, with implicit dependency on $\strat$.
}%

\new{First, by leveraging the vector $q_t$ of Definition~\ref{def:qt}}, we introduce the \emph{lifted discrete-time state}, $\xi_t\in\mathbb{R}^{n\abs{\VV{\Lambda^\strat}}}$, defined as 
$$\xi_t = q_t\otimes x_t.$$
By construction, $\xi_t$ is a vector composed of $\abs{\VV{\Lambda^\strat}}$ blocks of size $n$, where at most one block is equal to $x_t$ and all \removed{the}other blocks are equal to the $0$ vector.

\removed{All the possible lifted closed-loop system dynamic matrix associated to $\xi_t$ are defined as}\new{Second, we build a set of lifted matrices $P_{\event} ( \GG{\Lambda^\strat} )\in\mathbb{R}^{n\abs{\VV{\Lambda^\strat}}\times n\abs{\VV{\Lambda^\strat}}}$} that includes \removed{the}information \removed{of}\new{about} both the system dynamics (\removed{closed-loop matrix}$A^\strat_\event$ \new{of Section~\ref{sec:matrices}}) and the possible transitions (\removed{transition matrix}$F_{\event} ( \GG{\Lambda^\strat} )$ \new{of Definition~\ref{def:transition}}) \removed{of a}given \new{a certain} outcome $\event\in\Sigma\left(\strat\right)$, as follows
%
\begin{equation}
    \label{eq:lifted_matrix}
    P_{\event} ( \GG{\Lambda^\strat} ) = F_{\event} ( \GG{\Lambda^\strat} )\otimes A_\event^\strat,\quad \event \in \Sigma \left( \strat \right).
\end{equation}
%
\removed{where $P_\event\in\mathbb{R}^{n\abs{\VV{\Lambda^\strat}}\times n\abs{\VV{\Lambda^\strat}}}$.}%
%
The lifted dynamics of the closed loop system then become%
%
\removed{\begin{equation*}
    \xi_{t+1} = P_\event\,\xi_t, \quad \event=\alpha_t.
\end{equation*}}%
\new{%
$$
    \xi_{t+1} = P_{\event} ( \GG{\Lambda^\strat} )\,\xi_t.
$$
}%
\removed{To better understand how this works, we here apply the theory to Example~$1$.
\begin{example}%
    \emph{Consider the graph of Example~$1$ in Figure X (left). The transition matrices $F_{\cH}$ and $F_{\cM}$, computed using Definition X, are:}
    \begin{equation}
        F_{\cH} =
        \begin{bmatrix}
            1 & 0 & 1 \\
            0 & 0 & 0 \\
            0 & 1 & 0
        \end{bmatrix}, \quad
        F_{\cM} =
        \begin{bmatrix}
            0 & 0 & 0 \\
            1 & 0 & 0 \\
            0 & 0 & 0
        \end{bmatrix}.
    \end{equation}
    \emph{The lifted matrices corresponding to $\cH$ and $\cM$ are then computed using Equation X as follows:}
    \begin{equation}
        P_{\cH} =
        \begin{bmatrix}
            A_\cH & 0 & A_\cH \\
            0 & 0 & 0 \\
            0 &  A_\cH & 0
        \end{bmatrix}, \quad
        P_{\cM} =
        \begin{bmatrix}
            0 & 0 & 0 \\
            A_\cM & 0 & 0 \\
            0 & 0 & 0
        \end{bmatrix}.
    \end{equation}
    \emph{Let consider an initial G-state $q_1 = \left[ 1,\, 0,\, 0 \right]^T$ and an initial plant state $x_1$.  Given a sequence $\alpha = \cH\cM$, we obtain:}
    \begin{equation*}
        \begin{aligned}
            \xi_3 &= P_\cM\, P_\cH\, \xi_1 \\
            &=\begin{bmatrix}
                0 & 0 & 0 \\
                A_\cM & 0 & 0 \\
                0 & 0 & 0
            \end{bmatrix}
            \begin{bmatrix}
                A_\cH & 0 & A_\cH \\
                0 & 0 & 0 \\
                0 &  A_\cH & 0
            \end{bmatrix}
            \begin{bmatrix}
                x_1 \\
                0   \\
                0   \\
            \end{bmatrix} =
            \begin{bmatrix}
                0 \\
                A_\cM\, A_\cH\, x_1 \\
                0 \\
            \end{bmatrix}
        \end{aligned}
    \end{equation*}
    \emph{Instead, if the sequence $\alpha = \cM\cM$ was given, which is infeasible for this example (}$\alpha \notin \sset{}{(1,3)_{\text{Kill}} }$\emph{), the dynamics will converge to $\xi=0$}.
    \begin{equation*}
        \xi_3 = P_\cM\, P_\cM\, \xi_1 =
        \begin{bmatrix}
            0 & 0 & 0 \\
            A_\cM & 0 & 0 \\
            0 & 0 & 0
        \end{bmatrix}
        \begin{bmatrix}
            0 & 0 & 0 \\
            A_\cM & 0 & 0 \\
            0 & 0 & 0
        \end{bmatrix}
        \begin{bmatrix}
            x_1 \\
            0   \\
            0
        \end{bmatrix} =
        \begin{bmatrix}
            0 \\
            0 \\
            0
        \end{bmatrix}
    \end{equation*}%\qed
\end{example}}%
%
Formally, we obtain a system composed of a set of switching dynamic matrices, $\Alifted_{\Lambda^\strat}$, which is defined as follows.
%
\begin{definition}[Lifted switching set $\Alifted_{\Lambda^\strat}$]%
    \label{def:switching_set}%
    Given a set of dynamic matrices $\Aa^{\strat}$ and a graph $\GG{\Lambda^\strat}$, the switching set $\Alifted_{\Lambda^\strat}$ is defined as the set of all lifted matrices corresponding to $\Aa^{\strat}$.
    Formally,
    %
    \removed{$$
    \Alifted_{\Lambda^\strat} = \left\{ F_\event \otimes A^{\strat}_\event \,\, | \,\, \event \in \Sigma\left(\strat\right),\, A^{\strat}_\event \in \Aa^{\strat} \right\}.
    $$}
    $$
    \Alifted_{\Lambda^\strat} = \left\{ P_{\event} ( \GG{\Lambda^\strat} ) \,\, | \,\, \event \in \Sigma\left(\strat\right) \right\}.
    $$
    \end{definition}
%
\removed{Proving then that $\rho\left(\Alifted_{\Lambda^{\strat}}\right)= \rho\left(\Aa^{\strat},\GG{\Lambda^{\strat}}\right)$ requires first finding a proper norm for the lifted system.
For an arbitrary sequence $\astring\in\Sigma\left(\strat\right)^N$, the mixed-product property~\cite{horn2012matrix} of $\otimes$ guarantees the following relationship:
%
\begin{equation*}
\begin{aligned}
P_\astring &= P_{\alpha_N}\cdots P_{\alpha_2}P_{\alpha_1} \\ &= \left(F_{\alpha_N}\cdots F_{\alpha_2}F_{\alpha_1}\right)\otimes\left(A^{\strat}_{\alpha_N}\cdots A^{\strat}_{\alpha_2}A^{\strat}_{\alpha_1}\right)
\end{aligned}
\end{equation*}
%
Then, we introduce a submultiplicative norm $\vertiii{ \cdot}$ such that, for an arbitrary matrix $M$,
it holds that
%
$$\vertiii{M} = \max_j\sum_i\Vert m_{i,j} \Vert.$$
%
Finally, recalling from Section~\ref{ssec:dynamicgraph} that there may be at most one entry 1 in each column of $F_\alpha$, and from Lemma~\ref{cor:Fseqnotinlambda} that $F_\alpha=0$ if $\alpha \notin \sset{}{\Lambda }$, we obtain that:
\begin{equation}
\vertiii{P_\alpha} = \left\{ \begin{array}{cc}
    \Vert A_\alpha\Vert & \mbox{when } \alpha \in \sset{}{\Lambda } \\
    0 & \mbox{when } \alpha \notin \sset{}{\Lambda }
 \end{array}\right.
\end{equation}

As a consequence, it follows that $\rho\left(\Alifted_{\Lambda^\strat}\right)= \rho\left(\Aa^{\strat},\GG{\lambda^\strat}\right)$, which proves the viability of our approach.%
}\new{By leveraging the mixed-product property~\cite{horn2012matrix} of $\otimes$ and by introducing a proper submultiplicative norm, it is possible to prove that $\rho\left(\Alifted_{\Lambda^\strat}\right)= \rho\,(\Aa^{\strat},\GG{\Lambda^\strat})$.
For more details and a formal proof we refer the interested reader to~\cite{xu2020approximation}.}

\subsection{Extended weakly hard and JSR properties}
\label{sec:analytic_results}
\removed{In Section~\ref{sec:existing} we recalled existing stability results on the $\overbar{\left< m \right>}$ model obtained with the JSR~\cite{Maggio:2020}.}%
We now provide a general relation between \emph{all} \ewhc{}s in terms of the joint spectral radii, by leveraging the model presented earlier in the paper.

\begin{theorem}[JSR dominance]%
    \label{th:rho_dominance_general}%
    Given two arbitrary \ewhc{}, $\lambda_1^\strat$ and $\lambda_2^\strat$, if $\lambda_2^\strat \preceq \lambda_1^\strat$ then
    \begin{equation*}
        \rho\bigl(\Alifted_{\lambda_2^\strat}\bigr) \leq \rho\bigl(\Alifted_{\lambda_1^\strat}\bigr).
    \end{equation*}

    \begin{proof}
        From Equation~\eqref{jsr}, for a generic \ewhc{} $\lambda^\strat$,
        \begin{equation*}
            \rho\left(\Alifted_{\lambda^\strat}\right) = \lim_{\ell\rightarrow \infty}\rho_\ell\left(\Alifted_{\lambda^\strat}\right), \quad \rho_\ell\left(\Alifted_{\lambda^\strat}\right) = \max_{a \in \sset{\ell}{\lambda^\strat}}\norm{A_{a}}^{1/\ell}.
        \end{equation*}
        Definition~\ref{def:domination} gave us that $\lambda^\strat_2 \preceq \lambda^\strat_1$ if and only if $\sset{}{\lambda^\strat_2} \subseteq \sset{}{\lambda^\strat_1}$.
        Thus, if an arbitrary sequence is in the satisfaction set of $\lambda^\strat_2$, i.e., $b \in \sset{\ell}{ \lambda^\strat_2 }$, this means $b$ also belongs to the satisfaction set of $\lambda^\strat_1$, i.e., $b \in \sset{\ell}{ \lambda^\strat_1 }$.
        The set of all possible $A_{b}$ is thus included in the set of all possible $A_{a},\, a \in \sset{\ell}{ \lambda^\strat_1 }$.
        As a consequence it holds that
        \begin{equation*}
            \max_{b \in \sset{\ell}{\lambda^\strat_2}}\norm{A_{b}}^{1/\ell} \leq
            \max_{a \in \sset{\ell}{\lambda^\strat_1}}\norm{A_{a}}^{1/\ell}, \quad
            \forall \ell\in\mathbb{N}^{>}.
        \end{equation*}
        The theorem follows immediately when $\ell\rightarrow \infty$.
    \end{proof}
\end{theorem}

The intuition that a sequence of control activations containing a large number of deadline misses is less robust than one including fewer errors, is well-established.
However, Theorem~\ref{th:rho_dominance_general} is the first result that provides a clear, analytic, correlation between the control theoretical analysis and real-time implementation.
Primarily, it implies that the constraint dominance from Definition~\ref{def:domination} also carries on to the JSR, giving us a notion of \emph{JSR dominance}.
\removed{One of the most important consequences arising from the JSR dominance is the relation between the implementation and analysis of a weakly-hard constrained system.}If stability under a specific \ewhc{} is shown, Theorem~\ref{th:rho_dominance_general} guarantees stability for all \ewhc{} which are harder to satisfy.\footnote{This result also applies to the stability analysis presented in~\cite{Maggio:2020}, giving the user options for which methodology to use.}
We wish to emphasise that the results of Theorem~\ref{th:rho_dominance_general} are strategy independent (as long as $\lambda^\strat_1$ and $\lambda^\strat_2$ use the same strategy $\strat$), further reducing the coupling between the control analysis and implementation approach.
\new{The results are also independent of the dynamics, and apply to open-loop as well as closed-loop systems, including feedforward controllers and any linear feedback control strategy, provided that an appropriate reformulation of matrices $\Aa^\strat$ is given.}

\removed{Although Theorem~\ref{th:rho_dominance_general} holds for any \ewhc{}, $\lambda_1$ and $\lambda_2$, we here present a valuable special case of the theorem.}%
Due to the $\overbar{\left< m \right>}^{\strat}$ and \removed{$\mks{}{\strat}$}\new{$\overbar{\binom{m}{k}}^{\strat}$} constraints being the two most used models, we derive \new{two Corollaries of Theorem~\ref{th:rho_dominance_general}, highlighting} some practical relations between \removed{them}\new{such constraints} in terms of the corresponding joint spectral radii.
\begin{corollary}[\removed{$\mks{}{\strat}$}\new{$\overbar{\binom{m}{k}}^{\strat}$} dominance]%
    \label{cor:rho_dominance_mk}%
    Given two \ewhc{}, \removed{$\lambda_1 = \mkv{1}_{\strat}$ and $\lambda_2 = \mkv{2}_{\strat}$}\new{$\lambda^\strat_1 = \overbar{\binom{m}{k_1}}^{\strat}$ and $\lambda^\strat_2 = \overbar{\binom{m}{k_2}}^{\strat}$}, if $k_1 \leq k_2$ then
    $$
        \rho\bigl(\Alifted_{\lambda^\strat_2}\bigr) \leq \rho\bigl(\Alifted_{\lambda^\strat_1}\bigr).
    $$
%
    \begin{proof}
        According to~\cite{Wu:2020}, \removed{it holds that $\mkv{2}_{\strat} \preceq \mkv{1}_{\strat}$}\new{$\overbar{\binom{m}{k_2}}^{\strat}\preceq\overbar{\binom{m}{k_1}}^{\strat}$} if $k_1 \leq k_2$.
        The corollary follows directly from~\cite{Wu:2020} and Theorem~\ref{th:rho_dominance_general}.
    \end{proof}
\end{corollary}
%
\begin{corollary}[$\overbar{\left< m \right>}^{\strat}$ dominance]%
    \label{cor:rho_dominance_cons}%
    Given two \ewhc{}, $\lambda^\strat_1 = \overbar{\left<m\right>}^{\strat}$ and \removed{$\lambda_2 = \mks{}{\strat},\, k > m$}\new{$\lambda^\strat_2 =\overbar{\binom{m}{k}}^{\strat}$}, then
    $$
        \rho\bigl(\Alifted_{\lambda^\strat_2}\bigr) \leq \rho\bigl(\Alifted_{\lambda^\strat_1}\bigr).
    $$
%
    \begin{proof}
        According to~\cite{Maggio:2020} it holds that \removed{$\overbar{\left<m\right>}_{\strat} \equiv \overbar{\binom{m}{m+1}}^{\strat}$}\new{$\overbar{\left<m\right>}^{\strat} \equiv\overbar{\binom{m}{m+1}}^{\strat}$}. The corollary follows directly from~\cite{Maggio:2020} and Corollary~\ref{cor:rho_dominance_mk}.
    \end{proof}
\end{corollary}
%
The conclusions drawn from Theorem~\ref{th:rho_dominance_general} are theoretical and its practical applicability depends on the algorithms used to find lower and upper bounds for the JSR value $\rho^{LB}$ and $\rho^{UB}$.
Using these bounds we can determine the stability of the switching system.
However, it is not necessarily true that the upper bounds found using different algorithms follow the ordering presented above.
Still, we can bound the switching stability as
%
$$
\rho^{LB} \bigl( \Alifted_{\lambda^\strat_2} \bigr) \leq \rho \bigl( \Alifted_{\lambda^\strat_2}
\bigr) \leq \rho \bigl( \Alifted_{\lambda^\strat_1} \bigr) \leq \rho^{UB} \bigl(
\Alifted_{\lambda^\strat_1} \bigr).
$$
%
Regardless of the algorithm used to find the bounds, we can generally conclude that if $\rho^{UB} \left( \Alifted_{\lambda^\strat} \right) < 1$ \removed{we know that}the constrained system is switching stable.
Similarly, if $\rho^{LB} \left( \Alifted_{\lambda^\strat} \right) > 1$ the system is unstable.
Thus, if $\lambda^\strat_2 \preceq \lambda^\strat_1$, where $\rho^{UB} \left( \Alifted_{\lambda^\strat_1} \right) < 1$, we know that the system under $\lambda^\strat_2$ constraints is switching stable.
A similar relation holds for the lower bound.

We now extend the results of Theorem~\ref{th:rho_dominance_general} by relating the joint spectral radius of a single constraint to sets of constraints.
\begin{theorem}%
    \label{th:rho_dominance_set_general}%
    Given an arbitrary \ewhc{} $\lambda^\strat$, it holds that
    $$
        \rho\left(\Alifted_{\Lambda^\strat}\right) \leq \rho \left( \Alifted_{\lambda^\strat} \right) ,\,\, \forall \Lambda^\strat \ni \lambda^\strat.
    $$

    \begin{proof}
        \removed{%
        A relationship between the satisfaction set of a constraint set and the individual constraints' satisfaction sets was presented in Equation~\eqref{eq:satisfaction-multi}.
        Consequently, 
        }\new{Considering the relationship presented in Equation~\eqref{eq:satisfaction-multi},} for an arbitrary constraint $\lambda^\strat$ and constraint set $\Lambda^\strat$, where $\Lambda^\strat = \{ \Lambda^\strat_1, \lambda^\strat \}$, it holds that
        \begin{equation}
            \sset{\ell}{\Lambda^\strat } = \sset{\ell}{ \Lambda^\strat_1 } \cap \sset{\ell}{ \lambda^\strat } \subseteq \sset{\ell}{ \lambda^\strat }.
        \end{equation}
        \removed{%
            Thus, if an arbitrary sequence is in the satisfaction set of $\Lambda^\strat$, i.e., $\beta \in \sset{\ell}{\Lambda^\strat }$, this means $\beta$ also belongs to the satisfaction set of $\lambda^\strat$, i.e., $\beta \in \sset{\ell}{ \lambda^\strat }$.
        }%
        \new{%
            Thus, if an arbitrary sequence $b$ is in the satisfaction set of $\Lambda^\strat$ it also belongs to the satisfaction set of $\lambda^\strat$, i.e., $b \in \sset{\ell}{ \Lambda^\strat } \Rightarrow b \in \sset{\ell}{ \lambda^\strat }$.%
        }%
        The set of all possible $A_{b}$ is thus included in the set of all possible $A_{a},\, a \in \sset{\ell}{ \lambda^\strat }$.
        As a consequence it holds that
        \begin{equation*}
            \max_{b \in \sset{\ell}{ \Lambda^\strat } } \norm{A_{b}}^{1/\ell} \leq
            \max_{a \in \sset{\ell}{ \lambda^\strat } } \norm{A_{a}}^{1/\ell}, \quad
            \forall \ell\in\mathbb{N}^{>}.
        \end{equation*}
        The theorem follows immediately when $\ell\rightarrow \infty$.
    \end{proof}
\end{theorem}

Following the same intuition as Theorem~\ref{th:rho_dominance_general}, the more we restrict the execution pattern of the task constrained by $\lambda^\strat$, the lower the JSR will be.
In the case of constraint sets, the notion of JSR dominance is not as evident as it was for a single constraint.
However, the JSR dominance extension to sets of constraints further improves the relation between constraint specification and control verification.

One of the most important implications of Theorem~\ref{th:rho_dominance_set_general} is the practical significance it has on the switching stability of the system.
Specifically, enforcing tighter \ewhc{} to a stable system will \emph{never} destabilise it.
\removed{Thus, if a system has been shown to be stable under a distinct constraint $\lambda$, the system implementation may be changed arbitrarily as long as the control task's execution still satisfies $\lambda$. }The result is formally given by the following corollary.
\begin{corollary}%
    \label{cor:rho_dominance_set}%
    Given an arbitrary \ewhc{}, $\lambda^\strat$, if $\rho \left( \Alifted_{\lambda^\strat} \right) < 1$ then
    $$\rho\left(\Alifted_{\Lambda^\strat}\right) < 1 ,\,\, \forall \Lambda^\strat \ni \lambda^\strat. $$

    \begin{proof}
        The corollary follows immediately from Theorem~\ref{th:rho_dominance_set_general} when $\rho \left( \Alifted_{\lambda^\strat} \right) < 1$.
    \end{proof}
\end{corollary}
