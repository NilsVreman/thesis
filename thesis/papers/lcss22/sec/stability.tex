In this Section we illustrate the dynamics of the closed-loop system under \ewhc{}, an efficient way to implement a stability analysis and some interesting properties of the resulting system.

\subsection{Closed-loop system}%
\label{sec:matrices}

Using the alphabet $\Sigma\left(\strat\right)$, defined for a given strategy $\strat$, and the chosen actuator mode (Zero or Hold), we compute the closed-loop system's behaviour, recall Equations~\eqref{eq:plant} and~\eqref{eq:controller}.
In particular, we identify one matrix for each distinct dynamics corresponding to a character $\event \in \Sigma\left( \strat \right)$, building the set $\Aa^\strat$.
Again, we treat the cases of Kill and Skip-Next, but the same approach can be extended to other strategies.

\textbf{Kill: }%
%
The alphabet for the Kill strategy includes two characters, $\Sigma\left(\text{Kill}\right) = \{ \cH, \cM \}$. We define $\tilde x_t^{\,\text{K}} = \left[ x_t^T\,\, z_t^T\,\, u_t^T \right]^T$ as the state vector for the closed-loop system.
We compute the one-step closed-loop system dynamics $A^{\,\text{K}}_{\cH}$, corresponding to the character $\cH$ as follows.
\begin{equation*}
    \tilde x_{t+1}^{\,\text{K}} = A^{\,\text{K}}_{\cH}\,\tilde x_t^{\,\text{K}}, \quad
     A^{\,\text{K}}_{\cH} = \begin{bmatrix}
        A_p       & 0    & B_p       \\
        -B_cC_p   & A_c  & -B_cD_p   \\
        -D_cC_p   & C_c  & -D_cD_p   \\
    \end{bmatrix}
\end{equation*}
%
On the other hand, for the case of $\cM$, the controller terminates its execution prematurely by killing the job, thus not updating its states ($z_{t+1} = z_t$).
In this case, the controller output is forced to a default value, determined by the actuator mode and is thus either zeroed ($u_{t+1} = 0$) or held ($u_{t+1} = u_t$).
The resulting closed-loop system in state-space form is denoted with $A^{\,\text{K}}_{\cM}$ and defined as follows.
\begin{equation*}
    \tilde x_{t+1}^{\,\text{K}} = A^{\,\text{K}}_{\cM}\,\tilde x_t^{\,\text{K}}, \quad
    A^{\,\text{K}}_{\cM} = \begin{bmatrix}
        A_p & 0  & B_p \\
        0   & I  & 0   \\
        0   & 0  & \Delta
    \end{bmatrix}
\end{equation*}
Here, $I$ is the identity matrix of appropriate size, $\Delta = I$ if the control signal is held and $\Delta = 0$ if it is zeroed.
The set of dynamic matrices that a controlled system under Kill strategy may experience is $\Aa^{\text{Kill}}=\left\{A^{\,\text{K}}_{\cH},A^{\,\text{K}}_{\cM}\right\}$.

\textbf{Skip-Next: }%
%
For the Skip-Next strategy, we introduce two additional states $\hat x_t$ and $\hat u_t$ that store the corresponding old values of $x_t$ and $u_t$ while the controller awaits an update.
The resulting state vector then becomes $\tilde x_t^{\,\text{S}} = \left[ x_t^T\,\,z_t^T\,\, u_t^T\,\, \hat{x}_t^T\,\, \hat{u}_t^T \right]^T$.

When $\pi_t$ is associated to $\cH$ (i.e., a job is both activated and completed in $\pi_t$), the two additional states mirror the behaviour of the states of which they are storing data.
The resulting closed-loop system is described using $A^{\,\text{S}}_{\cH}$ as follows:
\begin{equation*}
    \tilde x_{t+1}^{\,\text{S}} = A^{\,\text{S}}_{\cH}\,\tilde x_t^{\,\text{S}}, \,\,\,
    A^{\,\text{S}}_{\cH} = \begin{bmatrix}
        A_p       & 0    & B_p      & 0 & 0 \\
        -B_cC_p   & A_c  & -B_cD_p  & 0 & 0 \\
        -D_cC_p   & C_c  & -D_cD_p  & 0 & 0 \\
        A_p       & 0    & B_p      & 0 & 0 \\
        -D_cC_p   & C_c  & -D_cD_p  & 0 & 0 \\
    \end{bmatrix}.
\end{equation*}
%
For the case of $\cM$ in $\pi_t$ (i.e., when a job is not completed) the two augmented states hold their previous values. The
resulting closed-loop system is described by $A^{\,\text{S}}_{\cM}$:
%
\begin{equation*}
    \tilde x_{t+1}^{\,\text{S}} = A^{\,\text{S}}_{\cM}\,\tilde x_t^{\,\text{S}}, \,\,\,
    A^{\,\text{S}}_{\cM}=\begin{bmatrix}
        A_p & 0  & B_p & 0 & 0 \\
        0   & I  & 0   & 0 & 0 \\
        0   & 0  & \Delta   & 0 & 0 \\
        0   & 0  & 0   & I & 0 \\
        0   & 0  & 0   & 0 & I \\
	\end{bmatrix}.
\end{equation*}
%
Finally, for the case of recovery $\cR$ in $\pi_t$ the new control command is calculated using the old measurement and control values stored in $\hat x_t$ and $\hat u_t$.
The resulting closed-loop system is described by $A^{\,\text{S}}_{\cR}$ as follows:
%
\begin{equation*}
    \tilde x_{t+1}^{\,\text{S}} = A^{\,\text{S}}_{\cR} \, \tilde x_{t}^{\,\text{S}},\,\,\,
    A^{\,\text{S}}_{\cR} = \begin{bmatrix}
        A_p & 0    & B_p & 0       & 0 \\
        0   & A_c  & 0   & -B_cC_p & -B_cD_p \\
        0   & C_c  & 0   & -D_cC_p & -D_cD_p \\
        A_p & 0    & B_p & 0       & 0 \\
        0   & C_c  & 0   & -D_cC_p & -D_cD_p \\
    \end{bmatrix}.
\end{equation*}
%
The set of dynamic matrices under Skip-Next strategy can then be finally defined as $\Aa^{\text{Skip-Next}}=\left\{A^{\,\text{S}}_{\cH},A^{\,\text{S}}_{\cM},A^{\,\text{S}}_{\cR}\right\}$.

\subsection{Kronecker lifted switching system}%
\label{sec:system_dynamics}
%
To analyse the system stability under any switching pattern constrained by $\Lambda^\strat$, we combine the set of system dynamics $\Aa^\strat$ with the automaton describing the allowed switches $\GG{\Lambda^\strat}$.

A straightforward approach would be configuring a CJSR problem in the form of Equation~\eqref{cjsr}, obtaining $\rho\,(\Aa^\strat,\GG{\Lambda^\strat})$.
%
Conversely, building upon the recent work of~\cite{xu2020approximation}, we seek to obtain an equivalent system model based on Kronecker lifting.
The model obtained with this approach is characterized by a set of matrices denoted by $\Alifted_{\Lambda^\strat}$ and behaves as an \emph{arbitrary switching system}, such that $\rho\,(\Alifted_{\Lambda^\strat})= \rho\,(\Aa^{\strat},\GG{\Lambda^\strat})$.
Therefore, powerful algorithms applicable to arbitrary switching system -- such as the ones implemented in~\cite{vankeerberghen2014jsr} and \cite{sparsejsr} -- can be used to find tight stability bounds.
For the sake of brevity, we only provide an intuitive explanation of the Kronecker lifting approach presented in~\cite{xu2020approximation}.
The Kronecker product is formally defined as follows~\cite{horn2012matrix}.

\begin{definition}[Kronecker product]%
    The Kronecker product between two real-valued matrices $A$ and $B$ \emph{of any size} is:%
    \begin{equation*}
        A \otimes B :=
        \begin{bmatrix}
            a_{11}\cdot B   & a_{12}\cdot B & \hdots \\
            a_{21}\cdot B   & a_{22}\cdot B & \hdots \\
            \vdots          & \vdots        & \ddots\\
      \end{bmatrix}.
    \end{equation*}
\end{definition}

First, by leveraging the vector $q_t$ of Definition~\ref{def:qt}, we introduce the \emph{lifted discrete-time state}, $\xi_t\in\R^{n\cdot n_{\VV{}}}$, defined as 
$$\xi_t = q_t\otimes \tilde x_t,$$
where $n$ is the number of states in $\tilde x_t$ and $n_{\VV{}} = \abs{\VV{\Lambda^\strat}}$.
By construction, $\xi_t$ is then a vector composed of $n_{\VV{}}$ blocks of size $n$, where at most one block is equal to $\tilde x_t$ and all other blocks are equal to the $0$ vector.

Second, we build a set of lifted matrices $P_{\event} ( \GG{\Lambda^\strat} )\in\R^{n\cdot n_{\VV{}}\times n\cdot n_{\VV{}}}$ that includes information about both the system dynamics ($A^\strat_\event$ of Section~\ref{sec:matrices}) and the possible transitions ($F_{\event} ( \GG{\Lambda^\strat} )$ of Definition~\ref{def:transition}) given a certain outcome $\event\in\Sigma\left(\strat\right)$, as follows
%
\begin{equation}
    \label{eq:lifted_matrix}
    P_{\event} ( \GG{\Lambda^\strat} ) = F_{\event} ( \GG{\Lambda^\strat} )\otimes A_\event^\strat,\quad \event \in \Sigma \left( \strat \right).
\end{equation}
%
The lifted dynamics of the closed loop system then become%
%
$$
    \xi_{t+1} = P_{\event} ( \GG{\Lambda^\strat} )\,\xi_t.
$$
%
Formally, we obtain a system composed of a set of switching dynamic matrices, $\Alifted_{\Lambda^\strat}$, which is defined as follows.
%
\begin{definition}[Lifted switching set $\Alifted_{\Lambda^\strat}$]%
    \label{def:switching_set}%
    Given a set of dynamic matrices $\Aa^{\strat}$ and an automaton $\GG{\Lambda^\strat}$, the switching set $\Alifted_{\Lambda^\strat}$ is defined as the set of all lifted matrices corresponding to $\Aa^{\strat}$.
    Formally,
    %
    $$
    \Alifted_{\Lambda^\strat} = \left\{ P_{\event} ( \GG{\Lambda^\strat} ) \,\, | \,\, \event \in \Sigma\left(\strat\right) \right\}.
    $$
    \end{definition}
%
By leveraging the mixed-product property~\cite{horn2012matrix} of $\otimes$ and by introducing a proper submultiplicative norm, it is possible to prove that $\rho\left(\Alifted_{\Lambda^\strat}\right)= \rho\,(\Aa^{\strat},\GG{\Lambda^\strat})$.
For more details and a formal proof we refer the interested reader to~\cite{xu2020approximation}.

\fix{Maybe include the example from the "longpaper".}

\subsection{Extended weakly hard and JSR properties}
\label{sec:analytic_results}
%
We now provide a general relation between \emph{all} \ewhc{}s in terms of the joint spectral radii, by leveraging the model presented earlier in the paper.

\begin{theorem}[JSR dominance]%
    \label{th:rho_dominance_general}%
    Given two arbitrary \ewhc{}, $\lambda_1^\strat$ and $\lambda_2^\strat$, if $\lambda_2^\strat \preceq \lambda_1^\strat$ then
    \begin{equation*}
        \rho\bigl(\Alifted_{\lambda_2^\strat}\bigr) \leq \rho\bigl(\Alifted_{\lambda_1^\strat}\bigr).
    \end{equation*}

    \begin{proof}
        From Equation~\eqref{jsr}, for a generic \ewhc{} $\lambda^\strat$,
        \begin{equation*}
            \rho\left(\Alifted_{\lambda^\strat}\right) = \lim_{\ell\rightarrow \infty}\rho_\ell\left(\Alifted_{\lambda^\strat}\right), \quad \rho_\ell\left(\Alifted_{\lambda^\strat}\right) = \max_{a \in \sset{\ell}{\lambda^\strat}}\norm{A_{a}}^{1/\ell}.
        \end{equation*}
        Definition~\ref{def:domination} gave us that $\lambda^\strat_2 \preceq \lambda^\strat_1$ if and only if $\sset{}{\lambda^\strat_2} \subseteq \sset{}{\lambda^\strat_1}$.
        Thus, if an arbitrary sequence is in the satisfaction set of $\lambda^\strat_2$, i.e., $b \in \sset{\ell}{ \lambda^\strat_2 }$, this means $b$ also belongs to the satisfaction set of $\lambda^\strat_1$, i.e., $b \in \sset{\ell}{ \lambda^\strat_1 }$.
        The set of all possible $A_{b}$ is thus included in the set of all possible $A_{a},\, a \in \sset{\ell}{ \lambda^\strat_1 }$.
        As a consequence it holds that
        \begin{equation*}
            \max_{b \in \sset{\ell}{\lambda^\strat_2}}\norm{A_{b}}^{1/\ell} \leq
            \max_{a \in \sset{\ell}{\lambda^\strat_1}}\norm{A_{a}}^{1/\ell}, \quad
            \forall \ell\in\mathbb{N}^{>}.
        \end{equation*}
        The theorem follows immediately when $\ell\rightarrow \infty$.
    \end{proof}
\end{theorem}

The intuition that a sequence of control activations containing a large number of deadline misses is less robust than one including fewer errors, is well-established.
However, Theorem~\ref{th:rho_dominance_general} is the first result that provides a clear, analytic, correlation between the control theoretical analysis and real-time implementation.
Primarily, it implies that the constraint dominance from Definition~\ref{def:domination} also carries on to the JSR, giving us a notion of \emph{JSR dominance}.
If stability under a specific \ewhc{} is shown, Theorem~\ref{th:rho_dominance_general} guarantees stability for all \ewhc{} which are harder to satisfy.\footnote{This result also applies to the stability analysis presented in~\cite{Maggio:2020}, giving the user options for which methodology to use.}
We wish to emphasise that the results of Theorem~\ref{th:rho_dominance_general} are strategy independent (as long as $\lambda^\strat_1$ and $\lambda^\strat_2$ use the same strategy $\strat$), further reducing the coupling between the control analysis and implementation approach.
The results are also independent of the dynamics, and apply to open-loop as well as closed-loop systems, including feedforward controllers and any linear feedback control strategy, provided that an appropriate reformulation of matrices $\Aa^\strat$ is given.

Due to the $\overbar{\left< x \right>}^{\strat}$ and $\eanymiss{}{\strat}$ constraints being the two most used models, we derive two Corollaries of Theorem~\ref{th:rho_dominance_general}, highlighting some practical relations between such constraints in terms of the corresponding joint spectral radii.
\begin{corollary}[$\eanymiss{}{\strat}$ dominance]%
    \label{cor:rho_dominance_mk}%
    Given two \ewhc{}, $\lambda^\strat_1 = \overbar{\binom{x}{k_1}}^{\strat}$ and $\lambda^\strat_2 = \overbar{\binom{x}{k_2}}^{\strat}$, if $k_1 \leq k_2$ then
    $$
        \rho\bigl(\Alifted_{\lambda^\strat_2}\bigr) \leq \rho\bigl(\Alifted_{\lambda^\strat_1}\bigr).
    $$
%
    \begin{proof}
        According to~\cite{Wu:2020}, $\overbar{\binom{m}{k_2}}^{\strat}\preceq\overbar{\binom{m}{k_1}}^{\strat}$ if $k_1 \leq k_2$.
        The corollary follows directly from~\cite{Wu:2020} and Theorem~\ref{th:rho_dominance_general}.
    \end{proof}
\end{corollary}
%
\begin{corollary}[$\overbar{\left< x \right>}^{\strat}$ dominance]%
    \label{cor:rho_dominance_cons}%
    Given two \ewhc{}, $\lambda^\strat_1 = \overbar{\left<x\right>}^{\strat}$ and $\lambda^\strat_2 =\eanymiss{}{\strat}$, then
    $$
        \rho\bigl(\Alifted_{\lambda^\strat_2}\bigr) \leq \rho\bigl(\Alifted_{\lambda^\strat_1}\bigr).
    $$
%
    \begin{proof}
        According to~\cite{Maggio:2020} it holds that $\overbar{\left<m\right>}^{\strat} \equiv\overbar{\binom{m}{m+1}}^{\strat}$.
        The corollary follows directly from~\cite{Maggio:2020} and Corollary~\ref{cor:rho_dominance_mk}.
    \end{proof}
\end{corollary}
%
The conclusions drawn from Theorem~\ref{th:rho_dominance_general} are theoretical and its practical applicability depends on the algorithms used to find lower and upper bounds for the JSR value $\rho^{LB}$ and $\rho^{UB}$.
Using these bounds we can determine the stability of the switching system.
However, it is not necessarily true that the upper bounds found using different algorithms follow the ordering presented above.
Still, we can bound the switching stability as
%
$$
\rho^{LB} \bigl( \Alifted_{\lambda^\strat_2} \bigr) \leq \rho \bigl( \Alifted_{\lambda^\strat_2}
\bigr) \leq \rho \bigl( \Alifted_{\lambda^\strat_1} \bigr) \leq \rho^{UB} \bigl(
\Alifted_{\lambda^\strat_1} \bigr).
$$
%
Regardless of the algorithm used to find the bounds, we can generally conclude that if $\rho^{UB} \left( \Alifted_{\lambda^\strat} \right) < 1$ the constrained system is switching stable.
Similarly, if $\rho^{LB} \left( \Alifted_{\lambda^\strat} \right) > 1$ the system is unstable.
Thus, if $\lambda^\strat_2 \preceq \lambda^\strat_1$, where $\rho^{UB} \left( \Alifted_{\lambda^\strat_1} \right) < 1$, we know that the system under $\lambda^\strat_2$ constraints is switching stable.
A similar relation holds for the lower bound.

We now extend the results of Theorem~\ref{th:rho_dominance_general} by relating the joint spectral radius of a single constraint to sets of constraints.
\begin{theorem}%
    \label{th:rho_dominance_set_general}%
    Given an arbitrary \ewhc{} $\lambda^\strat$, it holds that
    $$
        \rho\left(\Alifted_{\Lambda^\strat}\right) \leq \rho \left( \Alifted_{\lambda^\strat} \right) ,\,\, \forall \Lambda^\strat \ni \lambda^\strat.
    $$

    \begin{proof}
        Considering the relationship presented in Equation~\eqref{eq:satisfaction-multi}, for an arbitrary constraint $\lambda^\strat$ and constraint set $\Lambda^\strat$, where $\Lambda^\strat = \{ \Lambda^\strat_1, \lambda^\strat \}$, it holds that
        \begin{equation}
            \sset{\ell}{\Lambda^\strat } = \sset{\ell}{ \Lambda^\strat_1 } \cap \sset{\ell}{ \lambda^\strat } \subseteq \sset{\ell}{ \lambda^\strat }.
        \end{equation}
        %
        Thus, if an arbitrary sequence $b$ is in the satisfaction set of $\Lambda^\strat$ it also belongs to the satisfaction set of $\lambda^\strat$, i.e., $b \in \sset{\ell}{ \Lambda^\strat } \Rightarrow b \in \sset{\ell}{ \lambda^\strat }$.
        The set of all possible $A_{b}$ is thus included in the set of all possible $A_{a},\, a \in \sset{\ell}{ \lambda^\strat }$.
        As a consequence it holds that
        \begin{equation*}
            \max_{b \in \sset{\ell}{ \Lambda^\strat } } \norm{A_{b}}^{1/\ell} \leq
            \max_{a \in \sset{\ell}{ \lambda^\strat } } \norm{A_{a}}^{1/\ell}, \quad
            \forall \ell\in\mathbb{N}^{>}.
        \end{equation*}
        The theorem follows immediately when $\ell\rightarrow \infty$.
    \end{proof}
\end{theorem}

Following the same intuition as Theorem~\ref{th:rho_dominance_general}, the more we restrict the execution pattern of the task constrained by $\lambda^\strat$, the lower the JSR will be.
In the case of constraint sets, the notion of JSR dominance is not as evident as it was for a single constraint.
However, the JSR dominance extension to sets of constraints further improves the relation between constraint specification and control verification.

One of the most important implications of Theorem~\ref{th:rho_dominance_set_general} is the practical significance it has on the switching stability of the system.
Specifically, enforcing tighter \ewhc{} to a stable system will \emph{never} destabilise it.
The result is formally given by the following corollary.
\begin{corollary}%
    \label{cor:rho_dominance_set}%
    Given an arbitrary \ewhc{}, $\lambda^\strat$, if $\rho \left( \Alifted_{\lambda^\strat} \right) < 1$ then
    $$\rho\left(\Alifted_{\Lambda^\strat}\right) < 1 ,\,\, \forall \Lambda^\strat \ni \lambda^\strat. $$

    \begin{proof}
        The corollary follows immediately from Theorem~\ref{th:rho_dominance_set_general} when $\rho \left( \Alifted_{\lambda^\strat} \right) < 1$.
    \end{proof}
\end{corollary}
