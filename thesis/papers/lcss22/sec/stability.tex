Using the alphabet $\Sigma\,(\strat)$ and the chosen actuator mode (i.e., \tZ{}ing, or \tH{}ing the previous value), we compute the closed-loop behaviour of the controlled system.
We identify one matrix for each dynamics corresponding to an interval $\pi_k$ associated by $c \in \Sigma\,( \strat )$, building the set $\Aa^\strat$.

\textbf{\tK{}: }%
%
Defining $\tilde x_k^{\,\Kill} = [ x_k^\T{}\,\, z_k^\T{}\,\, u_k^\T{} ]^\T{}$ as the closed-loop state vector,
we compute the discrete time closed-loop system dynamics $\clmat{}^{\,\Kill}_{\cH}$, corresponding to the character $\cH$:
\begin{equation*}
    \tilde x_{k+1}^{\,\Kill} = \clmat{}^{\,\Kill}_{\cH}\,\tilde x_k^{\,\Kill}, \quad\;
     \clmat{}^{\,\Kill}_{\cH} = \begin{bmatrix}
        \Ap       & 0    & \Bp       \\
        -\Bc\Cp   & \Ac  & -\Bc\Dp   \\
        -\Dc\Cp   & \Cc  & -\Dc\Dp   \\
    \end{bmatrix}.
\end{equation*}
%
For the case of $\cM$, the controller execution terminates prematurely and its states are not updated ($z_{k+1} = z_k$).
Therefore, depending on the actuation mode (\tZ{} or \tH{}), the controller output is either zeroed ($u_{k+1} = 0$) or held ($u_{k+1} = u_k$).
The resulting closed-loop system in state-space form is denoted with $\clmat{}^{\,\Kill}_{\cM}$:
\begin{equation*}
    \tilde x_{k+1}^{\,\Kill} = \clmat{}^{\,\Kill}_{\cM}\,\tilde x_k^{\,\Kill}, \quad
   \clmat{}^{\,\Kill}_{\cM} = \begin{bmatrix}
        \Ap & 0  & \Bp \\
        0   & I  & 0   \\
        0   & 0  & \Delta
    \end{bmatrix}.
\end{equation*}
Here, $\Delta = I$ (identity matrix) if the control signal is held and $\Delta = 0$ if zeroed.
The set of dynamic matrices under the \tK{} strategy is then $\Aa^{\,\Kill}=\{\clmat{}^{\,\Kill}_{\cH},\clmat{}^{\,\Kill}_{\cM}\}$.

\textbf{\tS{}: }%
%
For the \tS{} strategy, we introduce two additional states $\hat x_k$ and $\hat u_k$ storing the old values of $x_k$ and $u_k$ while the controller awaits an update.
The resulting state vector then becomes $\tilde x_k^{\,\Skip} = [ x_k^\T{}\,\,z_k^\T{}\,\, u_k^\T{}\,\, \hat{x}_k^\T{}\,\, \hat{u}_k^\T{} ]^\T{}$.
When $\pi_k$ is associated to $\cH$, the two additional states mirror the behaviour of the states of which they are storing data.
The resulting closed-loop system is described using $\clmat{}^{\,\Skip}_{\cH}$:
\begin{equation*}
    \setlength\arraycolsep{3pt}
    \tilde x_{k+1}^{\,\Skip} = \clmat{}^{\,\Skip}_{\cH}\,\tilde x_k^{\,\Skip}, \quad
    \clmat{}^{\,\Skip}_{\cH} = \begin{bmatrix}
        \Ap       & 0    & \Bp      & 0 & 0 \\
        -\Bc\Cp   & \Ac  & -\Bc\Dp  & 0 & 0 \\
        -\Dc\Cp   & \Cc  & -\Dc\Dp  & 0 & 0 \\
        \Ap       & 0    & \Bp      & 0 & 0 \\
        -\Dc\Cp   & \Cc  & -\Dc\Dp  & 0 & 0 \\
    \end{bmatrix}.
\end{equation*}
%
For the case of $\cM$ in $\pi_k$, $\hat x_k$ and $\hat u_k$ maintain their previous values. The
resulting closed-loop is described by $\clmat{}^{\,\Skip}_{\cM}$:
%
\begin{equation*}
    \setlength\arraycolsep{3pt}
    \tilde x_{k+1}^{\,\Skip} = \clmat{}^{\,\Skip}_{\cM}\,\tilde x_k^{\,\Skip}, \quad
    \clmat{}^{\,\Skip}_{\cM}=\begin{bmatrix}
        \Ap & 0  & \Bp & 0 & 0 \\
        0   & I  & 0   & 0 & 0 \\
        0   & 0  & \Delta   & 0 & 0 \\
        0   & 0  & 0   & I & 0 \\
        0   & 0  & 0   & 0 & I \\
	\end{bmatrix}.
\end{equation*}
%
Finally, for the case of $\cR$, the new control command is calculated using the values stored in $\hat x_k$ and $\hat u_k$.
The resulting closed-loop system is described by $\clmat{}^{\,\Skip}_{\cR}$:
%
\begin{equation*}
    \setlength\arraycolsep{3pt}
    \tilde x_{k+1}^{\,\Skip} = \clmat{}^{\,\Skip}_{\cR} \, \tilde x_{k}^{\,\Skip},\quad
    \clmat{}^{\,\Skip}_{\cR} = \begin{bmatrix}
        \Ap & 0    & \Bp & 0       & 0 \\
        0   & \Ac  & 0   & -\Bc\Cp & -\Bc\Dp \\
        0   & \Cc  & 0   & -\Dc\Cp & -\Dc\Dp \\
        \Ap & 0    & \Bp & 0       & 0 \\
        0   & \Cc  & 0   & -\Dc\Cp & -\Dc\Dp \\
    \end{bmatrix}.
\end{equation*}
%
The resulting set of matrices under the \tS{} strategy is then defined as $\Aa^{\,\Skip}=\{\clmat{}^{\,\Skip}_{\cH},\clmat{}^{\,\Skip}_{\cM},\clmat{}^{\,\Skip}_{\cR}\}$.


\subsection{Kronecker lifted switching system}%
\label{sec:system_dynamics}

Combining the set of system dynamics $\Aa^\strat$ with the associated automaton $\GG{\Lambda^\strat}$, we seek to obtain an equivalent system model based on Kronecker lifting, characterized by a set of matrices denoted by $\Alifted_{\Lambda^\strat}$ and behaving as an \emph{arbitrary switching system}, such that $\rho\,(\Alifted_{\Lambda^\strat})= \rho\,(\Aa^{\strat},\GG{\Lambda^\strat})$.
In this way, powerful algorithms applicable to arbitrary switching system~\cite{vankeerberghen2014jsr,sparsejsr} can be used to find tight stability bounds.
%
We build upon the Kronecker lifting approach of~\cite{xu2020approximation}.
Leveraging the vector $q_k$ of Definition~\ref{def:qt}, we introduce the \emph{lifted discrete-time state} $\xi_k \in \R^{n\cdot n_V}$, defined as $\xi_k = q_k\otimes \tilde{x}_k$, where $n_V = \abs{\VV{\Lambda^\strat}}$ and $\otimes$ is the Kronecker product.
By construction, $\xi_k$ is a vector composed of $n_V$ blocks of size $n$, where at most one block is equal to $\tilde{x}_k$ and all other blocks are equal to the $0$ vector.
%
Then, we build a set of lifted matrices $L_{c} ( \GG{\Lambda^\strat} ) \in \R^{n\cdot n_V\times n\cdot n_V}$, which incorporates both the system dynamics and the possible transitions given a certain outcome $c\in\Sigma\,(\strat)$:
%
\begin{equation}\label{eq:lifted_matrix}
    L_{c} ( \GG{\Lambda^\strat} ) = F_{c} ( \GG{\Lambda^\strat} )\otimes \clmat^\strat_c, \quad c \in \Sigma \,( \strat ).
\end{equation}
%
The lifted dynamics of the closed loop system then become $ \xi_{k+1} = L_{c} ( \GG{\Lambda^\strat} )\cdot\xi_k.  $
Formally, we obtain a system composed of a set of switching dynamic matrices, $\Alifted_{\Lambda^\strat}$.
%
\begin{definition}[Lifted switching set $\Alifted_{\Lambda^\strat}$]%
    \label{def:switching_set}%
    Given a set of dynamic matrices $\Aa^{\strat}$ and an automaton $\GG{\Lambda^\strat}$, the switching set $\Alifted_{\Lambda^\strat}$ is defined as:
    $$
        \Alifted_{\Lambda^\strat} = \{ L_{c} ( \GG{\Lambda^\strat} ) \,\, | \,\, c \in \Sigma\,(\strat) \}.
    $$
\end{definition}%
%
Leveraging the mixed-product property of $\otimes$ and introducing a proper submultiplicative norm, it is possible to prove that $\rho\,(\Alifted_{\Lambda^\strat})= \rho\,(\Aa^{\strat},\GG{\Lambda^\strat})$.
For more details and a formal proof we refer the interested reader to~\cite{xu2020approximation}.

\subsection{Extended weakly hard and JSR properties}
\label{sec:analytic_results}
We now provide a general relation between \emph{all} \ewhc{}s in terms of the joint spectral radii.
%
\begin{theorem}[JSR dominance]
    \label{th:rho_dominance_general}
    Given $\lambda_1^\strat$ and $\lambda_2^\strat$ as arbitrary \ewhc{}s, if $\lambda_2^\strat \preceq \lambda_1^\strat$ then
    $$
        \rho\,(\Alifted_{\lambda_2^\strat}) \leq \rho\,(\Alifted_{\lambda_1^\strat}).
    $$
%
    \begin{proof}
        From Equation~\eqref{jsr}, for a generic \ewhc{} $\lambda^\strat$,
        \begin{equation*}
            \rho\,(\Alifted_{\lambda^\strat}) = \lim_{\ell\rightarrow \infty} \rho_\ell \,(\Alifted_{\lambda^\strat}), \quad 
            \rho_\ell\,(\Alifted_{\lambda^\strat}) = \max_{a \in \sset{\ell}{\lambda^\strat}}\norm{\clmat_{a}}^{1/\ell}.
        \end{equation*}
        Definition~\ref{def:domination} gave us that $\lambda^\strat_2 \preceq \lambda^\strat_1$ iff $\sset{}{\lambda^\strat_2} \subseteq \sset{}{\lambda^\strat_1}$.
        Thus, if for a word $b$ it holds that $b \in \sset{\ell}{\lambda^\strat_2}$, then it also holds that $b \in \sset{\ell}{\lambda^\strat_1}$.
        The set of all possible $\clmat_{b}$ is thus included in the set of all possible $\clmat_{a},\, a \in \sset{\ell}{\lambda^\strat_1}$, thus:
        \begin{equation*}
            \max_{b \in \sset{\ell}{\lambda^\strat_2}}\norm{\clmat_{b}}^{1/\ell} \leq
            \max_{a \in \sset{\ell}{\lambda^\strat_1}}\norm{\clmat_{a}}^{1/\ell}, \quad \forall \ell \in \N_{>}.
        \end{equation*}
        The theorem follows immediately when $\ell\rightarrow \infty$.
    \end{proof}
\end{theorem}
%
Theorem~\ref{th:rho_dominance_general} is the first result that provides an analytic, correlation between the control theoretical analysis and real-time implementation.
Primarily, it implies that the constraint dominance from Definition~\ref{def:domination} also carries on to the JSR, giving us a notion of \emph{JSR dominance}.
The results of Theorem~\ref{th:rho_dominance_general} are strategy-independent, further reducing the coupling between the control analysis and real-time implementation, and are also independent of the controlled system's dynamics.

Two Corollaries of Theorem~\ref{th:rho_dominance_general} are derived for the commonly used models $\erowmiss{}{\strat}$ and $\eanymiss{}{\strat}$, highlighting some practical relations between such constraints.
\begin{corollary}[$\eanymiss{}{\strat}$ dominance]%
    \label{cor:rho_dominance_mk}%
    Given $\lambda^\strat_1 = \overbar{\binom{x}{k_1}}^{\strat}$ and $\lambda^\strat_2 = \overbar{\binom{x}{k_2}}^{\strat}$, if $k_1 \leq k_2$ then
    $$
        \rho\,(\Alifted_{\lambda^\strat_2}) \leq \rho\,(\Alifted_{\lambda^\strat_1}).
    $$
\end{corollary}
\begin{corollary}[$\erowmiss{}{\strat}$ dominance]%
    \label{cor:rho_dominance_cons}%
    Given $\lambda^\strat_1 = \erowmiss{}{\strat}$ and $\lambda^\strat_2 = \eanymiss{}{\strat}$, then
    $$
        \rho\,(\Alifted_{\lambda^\strat_2}) \leq \rho\,(\Alifted_{\lambda^\strat_1}).
    $$
\end{corollary}
%
The conclusions drawn from Theorem~\ref{th:rho_dominance_general} are theoretical, but its practical applicability lies in the algorithm used to find $\rho^{LB}$ and $\rho^{UB}$, i.e., lower and upper bounds for the JSR value.
Using these bounds we can determine the stability of the corresponding switching systems, as follows:
%
$$
\rho^{LB} \,( \Alifted_{\lambda^\strat_2} ) \leq \rho \,( \Alifted_{\lambda^\strat_2} ) \leq \rho \,( \Alifted_{\lambda^\strat_1} ) \leq \rho^{UB} \,( \Alifted_{\lambda^\strat_1} ).
$$
%
Regardless of the algorithm used to find the bounds, if $\lambda^\strat_2 \preceq \lambda^\strat_1$ and $\rho^{UB} ( \Alifted_{\lambda^\strat_1} ) < 1$, the system under $\lambda^\strat_2$ is switching stable.
A similar relation holds for the lower bound.

Theorem~\ref{th:rho_dominance_general} can be further extended by relating the joint spectral radius of a single constraint to sets of constraints.
\begin{theorem}%
    \label{th:rho_dominance_set_general}%
    Given an arbitrary \ewhc{} $\lambda^\strat$, it holds that
    $$
        \rho\,(\Alifted_{\Lambda^\strat}) \leq \rho \,( \Alifted_{\lambda^\strat} ) ,\quad \forall \Lambda^\strat \ni \lambda^\strat.
    $$
    \begin{proof}
        For an arbitrary \ewhc{} set $\Lambda^\strat = \{\lambda^\strat_1, \dots, \lambda^\strat_N\}$, its satisfaction set is
        $$
            \sset{\ell}{\Lambda^\strat} = \bigcap_{i \in \{1,\dots, N\}} \sset{\ell}{\lambda^\strat_i}.
        $$
        Thus, for any $\lambda^\strat_i \in \Lambda^\strat$ it holds that 
        $$
            \sset{\ell}{ \Lambda^\strat } \subseteq \sset{\ell}{\lambda^\strat}.
        $$
        If a word $b$ is in $\sset{\ell}{\Lambda^\strat}$ it also belongs to $\sset{\ell}{\lambda^\strat}$. 
        The set of all possible $\clmat_{b}$ is thus included in the set of all possible $\clmat_{a},\, a \in \sset{\ell}{\lambda^\strat}$.
        As a consequence it holds that
        \begin{equation*}
            \max_{b \in \sset{\ell}{\Lambda^\strat}} \norm{\clmat_{b}}^{1/\ell} \leq
            \max_{a \in \sset{\ell}{\lambda^\strat}} \norm{\clmat_{a}}^{1/\ell}, \quad
            \forall \ell \in \N_{>}.
        \end{equation*}
        The theorem follows immediately when $\ell\rightarrow \infty$.
    \end{proof}
\end{theorem}

As in Theorem~\ref{th:rho_dominance_general}, the more we restrict the execution pattern of the control task with sets of constraints, the lower its JSR will be.
%
Theorem~\ref{th:rho_dominance_set_general} delivers the practical insight that enforcing tighter \ewhc{} to a stable system will \emph{never} destabilise it, as formally stated in the following corollary.
\begin{corollary}%
    \label{cor:rho_dominance_set}%
    Given an arbitrary \ewhc{} $\lambda^\strat$, if $\rho \,( \Alifted_{\lambda^\strat} ) < 1$ then
    $$
        \rho\,(\Alifted_{\Lambda^\strat}) < 1 ,\,\, \forall \Lambda^\strat \ni \lambda^\strat.
    $$
\end{corollary}
