Any \ewhc{}, as presented in Definition~\ref{def:new-mk}, can be systematically represented using an \emph{automaton}.
In this paper we build upon the \tool{} automaton model presented in~\cite{Vreman:2022}.
Here, a (minimal) automaton $\GG{\lambda^\strat} = \left(\VV{\lambda^\strat}, \EE{\lambda^\strat}\right)$ associated to $\lambda^\strat$ consists of a set of vertices ($\VV{\lambda^\strat}$) and a set of directed labeled edges ($\EE{\lambda^\strat}$). 
Each vertex $v_i \in \VV{\lambda^\strat}$ corresponds to a word of outcomes of the extended weakly-hard task executions. 
Trivially, there exists no vertices for words that do not satisfy the \ewhc{}.
A directed labeled edge $e_{i, j} = \left(v_i, v_j, c \right) \in \EE{\lambda^\strat}$ (also denoted \emph{transition}) connects two vertices iff the outcome $c \in \Sigma\left(\strat\right)$ -- the edge's label -- appended to the tail vertex's word representation ($v_i$) would result in the word equivalent to the one of the head vertex ($v_j$).
Thus, a random walk in the automaton corresponds to a random word satisfying the \ewhc{}.
In particular, all the walks in the automaton corresponds to \emph{all} words in $\sset{}{\lambda^{\strat}}$.

Since the \tool{} automaton model only uses the binary alphabet $\Sigma = \left\{\cM, \cH\right\}$, we require the additional character $\cR$ to handle the \tS{} strategy properly.
Recall that both a hit ($\cH$) and a recovery ($\cR$) are considered job completions.
Thus, for the \tS{} strategy, we post-process the automaton by enforcing that Rule~\ref{rule:R} is honoured and that the corresponding transitions are correct, i.e., switching the labels on some edges from $\cH$ to $\cR$. 
We emphasise that despite the extended automaton model appear similar for the \tK{} and \tS{} strategies, the differing transitions of the two automata significantly affect the corresponding closed-loop systems, as will be clear in Section~\ref{sec:stability}.

The \tool{} automaton model also allows for the case where the task $\tau$ is subject to a set of multiple constraints.
Since the stability analysis presented in this paper is invariant to the type (and amount) of the constraints acting on the control task $\tau$, we henceforth say that $\tau$ is subject to a set of \ewhc{} $\Lambda^\strat$ (unless stated otherwise).


Extracting all transitions in $\EE{\Lambda^\strat}$ corresponding to a character $c \in \Sigma\left(\strat\right)$ yields what is generally known as a \emph{directed adjacency matrix}~\cite{xu2012matrix}, denoted here as a \emph{transition matrix}.
\begin{definition}[Transition matrix]%
    \label{def:transition}%
    Given an automaton $\GG{\Lambda^\strat}$, the \emph{transition matrix} $F_{c} ( \GG{\Lambda^\strat} ) \in \R^{n_V\times n_V}$, with $n_V =\abs{\VV{\Lambda^\strat}} $ and $c \in \Sigma\left(\strat\right)$, is computed as $F_{c} ( \GG{\Lambda^\strat} ) = \{f_{i,j}(c)\}$ with
    %
    \begin{equation*}
        f_{i,j}\left(c\right)=
        \begin{cases}
            1, &\text{ if } \exists \, e_{i, j} = (v_i,v_j,c) \in \EE{\Lambda^\strat} \\
            0, &\text{ otherwise.}
        \end{cases}
        \end{equation*}%
\end{definition}
%
Since \emph{at most one} successor exists from each vertex with a transition labeled with $c \in \Sigma\left(\strat\right)$, matrix $F_c$ will have a column sum of either 1 or 0.
We now introduce a vector $q_t \in \R^{n_V}$ called \emph{G-state}, with $n_V =\abs{\VV{\Lambda^\strat}} $, representing the state of the given automaton $\GG{\Lambda^\strat}$ at interval $\pi_t$.
\begin{definition}[G-state $q_t$]%
    \label{def:qt}%
    Given an automaton $\GG{\Lambda^\strat}$ and a word $w \in \Sigma\left( \strat \right)^N$, $w=\langle c_1,c_2,\dots,c_N \rangle$, for $k = \abs{v},\,\, v\in\VV{\Lambda^\strat}$, we define $q_t\in \R^{n_V}$, where the $i$-th element $q_{t,i}$ is:
    \begin{equation*}
        q_{t,i}=
        \begin{cases}
            1, &\text{ if } \seq{c_{t-k},\dots,c_{t-1}} \equiv v_i \in \VV{\Lambda^\strat} \\
            0, &\text{otherwise}.
        \end{cases}
    \end{equation*}
\end{definition}
The G-state $q_t$ is the vector representation of the vertex \emph{left} at step $t$: here, $q_t=0$ means that the transition at step $t-1$ was infeasible for the automaton. 
Given an arbitrary word $w=\seq{c_1,\dots,c_t,\dots}$, the G-state dynamics is defined as $q_{t+1} = F_{c} ( \GG{\Lambda^\strat} )\cdot q_t$, and the following property holds~\cite{xu2012matrix}.
\begin{lemma}[Infeasible sequence]%
    \label{cor:Fseqnotinlambda}%
    If $w \notin \sset{N}{\Lambda^\strat}$, then $F_{w} ( \GG{\Lambda^\strat} ) = F_{c_N} ( \GG{\Lambda^\strat} )\cdots F_{c_2} ( \GG{\Lambda^\strat} )\cdot F_{c_1} ( \GG{\Lambda^\strat} ) = 0$
\end{lemma}
Thus, if $q_t=0$ for an arbitrary $t$, then $q_{t'}=0$ for $t' \geq t$.
