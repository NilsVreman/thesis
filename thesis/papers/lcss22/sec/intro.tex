Robustness is an essential concern in the design of control systems; they must be able to reliably handle nonlinear effects, unmodeled dynamics and noise, as well as delays in signal transmissions and dropped packets.
A lesser known problem concerns the assessment of robustness to \emph{computational issues} when controllers are implemented as periodic tasks in cheap embedded platforms.
Such tasks are expected to execute with real-time guarantees, i.e., their execution must be completed before a well-defined \emph{deadline}, when the control output must be sent to the actuator.
However, it is common in practice~\cite{akesson2020empirical} that tasks do not always complete within their deadline, causing what is called a \emph{deadline miss}.
This may be caused by delays in computation and memory accesses, transient overloads, bugs and other issues.

A popular model to describe real-time systems allowing deadline misses is the \emph{weakly-hard} model~\cite{Bernat:2001}. 
Weakly-hard tasks feature constraints defining a maximum number of deadlines that can be missed (alternatively, a minimum number to be satisfied) in a given number of consecutive periods.
This model is also the focus of this work.
To analyse the effects on the controlled plant, it is necessary to specify also \emph{what happens when the miss is experienced}, both in terms of changes to the control signal and of actions taken to deal with the failed computation~\cite{Pazzaglia:2019}.
An instance that experiences a deadline miss can be allowed to continue executing until completion (and possibly used later), while in other applications it is stopped and discarded instead.

There is however a mismatch between the guarantees that can be obtained for real-time tasks and platforms~\cite{Ernst:2015,choi2019job}, and the analysis available for \emph{control} tasks under the weakly-hard model.
Fewer works deal with \emph{stability} analysis of weakly-hard real-time control tasks, often targeting specific use-cases. 
For instance, the analysis in~\cite{Maggio:2020} is limited to constraints specifying a maximum number of \emph{consecutive} deadline misses.
The results in \cite{Linsenmayer:2017,linsenmayer2020linear}, obtained for networked linear control systems having packet dropouts bounded using the weakly-hard model, can not be generalised for \emph{late completions} or \emph{sets} of weakly-hard constraints.
The authors of~\cite{liang2019security,liang2020leveraging} studied safety guarantees of weakly-hard controllers, considering a miss as a discarded computation with a known periodic pattern.
%
In \cite{huang2020saw, huang2019formal}, an over-approximation-based approach is proposed to check the safety of nonlinear weakly-hard systems, where misses are treated as discarded computations and the actuator holds its previous value.
Convergence rates (providing sufficient stability guarantees) are analysed in~\cite{Gaukler:2019a}.
A Lyapunov-based stability analysis of nonlinear weakly-hard systems is studied in~\cite{hertneck2021efficient}, with deadline misses treated as packet dropouts.
However, the state-of-the-art listed above lack generalisability to more expressive real-time implementations, such as different deadline miss models or handling strategies.

This paper aims at filling the gap, by providing a stability analysis that can be applied to a class of generic weakly-hard models and deadline miss handling strategies.
First, we formally extend the weakly-hard model to explicitly consider the strategy used to handle the miss events. 
By leveraging an automaton representation of the sequences allowed by (a set of) extended weakly-hard constraints, we use Kronecker lifting and the joint spectral radius to properly express its stability conditions.
Using the concept of constraint dominance, we prove analytic bounds on the stability of a weakly-hard system with respect to \emph{less dominant} constraints.
Finally, we analyse the stability of the resulting closed-loop systems using \code{SparseJSR}~\cite{sparsejsr}, which exploits the sparsity pattern that naturally arises in the Kronecker lifted representation.
The proposed analysis calls for modularity and separation of concern, and can be a useful tool to decouple the constraint specification and the control verification.
%, the embedded system designer can extract a set of constraints to be used in the design phase, and the control engineer can verify that the proposed constraints satisfy all control requirements. 
