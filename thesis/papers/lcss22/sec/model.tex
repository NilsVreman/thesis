To provide a comprehensive analysis framework, we need to examine what occurs in each time interval $(\pi_k)_{k \in \N_{\geq}}$, with $\pi_k = [a_0 + k\cdot \Ts, a_0 + (k+1)\cdot \Ts)$.
In this context, an extension of the weakly-hard model is required to account for the given deadline miss handling strategy, denoted with the symbol $\strat$.
%
\begin{definition}[Extended weakly-hard model $\tau \vdash \lambda^{\strat}$]%
    \label{def:new-mk}%
    A task $\tau$ may satisfy any combination of the four \emph{extended weakly-hard constraints} (\ewhc{}) $\lambda^{\strat}$:
    \begin{enumerate}[label=(\roman*)]
        \item $\tau \vdash \elcssanymiss{}{\strat}$: in any window of $\ell$ consecutive jobs, at most $x$ intervals lack a job completion;
        \item $\tau \vdash \elcssanyhit{}{\strat}$:  in any window of $\ell$ consecutive jobs, at least $x$ intervals have a job completion;
        \item $\tau \vdash \elcssrowmiss{}{\strat}$: in any window of $\ell$ consecutive jobs, at most $x$ \emph{consecutive} intervals lack a job completion;
        \item $\tau \vdash \elcssrowhit{}{\strat}$: in any window of $\ell$ consecutive jobs, at least $x$ \emph{consecutive} intervals have a job completion
    \end{enumerate}
    with $x\in \N_{\geq}$, $\ell \in \N_{>}$, and $x\leq \ell$, while using strategy $\strat$ to handle potential deadline misses.
\end{definition}
%
The definition above differs from the original weakly-hard model of~\cite{Bernat:2001}, since
\begin{enumerate*}[label=(\roman*)]
    \item it explicitly introduces the handling strategy $\strat$; and
    \item it focuses on the presence of a new control command at the end of each time interval $\pi_k$, instead of checking the deadline miss events, which guarantees its applicability also for strategies different than \tK{}.
\end{enumerate*}

We now require an expressive alphabet $\Sigma\left(\strat\right)$ to characterize the behaviour of task $\tau$ in each possible time interval.
For both \tK{} and \tS{} strategies, each interval $\pi_k$ contains at most one activated and one completed job.
This restricts the possible behaviours to three cases:
\begin{enumerate}[label=(\roman*)]
    \item a time interval in which the same job is both released and completed is denoted by $\cH$ (\emph{hit});
    \item a time interval in which no job is completed is denoted by $\cM$ (\emph{miss});
    \item a time interval in which no job is released, but a job (released in a previous interval) is completed, is denoted by $\cR$ (\emph{recovery}).
\end{enumerate}
%
By checking all unique combinations of job activations and completions in each interval, we obtain the alphabets for \tK{} and \tS{} as $\Sigma\left(\tK{}\right) = \{ \cM, \cH \}$ and $\Sigma\left(\tS{}\right) = \{ \cM, \cH, \cR \}$, respectively.
The recovery character $\cR$ is used in the \tS{} alphabet to identify the late \emph{completion} of a job.
As a consequence, $\cR$ is treated equivalently to $\cH$ when verifying the extended weakly hard constraints (\ewhc{}).


The algebra presented in Section~\ref{ssec:whalgebra} is extended to the new alphabet.
We assign a character of the alphabet $\Sigma\left(\strat\right)$ to each interval $\pi_k$.
A word $w = \seq{c_1,c_2,\dots,c_N}$ is used to represent a sequence of $N$ outcomes for task $\tau$, with $c_k \in \Sigma\left(\strat\right)$ representing the outcome associated to the interval $\pi_k$. 
To enforce only feasible sequences, we introduce an order constraint for the $\cR$ character with the following Rule.
%
\begin{rule_}[Outcome ordering]%
    \label{rule:R}%
    For any word $w \in \Sigma\left(\tS{}\right)^N$, $\cR$ may only directly follow $\cM$, or be the initial element of the word.
\end{rule_}

The extended weakly-hard model also inherits all the properties of the original weakly-hard model.
In particular, the satisfaction set of $\lambda^\strat$ can be defined for $N\geq 1$ as $\sset{N}{\lambda^{\strat}} = \{ w \in \Sigma\left(\strat\right)^N \mid w \vdash \lambda^{\strat} \}$, and the constraint domination still holds as $\lambda^{\strat}_{i} \preceq \lambda^{\strat}_{j}$ if $\sset{}{\lambda^{\strat}_{i}} \subseteq \sset{}{\lambda^{\strat}_{j}}$.
