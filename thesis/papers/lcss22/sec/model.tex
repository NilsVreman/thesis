An important point raised in previous works~\cite{Pazzaglia:2018, Pazzaglia:2019, Maggio:2020, Vreman:2021}, is that the original weakly-hard model is not sufficiently descriptive to model the dynamics of a control task executing on an embedded platform.
The missing piece of the puzzle is represented by the strategy used to handle the control task's deadline misses, which directly affects the actual pattern of the control signal.
For instance, under the Kill strategy, a missed deadline represents a job that will never be completed;
in the case of the Skip-Next strategy, it represents a job that will complete \emph{later}.
%
This Section presents our proposal to extend the concepts related to weakly-hard constraints (Section~\ref{ssec:whalgebra}) to also include the deadline handling strategy. 

To provide a comprehensive analysis framework, we need to examine what occurs in each time interval $(\pi_i)_{i \in \N}$, with $\pi_i = [a_0 + i\cdot T, a_0 + (i+1)\cdot T)$. 
In fact, depending on the strategy that is used to handle deadline misses, the activation rate of jobs may be decoupled by the nominal periodic pattern.
In this context, the alphabet $\Sigma$ introduced in Definition~\ref{def:basic-alphabet} and the weakly-hard model need to be extended to account for the specific deadline handling procedure -- denoted hereafter with the symbol $\strat$.
%

\begin{definition}[Extended Weakly Hard Task $\tau \vdash \lambda^{\strat}$]%
    \label{def:new-mk}%
    An extended weakly-hard task $\tau$ satisfies any combination of these four constraints:
    \begin{enumerate}[label=(\roman*)]
        \item $\tau \vdash \overbar{\binom{m}{k}}^{\strat}$: at most $m$ intervals lack a job completion, in any window of $k$ consecutive jobs;
        \item $\tau \vdash \binom{h}{\!\:\!\:k\!\:\!\:}^{\strat}$: at least $h$ intervals contain a job completion, in any window of $k$ consecutive jobs;
        \item $\tau \vdash \overbar{\genfrac{<}{>}{0pt}{}{m}{\!\:\!\:k\!\:\!\:}}^{\strat}$: at most $m$ \emph{consecutive} intervals lack a job completion, in any window of $k$ consecutive jobs; and
        \item $\tau \vdash \genfrac{<}{>}{0pt}{}{h}{\!\:\!\:k\!\:\!\:}^{\strat}$: at least $h$ \emph{consecutive} intervals contain a job completion, in any window of $k$ consecutive jobs,
    \end{enumerate}
    with $m,h\in \N$, $k \in \N\setminus \left\{ 0 \right\}$, $m\leq k$, and $h\leq k$, while using strategy $\strat$ to handle potential deadline misses.
\end{definition}

The definition above differs from Definition~\ref{def:weakly-hard} in two points: firstly, it focuses on the presence of a new control command at the end of each time interval $\pi_i$, instead of checking the outcome of a job; secondly, it explicitly introduces the handling strategy $\strat$.
%
The dependency on the strategy $\strat$ is required by the necessity of introducing a more expressive alphabet $\Sigma\left(\strat\right)$ to characterise the behaviour of task $\tau$ in each possible time interval.
For both the Kill and Skip-Next strategies, each interval $\pi_i$ contains at most one activated and one completed job.
This restricts the possible behaviours to three cases, defined by the following symbols:
\begin{enumerate}[label=(\roman*)]
    \item a time interval in which the same job is both released and completed is denoted by $\cH$ (\emph{hit});
    \item a time interval in which either one or zero jobs are released, but no job is completed is denoted by $\cM$ (\emph{miss});
    \item a time interval in which no job is released, but a job (released in a previous interval) is completed, is denoted by $\cR$ (\emph{recovery}).
\end{enumerate}
An alphabet can thus be created for any arbitrary handling strategy by checking all unique combinations of job activations and completions in each interval. 
In the case of Kill and Skip-Next we obtain $\Sigma\left(\text{Kill}\right) = \{ \cM, \cH \}$ and $\Sigma\left(\text{Skip-Next}\right) = \{ \cM, \cH, \cR \}$.
Trivially, $\Sigma\left(\text{Kill}\right)$ uses the basic alphabet from Definition~\ref{def:basic-alphabet}.
The recovery character $\cR$ is used in the Skip-Next alphabet to identify the late \emph{completion} of a job.
As a consequence, $\cR$ is treated equivalently to $\cH$ when checking the extended weakly hard constraints (\ewhc{}).

We can now extend the algebra presented in Section~\ref{ssec:whalgebra} to the new alphabet.
We assign a character of the alphabet $\Sigma\left(\strat\right)$ to each interval $\pi_i$.
A word $\aword = \{\event_1,\event_2,\dots,\event_N \} \in \Sigma\left(\strat\right)^N$ is used to represent a sequence of $N$ outcomes for task $\tau$, with $\event_i$ representing the outcome associated to the interval $\pi_i$. 
Without loss of generality, we always consider ideal startup conditions, i.e., $\event_t=\cH, \, \forall t \leq 0$.

In principle, the word $\aword$ could be any combination of characters in $\Sigma\left(\strat\right)$.
However, we are only interested in analysing the set of \emph{feasible sequences} of the task $\tau$.
To this end, we introduce an order constraint for the $\cR$ character.
%
\begin{rule_}%
    \label{rule:R}%
    For any arbitrary word $\aword \in \Sigma\left(\text{Skip-Next}\right)^N$, $\cR$ may only directly follow $\cM$, or be the initial element of the word.
\end{rule_}
%
The rule is trivial when noticing that a recovery $\cR$ at interval $\pi_i$ corresponds to the completion of a job that was activated in an interval $\pi_j$ with $j<i$, that is not completed yet.
Hereafter, we will consider each arbitrary sequence $\aword\in \Sigma\left(\text{Skip-Next}\right)^N$ to always satisfy Rule~\ref{rule:R}.

The extended weakly hard model also inherits all the properties of the original weakly hard model.
In particular, the satisfaction set of $\lambda^\strat$ can be defined for $N\geq 1$ as $\sset{N}{\lambda^{\strat}} = \{ \aword \in \Sigma\left(\strat\right)^N \mid \aword \vdash \lambda^{\strat} \}$, and the constraint domination still holds as $\lambda^{\strat}_{i} \preceq \lambda^{\strat}_{j}$ if $\sset{}{\lambda^{\strat}_{i} } \subseteq \sset{}{\lambda^{\strat}_{j}}$.
