When applied to physical systems, the original weakly-hard model is not sufficiently descriptive to model the dynamics of, e.g., a control task.
The discrepancy is represented by the strategy used to handle the control task's deadline misses directly affecting the control actuation pattern.
For instance, under the \tK{} strategy, a missed deadline represents a job that will never be completed, while in the case of the \tS{} strategy, it represents a job that will complete \emph{later}.

To provide a comprehensive analysis framework, we therefore need to examine what occurs in each time interval $(\pi_i)_{i \in \N_{\geq}}$, with $\pi_i = [a_0 + i\cdot \Ts, a_0 + (i+1)\cdot \Ts)$. 
In this context, the alphabet $\Sigma$ introduced in Definition~\ref{def:basic-alphabet} and the weakly-hard model need to be extended to account for the specific deadline miss handling procedure, denoted with the symbol $\strat$.

\begin{definition}[Extended Weakly Hard Task $\tau \vdash \lambda^{\strat}$]%
    \label{def:new-mk}%
    An extended weakly-hard task $\tau$ satisfies any combination of these four constraints:
    \begin{enumerate}[label=(\roman*)]
        \item $\tau \vdash \erowmiss{}{\strat}$: at most $x$ intervals lack a job completion, in any window of $k$ consecutive jobs;
        \item $\tau \vdash \erowhit{}{\strat}$: at least $x$ intervals contain a job completion, in any window of $k$ consecutive jobs;
        \item $\tau \vdash \erowmiss{}{\strat}$: at most $x$ \emph{consecutive} intervals lack a job completion, in any window of $k$ consecutive jobs; and
        \item $\tau \vdash \erowhit{}{\strat}$: at least $x$ \emph{consecutive} intervals contain a job completion, in any window of $k$ consecutive jobs,
    \end{enumerate}
    with $x\in \N_{\geq}$, $k \in \N_{>}$, and $x\leq k$, while using strategy $\strat$ to handle potential deadline misses.
\end{definition}

The definition above differs from Definition~\ref{def:weakly-hard} in two points:
\begin{enumerate*}[label=(\roman*)]
    \item it explicitly introduces the handling strategy $\strat$; and
    \item it focuses on the presence of a new control command at the end of each time interval $\pi_i$, instead of checking the deadline outcome, thus guaranteeing its applicability also for strategies different than \tK{}.
\end{enumerate*}
To simplify notation, we indicate the \tK{} handling strategy with $K$ and \tS{} with $S$.

The strategy dependence necessitates a more expressive alphabet $\Sigma\funof{\strat}$ to characterise the behaviour of task $\tau$ in each possible time interval.
For both the \tK{} and \tS{} strategies, each interval $\pi_i$ contains at most one activated and one completed job.
This restricts the possible behaviours to three cases:
%
\begin{enumerate}[label=(\roman*)]
    \item a time interval in which the same job is both released and completed is denoted by $\cH$ (\emph{hit});
    \item a time interval in which no job is completed is denoted by $\cM$ (\emph{miss});
    \item a time interval in which no job is released, but a job (released in a previous interval) is completed, is denoted by $\cR$ (\emph{recovery}).
\end{enumerate}
%
By checking all unique combinations of job activations and completions in each interval, we obtain the alphabets for \tK{} and \tS{} as $\Sigma\left(\text{\tK{}}\right) = \{ \cM, \cH \}$ and $\Sigma\left(\text{\tS{}}\right) = \{ \cM, \cH, \cR \}$, respectively.
Trivially, $\Sigma\left(\text{\tK{}}\right)$ uses the basic alphabet from Definition~\ref{def:basic-alphabet}.
The recovery character $\cR$ is used in the \tS{} alphabet to identify the \emph{late completion} of a job.
As a consequence, $\cR$ is treated equivalently to $\cH$ when verifying the extended weakly hard constraints (\ewhc{}).

The algebra presented in Section~\ref{ssec:whalgebra} is extended to the new alphabet.
We assign a character of the alphabet $\Sigma\left(\strat\right)$ to each interval $\pi_i$.
A word $w = \seq{c_1,c_2,\dots,c_N} \in \Sigma\left(\strat\right)^N$ is used to represent a sequence of $N$ outcomes for task $\tau$, with $c_i$ representing the outcome associated to the interval $\pi_i$. 
To enforce only feasible sequences, we introduce an ordering constraint for the $\cR$ character with the following Rule.
%
\begin{rule_}[Outcome Ordering]%
    \label{rule:R}%
    For any word $w \in \Sigma\left(\text{\tS{}}\right)^N$, $\cR$ may only directly follow $\cM$, or be the initial element of the word.
\end{rule_}

The extended weakly hard model also inherits all the properties of the original weakly hard model.
In particular, the satisfaction set of $\lambda^\strat$ can be defined for $N\geq 1$ as $\sset{N}{\lambda^{\strat}} = \{ w \in \Sigma\left(\strat\right)^N \mid w \vdash \lambda^{\strat} \}$, and the constraint domination still holds as $\lambda^{\strat}_{i} \preceq \lambda^{\strat}_{j}$ if $\sset{}{\lambda^{\strat}_{i} } \subseteq \sset{}{\lambda^{\strat}_{j}}$.
