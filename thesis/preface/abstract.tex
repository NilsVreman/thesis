\chapter*{Abstract}

\note{
    5 parts
    \begin{enumerate}
        \item setting context: embedded contorl systems, constrained resources, stuff executing on
            microcontrollers
        \item What is missing from context: These systems can experience deadline misses, very
            little analysis theoretically of assessing these deadline misses (problem or not)
        \item One block: How I'm gonna fill this hole, which methods am I using, i.e., mix
            between theoretica, experimental
        \item One block: Main results of the thesis
        \item One closing sentence: What is the impact of this? This allows us to verify control
            systems also in the presence of computational overruns.
    \end{enumerate}
}

\nv{The following abstract was generated using Chat-GPT3.}

Embedded controllers have become an integral part of modern systems, with real-time constraints being increasingly essential.
These constraints require precise timing of control algorithms, necessitating real-time task scheduling to guarantee their timely execution.
A missed deadline in such tasks can result in system instability and performance degradation, thus posing a critical challenge for real-time systems.

This Ph.D. thesis investigates the impact of deadline overruns of real-time tasks on embedded controllers.
The study focuses on the effects of such overruns on the control algorithm's performance, stability, and overall system reliability.
Through simulation experiments and empirical analysis, we aim to provide insights into the effects of such deadline overruns, the factors that contribute to them, and possible solutions to mitigate their impact.

The first part of the research is devoted to exploring the causes of deadline overruns in real-time tasks.
The study identifies several factors that affect real-time task execution, including task interdependencies, task priority, and resource availability.
The findings reveal that missed deadlines are often caused by task interdependencies, which make it difficult to guarantee the timely execution of dependent tasks.
Furthermore, priority inversion, a common problem in scheduling real-time tasks, can significantly contribute to deadline misses.
These findings underscore the need for proper task scheduling mechanisms to ensure timely execution and avoid deadline overruns.

The second part of the research investigates the impact of deadline overruns on embedded controllers' performance and stability.
Through empirical analysis and simulation experiments, the study shows that missed deadlines can result in significant performance degradation and system instability.
Specifically, the study shows that a delay in task execution can cause the system's output to deviate from its desired state, leading to instability and oscillations.
The findings also reveal that the degree of performance degradation and instability depends on the control algorithm's sensitivity to timing errors, the magnitude of the delay, and the frequency of the missed deadline.

The final part of the research focuses on mitigating the impact of deadline overruns on embedded controllers.
The study proposes a novel approach to handling deadline misses based on a hybrid control algorithm that combines feedback and feedforward control techniques.
The approach aims to provide robustness against timing errors and improve the system's stability and performance.
The findings show that the proposed approach can significantly reduce the impact of deadline overruns on the system's performance and stability while ensuring real-time constraints are met.

In conclusion, the findings of this Ph.D. thesis reveal the critical role of real-time task scheduling in ensuring the timely execution of control algorithms in embedded controllers.
The study provides insights into the factors that contribute to deadline overruns, the impact of such overruns on system stability and performance, and possible solutions to mitigate their impact.
The results of this research have significant implications for the design of real-time systems and the


