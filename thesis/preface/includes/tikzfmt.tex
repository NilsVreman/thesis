% TIKZ settings
\usepackage{tikz, pgfplots, pgfplotstable}
\usepackage{shellesc} %As suggested in the wiki

% externalization
\usetikzlibrary{external}
\tikzexternalize[prefix=externalised/]

%%%%%%%%%%%%%%%%%%%%%%%%%
%%% packages copy pasted from:
%%%    ecrts21/og/pkg.tex
%%% as you add the papers you'll have to adjust them
%%%%%%%%%%%%%%%%%%%%%%%%%

%%%%%%%%%%%%%%%%%%%%%%%%%%
% libraries
\pgfplotsset{compat=1.16}
\usetikzlibrary{shapes}
\usetikzlibrary{fit}
\usetikzlibrary{arrows, arrows.meta}
\usetikzlibrary{decorations.markings}
\usepgfplotslibrary{groupplots}

% Extracts the max value from the input and marks it in the plot
\pgfplotsset{
  mark max/.style={
    scatter/@pre marker code/.code={%
    \ifx\pgfplotspointmeta\pgfplots@metamax
        \def\markopts{}%
        \node [anchor=south, xshift=1mm] { \pgfmathprintnumber[precision=1, fixed zerofill, ultra thick]{\pgfplotspointmeta} };
    \else
        \def\markopts{mark=none}
    \fi
        \expandafter\scope\expandafter[\markopts]
    },%
    scatter/@post marker code/.code={ \endscope },
    scatter
  }
}

% Read element from specific place in data table
\newcommand*{\ReadOutElement}[4]{%
    \pgfplotstablegetelem{#2}{[index]#3}\of{#1}%
    \let#4\pgfplotsretval
}

%%%%%%%%%%%%%%%%%%%%%%%%%
%%% packages copy pasted from:
%%%    rtas22a/og/pkg.tex
%%% as you add the papers you'll have to adjust them
%%%%%%%%%%%%%%%%%%%%%%%%%

\usetikzlibrary{shapes.misc}
\usetikzlibrary{backgrounds}
\pgfplotsset{
    ylabel right/.style={
        after end axis/.append code={
            \node [anchor=west, xshift=1.5mm] at (rel axis cs:1,0.5) {#1};
        }
    }
}

%%%%%%%%%%%%%%%%%%%%%%%%%
%%% packages copy pasted from:
%%%    rtas22b/og/pkg.tex
%%% as you add the papers you'll have to adjust them
%%%%%%%%%%%%%%%%%%%%%%%%%

%\usetikzlibrary{positioning}
\usepgfplotslibrary{fillbetween}

\tikzset{Dom Node/.style={draw,
                        thick, 
                        circle, 
                        inner sep=0pt,
                        minimum size=12mm}%
}%

% Comparison figure related
\def\xstart{1}
\def\xend{10} % change according to how many plots you have

% this is the list of styles define as many colours as the number of lines you could also change the marker etc
\pgfplotscreateplotcyclelist{blue10}{
    {blue!95!black, mark=*, mark size=2pt,mark options={fill=white}},
    {blue!85!black, mark=*, mark size=2pt},
    {blue!75!black, mark=*, mark size=2pt},
    {blue!65!black, mark=*, mark size=2pt},
    {blue!55!black, mark=*, mark size=2pt},
    {blue!45!black, mark=*, mark size=2pt},
    {blue!35!black, mark=*, mark size=2pt},
    {blue!25!black, mark=*, mark size=2pt},
    {blue!15!black, mark=*, mark size=2pt},
    {blue! 5!black, mark=*, mark size=2pt},
}

% argument #1: any options
\makeatletter
\newenvironment{customlegend}[1][]{%
    \begingroup
    % inits/clears the lists (which might be populated from previous axes):
    \pgfplots@init@cleared@structures
    \pgfplotsset{#1}%
}{%
    % draws the legend:
    \pgfplots@createlegend
    \endgroup
}%

% makes \addlegendimage available (typically only available within an axis environment):
\def\addlegendimage{\pgfplots@addlegendimage}
\makeatother
