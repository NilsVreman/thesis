% TIKZ settings
\usepackage{tikz, pgfplots, pgfplotstable}
\usepackage{shellesc} %As suggested in the wiki

% externalization
\usetikzlibrary{external}
\tikzexternalize[prefix=externalised/]

%%%%%%%%%%%%%%%%%%%%%%%%%
%%% packages copy pasted from:
%%%    ecrts21/og/pkg.tex
%%% as you add the papers you'll have to adjust them
%%%%%%%%%%%%%%%%%%%%%%%%%

%%%%%%%%%%%%%%%%%%%%%%%%%%
% libraries
\pgfplotsset{compat=1.16}
\usetikzlibrary{shapes, fit}
\usetikzlibrary{arrows.meta}
\usetikzlibrary{decorations.markings}
\usepgfplotslibrary{groupplots}

% Extracts the max value from the input and marks it in the plot
\pgfplotsset{
  mark max/.style={
    scatter/@pre marker code/.code={%
    \ifx\pgfplotspointmeta\pgfplots@metamax
        \def\markopts{}%
        \node [anchor=south, xshift=1mm] { \pgfmathprintnumber[precision=1, fixed zerofill, ultra thick]{\pgfplotspointmeta} };
    \else
        \def\markopts{mark=none}
    \fi
        \expandafter\scope\expandafter[\markopts]
    },%
    scatter/@post marker code/.code={ \endscope },
    scatter
  }
}

% Read element from specific place in data table
\newcommand*{\ReadOutElement}[4]{%
    \pgfplotstablegetelem{#2}{[index]#3}\of{#1}%
    \let#4\pgfplotsretval
}

%%%%%%%%%%%%%%%%%%%%%%%%%
%%% packages copy pasted from:
%%%    rtas22a/og/pkg.tex
%%% as you add the papers you'll have to adjust them
%%%%%%%%%%%%%%%%%%%%%%%%%
