%%%%%%%%%%%%%%%%%%%%%%%%%
%%% packages copy pasted from:
%%%    ecrts21/preamble/pkg.tex
%%% as you add the papers you'll have to adjust them
%%%%%%%%%%%%%%%%%%%%%%%%%

% TIKZ settings
\usepackage{tikz, pgfplots, pgfplotstable}
\usepackage{shellesc} %As suggested in the wiki

% externalization
\usetikzlibrary{external}
\tikzexternalize[prefix=externalised/]

%%%%%%%%%%%%%%%%%%%%%%%%%%
% libraries
\pgfplotsset{compat=1.16}
\usetikzlibrary{shapes, fit}
\usetikzlibrary{arrows.meta}
\usetikzlibrary{decorations.markings}
%\usepgfplotslibrary{groupplots}

%
%
%%% TIKZ commands
%\makeatletter
%\pgfplotsset{
%  every axis plot/.append style =
%    {cyan, ultra thick, mark=none, mark options={fill=white}},
%  mark max/.style={
%    scatter/@pre marker code/.code={%
%    \ifx\pgfplotspointmeta\pgfplots@metamax
%        \def\markopts{}%
%        \node [anchor=south, xshift=1mm] { \pgfmathprintnumber[precision=1, fixed zerofill]{\pgfplotspointmeta} };
%    \else
%        \def\markopts{mark=none}
%    \fi
%        \expandafter\scope\expandafter[\markopts]
%    },%
%    scatter/@post marker code/.code={ \endscope },
%    scatter
%  }
%}
%% Style to select only points from #1 to #2 (inclusive)
%\pgfplotsset{select coords between index/.style 2 args={
%    x filter/.code={
%        \ifnum\coordindex<#1\def\pgfmathresult{}\fi
%        \ifnum\coordindex>#2\def\pgfmathresult{}\fi
%    }
%}}
%\makeatother
%
%\newcommand{\findmax}[1]{
%    \pgfmathsetmacro\buffer{0.0}
%    \pgfplotstableforeachcolumnelement{#1}\of\extdata\as\cellValue{%
%    \pgfmathsetmacro{\buffer}{max(\buffer,\cellValue)}}
%}
%\newcommand*{\ReadOutElement}[4]{%
%    \pgfplotstablegetelem{#2}{[index]#3}\of{#1}%
%    \let#4\pgfplotsretval
%}
