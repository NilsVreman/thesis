\chapter{Contribution}
\nv{Don't forget to add note that you have unified language between papers.
Add it at the bottom of each paper front page as well}

The main contribution of this thesis consists of the \note{five} papers following this chapter.
This chapter provides a lucid overview of the individual publications' contributions to the real-time control domain.
In Section~\ref{sec:paper-summaries}, we provide a succinct summary of the scientific contribution of each paper in the order of appearance.
Additionally, each author's contribution to the specific paper is stated.
We conclude the chapter with Section~\ref{sec:additional-publications}, listing additional publications by the thesis' author that were excluded from the thesis.

%%%%%%%%%%%%%%%%%%%%%%%%%%%% NOTE %%%%%%%%%%%%%%%%%%%%%%%%%%%%
\clearpage %%%%%%%%%%%%%%%%% NOTE %%%%%%%%%%%%%%%%%%%%%%%%%%%%
%%%%%%%%%%%%%%%%%%%%%%%%%%%% NOTE %%%%%%%%%%%%%%%%%%%%%%%%%%%%

\section{Summary of Papers}%
\label{sec:paper-summaries}%
%
\subsection*{Paper I}%
%
\begin{quote}
    \fullcite{Vreman:2021}
\end{quote}

\subsubsection*{Scientific Contribution:}%
%
In this paper, linear control system stability and performance is analysed when the controller experiences persistent computational blackouts.
Historically, the models to analyse such events have been either stochastic or overly conservative.
These models have been thoroughly investigated in the literature, but few have been shown to achieve similar results on industrial applications.

Instead, the paper proposes a new fault model that handles consecutive computational overruns whilst also encapsulating the graceful recovery back to nominal execution conditions.
The fault model is coupled together with different actuator (\tZ{} and \tH{}) and scheduling strategies (\tK{} and \tS{}) to provide a holistic system model.
A methodology for analysing the stability and performance of systems subject to this specific fault model is also derived. \nv{I should write more specifics about the analysis here maybe?}

To reinforce the analysis methodology's applicability to industrial application, an experimental campaign is carried out.\nv{Also be more clear about what experiments are performed}
We show that, for a physical system, the analysis -- performed on a \emph{linear model} of the \emph{non-linear physical system} -- is enough to draw conclusions about the physical process'.

\subsubsection*{Authors' Contribution:}%
%
M. Maggio proposed the idea of analysing control system stability for fault models that are well-suited for industrial use.
The physical testing environment and the experiments were implemented and executed by N. Vreman, who also developed the new fault model.
After discussions between M. Maggio and N. Vreman the switching stability analysis presented in the paper was derived.
Together, A. Cervin and N. Vreman derived a method for analysing the performance of systems subject to the proposed fault model.
The manuscript writing was shared among all three authors.


%%%%%%%%%%%%%%%%%%%%%%%%%%%% NOTE %%%%%%%%%%%%%%%%%%%%%%%%%%%%
\clearpage %%%%%%%%%%%%%%%%% NOTE %%%%%%%%%%%%%%%%%%%%%%%%%%%%
%%%%%%%%%%%%%%%%%%%%%%%%%%%% NOTE %%%%%%%%%%%%%%%%%%%%%%%%%%%%

\subsection*{Paper II}%
%
\begin{quote}
    \fullcite{Vreman:2022a}
\end{quote}

\subsubsection*{Scientific Contribution:}%
%
Many different robust controllers exists, specifically targeting control systems where the controller can experience timing faults.
The main advantage of using these synthesis methods is that they can guarantee a priori system stability, under their respective fault model.
However, this paper identifies some limitations of previous work, such as
\begin{enumerate*}[label=(\roman*)]
    \item complex design methods,
    \item poor performance in nominal conditions, and
    \item strong assumptions on the control law's structure.
\end{enumerate*}

To address the shortcomings of previous synthesis methods, we propose an adaptive control law that
\begin{enumerate*}[label=(\roman*)]
    \item is simple to design and implement,
    \item requires minimal system knowledge, and
    \item does not affect the nominal control performance.
\end{enumerate*}
The adaptive controller assumes that there exists a controller that has been synthesised with respect to a specific control system's specification under nominal conditions.
Based on the number of computational overruns, the adaptation then alters the controller matrices to quickly restore nominal control performance.
\nv{Mention something about \tK{}}.
The proposed method does not guarantee any a priori stability guarantees; thus, we also derive a stochastic mean-square stability analysis to be performed a posteriori.

\nv{Write something about the experimental campaign}




\subsubsection*{Authors' Contribution:}%
%


%%%%%%%%%%%%%%%%%%%%%%%%%%%% NOTE %%%%%%%%%%%%%%%%%%%%%%%%%%%%
\clearpage %%%%%%%%%%%%%%%%% NOTE %%%%%%%%%%%%%%%%%%%%%%%%%%%%
%%%%%%%%%%%%%%%%%%%%%%%%%%%% NOTE %%%%%%%%%%%%%%%%%%%%%%%%%%%%

\subsection*{Paper III}%
%
\begin{quote}
    \fullcite{Vreman:2022b}
\end{quote}

\subsubsection*{Scientific Contribution:}%
%
\subsubsection*{Authors' Contribution:}%
%
    

%%%%%%%%%%%%%%%%%%%%%%%%%%%% NOTE %%%%%%%%%%%%%%%%%%%%%%%%%%%%
\clearpage %%%%%%%%%%%%%%%%% NOTE %%%%%%%%%%%%%%%%%%%%%%%%%%%%
%%%%%%%%%%%%%%%%%%%%%%%%%%%% NOTE %%%%%%%%%%%%%%%%%%%%%%%%%%%%

\subsection*{Paper IV}%
%
\begin{quote}
    \fullcite{Vreman:2022c}
\end{quote}

\subsubsection*{Scientific Contribution:}%
%
\subsubsection*{Authors' Contribution:}%
%


%%%%%%%%%%%%%%%%%%%%%%%%%%%% NOTE %%%%%%%%%%%%%%%%%%%%%%%%%%%%
\clearpage %%%%%%%%%%%%%%%%% NOTE %%%%%%%%%%%%%%%%%%%%%%%%%%%%
%%%%%%%%%%%%%%%%%%%%%%%%%%%% NOTE %%%%%%%%%%%%%%%%%%%%%%%%%%%%

\subsection*{Paper V}%
%
\begin{quote}
    \fullcite{Vreman:2023}
\end{quote}

\subsubsection*{Scientific Contribution:}%
%
\subsubsection*{Authors' Contribution:}%
%


%%%%%%%%%%%%%%%%%%%%%%%%%%%% NOTE %%%%%%%%%%%%%%%%%%%%%%%%%%%%
\clearpage %%%%%%%%%%%%%%%%% NOTE %%%%%%%%%%%%%%%%%%%%%%%%%%%%
%%%%%%%%%%%%%%%%%%%%%%%%%%%% NOTE %%%%%%%%%%%%%%%%%%%%%%%%%%%%

\section{Additional Publications}%
\label{sec:additional-publications}%
%
The author of this thesis has also contributed to the following publications.
The publications have been omitted to improve cohesion and clarity of the thesis' presentation, but also as they cover topics outside the scope of this thesis.

\begin{quote}
    \fullcite{Gunnarsson:2023}
\end{quote}

\begin{quote}
    \fullcite{Kruger:2021}
\end{quote}

\begin{quote}
    \fullcite{Vreman:2020}
\end{quote}

\begin{quote}
    \fullcite{Vreman:2019a}
\end{quote}

\begin{quote}
    \fullcite{Vreman:2019b}
\end{quote}
