\chapter{Contribution}
\nv{Don't forget to add note that you have unified language between papers.
Add it at the bottom of each paper front page as well}

The main contribution of this thesis consists of the \note{five} papers following this chapter.
This chapter provides a lucid overview of the individual publications' contributions to the real-time control domain.
In Section~\ref{sec:paper-summaries}, we provide a succinct summary of the scientific contribution of each paper in the order of appearance.
Additionally, each author's contribution to the specific paper is stated.
We conclude the chapter with Section~\ref{sec:additional-publications}, listing additional publications by the thesis' author that were excluded from the thesis.

%%%%%%%%%%%%%%%%%%%%%%%%%%%% NOTE %%%%%%%%%%%%%%%%%%%%%%%%%%%%
\clearpage %%%%%%%%%%%%%%%%% NOTE %%%%%%%%%%%%%%%%%%%%%%%%%%%%
%%%%%%%%%%%%%%%%%%%%%%%%%%%% NOTE %%%%%%%%%%%%%%%%%%%%%%%%%%%%

\section{Summary of Papers}%
\label{sec:paper-summaries}%
%
\subsection*{Paper I}%
%
\begin{quote}
    \fullcite{Vreman:2021}
\end{quote}

\subsubsection*{Scientific Contribution:}%
%
In this paper, linear control system stability and performance is analysed when the controller experiences persistent computational blackouts.
Historically, the models to analyse such events have been either stochastic or overly conservative.
These models have been thoroughly investigated in the literature, but few have been shown to achieve similar results on industrial applications.

Instead, the paper proposes a new fault model that handles consecutive computational overruns whilst also encapsulating the graceful recovery back to nominal execution conditions.
The fault model is coupled together with different actuator (\tZ{} and \tH{}) and scheduling strategies (\tK{} and \tS{}) to provide a holistic system model.
A methodology for analysing the stability and performance of systems subject to this specific fault model is also derived.

To reinforce the analysis methodology's applicability to industrial application, an experimental campaign is carried out.
We show, for a physical system, that the analysis -- performed on a \emph{linear model} of the \emph{non-linear physical system} -- is enough to draw conclusions about the physical process'.

\subsubsection*{Authors' Contribution:}%
%
M. Maggio proposed the idea of analysing control system stability for fault models that are well-suited for industrial use.
The physical testing environment and the experiments were implemented and executed by N. Vreman, who also developed the new fault model.
After discussions between M. Maggio and N. Vreman the switching stability analysis presented in the paper was derived.
Together, A. Cervin and N. Vreman derived a method for analysing the performance of systems subject to the proposed fault model.
The manuscript writing was shared among all three authors.


%%%%%%%%%%%%%%%%%%%%%%%%%%%% NOTE %%%%%%%%%%%%%%%%%%%%%%%%%%%%
\clearpage %%%%%%%%%%%%%%%%% NOTE %%%%%%%%%%%%%%%%%%%%%%%%%%%%
%%%%%%%%%%%%%%%%%%%%%%%%%%%% NOTE %%%%%%%%%%%%%%%%%%%%%%%%%%%%

\subsection*{Paper II}%
%
\begin{quote}
    \fullcite{Vreman:2022a}
\end{quote}

\subsubsection*{Scientific Contribution:}%
%

\subsubsection*{Authors' Contribution:}%
%


%%%%%%%%%%%%%%%%%%%%%%%%%%%% NOTE %%%%%%%%%%%%%%%%%%%%%%%%%%%%
\clearpage %%%%%%%%%%%%%%%%% NOTE %%%%%%%%%%%%%%%%%%%%%%%%%%%%
%%%%%%%%%%%%%%%%%%%%%%%%%%%% NOTE %%%%%%%%%%%%%%%%%%%%%%%%%%%%

\subsection*{Paper III}%
%
\begin{quote}
    \fullcite{Vreman:2022b}
\end{quote}

\subsubsection*{Scientific Contribution:}%
%
\subsubsection*{Authors' Contribution:}%
%
    

%%%%%%%%%%%%%%%%%%%%%%%%%%%% NOTE %%%%%%%%%%%%%%%%%%%%%%%%%%%%
\clearpage %%%%%%%%%%%%%%%%% NOTE %%%%%%%%%%%%%%%%%%%%%%%%%%%%
%%%%%%%%%%%%%%%%%%%%%%%%%%%% NOTE %%%%%%%%%%%%%%%%%%%%%%%%%%%%

\subsection*{Paper IV}%
%
\begin{quote}
    \fullcite{Vreman:2022c}
\end{quote}

\subsubsection*{Scientific Contribution:}%
%
\subsubsection*{Authors' Contribution:}%
%


%%%%%%%%%%%%%%%%%%%%%%%%%%%% NOTE %%%%%%%%%%%%%%%%%%%%%%%%%%%%
\clearpage %%%%%%%%%%%%%%%%% NOTE %%%%%%%%%%%%%%%%%%%%%%%%%%%%
%%%%%%%%%%%%%%%%%%%%%%%%%%%% NOTE %%%%%%%%%%%%%%%%%%%%%%%%%%%%

\subsection*{Paper V}%
%
\begin{quote}
    \fullcite{Vreman:2023}
\end{quote}

\subsubsection*{Scientific Contribution:}%
%
\subsubsection*{Authors' Contribution:}%
%


%%%%%%%%%%%%%%%%%%%%%%%%%%%% NOTE %%%%%%%%%%%%%%%%%%%%%%%%%%%%
\clearpage %%%%%%%%%%%%%%%%% NOTE %%%%%%%%%%%%%%%%%%%%%%%%%%%%
%%%%%%%%%%%%%%%%%%%%%%%%%%%% NOTE %%%%%%%%%%%%%%%%%%%%%%%%%%%%

\section{Additional Publications}%
\label{sec:additional-publications}%
%
The author of this thesis has also contributed to the following publications.
The publications have been omitted to improve cohesion and clarity of the thesis' presentation, but also as they cover topics outside the scope of this thesis.

\begin{quote}
    \fullcite{Gunnarsson:2023}
\end{quote}

\begin{quote}
    \fullcite{Kruger:2021}
\end{quote}

\begin{quote}
    \fullcite{Vreman:2020}
\end{quote}

\begin{quote}
    \fullcite{Vreman:2019a}
\end{quote}

\begin{quote}
    \fullcite{Vreman:2019b}
\end{quote}
