\chapter{Contribution}
\nv{Don't forget to add note that you have unified language between papers.
Add it at the bottom of each paper front page as well}

The main contribution of this thesis consists of the \note{five} papers following this chapter.
This chapter provides a lucid overview of the individual publications' contributions to the real-time control domain.
In Section~\ref{sec:paper-summaries}, we provide a succinct summary of the scientific contribution of each paper in the order of appearance.
Additionally, each author's contribution to the specific paper is stated.
We conclude the chapter with Section~\ref{sec:additional-publications}, listing additional publications by the thesis' author that were excluded from the thesis.

%%%%%%%%%%%%%%%%%%%%%%%%%%%% NOTE %%%%%%%%%%%%%%%%%%%%%%%%%%%%
\clearpage %%%%%%%%%%%%%%%%% NOTE %%%%%%%%%%%%%%%%%%%%%%%%%%%%
%%%%%%%%%%%%%%%%%%%%%%%%%%%% NOTE %%%%%%%%%%%%%%%%%%%%%%%%%%%%

\section{Summary of Papers}%
\label{sec:paper-summaries}%
%
\subsection*{Paper I}%
%
\begin{quote}
    \fullcite{Vreman:2021}
\end{quote}

\subsubsection*{Scientific Contribution:}%
%
In Paper I, linear control system stability and performance is analysed when the controller experiences persistent computational blackouts.
Historically, the models to analyse such events have been either stochastic or overly conservative.
These models have been thoroughly investigated in the literature, but few have been shown to achieve similar results on industrial applications.

Instead, Paper I proposes a new fault model that handles consecutive computational overruns whilst also encapsulating graceful recovery back to nominal execution conditions.
The fault model is coupled together with different actuator (\tZ{} and \tH{}) and scheduling strategies (\tK{} and \tS{}) to provide a holistic system model.
To properly analyse the specific fault model, a methodology for analysing the stability and performance is also derived.
The stability analysis is based on a switched system stability approach, whilst the performance analysis is based on evolving a quadratic cost function in time.

To reinforce the analysis methodology's applicability to industrial application, an experimental campaign is carried out.
We analyse both a non-linear physical system (a Furuta pendulum) and a set of linear system models representative for the process industry.
From the analysis performed on the physical system, we show that it is possible to draw accurate conclusions about the physical process' behaviour.

\subsubsection*{Authors' Contribution:}%
%
M. Maggio proposed the idea of analysing control system stability for fault models that are well-suited for industrial use.
The physical testing environment and the experiments were implemented and executed by N. Vreman, who also developed the new fault model.
After discussions between M. Maggio and N. Vreman the switching stability analysis was derived.
Together, A. Cervin and N. Vreman derived a method for analysing the performance of systems subject to the proposed fault model.
The manuscript writing was shared among all three authors.



%%%%%%%%%%%%%%%%%%%%%%%%%%%% NOTE %%%%%%%%%%%%%%%%%%%%%%%%%%%%
\clearpage %%%%%%%%%%%%%%%%% NOTE %%%%%%%%%%%%%%%%%%%%%%%%%%%%
%%%%%%%%%%%%%%%%%%%%%%%%%%%% NOTE %%%%%%%%%%%%%%%%%%%%%%%%%%%%

\subsection*{Paper II}%
%
\begin{quote}
    \fullcite{Vreman:2022a}
\end{quote}

\subsubsection*{Scientific Contribution:}%
%
Many different robust controllers exists, specifically targeting control systems where the controller can experience timing faults.
The main advantage of using these synthesis methods is that they can guarantee a priori system stability, under their respective fault model.
However, Paper II identifies some limitations of previous work, such as
\begin{enumerate*}[label=(\roman*)]
    \item complex design methods,
    \item poor performance in nominal conditions, and
    \item strong assumptions on the control law's structure.
\end{enumerate*}

To address the shortcomings of previous synthesis methods, we propose an adaptive control law that
\begin{enumerate*}[label=(\roman*)]
    \item is simple to design and implement,
    \item requires minimal system knowledge, and
    \item does not affect the nominal control performance.
\end{enumerate*}
The adaptive controller assumes that there exists a controller that has been synthesised with respect to a specific control system's specification under nominal conditions.
Assuming that the \tK{} strategy is employed, the adaptation then alters the controller matrices (based on the number of computational overruns) to quickly restore nominal control performance.
To analyse the performance gain of the adaptive controller, we extend existing stochastic performance analyses based on Markov jump linear system's theory to compute the \emph{relative performance degradation}, i.e., the performance degradation following from using a controller other than the ideal controller.

We validate the adaptive controller on a physical process (a Ball and Beam) and the set of process industrial models from Paper I.
Significant performance gain is shown despite having an immense number of computational overruns.
Additionally, since the proposed method does not guarantee any a priori stability guarantees, we also analyse the switched system stability a posteriori.

\subsubsection*{Authors' Contribution:}%
%
The initial observation, that many fault-tolerant controllers are restricted to overly specific system conditions, was made by C. Mandrioli.
After discussions between C. Mandrioli and N. Vreman a set of limitations was identified; together, they derived the adaptive controller addressing the set of limitations.
A. Cervin derived the performance analysis to validate the controller.
The adaptive controller was implemented on the Ball and Beam by N. Vreman and the data was collected by both N. Vreman and C. Mandrioli.
The three authors shared the manuscript writing.



%%%%%%%%%%%%%%%%%%%%%%%%%%%% NOTE %%%%%%%%%%%%%%%%%%%%%%%%%%%%
\clearpage %%%%%%%%%%%%%%%%% NOTE %%%%%%%%%%%%%%%%%%%%%%%%%%%%
%%%%%%%%%%%%%%%%%%%%%%%%%%%% NOTE %%%%%%%%%%%%%%%%%%%%%%%%%%%%

\subsection*{Paper III}%
%
\begin{quote}
    \fullcite{Vreman:2022b}
\end{quote}

\subsubsection*{Scientific Contribution:}%
%
Ever since their introduction, the weakly-hard models have been steadily increasing in both academic and industrial popularity.
The existing tools and research have focused on single constraints, in particular the \tAM{} constraint.
In certain domains, e.g., control systems, this choice is likely motivated by the \tAM{} constraints popularity rather than its fitness to the problem statement.

Paper III aspires to
\begin{enumerate*}[label=(\roman*)]
    \item change the focus from the \tAM{} constraint to the \tRH{} constraint and
    \item propose a scalable, open-source tool (\tool{}) that can be used to analyse \emph{all} the weakly-hard constraints.
\end{enumerate*}
In order to switch the attention to the \tRH{} constraint, we provide two novel theorems on the relation between it and the \tAH{} constraint, finalising the relationship graph between all the weakly-hard constraints.
The \tool{} tool employ an automaton model to describe the discrete-time execution of a task subject to a specific weakly-hard constraint.
Additionally, \tool{} is the first tool to address \emph{sets} of weakly-hard constraints.
\note{Don't know if I should mention anything about the experimental campaign here?}

\subsubsection*{Authors' Contribution:}%
%
The tool and its underlying algorithm was fully developed by N. Vreman, with input from discussions with M. Maggio.
N. Vreman came up with the idea for the theoretical results.
Together, N. Vreman and R. Pates formulated the theorems relating the \tAH{} and \tRH{} constraints, which R. Pates proved mathematically.
The manuscript was written by N. Vreman and M. Maggio with revisions from R. Pates.



%%%%%%%%%%%%%%%%%%%%%%%%%%%% NOTE %%%%%%%%%%%%%%%%%%%%%%%%%%%%
\clearpage %%%%%%%%%%%%%%%%% NOTE %%%%%%%%%%%%%%%%%%%%%%%%%%%%
%%%%%%%%%%%%%%%%%%%%%%%%%%%% NOTE %%%%%%%%%%%%%%%%%%%%%%%%%%%%

\subsection*{Paper IV}%
%
\begin{quote}
    \fullcite{Vreman:2022c}
\end{quote}

\subsubsection*{Scientific Contribution:}%
%
In Paper IV, we identify that despite the weakly-hard models having steadily gained more traction, they are lacking when it comes to describing more expressive implementation details, e.g., deadline handling strategies.
We rectify this discrepancy by extending the formal language defining the weakly-hard constraints, improving the models' real implementability.

The \tool{} tool, introduced in Paper III, unveiled new possibilities for analysing real-time control systems under weakly-hard constraints.
Utilising this automaton model, we propose a post-processing step refactoring the automaton according to the formal language specifications in order to facilitate the new \emph{extended weakly-hard constraints}.
For the refactored automaton, we present a switched linear systems stability analysis.

The dynamics of the linear control system is Kronecker lifted with the adjacency matrix of the automaton to obtain an equivalent system model, but now also subject to the extended weakly-hard constraint.
This system model can then be analysed using any method applicable to arbitrary switching systems; Paper IV utilises a JSR approach.
In addition to the stability analysis, we propose two novel theorems relating the real-time constraints with the control system stability.

\subsubsection*{Authors' Contribution:}%
%
P. Pazzaglia came up with the idea to the paper after discussions with N. Vreman; together, they derived the theoretical results presented in the paper.
N. Vreman refactored the \tool{} tool to also facilitate a more expressive formal language.
The discussions around the JSR analysis and evaluation were led by V. Magron and M. Maggio.
J. Wang implemented the JSR evaluation presented in the paper.
The manuscript was then written by P. Pazzaglia and N. Vreman with input and revisions from V. Magron and M. Maggio.



%%%%%%%%%%%%%%%%%%%%%%%%%%%% NOTE %%%%%%%%%%%%%%%%%%%%%%%%%%%%
\clearpage %%%%%%%%%%%%%%%%% NOTE %%%%%%%%%%%%%%%%%%%%%%%%%%%%
%%%%%%%%%%%%%%%%%%%%%%%%%%%% NOTE %%%%%%%%%%%%%%%%%%%%%%%%%%%%

\subsection*{Paper V}%
%
\begin{quote}
    \fullcite{Vreman:2023}
\end{quote}

\subsubsection*{Scientific Contribution:}%
%
\subsubsection*{Authors' Contribution:}%
%


%%%%%%%%%%%%%%%%%%%%%%%%%%%% NOTE %%%%%%%%%%%%%%%%%%%%%%%%%%%%
\clearpage %%%%%%%%%%%%%%%%% NOTE %%%%%%%%%%%%%%%%%%%%%%%%%%%%
%%%%%%%%%%%%%%%%%%%%%%%%%%%% NOTE %%%%%%%%%%%%%%%%%%%%%%%%%%%%

\section{Additional Publications}%
\label{sec:additional-publications}%
%
The author of this thesis has also contributed to the following publications.
The publications have been omitted to improve cohesion and clarity of the thesis' presentation, but also as they cover topics outside the scope of this thesis.

\begin{quote}
    \fullcite{Gunnarsson:2023}
\end{quote}

\begin{quote}
    \fullcite{Kruger:2021}
\end{quote}

\begin{quote}
    \fullcite{Vreman:2020}
\end{quote}

\begin{quote}
    \fullcite{Vreman:2019a}
\end{quote}

\begin{quote}
    \fullcite{Vreman:2019b}
\end{quote}
