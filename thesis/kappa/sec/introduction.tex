\chapter{Introduction}%
\label{ch:intro}%
%%% WHAT
% Introduction to digitalisation
Entering the digital age has forever changed how we interact with the world and how it interacts with us.
Unlike only 20 years ago, from the moment we wake up in the morning till the moment we close our eyes at night, we interact with advanced computer systems.
Our cellphones, work computers, and even our cars contain many computational devices, performing everything from menial tasks, such as checking the weather and accessing mail clients, to safety critical tasks, such as the car's ABS breaks and most of the engine's functionality.
To put the digital growth rate in perspective, the semiconductor\footnote{Semiconductors are components constituting the foundation of generally \emph{all} electronic devices.} market share has more than quadrupled over the last 20 years~\cite{statista:2022}.

% The hardware architecture is getting more powerful and cheaper to use
Not only is the number of computational devices increasing, but their independent capabilities, functionalities, and complexities are growing steadily, all while the cost to buy and manufacture them has become cheaper.
Obviously, the increased efficiency and reduced cost opened up new businesses and domains, in particular within the IT-domain, whilst also consolidating and automating preexisting industry.
Integrating digital components and software solutions is nowadays the norm rather than the exception; this does not come as a surprise, considering that automating and simplifying the decision making and data collection yield both economical and safety benefits.
Generally, integrating software into any domain help monitor system safety, log and transmit important data, orchestrate the execution of different components, and remotely micromanage system updates.
Subsequently, software integration is a powerful tool that both improves efficiency and increases revenue, assuming everything behaves as intended.

% The complexity and reduced prices increase error surfaces
Interconnecting multiple systems is, however, not a trivial task.
As the systems are getting increasingly more complex, the surface for possible errors is also growing.
After connecting two components together, new problems can be encountered in addition to the components' individual faults; for instance, problems with the coupling or new problems in the individual components.
A motorbike can experience all the same problems that a normal bike can encounter (such as a loose chain), but it can also experience issues from connecting the bike together with a motor (such as electric clutch).
Similarly to the motorbike, systems relying on the interconnection of computational devices and digital components can experience complex coupling issues.
For example, data transmissions can easily be delayed or stall indefinitely if data is lost, a computer's orchestrator can get overloaded, and systems with remote updates have the potential to break every time a new patch is installed.
These problems are neither easy to detect nor troubleshoot; particularly since their origin can be obfuscated by complex software and hardware interconnections.

%%% WHY
% Costs
The effects of such errors can be extremely expensive and in the worst case cause companies to lose billions of dollars.
Typically the outcome of system faults is that the normal operation of the device (or machine) is degraded.
The degradation can accumulate over time and either wear down the device or affect the end product.
Obviously, there is a lot of money to be gained by extending the devices' lifetime through proper fault analysis.
Furthermore, if the end product is inferior to the promised product, the consumers would go elsewhere --- no matter whether the product is a service, such as cloud storage, or a physical product, such as a cellphone.

% Safety and security
Arguably more important than the economic consequences are the risks to personal safety, security, and privacy.
One of the modern era's most famous examples are the Boeing's 737 MAX\footnote{\url{https://en.wikipedia.org/wiki/Boeing_737_MAX_groundings}} crashes, killing 346 people in two subsequent crashes.
The crashes were caused by erroneous sensor readings being misinterpreted by the flight control system, resulting in the planes nosediving into the ground.
Another relevant (although less lethal) example is the infamous Stuxnet worm.\footnote{\url{https://en.wikipedia.org/wiki/Stuxnet}}
Stuxnet infiltrated the system controlling the gas centrifuges in multiple Iranian nuclear plants, significantly damaging them whilst also collecting critical information. 

It is generally impossible to guarantee that today's complex computer systems are absolutely safe, secure, and performs according to specifications under all conditions.
Additionally, testing for all possible future problems is expensive and time consuming at best and infeasible in practice.
It is therefore crucial to develop easy-to-use, powerful tools to simplify the analysis of both the systems' performance and safety properties.

%%% STATEMENT & CONSEQUENCES
% Concluding paragraph summarising what the thesis is tackling and what the consequences of this might be
The purpose of this thesis is to provide tools and methods for analysing systems experiencing faults.
In particular, the focus is to analyse software integrated systems where the faults are occurring in the interconnection between software and hardware.
By treating accessibility, clarity, and generalisability as first-class citizens we aim to lower the threshold for using the powerful tools provided.
More specifically, we provide tools to analyse real-time control system performance and stability when the real-time tasks governing the control computations are subject to deadline overruns.
The following subsection introduces the basic context for the real-time control system constituting this thesis' principal theme.

%\question%
%{Add something more to the end here?
%Maybe why this thesis help solve the problems stated in 3rd and 4th paragraph?}%
%{I think you want to add something more here, yes.
%My take would be to add something quite technical at this point, that will be explained later.
%And then add a sentence that says that in the following subsection you will delve into the basic components of the technical sentence that you wrote just above.
%Basically, tell me why I should read the list of things below, otherwise I am not sure why I should be interested.}

\section{Real-Time Control Systems}%
\label{sec:intro:rts}%
%
Fundamentally all systems today contain a certain level of automation, whether it is automatic heat control in buildings or memory allocation in the cloud for storing photos.
The science of making systems automatically behave according to predefined specifications is called \emph{automatic control}.
%\mm{I think I would try to restrict to cyber-physical systems (or control systems, or systems that interact with the environment). So far you are talking about ``all systems'' and I can imagine for example a travel booking website as a ``system''. I am not sure I would say that abstracting such a system (that still is a valid action, or thing to do) is called automatic control.}\nv{But control is everything where we \emph{control} a system using predefined rules, i.e., control can be open-loop control, closed-loop control, logic control, event-based control, you name it. Shouldn't that mean that even the travel booking website is a control system (just a very loosely defined one)?}\nv{Hmm... but in that case the following sentences do not hold anymore.}
Characteristic for many automatic control systems (\emph{control systems} for short) is that they employ \emph{feedback}, i.e., data collected from the system is routed back and used in the decision mechanisms to control the system.
As an example, consider the temperature control in a room, if the actual temperature is known, it can be used (fed back) to determine whether the heating should be turned up or down to meet the desired temperature.
A specific class of control systems are the \emph{real-time control systems}, which are defined by guaranteeing the timely execution of software in the control system.
A common misconception is that real-time systems are inherently very fast; however, the definition only relates the timeliness of the system to a precise notion of correctness.
The real-time system's correctness is then expressed as guarantees that a set of predefined temporal constraints are met.
To enforce the satisfaction of these constraints, a \emph{real-time operating system} (RTOS) is typically employed.

A high-level abstraction of a real-time control system is depicted in Figure~\ref{fig:high-level-abstraction}.
Next, the individual components seen in the figure are introduced. 
%
\begin{figure}[t]
    \centering
    \def \delta {0.15}
\def \circlesizecm {0.5cm}
\def \circleshiftcm {0.125cm}
\def \armlength {0.625}
\def \armwidthcm {0.1cm}
\def \bodywidthcm {0.425cm}

\begin{tikzpicture}
\tikzstyle{task} = [draw,thick,fill=white,align=center]
\tikzstyle{turbine} = [circle,ultra thick,draw,fill=white,minimum size=\circlesizecm,inner sep=0pt,outer sep=0pt]

%%% TASKS %%%

\node[task,opacity=0.3] (t1) at (-1.5+0*\delta,1.6-0*\delta) {\textcolor{white}{Task $\#3$} \\\textcolor{white}{\faFileCode[regular]}};
\node[task,opacity=0.6] (t2) at (-1.5+1*\delta,1.6-1*\delta) {\textcolor{white}{Task $\#2$} \\\textcolor{white}{\faFileCode[regular]}};
\node[task,opacity=1.0] (t3) at (-1.5+2*\delta,1.6-2*\delta) {Task $\#1$ \\\faFileCode[regular]};

\node[task,opacity=0.3] (ct1) at (1+0*\delta,1.6-0*\delta) {\textcolor{white}{Control Task $\#3$} \\\textcolor{white}{\faFileCode[regular]}};
\node[task,opacity=0.6] (ct2) at (1+1*\delta,1.6-1*\delta) {\textcolor{white}{Control Task $\#2$} \\\textcolor{white}{\faFileCode[regular]}};
\node[task,opacity=1.0] (ct3) at (1+2*\delta,1.6-2*\delta) {Control Task $\#1$ \\\faFileCode[regular]};

%%% CYBER %%%

\node[thick, align=center] (rtos) at (-0.1,0.25) {Real-Time Operating System};
\node[thick, draw, align=center, rotate=90, text width=2.75cm] (hwi) at (3.15,0.87) {HW Interfaces};
\node[thick, fit=(rtos)(t1)(ct1)(ct3),draw,yshift=1.5mm,xshift=0.75mm] (sw) {};
\node[thick, draw, above left] (clock) at (sw.south east) {\faClock[regular]};
\node[thick, fit=(sw)(hwi), inner sep=7pt, draw] (hw) {};
\node[thick, above left, xshift=2.3cm, yshift=0.5mm] (hw-label) at (hw.south west) {Hardware};
\node[thick, draw, above right] (hwclock) at (hw.south west)  {\faClock[regular]};

%%% PHYSICAL %%%

\node[task, minimum width=2.125cm, minimum height=2.125cm] (phys) at (6,0.875) {};
% body
\node[
    draw,
    rounded corners=3pt,
    fill=black,
    minimum width=\bodywidthcm,
    minimum height=\bodywidthcm,
    name path=B] (body) at (phys) {};

% upper left turbine
\node[turbine, anchor=south east] (dronenw) at ([xshift=-\circleshiftcm, yshift=\circleshiftcm]body.north west) {};
\draw[name path=NW] ([yshift=-\armwidthcm]body.north west)..controls($(phys) + (-\armlength, \armlength)$)..([xshift=\armwidthcm]body.north west);
\tikzfillbetween [of=NW and B] {};
\draw[fill=black, rotate=75] (dronenw) ellipse (0.175cm and 0.025cm);
\draw[fill=black, rotate=165] (dronenw) ellipse (0.175cm and 0.025cm);
        
% upper right turbine
\node[turbine, anchor=south west] (dronene) at ([xshift=\circleshiftcm, yshift=\circleshiftcm]body.north east) {};
\draw[name path=NE] ([xshift=-\armwidthcm]body.north east)..controls($(phys) + (\armlength, \armlength)$)..([yshift=-\armwidthcm]body.north east);
\tikzfillbetween [of=NE and B] {};
\draw[fill=black, rotate=75] (dronene) ellipse (0.175cm and 0.025cm);
\draw[fill=black, rotate=165] (dronene) ellipse (0.175cm and 0.025cm);

% lower right turbine
\node[turbine, anchor=north west] (dronese) at ([xshift=\circleshiftcm, yshift=-\circleshiftcm]body.south east) {};
\draw[name path=SE] ([yshift=\armwidthcm]body.south east)..controls($(phys) + (\armlength, -\armlength)$)..([xshift=-\armwidthcm]body.south east);
\tikzfillbetween [of=SE and B] {};
\draw[fill=black, rotate=75] (dronese) ellipse (0.175cm and 0.025cm);
\draw[fill=black, rotate=165] (dronese) ellipse (0.175cm and 0.025cm);

% lower left turbine
\node[turbine, anchor=north east] (dronesw) at ([xshift=-\circleshiftcm, yshift=-\circleshiftcm]body.south west) {};
\draw[name path=SW] ([xshift=\armwidthcm]body.south west)..controls($(phys) + (-\armlength, -\armlength)$)..([yshift=\armwidthcm]body.south west);
\tikzfillbetween [of=SW and B] {};
\draw[fill=black, rotate=75] (dronesw) ellipse (0.175cm and 0.025cm);
\draw[fill=black, rotate=165] (dronesw) ellipse (0.175cm and 0.025cm);

% Clock
\node[task, above left] (time) at (phys.south east) {\faClock[regular]};

%%% ARROWS %%%

\draw[thick, -latex] ([yshift=0.65cm]hwi.south) to node[yshift=0.85cm,xshift=1mm,rotate=90] {Actuation} ([yshift=0.65cm]phys.west);
\draw[thick, -latex] ([yshift=-0.65cm]phys.west) to node[yshift=-0.75cm,xshift=1mm,rotate=90] {Sensing} ([yshift=-0.65cm]hwi.south);

\end{tikzpicture}
%
    \caption{A control system represented at a high level of abstraction.
    The plant is represented on the right and its digital control structure is shown on the left.
    The control structure comprises hardware and its interfaces with the plant as well as a real-time operating system and its running tasks, among which are the control tasks.}%
    \label{fig:high-level-abstraction}%
\end{figure}

\subsubsection{Plant}% Plant, sensors, HW interface
%
The right part of the figure depict the process we are trying to control (denoted the \emph{plant}).
This could be anything from an airplane's taxiing system, indoor heating systems, or the load on a server in a data centre.
In this and following chapters, the taxiing system will be used as a recurring example to illustrate the different concepts.
The arrows going to and from the plant indicate the flow of data; actuation data goes into the plant and sensor data is collected from the plant.
Actuation data refers to the commands sent to the components responsible for movement or change in the plant, i.e., the actuators.
Similarly, the sensor data is all information collected by the sensors, e.g., the rotational velocity of the wheel or the friction between the wheel and the ground in the taxiing system.
These signals are transmitted via the hardware interfaces on the computational unit responsible for controlling the plant.
Historically, these signals were transmitted via wire, but in the last couple of decades wireless communication has become more common~\cite{Park:2018}.

A communication protocol\footnote{A communication protocol is a set of rules setup in order for two or more actors in a network to be able to transmit information to one another. The rules include (but are not limited to) semantics, i.e., how to format the information, and synchronisation, i.e., how much and how fast the information is transmitted.} is used for the plant to communicate with the hardware interfaces.
The choice of protocol is domain dependent, e.g., the controller area network (CAN) is very common in the automotive industry~\cite{Voss:2005}.
There exists a plethora of domain specific communication protocols, but some established ones include Profibus, Modbus, Ethernet/IP, and (as already mentioned) CAN.

\subsubsection{Hardware}% Hardware
%
Depending on the application, the hardware used to control the plant can be anything from a logic-based system (e.g., programmable logic controllers) to a general purpose computer (e.g., laptops or server systems).
We mainly refer to \emph{microcontrollers} (MCUs), i.e., small computers with integrated memory, central processing units (CPUs), graphical processing units (GPUs), and programmable input/output peripherals (PIOs) all on a single chip; however, we emphasise that the presented results are not bound to a specific hardware architecture.
It is also common to connect multiple levels of control hardware together.
For instance, having a high-level trajectory planner communicate with a low-level control structure whose objective is to enforce that the desired trajectory is followed, e.g., the taxiways on the airfield.

The choice of computer architecture is often flexible and can be changed depending on the plant structure.
It is for instance not uncommon that the plant sensors include their own MCU to perform data processing before transmitting it to the central control hardware~\cite{Karray:2018}.
Another common architecture choice is having many single objective, specialised MCU nodes operating together towards a global target, e.g., controlling the rotational velocity of the individual wheels of a plane to move it along the taxiway.
%\question{Write something more about this, like distributed control?}{Yes. I think you want to state that depending on the hardware that is chosen the computer architecture can be distributed, and there could be for example sensors with their own MCUs, that execute some operations on the data before sending it to a controller etc.}

\subsubsection{Clocks}% Clocks
%
There exists another discrepancy between the components of the real-time control system: the time quantisation.
The plant, hardware, and RTOS are most likely executing in different time scales, where the plant (at least in the taxiing example) is likely to execute in continuous time while the hardware and RTOS are both executing in (different) discrete quanta.
Extremely simplified, the hardware contains a \emph{clock} that measures the progress of physical time in \emph{ticks} rather than seconds (like a wall-clock would).
For the hardware clock, these ticks are physical events (typically the oscillations of a crystal) occurring with a known frequency.
The RTOS clock is then based on the hardware clock; it reads the hardware clock at a frequency specified by the user and wraps it in a virtual layer to improve applicability. 
The granularity of the RTOS clock is thus quite coarse, which in turn introduces \emph{release jitter} and \emph{execution time variations} on the tasks.
Additionally, the inconsistencies between different clocks is a difficult problem and it has in fact warranted its own research domain, i.e., clock synchronisation.

\subsubsection{Real-Time Operating System}% RTOS, Tasks, and control task
%
Commonly, real-time control systems rely on a real-time operating system to schedule and constraint the temporal execution of a set of \emph{tasks}.
Each task is assigned a \emph{priority} (i.e., a value to describe how important the task is), a dedicated function to execute (e.g., compute a value, transmit data, or log data), and a \emph{deadline} before which the function is supposed to complete its execution.
Typically there exists more than one task executing in the RTOS.
To guarantee that each task is assigned the correct amount of processor time, a \emph{scheduler} is used to orchestrate the tasks' execution.
Specifically, the scheduler
\begin{enumerate*}[label = (\roman*)]
    \item swaps tasks in and out,
    \item wakes up tasks that are currently not executing but should start executing, and
    \item interrupt tasks that are currently executing when something with higher priority requires the processor.
\end{enumerate*}
The orchestration is based on a scheduling algorithm, where some of the most popular algorithms include: fixed-priority, earliest-deadline-first, and round-robin.
Additionally, the tasks are not supposed to have any information about the RTOS orchestration and it is thus the RTOS responsibility to ensure that the task has access to its own \emph{context}, i.e., the resources it requires. 

The time it takes for a task to finish executing its corresponding function can vary greatly between iterations.
For instance, if a task contains different conditional branches\footnote{If a function's behaviour change depending on a logical condition, it is said to have conditional branches.} its execution time may be dependent on the branch taken.
To quantify a task's execution time, the simplest task models approximate it as the \emph{worst-case execution time} (WCET), i.e., the maximum length of time the task could execute on the specific hardware.
The WCET is typically pessimistic, but it is also important for for guaranteeing reliability in safety-critical real-time systems.

Since there may exist many tasks in the RTOS and the scheduler can swap them in and out arbitrarily, there are no guarantees that a task will execute its entire function consecutively.
Firstly, as tasks can be dependent of one another, one task may have to wait for another task to change its state (or compute a value) before completing its execution.
Secondly, if the executing task gets \emph{preempted} by the scheduler in favour of another higher-priority process (e.g., an interrupt or a higher-priority task), it will again have to wait for the scheduler to switch it back in.
The maximum length of time from that a task starts executing until it finishes is called the \emph{worst-case response time} (WCRT).
If a task's WCRT is smaller than its deadline, the task will be guaranteed to always meet its deadline.

\subsubsection{Controller}% Controller and Actuator
%
As can be seen in Figure~\ref{fig:high-level-abstraction}, we conceptually distinguish \emph{control tasks} from normal tasks, even though there is no discernible difference between these tasks from the RTOS perspective.
The control tasks are all the tasks responsible for controlling the plant, while the normal tasks take care of everything else.
Taking the taxiing system as an example, the control tasks can be two tasks where one is controlling the velocity of the landing gear while the other is controlling its retraction mechanism.
Additionally, there are other tasks communicating sensor data (e.g., tire-pressure, temperature, etc.) to the aircraft's central computer.

As is elaborated upon in Chapter~\ref{ch:background}, we partition the tasks in these two categories because we are specifically interested in analysing the control tasks' effect on the real-time control system when their temporal execution is unreliable.
In particular, the aim of the thesis is to investigate the control system's behaviour when the control tasks overrun their respective deadlines.
This is particularly relevant for control systems, because enforcing that the control task's WCRT is shorter than its deadline (i.e., that the task \emph{never} overruns its deadline) would involve postponing the deadline.
Holding off the deadline results in fewer control updates, hence severely degrading the control system's performance.
Consequently, allowing a certain number of deadline overruns can improve the control system's overall performance.

\subsection{Timeliness}%
\label{sec:intro:timeliness}%
%
In some situations, real-time control systems are executed under suboptimal timing conditions.
As already mentioned, faults typically lead to reduced quality of the end product and can in the worst case be lethal.
However, the term ``timing faults'' is an oversimplification of a complex class of problems.
When analysing timing faults and irregularities it is thus important to properly define the fault type under analysis.
The following list include some of the most general timing irregularities analysed in literature.
%
\begin{itemize}
    \item \emph{Computational overruns} -- When there exists computational elements in the system, i.e., something that takes the current state of the system and transforms or translates it, there is the possibility that it will not complete its execution, either on time or at all. % TODO: Skeptical about this statement
        Generally, this is connected to a real-time system if the computation has to complete before a predetermined \emph{deadline}.

    \item \emph{Time delays} -- Time delays might be some of the oldest timing problems analysed in the literature.
        Nowadays, time delays involve both internal time delays and input/output delays, i.e., respectively when the plant includes time delays or when the sensing, actuation, or control contains time delays.
        The control community has developed ways to both analyse and design controllers for systems with known time delays~\cite{Mirkin:2004, Mirkin:2005}.

    \item \emph{Jitter} -- Variations in the time delay are called jitter.
        Note that the previously mentioned \emph{release jitter}, originating from the coarse granularity of the RTOS clock, is just \emph{one} type of jitter.
        Another type of jitter appear when transmitting data over a network; if the packet latency (time delay) is constant in time, there is no jitter in the system.
        Instead, if the latency varies over time (as it usually does), there exists jitter in the system.
        The jitter intensity depends on the size of the latency fluctuations.
        In the control literature, methods to compensate for jitter have been proposed~\cite{Cervin:2004}.

    \item \emph{Communication losses} -- If data packets transmitted over a network are lost along their route or too delayed to be useful, they are considered \emph{lost}.
        Another case when the packets are dropped is when the network is shut down or overloaded, either by too much traffic or an attack.
        The packet loss can occur both on the way to and from the hardware.
\end{itemize}
%
Conceptually, it may be natural to view some of the timing irregularities as equivalent from a system-wide perspective.
Distinguishing time delays from jitter or communication losses from computation overruns in this context is difficult.
For instance, both communicational losses and computational overruns affect the underlying software implementation; thus, it is only natural to assume that one can be substituted for the other when analysing the system in order to save both time and effort.
However, the different models hold unique information that does not necessarily overlap.
It is therefore crucial to analyse all the appropriate models in order to get a comprehensive picture of the specific system's behaviour.

This thesis specifically targets systems subject to computational overruns, and (to a minor extent) communication losses.
Of particular interest are the problems that occur in real-time control systems due to control tasks overrunning their corresponding deadlines.
It is convenient to blame computational overruns on poor system design and programming errors; however, even a flawless system design can experience overruns due to, for instance, cache memory misses~\cite{Gracioli:2015} or radiation-induced faults~\cite{Tsog:2021}.
More commonly, it is known that the real-time control system's nominal performance is degraded if a design that completely avoids overruns is used, i.e., if the deadline is postponed until after the WCRT of the controller.
Thus, with the awareness that it can cause transient faults, the control system engineers tolerate a few overruns when the control task's execution time is near its worst case, in order to improve the system's nominal performance.



\section{Outline}%
\label{sec:intro:outline}%
%
This thesis is a collection of papers and is divided into two parts.
The first part comprises the first three chapters and serves both as a summary and as an extension of the motivation for the research work.
The second part includes the papers that constitute the major scientific contribution of this thesis.

The first chapter presented a high-level introduction to the relevant concepts and problems.
Next, Chapter~\ref{ch:background} provide a rigorous presentation of the concepts that were loosely introduced in Chapter~\ref{ch:intro}.
In addition to the proper problem description, the chapter also outlines the relevant background and related work from both the real-time and control theoretical domains.
For each of the five papers constituting the main part of the thesis, Chapter~\ref{ch:contribution} contains a short summary of its content, a brief description of its scientific contribution, and the respective authors' individual contribution.
