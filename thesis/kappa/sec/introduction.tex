\chapter{Introduction}%
\label{ch:intro}%

\nv{%
In this section, I'll try to portray my ``elevator pitch''.

\textbf{Pitch:}
\begin{enumerate}
    \item \label{item:pitch1} What is the problem this thesis is trying to solve?
        \begin{itemize}
            \item The hardware and system architecture is getting more powerful.
            \item This increases the complexity of the designed system and increases the error probability.
            \item Especially since more and more junior developers can use this powerful and complicated hardware for their designs.
        \end{itemize}
    \item Why is the problem an important one to analyse?
        \begin{itemize}
            \item This can introduce huge risks in safety critical systems, risking lives and damage to humans
            \item Can cause major damage on infrastructure and commercial systems
            \item Security issues are introduced, leaking private data
        \end{itemize}
    \item Give a positive and bold statement of what this thesis is doing.
        \begin{itemize}
            \item We present novel theoretical and practically applicable analytical tools to help analyse fault-prone systems
            \item We focus on making these tools easy to use, lowering the threshold for developers to use them in everyday life
            \item \textbf{If nothing else is remembered from the thesis, this should be it.}
        \end{itemize}
    \item The consequences of the thesis (preferably ``solve'' the problem in~\ref{item:pitch1})
        \begin{itemize}
            \item By applying the tools presented in this thesis to embedded real-time control systems, we can identify and mitigate a lot of trivial bugs and errors that occur.
        \end{itemize}
\end{enumerate}
}%

\subsection*{Embedded Systems}%
%
\begin{figure}[t]
    \centering
    \def \delta {0.15}

\begin{tikzpicture}
\tikzstyle{task} = [draw,thick,fill=white,align=center]

%%% TASKS %%%

\node[task,opacity=0.4] (t1) at (-2+0*\delta,1.6-0*\delta) {Task $\#1$ \\\faFileCode[regular]};
\node[task,opacity=0.7] (t2) at (-2+1*\delta,1.6-1*\delta) {Task $\#2$ \\\faFileCode[regular]};
\node[task,opacity=1.0] (t3) at (-2+2*\delta,1.6-2*\delta) {Task $\#3$ \\\faFileCode[regular]};

\node[task,opacity=0.4] (ct1) at (1+0*\delta,1.6-0*\delta) {Control-Task $\#1$ \\\faFileCode[regular]};
\node[task,opacity=0.7] (ct2) at (1+1*\delta,1.6-1*\delta) {Control-Task $\#2$ \\\faFileCode[regular]};
\node[task,opacity=1.0] (ct3) at (1+2*\delta,1.6-2*\delta) {Control-Task $\#3$ \\\faFileCode[regular]};

%%% CYBER %%%

\node[thick, align=center] (rtos) at (-0.2,0.15) {Real-Time Operating System};
\node[thick, draw, align=center, rotate=90, text width=2.92cm] (hw) at (3.15,0.775) {HW interfaces};
\node[thick, fit=(rtos)(t1)(ct1)(ct3),draw,yshift=1.5mm] (sw) {};
\node[thick, draw, above left] (clock) at (sw.south east) {\faClock[regular]};
\node[thick, fit=(sw)(hw), inner sep=5pt, draw] (board) {};
\node[thick, above left, xshift=1.8cm] (borad-label) at (board.south west) {Board};
\node[thick, draw, above right] (clockboard) at (board.south west)  {\faClock[regular]};

%%% PHYSICAL %%%

\node[thick, draw ,align=center] (phys) at (6,0.775) {\includegraphics[scale=4]{\figdir/airplane.jpg}};
\node[thick, draw, above left] (time) at (phys.south east) {\faClock[regular]};

%%% ARROWS %%%

\draw[-latex] ([yshift=0.65cm]hw.south) to node[yshift=0.85cm,rotate=90]{actuation} ([yshift=0.65cm]phys.west);
\draw[-latex] ([yshift=-0.65cm]phys.west) to node[yshift=-0.85cm, rotate=90]{sensing} ([yshift=-0.65cm]hw.south);

\end{tikzpicture}
%
    \caption{\fix{Add caption, and maybe change font?}}%
    \label{fig:embedded-system}%
\end{figure}

\nv{%
Figure (somewhere) of a general embedded setup with a brief introduction to the different
parts of the figure (nothing technical).
}%

\nv{%
\textbf{The What:}
\begin{itemize}
    \item Connectivity, Distributed systems, NCS, IoT, Wireless
    \item Communication hiccups
    \item packet drops
    \item deadline misses
\end{itemize}
}%

\nv{%
\textbf{The why:}
\begin{itemize}
    \item Huge costs, damages
    \item Safety
    \item Security
    \item Paper by~\cite{Akesson:2020}
\end{itemize}
}%


\subsection*{Control Theoretical Benefits}%
%
\nv{%
Short introduction to what has historically been done to mitigate the problems
}%

\nv{%
\textbf{The Statement:}
\begin{itemize}
    \item Theoretical and practical tools for analysing and synthesising
        fault-tolerant controllers, with significant performance gains.
    \item Lowering the entry-level knowledge for applying the tools to real
        processes.
\end{itemize}
}%

\nv{%
\textbf{The Consequences:}
\begin{itemize}
    \item Decoupling and clarifying real-time and control theoretical models.
    \item Making them more accessible to a broader class of practitioners.
\end{itemize}
}%

\subsection*{Outline}%
%
Thesis outline.
