\chapter{Background}%
\label{ch:background}%

This chapter presents the necessary background and motivation for the remainder of the thesis.
We divide the chapter in two primary parts.
First, a discussion on the real-time theoretical aspects is provided.
An introduction to how real-time operating system operates is presented, e.g., processor sharing, task states, scheduling strategies, etc.
However, the main focus is dedicated to the most commonly used task models and their respective advantages and disadvantages, with respect to deadline overruns.
Additionally, we provide a brief discussion on state-machine applicability to the aforementioned task models. 
Next, the relevant control theoretical background is presented based on the theory of real-time systems.
Two different system modelling approaches, relevant to real-time control systems, are introduced: switching systems and Markov jump linear systems.
For these systems, we present and discuss different stability and performance analyses.

\section{Real-Time Systems}%
\label{sec:background:rts}%
%


\nv{%

\section*{Embedded real-time control Systems}%
%
\begin{itemize}
    \item Refer to figure from Chapter~\ref{ch:intro} and then ``zoom'' in on the
        different aspects treated in the different subsections.
    \item Simple description of hardware components: Plant, Sensors, Network
        (wired/wireless), control hardware, actuators.
    \item Simple description of software components: network protocol, RTOS,
        tasks, memory, interrupts, etc.
    \item Maybe create a simple practical example (e.g., taxi of a plane) that
        can be followed throughout this section.
\end{itemize}


\subsection*{Real-Time Model}%
%
\begin{itemize}
    \item Start with a historical perspective
    \item Hard, Soft, Weakly-Hard, More expressive (\cite{Stigge:2011}), other?
    \item Describe advantages and disadvantages with each of the models
    \item recall great example of real-time system in rust book
\end{itemize}

\subsubsection*{Modelling Execution using finite state-machine}%
%
\begin{itemize}
    \item This section should maybe be moved?
    \item Entire subsubsection requested by KE
    \item More automata theory
    \item "typically in control fsm have been used to design highlevel control
        (e.g., taxi takeoff and landing), in principle (computer science)
        automata have been used to represent more complicated things (for
        instance regular languages). This is the basis of what the WeaklyHard.jl
        does."
    \item Include markov theory here. "In CS when the transition was
        non-deterministic, i.e., probabilistic, then you have the concept of
        Markov chains."
\end{itemize}


\subsection*{Control System Model}%
%
\begin{itemize}
    \item Start from plant, non-linear model, linearisation
    \item Sensors and actuator models included here.
    \item Control model (non-linear, more common ones), relate to real-time
        tasks
    \item Switching systems! Markov Jump Linear Systems!
\end{itemize}


\subsubsection*{Stability Analysis}%
%
\begin{itemize}
    \item Binary: Stable or not
    \item Stability definitions: nominal, MS, MSS, JSR, Lyapunov
    \item differences (e.g., JSR vs. Lyapunov), applicability
\end{itemize}


\subsubsection*{Performance Analysis}%
%
\begin{itemize}
    \item Gradient: varying degree of performance
    \item Why Performance Analysis?
    \item Metrics
\end{itemize}

}%


% MENTION SOMETHING ABOUT {SYNCHRONOUS, ASYNCHRONOUS, PERIODIC, APERIODIC} TASKS

% MENTION TASK STATES (INTERNAL AND EXTERNAL)
