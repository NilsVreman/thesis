%%%%%%%%%%%%%%%%%%%%%%%%%%%%%%%%%%%%%%%%%%%%%%%%%%%%%%%%%%%%%%%%%%%%%%%%%%%%%%%%
% In this file, only packages are allowed. These packages should be explained to
% greatest possible extent.
%%%%%%%%%%%%%%%%%%%%%%%%%%%%%%%%%%%%%%%%%%%%%%%%%%%%%%%%%%%%%%%%%%%%%%%%%%%%%%%%

% Document encodings
\usepackage[english]{babel}
\usepackage[utf8]{inputenc}
\usepackage[T1]{fontenc} % This can slightly change font appearance
\usepackage{xcolor} 

% Beamer theme
\usepackage{style/beamertheme} % Needes XeLaTeX

% Math related
\usepackage{amsmath, amssymb, amsthm, mathtools}
\usepackage{xfrac} %for nice inline fractions

% Char. related 
\usepackage{microtype}

% Bibliography related
\usepackage{cite}

\usepackage{multirow} % For multirow tables
\usepackage{colortbl} % For coloured tables

% Colordefinitions
\definecolor{misscolour}{RGB}{255, 0, 0}
\definecolor{lqrcolour}{RGB}{0,28,255}
\definecolor{lqrnomcolour}{RGB}{0,28,255}
\definecolor{lqgcolour}{RGB}{0,139,0}
\definecolor{lqgnomcolour}{RGB}{0,139,0}
\definecolor{adacolour}{RGB}{239,133,16}

\definecolor{baselinecolor}{rgb}{0.9, 0.78, 0.07}
\definecolor{markcolor}{rgb}{0.6, 0.64, 0.69}

% Tikz packages and related
\usepackage{tikz} 
\usepackage{pgfplots, pgfplotstable} 
\usepackage{fontawesome5} 

% Tikz Libraries
\usepgfplotslibrary{external}
    \tikzexternalize[prefix=tikz/]
    \tikzset{external/system call={xelatex \tikzexternalcheckshellescape
        -halt-on-error -interaction=batchmode -jobname "\image" "\texsource"}
    } % to let pdflatex work

\usepgfplotslibrary{groupplots}
%\usepgfplotslibrary{fillbetween}
\usetikzlibrary{decorations.markings}
\usetikzlibrary{shapes}
\usetikzlibrary{backgrounds}
\usetikzlibrary{fit}

\usetikzlibrary{calc}
\usetikzlibrary{arrows}
\usetikzlibrary{arrows.meta}
\usetikzlibrary{patterns}
\usetikzlibrary{shapes.misc}
%\pgfplotsset{compat=1.16}

%\usepgfplotslibrary{fillbetween}
\usetikzlibrary{positioning}
%\usepackage{makecell} 

%%% RTAS22B
\tikzset{Dom Node/.style={draw,
                        thick,
                        circle,
                        inner sep=0pt,
                        minimum size=12mm}%
}%
\tikzset{
    old inner xsep/.estore in=\oldinnerxsep,
    old inner ysep/.estore in=\oldinnerysep,
    Init Node/.style 2 args={draw,
                    thick,
                    circle,
                    minimum size=12mm,
                    old inner xsep=\pgfkeysvalueof{/pgf/inner xsep},
                    old inner ysep=\pgfkeysvalueof{/pgf/inner ysep},
                    /pgf/inner xsep=\oldinnerxsep+#1,
                    /pgf/inner ysep=\oldinnerysep+#1,
                    alias=sourcenode,
                    append after command={
                    let     \p1 = (sourcenode.center),
                            \p2 = (sourcenode.east),
                            \n1 = {\x2-\x1-#1-0.5*\pgflinewidth}
                    in
                        node [inner sep=0pt, draw, circle, minimum width=2*\n1,at=(\p1),#2] {}
                    }
    },
    Init Node/.default={2pt}{black}%
}%

%%% Comparison figure related
\def\xstart{1}
\def\xend{10} % change according to how many plots you have

% this is the list of styles
% define as many colours as the number of lines
% you could also change the marker etc
\pgfplotscreateplotcyclelist{blue10}{
    {blue!95!black, mark=*, mark size=2pt,mark options={fill=white}},
    {blue!85!black, mark=*, mark size=2pt},
    {blue!75!black, mark=*, mark size=2pt},
    {blue!65!black, mark=*, mark size=2pt},
    {blue!55!black, mark=*, mark size=2pt},
    {blue!45!black, mark=*, mark size=2pt},
    {blue!35!black, mark=*, mark size=2pt},
    {blue!25!black, mark=*, mark size=2pt},
    {blue!15!black, mark=*, mark size=2pt},
    {blue! 5!black, mark=*, mark size=2pt},
}

% argument #1: any options
\makeatletter
\newenvironment{customlegend}[1][]{%
    \begingroup
    % inits/clears the lists (which might be populated from previous
    % axes):
    \pgfplots@init@cleared@structures
    \pgfplotsset{#1}%
}{%
    % draws the legend:
    \pgfplots@createlegend
    \endgroup
}%

% makes \addlegendimage available (typically only available within an
% axis environment):
\def\addlegendimage{\pgfplots@addlegendimage}
\makeatother

%%% ECRTS 
\tikzset{cross/.style={%
    cross out,
    draw,
    minimum size=2*(#1-\pgflinewidth),
    inner sep=0pt, outer sep=0pt}}

\makeatletter
\pgfplotsset{
    every axis plot/.append style =
    {mark=none, mark options={fill=white}},
    mark max/.style={
        scatter/@pre marker code/.code={%
            \ifx\pgfplotspointmeta\pgfplots@metamax
                \def\markopts{}%
                \node [anchor=south, xshift=1mm, yshift=-0.4mm] { \pgfmathprintnumber[precision=1, fixed zerofill]{\pgfplotspointmeta} };
            \else
                \def\markopts{mark=none}
            \fi
                \expandafter\scope\expandafter[\markopts]
        },%
        scatter/@post marker code/.code={ \endscope },
        scatter
    }
}
% Style to select only points from #1 to #2 (inclusive)
\pgfplotsset{select coords between index/.style 2 args={
    x filter/.code={
        \ifnum\coordindex<#1\def\pgfmathresult{}\fi
        \ifnum\coordindex>#2\def\pgfmathresult{}\fi
    }
}}
\makeatother

\newcommand{\findmax}[1]{
    \pgfmathsetmacro\buffer{0.0}
    \pgfplotstableforeachcolumnelement{#1}\of\extdata\as\cellValue{%
        \pgfmathsetmacro{\buffer}{max(\buffer,\cellValue)}}
}
\newcommand*{\ReadOutElement}[4]{%
    \pgfplotstablegetelem{#2}{[index]#3}\of{#1}%
    \let#4\pgfplotsretval
}
