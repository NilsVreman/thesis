\pgfplotstableread[header=false,col sep=comma]{figs/ecrts21/data/KH-stability.csv}\kh
\pgfplotstableread[header=false,col sep=comma]{figs/ecrts21/data/KZ-stability.csv}\kz
\pgfplotstableread[header=false,col sep=comma]{figs/ecrts21/data/SH-stability.csv}\sh
\pgfplotstableread[header=false,col sep=comma]{figs/ecrts21/data/SZ-stability.csv}\sz

% they all have the same number of columns and rows
\pgfplotstablegetrowsof{\kh}
\pgfmathtruncatemacro{\numrows}{\pgfplotsretval}
\pgfplotstablegetcolsof{\kh}
\pgfmathtruncatemacro{\numcols}{\pgfplotsretval}

\centering

\begin{tikzpicture}[scale=0.09]
\small

\draw[thick] (0,0) -- (0,50) -- (25,50) -- (25,0) -- cycle;
\draw[dashed] (-1,0.5) node[left] {$1$} -- (0,0.5);
\draw[dashed] (-1,9.5) node[left] {$10$} -- (0,9.5);
\draw[dashed] (-1,19.5) node[left] {$20$} -- (0,19.5);
\draw[dashed] (-1,29.5) node[left] {$30$} -- (0,29.5);
\draw[dashed] (-1,39.5) node[left] {$40$} -- (0,39.5);
\draw[dashed] (-1,49.5) node[left] {$50$} -- (0,49.5);

\draw[dashed] (0.5,-1) node[below] {$1$} -- (0.5,0);
\draw[dashed] (4.5,-1) node[below] {$5$} -- (4.5,0);
\draw[dashed] (9.5,-1) node[below] {$10$} -- (9.5,0);
\draw[dashed] (14.5,-1) node[below] {$15$} -- (14.5,0);
\draw[dashed] (19.5,-1) node[below] {$20$} -- (19.5,0);
\draw[dashed] (24.5,-1) node[below] {$25$} -- (24.5,0);

\foreach \Y [evaluate=\Y as \PrevY using {int(\numrows-\Y)},
evaluate=\Y as \NewY using {int(\numrows-\Y+1)}] in {1,...,\numrows}{
\foreach \X  [evaluate=\X as \PrevX using {int(\X-1)}] in {1,...,\numcols}{
  \ReadOutElement{\kh}{\PrevY}{\PrevX}{\Current}
  % our class
  \ifnum\Current=0 \def\colorcell{white} \fi
  \ifnum\Current=2 \def\colorcell{blue!30} \fi
  \ifnum\Current=3 \def\colorcell{blue} \fi

  \draw[black,densely dotted,fill=\colorcell] (\PrevX,\numrows-\PrevY-1) rectangle +(1,1);
  }
}
% title and axis
\node at (12.5,52.5) {\textbf{Kill\&Hold}};
%\node[rotate=90] at (-8.5,25) {$m$};
\node[] at (12.5,-8.5) {$n$};
\node[rotate=90] at (-9.5,25) {$m$};
\end{tikzpicture}
\hspace{-1mm}
\begin{tikzpicture}[scale=0.09]
\small

\draw[thick] (0,0) -- (0,50) -- (25,50) -- (25,0) -- cycle;
\draw[dashed] (-1,0.5) node[left] {$1$} -- (0,0.5);
\draw[dashed] (-1,9.5) node[left] {$10$} -- (0,9.5);
\draw[dashed] (-1,19.5) node[left] {$20$} -- (0,19.5);
\draw[dashed] (-1,29.5) node[left] {$30$} -- (0,29.5);
\draw[dashed] (-1,39.5) node[left] {$40$} -- (0,39.5);
\draw[dashed] (-1,49.5) node[left] {$50$} -- (0,49.5);

\draw[dashed] (0.5,-1) node[below] {$1$} -- (0.5,0);
\draw[dashed] (4.5,-1) node[below] {$5$} -- (4.5,0);
\draw[dashed] (9.5,-1) node[below] {$10$} -- (9.5,0);
\draw[dashed] (14.5,-1) node[below] {$15$} -- (14.5,0);
\draw[dashed] (19.5,-1) node[below] {$20$} -- (19.5,0);
\draw[dashed] (24.5,-1) node[below] {$25$} -- (24.5,0);

\foreach \Y [evaluate=\Y as \PrevY using {int(\numrows-\Y)},
evaluate=\Y as \NewY using {int(\numrows-\Y+1)}] in {1,...,\numrows}{
\foreach \X  [evaluate=\X as \PrevX using {int(\X-1)}] in {1,...,\numcols}{
  \ReadOutElement{\sh}{\PrevY}{\PrevX}{\Current}
  % our class
  \ifnum\Current=0 \def\colorcell{white} \fi
  \ifnum\Current=2 \def\colorcell{pink} \fi
  \ifnum\Current=3 \def\colorcell{pink!75!black} \fi

  \draw[black,densely dotted,fill=\colorcell] (\PrevX,\numrows-\PrevY-1) rectangle +(1,1);
  }
}
% title and axis
\node at (12.5,52.5) {\textbf{Skip\&Hold}};
%\node[rotate=90] at (-8.5,25) {$m$};
\node[] at (12.5,-8.5) {$n$};
\end{tikzpicture}
\hspace{-1mm}
\begin{tikzpicture}[scale=0.09]
\small

\draw[thick] (0,0) -- (0,50) -- (25,50) -- (25,0) -- cycle;
\draw[dashed] (-1,0.5) node[left] {$1$} -- (0,0.5);
\draw[dashed] (-1,9.5) node[left] {$10$} -- (0,9.5);
\draw[dashed] (-1,19.5) node[left] {$20$} -- (0,19.5);
\draw[dashed] (-1,29.5) node[left] {$30$} -- (0,29.5);
\draw[dashed] (-1,39.5) node[left] {$40$} -- (0,39.5);
\draw[dashed] (-1,49.5) node[left] {$50$} -- (0,49.5);

\draw[dashed] (0.5,-1) node[below] {$1$} -- (0.5,0);
\draw[dashed] (4.5,-1) node[below] {$5$} -- (4.5,0);
\draw[dashed] (9.5,-1) node[below] {$10$} -- (9.5,0);
\draw[dashed] (14.5,-1) node[below] {$15$} -- (14.5,0);
\draw[dashed] (19.5,-1) node[below] {$20$} -- (19.5,0);
\draw[dashed] (24.5,-1) node[below] {$25$} -- (24.5,0);

\foreach \Y [evaluate=\Y as \PrevY using {int(\numrows-\Y)},
evaluate=\Y as \NewY using {int(\numrows-\Y+1)}] in {1,...,\numrows}{
\foreach \X  [evaluate=\X as \PrevX using {int(\X-1)}] in {1,...,\numcols}{
  \ReadOutElement{\kz}{\PrevY}{\PrevX}{\Current}
  % our class
  \ifnum\Current=0 \def\colorcell{white} \fi
  \ifnum\Current=2 \def\colorcell{cyan!30} \fi
  \ifnum\Current=3 \def\colorcell{cyan} \fi

  \draw[black,densely dotted,fill=\colorcell] (\PrevX,\numrows-\PrevY-1) rectangle +(1,1);
  }
}
% title and axis
\node at (12.5,52.5) {\textbf{Kill\&Zero}};
\node[] at (12.5,-8.5) {$n$};
\end{tikzpicture}
\hspace{-1mm}
\begin{tikzpicture}[scale=0.09]
\small

\draw[thick] (0,0) -- (0,50) -- (25,50) -- (25,0) -- cycle;
\draw[dashed] (-1,0.5) node[left] {$1$} -- (0,0.5);
\draw[dashed] (-1,9.5) node[left] {$10$} -- (0,9.5);
\draw[dashed] (-1,19.5) node[left] {$20$} -- (0,19.5);
\draw[dashed] (-1,29.5) node[left] {$30$} -- (0,29.5);
\draw[dashed] (-1,39.5) node[left] {$40$} -- (0,39.5);
\draw[dashed] (-1,49.5) node[left] {$50$} -- (0,49.5);

\draw[dashed] (0.5,-1) node[below] {$1$} -- (0.5,0);
\draw[dashed] (4.5,-1) node[below] {$5$} -- (4.5,0);
\draw[dashed] (9.5,-1) node[below] {$10$} -- (9.5,0);
\draw[dashed] (14.5,-1) node[below] {$15$} -- (14.5,0);
\draw[dashed] (19.5,-1) node[below] {$20$} -- (19.5,0);
\draw[dashed] (24.5,-1) node[below] {$25$} -- (24.5,0);

\foreach \Y [evaluate=\Y as \PrevY using {int(\numrows-\Y)},
evaluate=\Y as \NewY using {int(\numrows-\Y+1)}] in {1,...,\numrows}{
\foreach \X  [evaluate=\X as \PrevX using {int(\X-1)}] in {1,...,\numcols}{
  \ReadOutElement{\sz}{\PrevY}{\PrevX}{\Current}
  % our class
  \ifnum\Current=0 \def\colorcell{white} \fi
  \ifnum\Current=2 \def\colorcell{red!30} \fi
  \ifnum\Current=3 \def\colorcell{red} \fi

  \draw[black,densely dotted,fill=\colorcell] (\PrevX,\numrows-\PrevY-1) rectangle +(1,1);
  }
}
% title and axis
\node at (12.5,52.5) {\textbf{Skip\&Zero}};
%\node[rotate=90] at (-8.5,25) {$m$};
\node[] at (12.5,-8.5) {$n$};
\end{tikzpicture}
